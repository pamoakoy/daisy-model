%%% introduction.tex --- Daisy Reference Manual introduction.

\chapter{Introduction}
\label{cha:introduction}

Welcome to the Daisy Program Reference Manual.  This manual provides
you with a reference to the Daisy software.  That is, it describes the
command line version of Daisy, and the format of the setup files.  As
a start, let's look at what this manual isn't.
\begin{itemize}
\item It is not an introduction to the Daisy simulation model itself.
  See \cite{daisy-new}, \cite{daisy-def} and \cite{daisy-fertilizer}
  instead.
\item It is not a reference manual for the mathematical or physical
  aspects of the various models in Daisy.  The models are listed
  together with their parameters, but not described in any detail.
  Some models are explained in \cite{daisy-new}, \cite{daisy-def} and
  \cite{daisy-fertilizer}. 
\item It is not a tutorial for the Daisy software.  The manual is
  intended as a reference for people who already know the system.  A
  tutorial can be found in~\cite{daisy-tut}.
\item It is not a programmers manual.  See \cite{daisy-guide} for how
  to add new models to Daisy, or \cite{c-api} for a description of how
  to add Daisy to larger systems.  An overview of the program design
  can be found in \cite{daisy-ems}.
\end{itemize}

Enough negative statements, what this manual \emph{does} provide is a
complete reference to all components and models provided by Daisy at
the time the manual was generated, as well as all the parameters,
state variables and log variables supported by these models.  These
are collected in part~\ref{part:components} of the manual.  The reason
why we can state with absolute confidence that the listing is
complete, is that it is generated automatically from the code.  Which
means, if Daisy accepts a parameter in the setup files, it will also
be listed in the manual.  It also means that the components, models
and parameters cannot be organized into logical groups with helpful
descriptions of each.  Instead they are presented in alphabetical order.
This manual is generated from the version of Daisy listed at
page~\pageref{version}.

Part~\ref{part:simulation} of the manual is written by humans though.
Here, we will try to describe enough of the concepts, format, and
nomenclature to be able to use part~\ref{part:components} efficiently.

\section{Concepts}
\label{sec:concepts}

In order to use the Daisy model, as well as this manual, you must
already possess a fairly good understanding of the agronomical and
physical processes that are modelled by Daisy.  However, when
describing the organizing and principles behind the Daisy software, we
have to introduce some new concepts, which are used throughout this
manual.  This section is somewhat abstract and dry.  The best way to
read it might be to read it once, without worrying too much about
understanding it, then get some experience with the Daisy program and
the setup files, and then read this section again.

\subsection{Components, Models, and Parameterizations}

The two most significant concepts behind the organization of the
Daisy, are \emph{components} and \emph{models}.  Typically, a
component corresponds to some physical process, and a model is a
particular way to describe (or simulate) the process.  For example,
water transport in the unsaturated zone is a process, while Richards
Equation is a particular way to describe that process.  In Daisy,
water transport is a component (described in
chapter~\ref{component:uzmodel}), while Richards Equation is a model
(described in section~\ref{model:uzmodel-richards}).

Just about everything in Daisy are either components or models, not
just what you naturally think of as physical processes.  For example,
crops (chapter~\ref{component:crop}), management
(chapter~\ref{component:action}), and even log files
(chapter~\ref{component:log}) are all components, each having several
different models. Which leads us to the next issue, \emph{fixed} vs.\
\emph{library} components.

The general component/model framework makes it very easy to add new
models to existing Daisy components.  All that is required is a
\cplusplus{} file with an implementation of the model.  No other code
has to be changed.  To activate the model, the user just selects it in
the setup file.  However, for some components we only support a single
model.  This makes them slightly simpler, for example the user will
not have to specify a model in the setup file, as there can only be
one.  The cost is less flexibility.  We call the first kind of
components, which can (theoretically) can have many different models,
for \emph{library components}.  Each library component has its own
chapter in part~\ref{part:components}, with a section for each model
of that component.  The second less flexible kind of components is
called \emph{fixed components}.  All the fixed components are described
in chapter~\ref{cha:fixed}.

\subsubsection{Parameterizations}

Library components have another important feature, which fixed
components lack.  They allow the user to save standard
parameterizations of the model in text files, and refer to these
parameterizations by name.  Daisy is distributed with a large number
of such standard parameterizations.  The most obvious example where
they are useful is parameterization of crop models.  The default crop
model (\ref{model:crop-default}) is sufficiently generic to be able to
describe crops ranging from grass to potatoes, but it takes lot of
parameters to do so.  This is why Daisy comes with already
parameterized versions of the model, with names like ``Grass'' and
``Potato''.

All in all, this gives a three layered hierarchy, with library
components in top.  For each library component, there can be one or
more models implemented in \cplusplus{}.  For each model, there can be
one or more named parameterizations.

\subsection{Model Attributes}

A parameterization can set \emph{parameters} (as the name implies) for
the model, as well as initial values for the models \emph{state
  variables}.  The parameters are constant during the simulation,
while state variables can change as the simulation progresses.  A
typical parameter could be the hydraulic conductivity for saturated
soil, or the maximum height for a crop.  Typical state variables could
then be the water content in the soil, or the current height of a
crop.

\subsubsection{Optional and Default Parameters}

In general, all the parameters and state variables must be specified
before a given model can be used in the simulation.  There are two
exceptions to that rule though.  

First, some parameters and state variables are marked \emph{optional}.
This can mean one of two things.  Sometimes the model will be able to
infer the value of one variable (or parameter) from the values of
other variables or parameters.  For example, the `SoilNO3' component
(section~\ref{fixed:SoilNO3}) has both an `M' state variable,
indicating the total nitrate content in the soil, and a `C' state
variable, indicating the nitrate concentration in the soil water.
Given one of these, and information about the amount of water in the
soil, the `SoilNO3' component is able to calculate the other.

The other exception is parameters or state variables which have a
\emph{default value}.  Many models have parameters that the typical
user will rarely change, because the default value is well
established, but where the parameter is still provided, for those
people who want to fiddle with the model.  For state variables, there
can also be reasonable values provided by default.  For example the
`crops' vegetation model (section~\ref{model:vegetation-crops}) will
by default start with an empty list of crops, i.e.\ there will be no
crops in the field at the start of the simulation.  This is what most
people will expect.

\subsubsection{Partial Parameterizations}

One can say that the model itself can provide a \emph{partial
  parameterization} of itself this way, i.e.\ a parameterization of
some, but not necessarily all, parameters in the model.  The user can
also create partial parameterizations, which is a third example where
you don't need to specify all the parameters and state variables.  A
partial parameterization cannot be used directly in the simulation
without specifying the missing parameters and state variables, but can
be used as a \emph{base} for other \emph{specialized}
parameterizations.  For example, the standard fertilizer library in
the Daisy distribution includes a partial parameterization named
`slurry'.  This parameterization is used as the base for two
specialized parameterizations, namely `pig\_slurry' and
`cattle\_slurry', and contains the parameters that are common for
both kinds of slurry.  With partial parameterizations, the user can
build a hierarchy of parameterizations with the most general on top,
and the most specialized (complete) parameterizations at the bottom.

\subsubsection{Log Variables}

A model can also support a second kind of variables, namely \emph{log
  variables}.  Log variables are computed during the simulation, but
are not part of the model state, i.e.\ they are nor stored between
time steps.  This also means that it makes no sense to initialize
them, so they cannot appear in a parameterization of the model.
However, log variables and state variables can be written to output
files (or other external media) by the various `log' components
(chapter~\ref{component:log}).  Thus, we have parameters and state
variables in the input files, and state variables and log variables in
the output files.

In the Daisy vocabulary, parameters, state variables, and log
variables are three different \emph{categories} of model
\emph{attributes}.  One can think of the attributes in general as the
users mean to interact with the model, and each of the three
categories as specifications of how the particular attribute can be
used, for input, for output, or for both.

\subsection{Attribute Types}
\label{sec:attribute-types}

Beside having a category, attributes also have a \emph{type} and a
\emph{size}.

A type is the set of legal values.  For example, an attribute whose
type is `integer' (see section~\ref{type:integer}) cannot hold a
string (see section~\ref{type:string}) value.  This prevents you from
setting the attribute `\texttt{NoOfIntervals}' to `\texttt{"Gnat"}' in
the input file.  For log variables, the type will give you an idea of
what kind of values to expect in the output files.  The various types
are described in section~\ref{sec:types}.

The attribute size specifies the number of values to expect.  If the
attribute is a \emph{singleton}, only a single value can appear.  If
the attribute is a \emph{sequence}, any number of values can appear.
The attribute can also have an integer size, which means exactly that
number of values should appear.

Each attribute has an associated description, which hopefully will give
some idea of what the attribute is used for.  Attributes with the type
`number' (see section~\ref{type:number}) will also have a dimension,
so you for example will know whether the length should be given in
millimeter or kilometer.  This can make a difference.  Trust me on
that one.

\subsubsection{Ordered Attributes}

Finally, some attributes may be \emph{ordered} within the model.  This
order is used by the `file' parser model
(section~\ref{model:parser-file}) for parsing the setup files.
Normally, you will have to specify both the name and the value of the
attributes in the setup files.  However, ordered attributes are
identified by the order they appear in the setup file, so you will not
have to specify their names.  In fact, you are not allowed to specify
their names.  Ordered attributes are typically used for ``obvious''
parameters, for example the `file' parser model has a single
parameter, `where', the name of the file to parse.  That parameter is
ordered, so you can write
\begin{verbatim}
  (input file "filename.dai")         ; Correct.
\end{verbatim}
instead of
\begin{verbatim}
  (input file (where "filename.dai")) ; Wrong.
\end{verbatim}
as you would have had to write, if the `where' attribute had been
unordered. 

\section{Nomenclature}

Having covered the concepts, it is now possible to explain the
nomenclature used in part~\ref{part:components} of the manual.  As
already explained, each library component has its own chapter, while
all the fixed components are collected in chapter~\ref{cha:fixed}.
All the library component chapters start with a short explanation of
the purpose of that component.

Each model (including those models that are part of a fixed component)
has its own section.  Here, we will explain how to read the content
of these sections.  This is best done with an example.  Now, keep a
finger here, and find section~\ref{model:action-fertilize} (that's on
page~\pageref{model:action-fertilize}).
Section~\ref{model:action-fertilize} is a description of the
`fertilize' model, which is part of the action component.  You specify
this model when you want the manager to fertilize the field.  

\subsection{Name and Description}

The name of the section is `fertilize', which is the name you have to
specify in the setup file when you want to refer to the model.  The
text in the start of the section is a brief description of the model
and what it does.

\subsection{Sample}
\label{sec:sample}

The third element in the section is a \emph{sample}, which shows the
format recognized by the file parser model when it reads the setup
files.  The sample starts with `\texttt{$<$}' and ends with
`\texttt{$>$}', these should not be included in the setup files, they
are only there to show the start and end of the sample.  The sample
text written in a normal typewriter should be included directly in the
setup files, while the text written in slanted text should be replaced
with the parameter (or state variable) values, as listed after the
sample.

All parameters and state variables will be listed in the sample.  Log
variables will not be listed, since they cannot be set in input files.

Ordered attributes will be listed first, in order, since that is how
they must appear in the setup files.  In the example (that's
section~\ref{model:action-fertilize}, your finger should still be
there!), \texttt{\textit{am}} is such an ordered attribute.  You will
note that the text \emph{am} only occurs once in the sample, while the
texts \emph{to} and \emph{from} both occur twice.  This is because you
only have to specify the value for the `am' component in the setup
files, since it is ordered, while you have to specify both the name
and value for the unordered `to' and `from' attributes.

For the unordered attributes, it doesn't matter what order they appear
in the setup files.  In the sample, they will be shown in alphabetic
order, `from' will appear before `to'.  However, there is no reason
(except good taste) why you shouldn't set the `to' parameter before
the `from' parameter in the setup files.

All three attributes (`am', `from', and `to') are singletons (reread
section~\ref{sec:attribute-types} if you have forgotten what that
means).  If any of the attributes had been sequences, they would have
been followed by the string `\texttt{\textit{\ldots{}}}'.  This does
not mean you should write `\texttt{...}' in the setup files, but that
you can write any number of values at that point, separated by
whitespace.  Section~\ref{model:action-cond} has an example of that.

If an unordered attribute has a simple value, it will be printed
directly in the sample.  

\subsection{The Attribute List}
\label{sec:alist-doc}

After the sample comes a list with information about each attribute in
the model.  The parameters and state variables are listed in the same
order as in the sample, that is, first the ordered attributes (in the
specified order), and then the rest of the parameters and state
variables in alphabetic order.  After that, the log variables are
listed, if any, in their own section.  The `fertilize' management
action model doesn't have any log variables, so look at the default
bioclimate model (section~\ref{model:bioclimate-default}) instead.  It
has plenty.

The documentation for each attribute is divided in three parts, which
are explained below.

\subsubsection{The Type Line}

The first line (the line with the bullet) contains the name and the
type (section~\ref{sec:attribute-types}, remember?) of the attribute.
The name is written in slanted italic, to match the way it is written
in the sample (section~\ref{sec:sample}).  Two types are treated
specially: Numbers have a dimension, which is written in bold in
square brackets; Components have their name written in bold, and a
reference to the chapter (or section, for fixed components) in the
manual where they are described.  Other types (such as integers and
strings) have a reference to the subsection of section~\ref{sec:types}
where they are explained.

\subsubsection{The Category Line}

The second line contains the following information:
\begin{enumerate}
\item Whether the attribute is optional.  If so, the line will start
  with the string `Optional'.
\item The category of the attribute, that is, whether it is a
  parameter or a state variable.  The category line is not included
  for log variables.
\item The default value, if any.  If the value is something simple
  (like a number or a very short string), the value will be shown on
  the line.  Otherwise, the manual will just state that the attribute
  has a default value, but not what it is.
\end{enumerate}

Sometimes the category line is left out.  It is never included for log
variables, since their category it is obvious from context, and they
never have default values.  The category line is also left out for
components, unless the component is optional or has a default value.
This is because components can contain both state and log variables, as
well as parameters. So the category of a component is not a very
useful concept.  If a component does a default value or is optional,
the category line will be printed.  However, instead of `parameter' or
`state variable', the attribute will be categorized as a `component'.

\subsubsection{The Textual Description}

On the third line (or second, if the category line is missing) there
is a brief textual description of the attribute.  The description may
continue on the following lines.

\subsection{Submodels}

Sometimes it is useful to combine attributes in a model into groups.
One example of this is the `cond' management action model
(section~\ref{model:action-cond}), where we use the `clauses' group to
emphasize the strong association between each condition, and the list
of actions to perform when the condition is true.  Another example is
the `default' crop model (section~\ref{model:crop-default}), where
there simply are so many attributes that grouping them logically
becomes a necessity.

If you have looked at the two examples, you will see the attribute
groups are categorized and typed as `submodel' in the manual.  After
the textual description of the submodel follows a sample illustrating
the formatting of the submodel attributes.  The submodel sample
follows the same conventions as the model sample, described in
section~\ref{sec:sample}.  After the sample follows a nested attribute
list, documenting attributes of the submodel in the exact same way as
the attributes of the main model are documented, as described in
section~\ref{sec:alist-doc}.  You can even have submodels nested
inside submodels, ad infinitum.

\section{The Daisy Model and Named Parameterizations}
\label{sec:daisy}

Try to look at section~\ref{model:program-Daisy}, which describes a program
model named `Daisy'.  That model is actually the top level description
of the simulation.  All you need to do in a setup file, is to specify
values for the specified attributes.  Then you will be ready to run
it.

The main attribute is `column', which describes the physical model
itself.  To run the model, you also need to specify `time', which is
the starting time of the simulation, `manager', which specifies what
management operations should be done to the column, and `weather'
which provides the weather during the simulation.  Furthermore, the
`output' attribute should be set to specify what data you want logged
during the simulation.  You can also use the `input' command to split
the setup description into multiple files. 

In this section we will describe the individual Daisy attributes in
more detail.  It helps having an example to look at, the Daisy
distribution (see section~\ref{sec:distribution}) contains a sample
setup file named ``\texttt{test.dai}''.  We do change that file from
time to time, but the content shouldn't be too different from what you
see here.

\subsection{Input Files}
\label{sec:daisy-input}

The `input' attribute isn't really an attribute.  It is a command to
read some more parameterizations or, more rarely, set some of the top
level attributes.  However, the format is the same as for singleton
parser components (see
sections~\ref{component:parser},~\ref{sec:attribute-types}
and~\ref{type:object}), except that you can have multiple `input'
statements, all taking effect.

The parser component is supposed to read parameterizations or top
level attributes from an external source.  The `file' parser model
(section~\ref{model:parser-file}) does this by reading a file in
exactly the same format as the main setup file.  It has a single
(ordered) attribute.  In the example below, we use the input command
to read files containing parameterizations of management actions
(tillage), crops, output (log) file formats, and soil types
(horizons).  The three first contain parameterizations that are used
in most simulations.

\begin{verbatim}
  (input file "tillage.dai")
  (input file "crop.dai")
  (input file "log.dai")
  (input file "test-horizons.dai")
\end{verbatim}

These parameterizations will be used when setting the attributes
described in the rest of this section.

The file parser will search for files in the directories specified by
the \texttt{DAISYPATH} environment variable
(see~\ref{sec:distribution}), or by the `path' attribute
(section~\ref{model:program-Daisy}. 

\subsection{Simulation Time}

The simplest attribute is `time', which is a state variable holding
the current time in the simulation.  The value you specify in the
setup file will be its value at the start of the simulation.
However, if you output the time attribute value in a log file, you will
see that time changes constantly during the simulation in steps of one
hour.

The type of the `time' attribute is a Time fixed component, as
described in section~\ref{fixed:Time}.  You specify four integer
values separated by whitespace after the attribute name (time), which
indicate the year, the month, the day in the month, and the hour (from
0 to 23) in that sequence.

\begin{verbatim}
  (time 1986 12 1 1)
\end{verbatim}


\subsection{Weather}
\label{sec:weather}

The type of the `weather' attribute is a singleton library component,
so the format is the same as for `input'
(section~\ref{sec:daisy-input}).

There are several weather models, but the main one is named `default',
and read files containing daily weather data, as described in
appendix~\ref{cha:dwf}.  The first (ordered) attribute specifies the
name of the weather file; ``weather.dwf'' is a file distributed
together with Daisy, containing many years worth of weather data from
KVL's research farm in Taastrup, near Copenhagen.

\begin{verbatim}
  (weather file "weather.dwf")
\end{verbatim}

\subsection{Field}
\label{sec:column}

The value of the `column' attribute should be a sequence of column
components.  The `column' component is the main component of Daisy,
most of the physical processes modelled by the simulation are
attributes of the `default' column component (see
section~\ref{model:column-default}).  The `default' column model is
the only column model implemented at the time of writing.

Conventionally, the columns are specified by first defining a named
parameterization for the default column, and then using that
parameterization when setting the `column' attribute.  This is also
what we do in the example.

First we define a parameterization named `JB1\_Andeby' of the
`default' column model.  The command `defcolumn' is used for this.
The first argument to the `defcolumn' is the name of the new
parameterization (in this case: `JB1\_Andeby'), and the next argument
is the name of the column model to parameterize (in this case
`default').  The second argument could also be a name of an existing
parameterization, which you wanted to change or further refine.  If
there are no ordered attributes, the optional third argument is a
documentation string.  Otherwise, the second argument is followed by
the ordered attributes (the default column model has no ordered
attributes).  Finally comes the unordered attributes.  All separated
by whitespace.

\begin{verbatim}
  (defcolumn JB1_Andeby default
    "Data for the farm of B.And, Andeby, as collected by F.Guf, 2002."
    (Groundwater static)
    (Soil (horizons (-20 [cm] top_soil) (-250 [cm] bottom_soil))
          (zplus -2.5 -5 -10 -15 -20
                 -25 -30 -40 -50 -60 -70
                 -80 -90 -100 -125 -150 -175
                 -200 -225 -250 [cm])
          (MaxRootingDepth 60.0 [cm])
          (dispersivity 8.0 [cm]))
    (OrganicMatter (initial_SOM (-20  [cm] 4.86 [kg C/m^2])
                                (-40  [cm] 2.92 [kg C/m^2])
                                (-60  [cm] 1.46 [kg C/m^2])
                                (-80  [cm] 0.49 [kg C/m^2])
                                (-100 [cm] 0.16 [kg C/m^2])
                                (-120 [cm] 0.03 [kg C/m^2]))))
\end{verbatim}

You can define named parameterizations for all components this way.
The name of the command to define a new parameterization starts with
`def', followed by the name of the component.  For example, the
`defweather' command can be used for defining named weather
parameterizations.

The default column model has many attributes, namely one for each
major physical process in the model.  However, only two attributes
have to be specified in the setup file, the rest have default or
optional values.
\begin{itemize}
\item You need to specify a groundwater model.  In this case we have
  specified static (unchanging) groundwater conditions with free
  drainage.
\item You need to specify the numeric and physical properties of the
  soil.  
  \begin{itemize}
  \item Here we claim that the soil is divided in two horizons, the
    top horizon reaching down to 20 cm and using the physical
    properties specified by the `top\_soil' horizon parameterization,
    and the second reaching from 20 cm down to 250 cm, using the
    `bottom\_soil' horizon parameterization.  You will usually have to
    define the horizon parameterizations yourself, the real
    `\texttt{test.dai}' file should give you an example of how to do
    that.
    
    Technically, the `horizons' attribute is a sequence of submodels,
    each submodel having two ordered attributes.  The first attribute
    specifies the endpoint of that horizon (a number), while the second
    attribute specifies the horizon parameterization to use.
  \item The `zplus' soil attribute contains a sequence of numbers,
    each specifying the endpoint of one of the numeric layers.
  \item The two remaining attributes are physical properties that
    relate to the soil as a whole, rather than the individual
    horizons.
  \end{itemize}  
\item In the example, we also specify `initial\_SOM' for the
  `OrganicMatter' fixed component.  The initial SOM is another
  sequence of submodels with two ordered attributes.  In this case the
  first number is the depth in cm, and the second number is the soil
  carbon content in kg/$\mbox{m}^2$.  Each submodel specifies a soil
  layer, with the first attribute indicating the layer's end point, and
  the second attribute the layer's carbon content.  If the initial SOM
  is not explicitly specified, Daisy will use the humus content
  specified in the horizon to deduce it.
\end{itemize}

After defining the column parameterization, we can set the column
attribute itself.

\begin{verbatim}
  (column JB1_Andeby)
\end{verbatim}

If we had defined two column parameterizations which we wanted to run
concurrently, we could have set the column attribute like this:

\begin{verbatim}
  (column JB1_Andeby JB1_Gaaseroed)
\end{verbatim}

The most common case is to run a single column though, and at the time
of writing very few of the standard log component parameterizations
support logging more than a single column.

\subsection{Management}
\label{sec:management}

The value of the `manager' attribute should be a single
parameterization of the `action' library component.  What makes the
`manager' attribute special, is that the `action' library component
contains a large number of primitive action models like `sow' or
`fertilize', and a number of generic models which can be used for
creating crop rotations out of these primitive actions.

The most common way to specify a manager is to use the `activity' model
(section~\ref{model:action-activity}).  The `activity' action model has a
single ordered attribute, whose value is a sequence of actions.  Each
action will be performed in each time step, until done.  Most actions
are done after a single time step, but the `wait' action is only done
when the corresponding condition is true.

A common condition is `at' (section~\ref{model:condition-at}), which
is fulfilled only when the time of the simulation is equal to the time
value specified in the condition.  However, in the example we also use
`crop\_ds\_after' (section~\ref{model:condition-crop_ds_after}), which
is fulfilled after the specified crop in the field has reached the
specified development stage.  The value specified here, 2.0, is
reached when a crop is ripe.  The action we then perform is to
immediately harvest the crop, which means the condition will only be
fulfilled once.

\begin{verbatim}
(manager activity
  (wait (at 1987 3 20 1))
  (plowing)
  (wait (at 1987 4 4 1))
  (fertilize (mineral 
              (weight 100.0 [kg N/ha])
              (NH4_fraction 0.5 [])))
  (wait (at 1987 4 5 1))
  (progn 
    (sow "Grass")
    (sow "Spring Barley"))
  (wait (crop_ds_after "Spring Barley" 2.0))
  (harvest "Spring Barley")
  (wait (at 1987 10 10 1))
  (harvest "Grass"
           (stub 15 [cm])                    ;Leave 15 cm stub.
           (stem 1.00 []))                 ;Harvest everything above stub.
  (wait (at 1987 9 8 1))
  (fertilize (mineral
              (weight 80.0 [kg N/ha])
              (NH4_fraction 0.5 [])))
  (wait (at 1987 10 10 1))
  (harvest "Grass"
           (stub 8. [cm])                    ;Leave 08 cm stub.
           (stem 1.00 []))                 ;Harvest everything above stub.
  (wait (at 1988 4 1 1))
  (stop))
\end{verbatim}

Read chapter~\ref{component:action} and~\ref{component:condition} to
learn more about the available action and condition models.

\subsection{Log File Generation}

The value of the `output' attribute should be a sequence of log
component parameterizations.  Each parameterization should specify
some state or log variables to be output during the simulation.

The `\texttt{log.dai}' file contains a number of standard log
parameterizations, for example `Crop Production', which will write
information about the development and production of any crops in the
field to a file named `\texttt{crop\_prod.dlf}' located in the
directory specified by the `directory' attribute, or, by default, the
directory where daisy was started.  This file will contain a header
with some meta-information about the simulation, a line of dashes, and
then a large number of tab separated columns, with each column
describing one aspect of the crop development (e.g.\ the stem nitrogen
content), and each row a specific time step.  The two first rows are
special, the first will contain a label identifying the column (e.g.\ 
NStem), and the second row the dimension used for the numbers in that
column (e.g.\ kg~N/ha).

You can obviously also define your own parameterizations.  If you wish
to do so, the `table' log model (see section~\ref{model:log-table})
is far the most useful.  In the example, we specify two standard log
parameterizations (`Crop Production' and `Harvest') and one home-made
parameterization (`Pond').

\begin{verbatim}
  (deflog Pond table
    (where "pond.dlf")
    (when daily)
    (entries (number (path time year))
             (number (path time month))
             (number (path time mday))
             (number (path column "*" Surface pond)
                     (dimension "mm"))))
\end{verbatim}

The `table' log model has a number of parameters, most of these have
default values.  The three listed here are `where', which specifies
the name of the file to log the results in, `when', which specifies
how often to log the results, and `entries', which specifies what to
log.  The `entries' attribute is a sequence of `select' (see
section~\ref{component:select}) models, where each entry specifies one
column in the log file.  The first entry listed will correspond to the
first column in the log file.

Each log entry model has a large number of attributes, allowing you
to accumulate results, extract values from the soil array, specify the
tag and dimension printed in the two first rows of the column, and
more.  Most of these have default values, and only the fourth (and
last) log entry submodel specifies one of these attributes, namely the
dimension string to print in the second row.

The most important log entry parameter is `path', which specifies the
state or log variable we want to include in the log file.  The value
is a sequence of attribute names, which should be read as follows:
Start with the Daisy model (section~\ref{model:program-Daisy}), and
select an attribute in that model.  That should be the first name
listed in the path sequence.
\begin{itemize}
\item If the attribute is a time value, as in the three first log
  entries, you list one of `year', `month', `mday', or `hour' next in
  the path.  The table log model will then log the selected part of
  the time value.
\item If the selected attribute is a component sequence, as in the
  fourth log entry in the example, you should list the name of the
  model or parameterization you want to log next in the path list.  In
  this case `JB1\_Andeby' would do, but instead we use the special
  value `*', which will match \emph{all} entries in the sequence.
  However, their values will be accumulated into a single number in
  the log file, which is why we say most of the standard log models
  don't work well with multiple columns.
\item After listing the name of the model or parameterization in the
  sequence to log, we must list an attribute of the model.  In this
  case we list the `Surface' fixed component.
\item As long as the selected attribute itself is a submodel or fixed
  component, we keep listing a name of an attribute in \emph{that}
  model next.
\item In this case, it ends with the `pond' attribute of the `Surface'
  fixed component.  The value of the `pond' attribute is a single
  number, which will be logged in the fourth column of the
  `\texttt{pond.dlf}' log file.
\end{itemize}
That is the general strategy for writing log file entry paths.  Keep
listing names of nested attributes, until you reach a simple value,
which will then be logged.

\section{Types and Formatting}
\label{sec:types}

There are a few general rules for the formatting accepted by the
`file' parser.  
\begin{enumerate}
\item Elements are separated by whitespace.
\item Grouping is done with parentheses.  You can always (but need
  not) have whitespace before and after parentheses.
\item Whitespace is linefeeds, spaces, and tabs.  You can have any
  number (greater than one) of these anywhere you can have whitespace.
\item Comments start with the character `;' and ends at the end of the
  line.  Comments are legal anywhere you can have whitespace.
\item All names (attribute names, component names, model names, and
  parameterization names) follow the same conventions as string
  values, described in section~\ref{type:string}.
\end{enumerate}

In this section, we will describe how each type of attribute values
should be formatted for each of the attribute types.  The formatting
does also depend slightly on whether the attribute is a singleton or
a sequence, and whether the attribute is ordered or not, so we will
cover that subject first.

\subsection{Order and Size}
\label{sec:order-size}

Booleans, integers, numbers, and strings have a common format though.
If \emph{val} denotes the attribute value as described in
sections~\ref{type:boolean},~\ref{type:integer},~\ref{type:number},
and~\ref{type:string}, and \emph{name} is the attribute name, then
the format for the attributes in the setup files depends on whether they
are ordered or not, and whether they are singletons or sequences, as
illustrated in table~\ref{tab:simple-types}.

\begin{table}[htbp]
  \begin{center}
    \begin{tabular}{|r|l|l|}\hline
                & singleton           
                & sequence \\\hline
      ordered   & \texttt{\emph{val}} 
                & \texttt{(\emph{val}\ldots{})} \\\hline
      unordered & \texttt{(\emph{name} \emph{val})}
                & \texttt{(\emph{name} \emph{val}\ldots{})}\\\hline
    \end{tabular}
  \end{center}
  \caption{Format for booleans, integers, numbers, and strings.}
  \label{tab:simple-types}
\end{table}

PLF's, submodels and fixed components are different though.  Unlike
the simpler types described above, these already contain several parts
that are white space separated.  Therefore, it is necessary to mark
their start and end with parentheses except for unordered singletons,
when there can be no ambiguity.  Again, \emph{name} is the attribute
name, and \emph{value} is the attribute value in the format described
in sections~\ref{type:plf} and~\ref{type:alist}.

\begin{table}[htbp]
  \begin{center}
    \begin{tabular}{|r|l|l|} \hline
                & singleton           
                & sequence \\\hline
        ordered & \texttt{(\emph{val})} 
                & \texttt{((\emph{val})\ldots{})} \\\hline
      unordered & \texttt{(\emph{name} \emph{val})}
                & \texttt{(\emph{name} (\emph{val})\ldots{})}\\\hline
  \end{tabular}

  \end{center}
\caption{Format for time values, PLF's, submodels, and fixed
    components.} 
  \label{tab:complex-types}
\end{table}

In general, library component values are written using the formatting
conventions in table~\ref{tab:complex-types}.  However, there are two
exceptions.
\begin{enumerate}
\item If you just specify the name of a complete parameterization, as
  in section~\ref{sec:column}, you do not need the extra parentheses,
  and can use the format conventions from
  table~\ref{tab:simple-types}. 
\item If there are no unordered attributes, then you do not need to
  put parentheses around the last attribute in the order.  You can see
  an example of that in section~\ref{sec:management}.  The action
  sequence is the last attribute in the `clauses' submodel in the
  `cond' action model.
\end{enumerate}

\subsection{Booleans}
\label{type:boolean}

The simplest attribute type is booleans.  A boolean value can be
either `\texttt{false}' or `\texttt{true}', spelled exactly that way,
with small letters.

\subsection{Integers}
\label{type:integer}

Integers should start with a non-zero digit, and be followed by a
sequence of digits.  Negative numbers should start with a `\texttt{-}'
sign, followed immediately with the digits.  No whitespace is allowed.
The only integer which should start with `\texttt{0}' is the integer
zero itself, don't use leading zeros for padding.  Use space instead.
Otherwise, Daisy might misinterpret the integer on some systems.

\subsection{Numbers}
\label{type:number}

The general format for writing numbers is

\begin{tt}
  $<$\emph{sign}$>$$<$\emph{integer-part}$>$.$<$\emph{fraction}$>$e$<$\emph{sign}$>$$<$\emph{exponent}$>$
\end{tt}

All of these can be left out, except for the integer part.  There can
be no whitespace between any of the parts.

After the number, you may optionally specify the dimension inside
square brackets, like this \verb|[cm]|.  Fractions and other
dimensionless numbers are specified as \verb|[]|.  Daisy will check
that the specified dimension is also the dimension is expects, or a
simple convertion thereof (like \verb|[m]| instead of \verb|[cm]|).
If the deimension is listed as ``unknown'' in this manual, Daisy will
be unable to check or convert it.  It can still be specified for
informational purposes, but the string must start with a question
mark, like this \verb|[?cm]|.

For lists of numbers, the dimension applies to all the number before
it. 

\subsection{Strings}
\label{type:string}

Daisy accepts two different string formats.  The first format is called
`identifier'.  Identifiers must start with a letter or an underscore,
and can be followed by letters, digits or underscores only.  They are
conventionally used for attribute names.

The other format is called `quoted strings', and are most often used
for attribute values.  Quoted strings start and begin with a double
quote (`\texttt{"}').  Any characters in between the two double quote
signs are part of the string, with these exceptions:
\begin{itemize}
\item To include a `\texttt{"}' in the string, you must write
  `\texttt{\mbox{$\backslash$}"}'.
\item To include a `\texttt{\mbox{$\backslash$}}', you must write
  `\texttt{\mbox{$\backslash$}\mbox{$\backslash$}}'.
\item If you put a backslash as the last character on the line,
  neither it, nor the following newline character will not be part of
  the string.
\item A `\texttt{\mbox{$\backslash$}n}' will be replaced by a newline
  character. 
\end{itemize}
It is an error to put any other character after backslash.  

Case is always significant.
\subsubsection{File Names}

Under some operating systems,  file names are conventionally written
like this:
\begin{verbatim}
  C:\daisy\lib\log.dai
\end{verbatim}

However, since `\texttt{\mbox{$\backslash$}}' is treated specially in
Daisy setup files, you will have to double all the backslashed, like
this:
\begin{verbatim}
  (input file "C:\\daisy\\sample\\test.dai")
\end{verbatim}

Since this looks rather awkward, Daisy also supports ordinary slashes
in file names:
\begin{verbatim}
  (input file "C:/daisy/sample/test.dai")
\end{verbatim}

\subsection{PLF}
\label{type:plf}

PLF is the name used in Daisy for piecewise linear functions.  In the
setup files, such a function is specified as a sequence of
\texttt{(\emph{x} \emph{y})} pairs, where the \emph{x} value must be
monotonically increasing.  The value of the PLF for any point in the x
axis is then calculated by linear interpolation between the two
closest points.  The value \emph{before} the first specified point
will be the same as the value \emph{at} the first specified point.
Similarly, The value \emph{after} the last specified point will be the
same as the value \emph{at} the last specified point.

An example might help.  If `f' is a unordered singleton PLF, we can
specify its value like this:
\begin{verbatim}
  (f (1.0 1.0) (3.0 2.0) (5.0 4.0))
\end{verbatim}
We then get the values in table~\ref{tab:plf}.

\begin{table}[htbp]
  \begin{center}
    \begin{tabular}{l|l}
      $x$ & $f(x)$ \\\hline
      0.0 & 1.0 \\\hline
      1.0 & 1.0 \\\hline
      2.0 & 1.5 \\\hline
      3.0 & 2.0 \\\hline
      4.0 & 3.0 \\\hline
      5.0 & 4.0 \\\hline
      6.0 & 4.0 
    \end{tabular}
    \caption{Interpolated values for f}
    \label{tab:plf}
  \end{center}
\end{table}

\subsection{Submodels and Fixed Components}
\label{type:alist}

A submodel or fixed component is basically a set of attribute names,
with a value associated to each value.  Some of the attributes are
ordered, when specifying the value of a submodel you must specify the
values for all the ordered attributes separated by whitespace,
followed by the values for each of those unordered attributes you wish
to set.  The unordered attributes can, as the name implies, come in
any order.  How to write the attributes in the file depends on whether
they are ordered or not, their type, and whether they are sequences or
singletons.  This is all described in section~\ref{sec:types}, in
particular~\ref{sec:order-size}.

The fixed components are found in chapter~\ref{cha:fixed}, and the
submodels are found with their parent models throughout
part~\ref{part:components}.  In the description of each submodel or
fixed component, there is a \emph{sample} showing the attributes you
can specify, and how they should be placed.  The sample is explained in
section~\ref{sec:sample}. 

Submodels and fixed components are special compared to the other
attribute values, in that they do not replace the original value.
Instead, the original value will be amended with the individual
attributes you specify.  If you don't specify the value for a
particular attribute, the original value will be used.  

\subsection{Library Components}
\label{type:object}

A library component value should always start with the name of model
or parameterization.  If you want to overwrite some of the attribute
values of the model or parameterization you just specified, you should
write all the values of all the ordered attributes immediately after
the name of the model or parameterization, separated by whitespace,
and then the unordered attributes, just like you would for a submodel
or fixed component (see section~\ref{type:alist}).

Each section of each chapter in part~\ref{part:components} describes a
specific model.  Part of that description is a \emph{sample}, which
shows how to specify that model in a setup file, and how to set the
parameters for that model.

If you want to make partial modifications to the original value of the
library component, you can specify the special keyword ``original''
instead of a name of a model or parameterization, followed by a list
of those attribute names and value you want to overwrite.

The format for library component values described here is also used
when defining named parameterizations.  In the setup file, you
write: 

\begin{tt}
  (def\emph{component} \emph{name} \emph{value})
\end{tt}

where \emph{component} is the name of the component you want to define
a named parameterization for, \emph{name} is the name of the new
parameterization you want to define, and \emph{value} is a library
component value as described in this section (\ref{type:object}).

\section{The Daisy Distribution}
\label{sec:distribution}

In this section we describe the executable files which are part of the
Daisy distribution, followed by a brief overview of the content of
each of the standard parameterization files.  Please note that these
changes frequently, so the information here is unlikely to be
complete and current.

\subsection{The Daisy Program}

The main Daisy executable is called `\texttt{daisy.exe}' on win32
systems, or just `\texttt{daisy}' on Unix.  This is the command line
version of Daisy.  

To run a simulation, you should start the daisy executable with a
setup file as an argument.  A sample setup file named
`\texttt{test.dai}' is distributed together with the sample
parameterizations.  You run it by typing
\begin{verbatim}
  daisy test.dai
\end{verbatim}
from the command line.  Please note that you \emph{must} include the
full name of the file, that is include `\texttt{.dai}' file suffix.
Setup files can be named anything, the `\texttt{.dai}' is only there
to make them easier to recognize for the user.

The command above will only work if you are located in the directory
containing the `\texttt{test.dai}' file, and all the parameterization
files used by `\texttt{test.dai}'.  However, we recommend that you
place all the standard parameterization files in one directory, and
use another for the setup files you create yourself.  The
\texttt{DAISYPATH} environment variable should then contain a list of
directories which Daisy will search for directories.  Under win32, the
directories should be separated by semicolons, while under UNIX they
should be separated by colons.  How to set the environment variable
depends on the system, here are some examples:
\begin{itemize}
\item Unix, using the C shell, in `\texttt{.cshrc}'.
\begin{verbatim}
  setenv DAISYPATH .:$HOME/daisy/sample:$HOME/daisy/lib
\end{verbatim}
\item Unix, using the Bourne shell, in `\texttt{.profile}'.
\begin{verbatim}
  DAISYPATH=.:$HOME/daisy/sample:$HOME/daisy/lib; export DAISYPATH
\end{verbatim}
\item Windows 95/98, in `\texttt{autoexec.bat}'.
\begin{verbatim}
  set DAISYPATH=.;C:\daisy\sample;C:\daisy\lib
\end{verbatim}
\item Under Windows NT, use \texttt{regedit} to add the environment
  variable. 
\end{itemize}

Alternatively, you can specify both where to run the executable (and
thus, where to create output files) and the path for searching for
files directly in the beginning of the setup file, with the
\emph{directory} and \emph{path} commands, like this:
\begin{verbatim}
  (directory "C:/daisy/jsmith/sim1")
  (path "." "C:/daisy/sample" "C:/daisy/lib")
\end{verbatim}
Here we specify that all files generated must be put in the ``sim1''
directory, and that \daisy{} should look for files there first,
followed by the ``sample'' and then the ``lib'' directories.  Note
that you must use slash (/) and not backslash (\mbox{$\backslash$}) to
separate directory components. 


\subsection{\texttt{tillage.dai}}

The `\texttt{tillage.dai}' file contains parameterizations for the
tillage operations listed below, whose nature should be apparent from
their names.
\begin{itemize}
\item plowing
\item rotavation
\item disk\_harrowing
\item stubble\_cultivation
\item seed\_bed\_preparation
\end{itemize}

\subsection{\texttt{crop.dai}}

The `\texttt{crop.dai}' file contains crop parameterizations.

\begin{itemize}
\item "Grass"
\item "Maize"
\item "Pioneer Maize"
\item "Ikuwala Maize"
\item "Silage Maize"
\item "Pea"
\item "Potato"
\item "Rye"
\item "Spring Barley"
\item "Spring Rape"
\item "Sugar Beet"
\item "Spring Wheat"
\item "Winter Barley"
\item "Winter Rape"
\item "Winter Wheat"
\end{itemize}

The above parameterizations all refer to the default crop model, we
also have some parameterizations of the old crop model.  Please note
that these cannot be used with multicrop systems.

\begin{itemize}
\item "Beet; V1"
\item "Grass; V1"
\item "Pea; V1"
\item "Potato; V1"
\item "Spring Barley; V1"
\item "Spring Rape; V1"
\item "Winter Barley; V1"
\item "Winter Rape; V1"
\item "Winter Wheat; V1"
\end{itemize}

\subsection{\texttt{fertilizer.dai}}

The parameterizations in the `\texttt{fertilizer.dai}' file are mostly
intended as examples, as the parameters for organic fertilizer vary
a lot.  
\begin{itemize}
\item pig\_slurry
\item cattle\_slurry
\item pig\_manure
\item cattle\_manure
\item horse\_manure
\end{itemize}
The `\texttt{fertilizer.dai}' file does also contain a partial
parameterization named `slurry', which contains the parameters common
to the `pig\_slurry' and `cattle\_slurry' parameterizations, and
partial parameterization named `manure', which contains the parameters
common to the `pig\_manure', `cattle\_manure' and `horse\_manure'
parameterizations

\subsection{\texttt{log.dai}}

The `\texttt{log.dai}' file contains a number of standard log file
descriptions.  With the exception of `Harvest', they only make sense
for single column fields.  Also with the exception of `Harvest', they
all use a tabular format, where the name and dimensions of each column
are written in the two first rows.

\begin{itemize}
\item "Total Soil Content"\\
  This log file format will print information about the total soil
  content at midnight each day. 
\item "Soil Nitrate Concentration"\\
  Midnight nitrate concentration in all numeric layers.
\item "Soil Temperature"\\
  Midnight soil temperature in all numeric layers.
\item "Soil Water Potential"\\
  Midnight water potential in all numeric layers.
\item "Surface Water Balance"\\
  Enough information to calculate a daily surface water balance. 
\item "Root Zone Water Balance"\\
  Enough information to calculate a daily root zone water balance. 
\item "N Balance"\\
  Enough information to calculate a daily nitrogen balance. 
\item "Soil Chemicals"\\
  Daily information about the fate of the soil chemicals.
\item "Surface Chemicals"\\
  Daily information about the fate of the surface chemicals.
\item "Crop Production"\\
  Daily crop development numbers for the default crop model.
\item "Old Crop Production"\\
  Daily crop development numbers for the old crop model.
\item "Bioclimate"\\
  All the information you ever wanted about the bioclimate, and more. 
\item "harvest"\\
  Information about all harvests during the simulation. 
\end{itemize}

\subsection{Error Messages}

Sometimes the simulation will fail to run for some reason or another.
In these cases, Daisy will attempt to give an error message, which
hopefully will give some indication of what went wrong.  There are
zillions of situations where Daisy might give some error message, in
the following we describe some of them.

\subsubsection{Parser errors}

If the format of the setup files is wrong, for example if there are
missing parentheses or unknown attributes, Daisy will give you
information about which file the error was found, the line number, the
column number, and what the problem was.
\begin{verbatim}
  test.dai:15:6: Unknown attribute 'kurt'
\end{verbatim}
This is the ``best'' kind of errors, since it will lead you directly
to the place where the error occurred.  

If you run Daisy with `\texttt{M-x~compile~RET}' inside the Emacs
text editor, you can go directly to the place the error occurred by
clicking on the error message with the mouse.

\subsubsection{Completeness checks}

After parsing the setup files, Daisy will check that all the required
parameters and state variables are defined, and make some very simple
consistency checks.  Errors during this phase looks as follows:
\begin{verbatim}
  * Daisy
  time missing
\end{verbatim}

\subsubsection{Consistency checks}

After checking that all the required attributes are there, Daisy will
create the simulation objects, and then perform a more throughout
consistency check.  Errors found during these phases will be reported
like the one below.
\begin{verbatim}
  * column
  ** Andeby
  *** SoilHeat
  You have 20 intervals but 60 T values
\end{verbatim}

\subsubsection{Runtime exceptions}

Now Daisy is ready to run the simulation.  However, some errors are
too hard to find before the simulation has started.  If Daisy finds
one of those during the simulation, it will throw an exception, which
normally causes the program to exit.  
\begin{verbatim}
  Exception: Cannot restrict already restricted field
\end{verbatim}

\subsubsection{Assertion failures}

During debugging, a lot of `assert' statements have been added to the
code.  Each of these statements will check if a certain condition is
true, and if not, exit the program with a message like
\begin{verbatim}
  field.C:71: failed assertion 'false'
\end{verbatim}
A failed assertion message will always print the file name, the line
number, and the assertion that failed, but the format may vary between
different platforms.  

The difference between assertion failures and runtime exceptions is
that the runtime exceptions are normally due to an error in the setup
files, while the assertion may also be caused by bugs in the program.
The printed information is a great help for debugging those.  So
failed assertions should be reported to KVL.

\subsubsection{Program errors}

Sometimes we can get errors that aren't even caught by the assertions,
they can manifest themselves as "Bus error" or "Segmentation
violation" errors, or weird pop-up windows under win32.  Or even
worse, as bogus numbers in the log files.  These can be due to program
errors, hardware errors, too little memory, or errors in the operating
system.  They can also be caused by errors in the setup files, but in
this case it is also a program error, as such errors should be caught
by one of the mechanisms above.

%%% Local Variables: 
%%% mode: latex
%%% TeX-master: "reference"
%%% End: 

%%% introduction.tex ends here
