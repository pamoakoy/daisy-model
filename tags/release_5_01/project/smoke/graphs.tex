\documentclass[a4paper]{article}
\usepackage[T1]{fontenc}
\usepackage[latin1]{inputenc}

\usepackage{graphicx}

\usepackage[top=3cm,bottom=2cm]{geometry}
%% \usepackage[left=1cm,top=1cm,right=1cm,nohead,nofoot]{geometry}

\usepackage{natbib}
\bibliographystyle{apalike}

\newcommand{\fig}[1]{\includegraphics{pdf/#1}}

\usepackage{fancyhdr}
\pagestyle{fancy}
\lhead{\today}
\chead{\thepage}
\rhead{MST Smoke}
\cfoot{}

\begin{document}

\section*{u4 Bromide in soil samples}

\noindent\fig{u4-Bromide}

%% \begin{tabular}{lllllll}
%%   R$^2$ & S$_{lim}$ & A$_{lim}$ & B$_{lim}$ & S$_\alpha$ & X$_\alpha$ \\
%%   & cm & cm & cm & h$^-1$ & h$^-1$ \\
%%   0.991763 & -9325.09 & -1202.38 & -338.895 & 1.44872e-11 & 8.33685e-05
%% \end{tabular}

\section*{u4 Brilliant Blue in soil samples}

\noindent\fig{u4-BB-base}

%% \begin{tabular}{lllll}
%%   R$^2$ & K$_{clay}$ & pore$_{min}$ & k & Applied  \\
%%   & cm$^3$/g & $\mu$m & h$^{-1}$ & g/ha \\
%%   0.953404 & 7.18551 &0.1 & 0.244831 & 9.09419e+06 ( 51.4 \%)
%% \end{tabular}

\section*{m19 Bromide in soil samples}

\noindent\fig{m19-Bromide}

%% \begin{tabular}{lllllll}
%%   R$^2$ & S$_{lim}$ & A$_{lim}$ & B$_{lim}$ & S$_\alpha$ & X$_\alpha$ \\
%%   & cm & cm & cm & h$^-1$ & h$^-1$ \\
%%   0.999541 & -13775.2 & -545.848 & -272.606 & 1.04634e-12 & 1.77239e-05
%% \end{tabular}

\section*{m19 Brilliant Blue in soil samples}

\noindent\fig{m19-BB-base}

%% \begin{tabular}{lllll}
%%   R$^2$ & K$_{clay}$ & pore$_{min}$ & k & Applied  \\
%%   & cm$^3$/g & $\mu$m & h$^{-1}$ & g/ha \\
%%   0.86986 & 1.44327 & 0.987659 & 1 & 6.97091e+06 (39.4 \%)
%% \end{tabular}

\section*{d1 Bromide in soil samples}

\noindent\fig{d1-Bromide}

%% \begin{tabular}{lllllll}
%%   R$^2$ & S$_{lim}$ & A$_{lim}$ & B$_{lim}$ & S$_\alpha$ & X$_\alpha$ \\
%%   & cm & cm & cm & h$^-1$ & h$^-1$ \\
%%   0.999409 & -5.37705e+13 & -429.103 & -444.339 & 0.0135308 & 1.22115e-05
%% \end{tabular}

\section*{d1 Brilliant Blue in soil samples}

\noindent\fig{d1-BB-base}

%% \begin{tabular}{lllll}
%%   R$^2$ & K$_{clay}$ & pore$_{min}$ & k & Applied  \\
%%   & cm$^3$/g & $\mu$m & h$^{-1}$ & g/ha \\
%%    0.967296 & 5.13366 & 1 & 0.914811 & 4.68095e+06 (26.5\%)
%% \end{tabular}

\section*{Initial soil water}

U4

\noindent\fig{u4-Bromide-end}

M19

\noindent\fig{m19-Bromide-end}

D1

\noindent\fig{d1-Bromide-end}

h

\noindent\fig{Pres}

\section*{Rain \& u4 Conductivity}

\noindent\fig{rain}\\
\noindent\fig{u4-conductivity}\\
\noindent\fig{u4-conductivity_2}


\pagebreak{}
\section*{Experiment}

An amount of 5 gram potassium bromide (KBr) and 5 gram of Brilliant
Blue was applied in 63 locations in the experimental field at
R�rrendeg�rd 2010-10-26.  Each application was done within a steel
ring with a diameter of 6 cm.  Three types of locations were chosen.
The first type was around a biopore where there has been strong smoke
(near the drain), the second around a biopore where there were
detected no smoke (5 or 9 meter from the drain), and the third at an
area of the soil without macropores (3.5 meter from the drain).  On
2010-11-13 the field was covered, and 2011-01-18 a cyldinder with 30
cm diameter and 40 cm depth were dug up. The cylinders were cut over
at 9, 18, 24, and 30 cm, giving a total of 5 samples per cylinder.
Each sample was analysed for content of bromide and brilliant blue
content, as well as dry bulk density ($\rho_b$).  The 9--18 cm sample
had lower average
dry bulk density, likely due to a higher degree of crop residuals.\\\\
\begin{tabular}{|c|c|ccccc|}\hline
  Interval & cm & 0--9 & 9--18 & 18--24 & 24--30 & 30--40\\
  $\rho_b$ & g/cm$^3$ & 1.60 & 1.53 & 1.60 & 1.72 & 1.72\\\hline
\end{tabular}\\

The mean and standard error was found for each combination of location
type and sample depth, and are shown on the preceeding page.

Weather data was provided by Jens Rauns� Jensen from a nearby weather
station.  The rain measurements from that station is known to
problematic.  The weather data was provided with 15 minutes intervals,
and aggregated to hourly values.  From 2010-10-21 14:10 rain data was
measured at the field with 5 minutes intervals.

\section*{Daisy setup}

The setup documented in~\citet{mst-agrovand} is mostly used as a base.

We chose to adjust the upper horizon borders so they match
our sampling depth.  So the Ap horizon is 0--24 cm and the plow pan
from 24-30 cm.  The C horizon is unchanged from~\cite{mst-agrovand}.
Texture is taken from \citet{mst-agrovand}. $\rho_b$ is set to
1.6~g/cm$^3$ in the Ap horizon (including the surface layer), and
1.72~g/cm$^3$ in the Bt horizon (including the plow pan).  Gydraulic
properties are estimated with \textsc{hypres}~\citep{hypres}.  We
chose an 1D setup.  Discretization was adjusted to match sampling
depths, and increased to Groundwater table variation was estimated
from old measured values \citep{mst-agrovand}.  All considered
biopores start at the surface, and have a diameter of 5 mm.  A single
biopore in a cicle with a diamter of 6 cm corresponds to a biopore
density of 354 per square meter.

Based on this, three profiles were created.
\begin{description}
\item[drain] Corresponding to locations with biopores with smoke.  A
  single biopore class was used, with biopores that ends in drain
  pipes.  The Bt horizon was replaced with the drain canyon from
  \citet{mst-agrovand}.  The drain connected biopores are responsible
  for draining the soil.
\item[macro] Corresponding to locations with biopores but without
  smoke.  A single biopore class was used, with biopores that ends in
  drain pipe depth, but not in drain pipes.  The 1D drain
  approximation was setup with a distance of 7 meter from the drain
  pipes.
\item[without] Corresponding to locations without biopores.  The
  tertiare transport system was disabled.  The 1D drain approximation
  was setup with a distance of 3.5 meter from the drain.
\end{description}

A warmup period with winter wheat was added to give reasonable initial
conditions for soil water, giving a total simulation period from
2009-6-2 to 2010-12-13.  Daisy was allowed to decrease timestep to 1
microsecond as needed.  The hourly weather data was used for the whole
period, suplemented with 5 minute rain values for the drain season.
Snow was disabled.

The division and exchange rate between primary and secondary domain
was found by calibration based on measurements from locations without
biopores.  

%%\addcontentsline{toc}{chapter}{\numberline{}References}
%%\bibliography{../../txt/daisy}

\end{document}

%%% Local Variables: 
%%% mode: latex
%%% TeX-master: t
%%% End: 
