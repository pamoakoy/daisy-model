\documentclass[a4paper]{article}
\usepackage[latin1]{inputenc}
\usepackage{hyperref}
\usepackage{a4}
\usepackage{verbatim}
\newcommand{\daisy}{Daisy}
\newcommand{\Daisy}{Daisy}
\newcommand{\cplusplus}%
{{\leavevmode{\rm{\hbox{C\hskip -0.1ex\raise 0.5ex\hbox{\tiny ++}}}}}}
\newcommand{\Cplusplus}{\cplusplus}
\newcommand{\mshe}{Mike/\textsc{she}}
\newcommand{\wintel}{\texttt{win32}}
\newcommand{\dll}{\textsc{dll}}
\newcommand{\Dll}{\textsc{Dll}}
\newcommand{\gui}{\textsc{gui}}
\newcommand{\Gui}{\textsc{Gui}}
\newcommand{\unix}{Unix}
\newcommand{\dhi}{\textsc{dhi}}
\newcommand{\Dhi}{\textsc{Dhi}}
\newcommand{\api}{\textsc{api}}
\newcommand{\Api}{\textsc{Api}}
\newcommand{\lai}{\textsc{lai}}
\newcommand{\Lai}{\textsc{Lai}}
%\newcommand{\url}[1]{\linebreak[4]\texttt{<URL:#1>}}

%%% Local Variables: 
%%% mode: latex
%%% TeX-master: t
%%% End: 

\usepackage{natbib}
%\bibliographystyle{plainnat}
\bibliographystyle{apalike}
\usepackage{graphicx}
\usepackage{latexsym}

\begin{document}

\section*{\Daisy{} notes}

This paper contain random \daisy{} notes I can't fit anywhere else.

\section{The \texttt{default} crop model}

This section should explain how to calibrate the \texttt{default} crop
model.


\subsection{Phenology: \textsc{ds}, development stage}
\label{sec:phenology}

Here we should describe how to calibrate the development stage of the
crop.

\subsection{Canopy: {\LAI{}, \textsc{CAI}, oh my\ldots}
\label{sec:canopy}

The \emph{leaf area index}, or \lai{} for short, is the total area of
all leafs on a field or plot, divided by the area of the field or plot
itself.  This term is commonly used in acgricultural science.
\Daisy{} also have a \emph{crop area index}, or \textsc{cai}, which is
less widely used.  The difference between the two is that the
\textsc{cai} also included the photosynthetically effect of green stem
and storage organ.  Since these later are rarely significant, you will
often see \lai{} and \textsc{cai} used for the same.  

When adding a parameterization of a \texttt{default} crop model to
\daisy{}, you should focus on the \lai{} right first, on only bother
with the \textsc{cai} when you know storage organ and stem gives a
significant contribution .  You should have completed calibrating the
\textsc{ds} (section~\ref{sec:phenology}) before calibrating \lai{}.

Depending on how much information you have, you may have to calibrate
the partioning (section~\ref{sec:patit}) simulatiously.

\subsubsection{The model}

How \lai{} (and \textsc{cai}) is calculated depend on whether the crop
has left the initialization phase (\texttt{InitLAI} or not.  In the
initialization phase, \lai{} is calculated like this:
\begin{equation}
  
\end{equation}



I den initialiseringsfasen g�lder

  Exit LAI = SpLAI * WLeaf

  Max LAI = SpLAIfac (DS) * SpLAI * WLeaf

  Fun LAI = 1.0/(1.0+exp(-15.0*(DS-DSLAI05)))

  Hvis DS < 0.07
   
    LAI = Fun LAI

  Ellers 

    LAI = min (Max LAI, Fun LAI)

  CAI = LAI

  Hvis LAI > Exit LAI s� forlader vi initialiseringsfasen

N�r vi er ude af inititaliseringsfasen, s� er

  LAI = LeafAIMod (DS) * SpLAI * WLeaf
  SOrgAI = SpSOrgAI * SOrgAIMod (DS) * WSOrg;
  StemAI = SpStemAI * StemAIMod (DS) * WStem;

  CAI = LeafAI + StemPhotEff * StemAI + SOrgPhotEff * SOrgAI;


\begin{displaymath}
\frac{1}{1+e^{-15 (\mbox{\sc ds}-\mbox{\sc dslai05})}}
\end{displaymath}

\subsection{Patitioning}
\label{sec:patit}



\section*{Crop \lai{} parameterization}

The \emph{leaf area index}, or \lai{} for short, is the total area of
all leafs on a field or plot, divided by the area of the field or plot
itself.  The \lai{} is important for photosynthesis and transpiration.
The larger the \lai{} is, the more sunlight will be caught.  This is
true even for \lai{} larger than 1, because the leafs are not
uniformly distributed over the field so there is some overlap
However, it is of course a case of dimishing returns.  A \lai{} of 5
is commong for fully developed crops.

The \textsc{default} crop model has three different mechanisms for
calculating \lai{}.  These are used during different development
phases of the crop.

The first phase used right after emergence is is a forced curve, which
is unique in that it depend solely on the development stage of the
crop.  In the two other phases, the \lai{} is a function of the leaf
dry matter.  But when the crop just emerge, it has no leaf dry
matter.  In order to create keaf dry matter, it needs assimilate.  In
order to create assimilate, it needs to catch sunligt.  Which again
require a leaf area.  In order to escape this catch-22, the \lai{} is
initialy calculated based on age alone.  The function is
\begin{displaymath}
\frac{1}{1+e^{-15 (\mbox{\sc ds}-\mbox{\sc dslai05})}}
\end{displaymath}
This function will be close to zero at DS = 0, 0.5 at DS =
\texttt{DSLAI05}, and grow towards one from there.  The only
calibration parameter for this phase is \texttt{DSLAI05}.  The smaller
you mae this number, the fast LAI will grow towards 0.5.

The crop will leave this phase when the LAI


\end{document}
