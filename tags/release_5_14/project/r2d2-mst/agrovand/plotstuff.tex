\documentclass[a4paper]{article}

\usepackage{graphicx}
\usepackage{natbib}

\usepackage[top=3cm,bottom=2cm]{geometry}
%% \usepackage[left=1cm,top=1cm,right=1cm,nohead,nofoot]{geometry}

\newcommand{\figl}{\hspace*{-2cm}}
\newcommand{\figright}[1]{\includegraphics{fig/#1}}
\newcommand{\fig}[1]{\figl\figright{#1}}
\newcommand{\figtop}[1]{\figl\includegraphics[trim=0mm 5mm 0mm 0mm,clip]{fig/#1}}
\newcommand{\figctop}[1]{\hspace*{-1cm}\figright{#1}} 
\newcommand{\figc}[1]{\vspace*{-1.5cm}\figctop{#1}}

\newcommand{\Hypres}{\textsc{Hypres}}
\newcommand{\MyID}{\today:~}

\usepackage{placeins}
\usepackage{fancyhdr}
\pagestyle{fancy}
\lhead{\today}

\begin{document}

\begin{figure}[htbp] 
  \fig{Rorrende-Ap-Theta}\figright{Rorrende-Ap-K}\\
  \fig{Rorrende-Bt-Theta}\figright{Rorrende-Bt-K}\\
  \fig{Rorrende-C-Theta}\figright{Rorrende-C-K}\\
  \fig{Rorrende-DC-Theta}\figright{Rorrende-DC-K}
  \caption{\MyID{}R{\o}rrende soil hydraulic properties.  \Hypres{} refers to
    parameters estimated according to \citet{hypres}, Daisy to the
    final parametrization (ignoring anisotropy and biopores), and
    Surface and plow pan to the conditions at the top of the A and Bt
    horizons.  DC is the drain canyon.}
  \label{fig:Rorrende-hor}
\end{figure}

\begin{figure}[htbp]
  \begin{center}
    \fig{gwt}
  \end{center}
  \caption{Groundwater table.  Simulated low value is calculated from
    pressure in lowest unsaturated numeric cell, typically located
    near drain.  Simulated high value is calculated from pressure in
    highest saturated cell, typically farthest away from the drain.}
  \label{fig:gw}
\end{figure}

\newgeometry{left=1cm,top=1cm,right=1cm,bottom=1cm,nohead,nofoot}
\pagestyle{empty}
\begin{figure}[htbp]
  \begin{center}
    \figctop{weather} \\
    \figc{theta4cm} \\
    \figc{theta8cm} \\
    \figc{theta12cm} \\
    \figc{theta16cm} \\
    \figc{theta20cm} \\
    \figc{theta24cm} \\
    \figc{theta36cm} \\
    \figc{theta60cm}
  \end{center}
  \caption{\MyID{}TDR measurements.}
  \label{fig:tdr}
  \label{fig:first}
\end{figure}

\begin{figure}[htbp]
  \begin{center}
    \figctop{weather_short} \\
    \figc{theta_short4cm} \\
    \figc{theta_short8cm} \\
    \figc{theta_short12cm} \\
    \figc{theta_short16cm} \\
    \figc{theta_short20cm} \\
    \figc{theta_short24cm} \\
    \figc{theta_short36cm} \\
    \figc{theta_short60cm}
  \end{center}
  \caption{\MyID{}Early TDR measurements.}
  \label{fig:tdr-zoom}
\end{figure}

\begin{figure}[htbp]
  \begin{center}
    \figctop{weather_brominf} \\
    \figc{infiltration}\\
    \figc{pondingdepth}\\
    \figc{brom-input} \\
    \figc{brom-0-25-output} \\
    \figc{brom-25-50-output} \\
    \figc{brom-50-75-output} \\
    \figc{brom-75-100-output}
  \end{center}
  \caption{\MyID{}Bromide dynamics.}
  \label{fig:bromide}
\end{figure}

\begin{figure}[htbp]
  \begin{center}
    \figctop{brom-total} \\
    \figc{brom-primary} \\
    \figc{brom-secondary} \\
    \figc{brom-input-acc} \\
    \figc{brom-0-25-acc} \\
    \figc{brom-25-50-acc} \\
    \figc{brom-50-75-acc} \\
    \figc{brom-75-100-acc}
  \end{center}
  \caption{\MyID{}Accumulated bromide.}
  \label{fig:bromide-acc}
\end{figure}

\begin{figure}[htbp]
  \begin{center}
    \figctop{weather-98-99} \\
    \figc{drainflow-98-99} \\
    \figc{drainflowacc-98-99} \\
    \figc{particles-98-99} \\
    \figc{particlesacc-98-99} \\
    \figc{bromide-98-99} \\
    \figc{brommass-98-99}
  \end{center}
  \caption{\MyID{}Drain season 1998 -- 1999.}
  \label{fig:season9899}
\end{figure}

\begin{figure}[htbp]
  \begin{center}
    \figctop{weather-98-99-zoom} \\
    \figc{drainflow-98-99-zoom} \\
    \figc{drainflowacc-98-99-zoom} \\
    \figc{particles-98-99-zoom} \\
    \figc{particlesacc-98-99-zoom} \\
    \figc{bromide-98-99-zoom} \\
    \figc{brommass-98-99-zoom}
  \end{center}
  \caption{\MyID{}Drain season 1998 --- 1999, single event.}
  \label{fig:season9899zoom}
\end{figure}

\begin{figure}[htbp]
  \begin{center}
    \figctop{weather-99-00} \\
    \figc{drainflow-99-00} \\
    \figc{drainflowacc-99-00} \\
    \figc{particles-99-00} \\
    \figc{particlesacc-99-00} \\
    \figc{bromide-99-00} \\
    \figc{brommass-99-00} \\
    \figc{pendconc-99-00} \\
    \figc{pendmass-99-00}
  \end{center}
  \caption{\MyID{}Drain season 1999 --- 2000.}
  \label{fig:season9900}
\end{figure}

\begin{figure}[htbp]
  \begin{center}
    \figctop{weather-99-00-zoom} \\
    \figc{drainflow-99-00-zoom} \\
    \figc{drainflowacc-99-00-zoom} \\
    \figc{particles-99-00-zoom} \\
    \figc{particlesacc-99-00-zoom} \\
    \figc{bromide-99-00-zoom} \\
    \figc{brommass-99-00-zoom} \\
    \figc{pendconc-99-00-zoom} \\
    \figc{pendmass-99-00-zoom}
  \end{center}
  \caption{\MyID{}Drain season 1999 --- 2000, single event.}
  \label{fig:season9900zoom}
\end{figure}

\begin{figure}[htbp]
  \begin{center}
    \figctop{weather-00-01} \\
    \figc{drainflow-00-01} \\
    \figc{drainflowacc-00-01} \\
    \figc{particles-00-01} \\
    \figc{particlesacc-00-01} \\
    \figc{ioxconc-00-01} \\
    \figc{ioxmass-00-01} \\
    \figc{pendconc-00-01} \\
    \figc{pendmass-00-01}
  \end{center}
  \caption{\MyID{}Drain season 2000 --- 2001.}
  \label{fig:season0001}
\end{figure}

\begin{figure}[htbp]
  \begin{center}
    \figctop{weather-00-01-zoom} \\
    \figc{drainflow-00-01-zoom} \\
    \figc{drainflowacc-00-01-zoom} \\
    \figc{particles-00-01-zoom} \\
    \figc{particlesacc-00-01-zoom} \\
    \figc{ioxconc-00-01-zoom} \\
    \figc{ioxmass-00-01-zoom} \\
    \figc{pendconc-00-01-zoom} \\
    \figc{pendmass-00-01-zoom}
  \end{center}
  \caption{\MyID{}Drain season 2000 --- 2001, single event.}
  \label{fig:season0001zoom}
  \label{fig:last}
\end{figure}

%%% Local Variables: 
%%% mode: latex
%%% TeX-master: nil
%%% End: 


\end{document}
