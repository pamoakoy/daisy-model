\newcommand{\figsilstrupl}[1]{\figl\includegraphics[trim=8mm 0mm 12mm 7mm,clip]{fig/#1}}
\newcommand{\figsilstrup}[1]{\includegraphics[trim=8mm 0mm 12mm 7mm,clip]{fig/#1}}
\newcommand{\fluxtop}[1]{\figl\includegraphics[trim=0mm 10mm 0mm 0mm,clip]{fig/#1}}
\newcommand{\figestrupl}[1]{\hspace*{-1cm}\includegraphics[trim=12mm 0mm 17mm 9mm,clip]{fig/#1}}
\newcommand{\figestrup}[1]{\includegraphics[trim=12mm 0mm 17mm 9mm,clip]{fig/#1}}

\chapter{2D plots}
\label{app:plot-2d}

In this appendix we present simulated 2D plots for water, bromide,
glyphosate, and metamitron.  There are no measurements to compare
with, a major caveat for both the results and discussion.  We use two
kinds of graphs to capture the 2D structure.

The first kind depict static distribution in the soil.  Each graph has
horizontal distance from drain on the x-axis and height above surface
on the y-axis, using the same scale for both axes.  The graph
represents the the computational soil area used in the simulation.
The right side is the center between two drains (9 meter for Silstrup
and 6.5 meter Estrup), and the bottom is 5 meter, where we use the
measured groundwater pressure table as the lower boundary.  The graphs
are color coded, where specific colors represent specific values for
the soil at the end of the month indicated by the graph title.  Each
numeric cell in the computation has a color representing the value
within that cell.  Since cells are rectangular, the graphs appear
blocky.

The second kind of graph depicts horizontal or vertical movement.  For
the graphs depicting horizontal movement, the y-axis specifies height
above surface (negative number) and the x-axis movement away from
drain (usually also negative).  The horizontal movement at different
distances from the drain pipes are shown as separate plots on each
graph.  For the graphs depicting vertical movement, the axes are
swapped.  The individual plots represent different depths.  We use the
same flow units as we used for the original input, so e.g.\ pesticide
transport is given in g/ha.

\FloatBarrier
\section{Water}

\subsection{Distribution}

The Silstrup soil water pressure potential
(figure~\ref{fig:Silstrup-pF-2000} and~\ref{fig:Silstrup-pF-2001})
rarely show any horizontal gradients, in contrast to Estrup
(figure~\ref{fig:Estrup-pF-2000} and~\ref{fig:Estrup-pF-2001}) where
there is a clear horizontal gradient in the drain season.  This
reflects the much higher conductivity of the Silstrup soil, where the
soil down to 3.5 m all have a high saturated horizontal conductivity
due to cracks.  The exception is the plow pan, on top of which we
several times see a build up of water.  The plow pan also acts as a
barrier the other direction, where we at Silstrup (unlike Estrup) see
the plow layer dry out to near wilting point both summers.

\subsection{Flow}

For both sites we see, unsurprisingly, large horizontal flow near the
drain in direction of the drains
(figure~\ref{fig:Silstrup-water-horizontal}
and~\ref{fig:Estrup-water-horizontal}).  For Estrup we also see an
even larger horizontal flow in the plow layer, largest one meter from
the drain.  At Estrup only the plow layer has a good horizontal
conductivity.  For Silstrup, the vertical flow graphs
(figure~\ref{fig:Silstrup-water-2000}
and~\ref{fig:Silstrup-water-2001}) show us that:
\begin{itemize}
\item The deep percolation (the -150 and -200 cm plots, top graph) are
  pretty much unaffected by the position of the drain pipes.
\item The effect of surface flow can be seen on the biopore activity
  (the 0 cm plot, bottom graph).
\item The plow pan contribute relatively little to the total biopore
  activity (-50 cm compared to 0 cm, bottom graph).
\item The area near the drain is far more active than the rest of the
  field for vertical movement, almost exclusively due to the biopores.
\end{itemize}

In contrast, on Estrup (figure~\ref{fig:Estrup-water-2000}
and~\ref{fig:Estrup-water-2001}) the higher groundwater means we get
significant contributions to the drains from below, there is no
significant surface flow or biopore activation on surface, and the
plow pan seems to be an important factor for biopore activation.  The
area above the drain is still much more active than the rest of the
field, and the biopores play a large role in this.

\begin{figure}[htbp]\centering
  \begin{tabular}{ccc}
    \figsilstrupl{Silstrup-pF-2000-5} & 
    \figsilstrup{Silstrup-pF-2000-6} & 
    \figsilstrup{Silstrup-pF-2000-7} \\
    \figsilstrupl{Silstrup-pF-2000-8} & 
    \figsilstrup{Silstrup-pF-2000-9} & 
    \figsilstrup{Silstrup-pF-2000-10} \\
    \figsilstrupl{Silstrup-pF-2000-11} & 
    \figsilstrup{Silstrup-pF-2000-12} & 
    \figsilstrup{Silstrup-pF-2001-1} \\
    \figsilstrupl{Silstrup-pF-2001-2} & 
    \figsilstrup{Silstrup-pF-2001-3} & 
    \figsilstrup{Silstrup-pF-2001-4}
  \end{tabular}
  
  \caption{Silstrup soil water pressure potential at the end of each
    month since first application of bromide.  The y-axis denotes
    depth, the x-axis distance from drain.  There are tick marks for
    every meter.  Blue denotes pF<0, white pF=1, yellow pF=2, orange
    pF=3, red pF=4, and black pF>5.}
\label{fig:Silstrup-pF-2000}
\end{figure}

\begin{figure}[htbp]\centering
  \begin{tabular}{ccc}
    \figsilstrupl{Silstrup-pF-2001-5} & 
    \figsilstrup{Silstrup-pF-2001-6} & 
    \figsilstrup{Silstrup-pF-2001-7} \\
    \figsilstrupl{Silstrup-pF-2001-8} & 
    \figsilstrup{Silstrup-pF-2001-9} & 
    \figsilstrup{Silstrup-pF-2001-10} \\
    \figsilstrupl{Silstrup-pF-2001-11} & 
    \figsilstrup{Silstrup-pF-2001-12} & 
    \figsilstrup{Silstrup-pF-2002-1} \\
    \figsilstrupl{Silstrup-pF-2002-2} & &
  \end{tabular}
  
  \caption{Silstrup soil water pressure potential at the end of each
    month second year after application of bromide.  The y-axis
    denotes depth, the x-axis distance from drain.  There are tick
    marks for every meter.  Blue denotes pF<0, white pF=1, yellow
    pF=2, orange pF=3, red pF=4, and black pF>5.}
\label{fig:Silstrup-pF-2001}
\end{figure}

\begin{figure}[htbp]\centering
  \begin{tabular}{ccc}
    \figestrupl{Estrup-pF-2000-5} & 
    \figestrup{Estrup-pF-2000-6} & 
    \figestrup{Estrup-pF-2000-7} \\
    \figestrupl{Estrup-pF-2000-8} & 
    \figestrup{Estrup-pF-2000-9} & 
    \figestrup{Estrup-pF-2000-10} \\
    \figestrupl{Estrup-pF-2000-11} & 
    \figestrup{Estrup-pF-2000-12} & 
    \figestrup{Estrup-pF-2001-1} \\
    \figestrupl{Estrup-pF-2001-2} & 
    \figestrup{Estrup-pF-2001-3} & 
    \figestrup{Estrup-pF-2001-4}
  \end{tabular}
  
  \caption{Estrup soil water pressure potential at the end of each
    month since first application of bromide.  The y-axis denotes
    depth, the x-axis distance from drain.  There are tick marks for
    every meter.  Blue denotes pF<0, white pF=1, yellow pF=2, orange
    pF=3, red pF=4, and black pF>5.}
\label{fig:Estrup-pF-2000}
\end{figure}

\begin{figure}[htbp]\centering
  \begin{tabular}{ccc}
    \figestrupl{Estrup-pF-2001-5} & 
    \figestrup{Estrup-pF-2001-6} & 
    \figestrup{Estrup-pF-2001-7} \\
    \figestrupl{Estrup-pF-2001-8} & 
    \figestrup{Estrup-pF-2001-9} & 
    \figestrup{Estrup-pF-2001-10} \\
    \figestrupl{Estrup-pF-2001-11} & 
    \figestrup{Estrup-pF-2001-12} & 
    \figestrup{Estrup-pF-2002-1} \\
    \figestrupl{Estrup-pF-2002-2} & 
    \figestrup{Estrup-pF-2002-3} & 
    \figestrup{Estrup-pF-2002-4}
  \end{tabular}
  
  \caption{Estrup soil water pressure potential at the end of each
    month second year after application of bromide.  The y-axis
    denotes depth, the x-axis distance from drain.  There are tick
    marks for every meter.  Blue denotes pF<0, white pF=1, yellow
    pF=2, orange pF=3, red pF=4, and black pF>5.}
\label{fig:Estrup-pF-2001}
\end{figure}

\begin{figure}[htbp]
  \centering
  \figtop{Silstrup-water-horizontal-2000}
  \fig{Silstrup-water-horizontal-2001}
  
  \caption{Silstrup total horizontal water flux between 2000-5-1 and
    2001-5-1 (top) and between 2001-5-1 and 2002-3-1 (bottom).  The
    flux is shown on the x-axis (positive away from drain) as a
    function of depth shown on the y-axis.  The graph labels are the
    distance from drain in centimeters.}
  \label{fig:Silstrup-water-horizontal}
\end{figure}

\begin{figure}[htbp]
  \centering
  \figtop{Estrup-water-horizontal-2000}
  \fig{Estrup-water-horizontal-2001}
  
  \caption{Estrup total horizontal water flux between 2000-5-1 and
    2001-5-1 (top) and between 2001-5-1 and 2002-5-1 (bottom).  The
    flux is shown on the x-axis (positive away from drain) as a
    function of depth shown on the y-axis.  The graph labels are the
    distance from drain in centimeters.}
  \label{fig:Estrup-water-horizontal}
\end{figure}

\begin{figure}[htbp]
  \centering
  \figtop{Silstrup-water-2000}
  \fig{Silstrup-water-biopore-2000}
  
  \caption{Silstrup vertical water flux between 2000-5-1 and
    2001-5-1.  Top graph show total flux, bottom graph only biopores.  The flux is shown on the y-axis (positive up) as a
    function of distance from drain shown on the x-axis.  The graph
    labels are depths in centimeters above surface.}
  \label{fig:Silstrup-water-2000}
\end{figure}

\begin{figure}[htbp]
  \centering
  \figtop{Silstrup-water-2001}
  \fig{Silstrup-water-biopore-2001}
  
  \caption{Silstrup vertical water flux between 2001-5-1 and 2002-3-1.
    Top graph show total flux, bottom graph only biopores.  The flux
    is shown on the y-axis (positive up) as a function of distance
    from drain shown on the x-axis.  The graph labels are depths in
    centimeters above surface.}
  \label{fig:Silstrup-water-2001}
\end{figure}

\begin{figure}[htbp]
  \centering
  \figtop{Estrup-water-2000}
  \fig{Estrup-water-biopore-2000}
  
  \caption{Estrup vertical water flux between 2000-5-1 and 2001-5-1.
    Top graph show total flux, bottom graph only biopores. The flux is
    shown on the y-axis (positive up) as a function of distance from
    drain shown on the x-axis.  The graph labels are depths in
    centimeters above surface.}
  \label{fig:Estrup-water-2000}
\end{figure}

\begin{figure}[htbp]
  \centering
  \figtop{Estrup-water-2001}
  \fig{Estrup-water-biopore-2001}
  
  \caption{Estrup vertical water flux between 2001-5-1 and 2002-5-1.
    Top graph show total flux, bottom graph only biopores. The flux is
    shown on the y-axis (positive up) as a function of distance from
    drain shown on the x-axis.  The graph labels are depths in
    centimeters above surface.}
  \label{fig:Estrup-water-2001}
\end{figure}

\FloatBarrier
\section{Bromide}

\subsection{Distribution}

Like for water, there is hardly any horizontal gradients worth
speaking of for bromide at Silstrup
(figure~\ref{fig:Silstrup-Bromide-2000}
and~\ref{fig:Silstrup-Bromide-2001}).  The bromide is mostly contained
within the plow layer the first summer, but at the end of the drain
season, the bromide is everywhere.  Estrup shows a different pattern
(figure~\ref{fig:Estrup-Bromide-2000}
and~\ref{fig:Estrup-Bromide-2001}).  At the end of summer, most of the
bromide has left the plow layer, and the upward direction of the water
flow below the drain pipes keep that part of the soil relatively clear
of bromide.  In the second year, the horizontal flow of of water in
the plow layer is resulting in the soil above drain pipes also being
cleared of bromide.

\subsection{Transport}

The most interesting thing to note about the horizontal bromide
transport is that the rather small horizontal flow of water depicted
on the top graph of figure~\ref{fig:Silstrup-water-horizontal}
translate into a much more significant transport of bromide shown on
figure~\ref{fig:Silstrup-Bromide-horizontal}.  This indicates that the
horizontal water flow happens early, when the bromide concentration of
the plow layer is still high.  The bottom graph of
figures~\ref{fig:Silstrup-Bromide-horizontal}
and~\ref{fig:Estrup-Bromide-horizontal} both show less horizontal transport
the second year, especially in the plow layer.

Figure~\ref{fig:Silstrup-Bromide-2000-vertical} shows us that all the
bromide enter through the matrix, and only half the bromide leaver the
top 25 cm.  We also see the biopores being activated between -25 and
-50 cm, indicating the plow pan being significant.  The drain pipes
only visibly affect the transport right on top of them (-100 cm),
where most of the transport is through biopores.  The second year
(figure~\ref{fig:Silstrup-Bromide-2001-vertical}) does not show much
transport at all, except right above the pipes like the year before.  For
Estrup, we see a strong matrix transport with right above the pipes, with
some contributions from biopores
(figure~\ref{fig:Estrup-Bromide-2000-vertical}).  The bromide leaching
from the top 25 cm is slightly higher than for Silstrup, and dominated
by matrix transport.  The second year
(figure~\ref{fig:Estrup-Bromide-2001-vertical}) we get contribution to
the drains from both above and below, almost exclusively through
matrix transport.

\begin{figure}[htbp]\centering
  \begin{tabular}{ccc}
    \figsilstrupl{Silstrup-M-Bromide-2000-5} & 
    \figsilstrup{Silstrup-M-Bromide-2000-6} & 
    \figsilstrup{Silstrup-M-Bromide-2000-7} \\
    \figsilstrupl{Silstrup-M-Bromide-2000-8} & 
    \figsilstrup{Silstrup-M-Bromide-2000-9} & 
    \figsilstrup{Silstrup-M-Bromide-2000-10} \\
    \figsilstrupl{Silstrup-M-Bromide-2000-11} & 
    \figsilstrup{Silstrup-M-Bromide-2000-12} & 
    \figsilstrup{Silstrup-M-Bromide-2001-1} \\
    \figsilstrupl{Silstrup-M-Bromide-2001-2} & 
    \figsilstrup{Silstrup-M-Bromide-2001-3} & 
    \figsilstrup{Silstrup-M-Bromide-2001-4}
  \end{tabular}
  
  \caption{Silstrup bromide soil content at the end of each month
    since first application of bromide.  The y-axis denotes depth, the
    x-axis distance from drain.  There are tick marks for every
    meter. The color scale is white<10 pg/l, yellow=1 ng/l, orange=0.1
    $\mu$g/l, red=10 $\mu$g/l, and black>1 mg/l}
\label{fig:Silstrup-Bromide-2000}
\end{figure}

\begin{figure}[htbp]\centering
  \begin{tabular}{ccc}
    \figsilstrupl{Silstrup-M-Bromide-2001-5} & 
    \figsilstrup{Silstrup-M-Bromide-2001-6} & 
    \figsilstrup{Silstrup-M-Bromide-2001-7} \\
    \figsilstrupl{Silstrup-M-Bromide-2001-8} & 
    \figsilstrup{Silstrup-M-Bromide-2001-9} & 
    \figsilstrup{Silstrup-M-Bromide-2001-10} \\
    \figsilstrupl{Silstrup-M-Bromide-2001-11} & 
    \figsilstrup{Silstrup-M-Bromide-2001-12} & 
    \figsilstrup{Silstrup-M-Bromide-2002-1} \\
    \figsilstrupl{Silstrup-M-Bromide-2002-2} &  & 
  \end{tabular}
  
  \caption{Silstrup bromide soil content at the end of each month
    second year after application of bromide.  The y-axis denotes
    depth, the x-axis distance from drain.  There are tick marks for
    every meter. The color scale is white<10 pg/l, yellow=1 ng/l,
    orange=0.1 $\mu$g/l, red=10 $\mu$g/l, and black>1 mg/l}
\label{fig:Silstrup-Bromide-2001}
\end{figure}

\begin{figure}[htbp]\centering
  \begin{tabular}{ccc}
    \figestrupl{Estrup-M-Bromide-2000-5} & 
    \figestrup{Estrup-M-Bromide-2000-6} & 
    \figestrup{Estrup-M-Bromide-2000-7} \\
    \figestrupl{Estrup-M-Bromide-2000-8} & 
    \figestrup{Estrup-M-Bromide-2000-9} & 
    \figestrup{Estrup-M-Bromide-2000-10} \\
    \figestrupl{Estrup-M-Bromide-2000-11} & 
    \figestrup{Estrup-M-Bromide-2000-12} & 
    \figestrup{Estrup-M-Bromide-2001-1} \\
    \figestrupl{Estrup-M-Bromide-2001-2} & 
    \figestrup{Estrup-M-Bromide-2001-3} & 
    \figestrup{Estrup-M-Bromide-2001-4}
  \end{tabular}
  
  \caption{Estrup bromide soil content at the end of each month since
    first application of bromide.  The y-axis denotes depth, the x-axis distance from drain.  There are tick marks for every
    meter. The color scale is white<10 pg/l, yellow=1 ng/l,
    orange=0.1 $\mu$g/l, red=10 $\mu$g/l, and black>1 mg/l}
\label{fig:Estrup-Bromide-2000}
\end{figure}

\begin{figure}[htbp]\centering
  \begin{tabular}{ccc}
    \figestrupl{Estrup-M-Bromide-2001-5} & 
    \figestrup{Estrup-M-Bromide-2001-6} & 
    \figestrup{Estrup-M-Bromide-2001-7} \\
    \figestrupl{Estrup-M-Bromide-2001-8} & 
    \figestrup{Estrup-M-Bromide-2001-9} & 
    \figestrup{Estrup-M-Bromide-2001-10} \\
    \figestrupl{Estrup-M-Bromide-2001-11} & 
    \figestrup{Estrup-M-Bromide-2001-12} & 
    \figestrup{Estrup-M-Bromide-2002-1} \\
    \figestrupl{Estrup-M-Bromide-2002-2} & 
    \figestrup{Estrup-M-Bromide-2002-3} & 
    \figestrup{Estrup-M-Bromide-2002-4}
  \end{tabular}
  
  \caption{Estrup bromide soil content at the end of each month second
    year after application of bromide.  The y-axis denotes depth, the
    x-axis distance from drain.  There are tick marks for every
    meter. The color scale is white<10 pg/l, yellow=1 ng/l, orange=0.1
    $\mu$g/l, red=10 $\mu$g/l, and black>1 mg/l}
\label{fig:Estrup-Bromide-2001}
\end{figure}

\begin{figure}[htbp]
  \centering
  \figtop{Silstrup-Bromide-horizontal-2000}
  \fig{Silstrup-Bromide-horizontal-2001}
  
  \caption{Silstrup total horizontal bromide transport between 2000-5-1 and
    2001-5-1 (top) and between 2001-5-1 and 2002-3-1 (bottom).  The
    transport is shown on the x-axis (positive away from drain) as a
    function of depth shown on the y-axis.  The graph labels are the
    distance from drain in centimeters.}
  \label{fig:Silstrup-Bromide-horizontal}
\end{figure}

\begin{figure}[htbp]
  \centering
  \figtop{Estrup-Bromide-horizontal-2000}
  \fig{Estrup-Bromide-horizontal-2001}
  
  \caption{Estrup total horizontal bromide transport between 2000-5-1 and
    2001-5-1 (top) and between 2001-5-1 and 2002-5-1 (bottom).  The
    transport is shown on the x-axis (positive away from drain) as a
    function of depth shown on the y-axis.  The graph labels are the
    distance from drain in centimeters.}
  \label{fig:Estrup-Bromide-horizontal}
\end{figure}

\begin{figure}[htbp]
  \centering
  \figtop{Silstrup-Bromide-2000}
  \fig{Silstrup-Bromide-biopore-2000}
  
  \caption{Silstrup total (top) and biopores (bottom) vertical bromide
    transport between 2000-5-1 and 2001-5-1.  The transport is shown on the
    y-axis (positive up) as a function of distance from drain shown on
    the x-axis.  The graph labels are depths in centimeters above
    surface.}
  \label{fig:Silstrup-Bromide-2000-vertical}
\end{figure}

\begin{figure}[htbp]
  \centering
  \figtop{Silstrup-Bromide-2001}
  \fig{Silstrup-Bromide-biopore-2001}
  
  \caption{Silstrup total (too) and biopore (bottom) vertical bromide
    transport between 2001-5-1 and 2002-3-1.  The transport is shown on the
    y-axis (positive up) as a function of distance from drain shown on
    the x-axis.  The graph labels are depths in centimeters above
    surface.}
  \label{fig:Silstrup-Bromide-2001-vertical}
\end{figure}

\begin{figure}[htbp]
  \centering
  \figtop{Estrup-Bromide-2000}
  \fig{Estrup-Bromide-biopore-2000}
  
  \caption{Estrup total (top) and biopore (bottom) vertical bromide
    transport between 2000-5-1 and 2001-5-1.  The transport is shown on the
    y-axis (positive up) as a function of distance from drain shown on
    the x-axis.  The graph labels are depths in centimeters above
    surface.}
  \label{fig:Estrup-Bromide-2000-vertical}
\end{figure}

\begin{figure}[htbp]
  \centering
  \figtop{Estrup-Bromide-2001}
  \fig{Estrup-Bromide-biopore-2001}
  
  \caption{Estrup total (top) and biopore (bottom) vertical bromide
    transport between 2001-5-1 and 2002-5-1.  The transport is shown on the
    y-axis (positive up) as a function of distance from drain shown on
    the x-axis.  The graph labels are depths in centimeters above
    surface.}
  \label{fig:Estrup-Bromide-2001-vertical}
\end{figure}

\FloatBarrier
\section{Metamitron}

\subsection{Distribution}

Figure~\ref{fig:Silstrup-M-Metamitron-2000} shows the metamitron
entering first the plow layer, and later being transported to the end
of the biopores, indicating that the plow pan could be important for
metamitron dynamics.  The metamitron eventually disappear from the
plow layer, but linger at the end of the biopores (where there is no
degradation).  It is more likely diluted than removed.
Figure~\ref{fig:Silstrup-C-Metamitron-2000} shows concentration in soil
water, where four months after application only the soil near the end
of the biopores show concentrations near the limit for drinking water
(0.1 $\mu$g/l).

\subsection{Transport}


Figure~\ref{fig:Silstrup-Metamitron-2000-vertical} shows that most of
the metamitron enter the soil through the matrix, and only above the
drains are there a significant contribution from the biopores. We can
also see that the vertical movement within the soil is almost
exclusively through biopores.  Since Daisy does not have a model for
transport of solutes on the surface, the reason for the decline in
metamitron entering the soil away from the drain pipes must be surface
degradation.

Figure~\ref{fig:Silstrup-Metamitron-2000-horizontal} shows the largest
horizontal transport near the top of the soil.  Likely because the majority
of the metamitron enters the soil through the matrix, and does not
move much further down.

\begin{figure}[htbp]
  \centering
  \fig{Silstrup-Metamitron-horizontal-2000}
  
  \caption{Silstrup total horizontal metamitron transport between 2000-5-1 and
    2001-5-1.  The transport is shown on the x-axis (positive away from
    drain) as a function of depth shown on the y-axis.  The graph
    labels are the distance from drain in centimeters.}
  \label{fig:Silstrup-Metamitron-2000-horizontal}
\end{figure}

\begin{figure}[htbp]\centering
  \begin{tabular}{ccc}
    \figsilstrupl{Silstrup-M-Metamitron-2000-5} & 
    \figsilstrup{Silstrup-M-Metamitron-2000-6} & 
    \figsilstrup{Silstrup-M-Metamitron-2000-7} \\
    \figsilstrupl{Silstrup-M-Metamitron-2000-8} & 
    \figsilstrup{Silstrup-M-Metamitron-2000-9} & 
    \figsilstrup{Silstrup-M-Metamitron-2000-10} \\
    \figsilstrupl{Silstrup-M-Metamitron-2000-11} & 
    \figsilstrup{Silstrup-M-Metamitron-2000-12} & 
    \figsilstrup{Silstrup-M-Metamitron-2001-1} \\
    \figsilstrupl{Silstrup-M-Metamitron-2001-2} & 
    \figsilstrup{Silstrup-M-Metamitron-2001-3} & 
    \figsilstrup{Silstrup-M-Metamitron-2001-4}
  \end{tabular}
  
  \caption{Silstrup metamitron soil content at the end of each month
    since first application of bromide.  The y-axis denotes depth, the
    x-axis distance from drain.  There are tick marks for every
    meter. The color scale is white<10 pg/l, yellow=1 ng/l, orange=0.1
    $\mu$g/l, red=10 $\mu$g/l, and black>1 mg/l}
\label{fig:Silstrup-M-Metamitron-2000}
\end{figure}

\begin{figure}[htbp]\centering
  \begin{tabular}{ccc}
    \figsilstrupl{Silstrup-C-Metamitron-2000-5} & 
    \figsilstrup{Silstrup-C-Metamitron-2000-6} & 
    \figsilstrup{Silstrup-C-Metamitron-2000-7} \\
    \figsilstrupl{Silstrup-C-Metamitron-2000-8} & 
    \figsilstrup{Silstrup-C-Metamitron-2000-9} & 
    \figsilstrup{Silstrup-C-Metamitron-2000-10} \\
    \figsilstrupl{Silstrup-C-Metamitron-2000-11} & 
    \figsilstrup{Silstrup-C-Metamitron-2000-12} & 
    \figsilstrup{Silstrup-C-Metamitron-2001-1} \\
    \figsilstrupl{Silstrup-C-Metamitron-2001-2} & 
    \figsilstrup{Silstrup-C-Metamitron-2001-3} & 
    \figsilstrup{Silstrup-C-Metamitron-2001-4}
  \end{tabular}
  
  \caption{Silstrup metamitron soil water concentration at the end of
    each month since first application of bromide.  The y-axis denotes
    depth, the x-axis distance from drain.  There are tick marks for
    every meter. The color scale is white<10 pg/l, yellow=1 ng/l, orange=0.1
    $\mu$g/l, red=10 $\mu$g/l, and black>1 mg/l}
\label{fig:Silstrup-C-Metamitron-2000}
\end{figure}

\begin{figure}[htbp]
  \centering
  \figtop{Silstrup-Metamitron-2000} 
  \fig{Silstrup-Metamitron-biopore-2000}
 
  \caption{Silstrup total (top) and biopore (bottom) vertical
    metamitron transport between 2000-5-1 and 2001-5-1.  The transport is shown
    on the y-axis (positive up) as a function of distance from drain
    shown on the x-axis.  The graph labels are depths in centimeters
    above surface.}
  \label{fig:Silstrup-Metamitron-2000-vertical}
\end{figure}

\FloatBarrier
\section{Glyphosate}

Unfortunately, the glyphosate was applied on different years for the
sites, making them less comparable.  Nonetheless, comparing with the
rest of the data, the differences seem to be more a result of the
respective soils than difference in weather.

\subsection{Distribution}

On figure~\ref{fig:Silstrup-M-Glyphosate-2001} (Silstrup) we can see
the glyphosate entering the soil in three different places.  The soil
surface, the bottom of the short biopores that end right above the
plow pan, and the end of the deep biopores than end 1.2 meter below
the surface.  The glyphosate within the plow layer is then mixed by a
soil tillage operation.  The leaching below 2 meter is hardly visible,
but there is clearly some redistribution within the biopore active
soil.  If we look at the concentration in soil
water~\ref{fig:Silstrup-C-Glyphosate-2001} we see a clear decrease in
the plow layer, which can be explained by a combination of degradation
and dilution as the water content is increasing (see
figure~\ref{fig:Silstrup-pF-2001}).

At Estrup, the glyphosate hardly even move out of the plow layer
(figure~\ref{fig:Estrup-M-Glyphosate-2000}).  If we look at the soil
water concentration (figure~\ref{fig:Estrup-M-Glyphosate-2000}), it is
only above the limit for drinking water within the plow layer, except
for the first month where it is near the limit in a area above the
drain pipes.  The reason for this is that the water table at the time
is lower above the drain pipes (see figure~\ref{fig:Estrup-pF-2000}),
and the biopores will mainly empty in unsaturated soil.  Looking one
year further ahead (figure~\ref{fig:Estrup-C-Glyphosate-2001}) we see
the glyphosate above 1 meter being degraded, and the glyphosate below
1 meter going nowhere.

\subsection{Transport}

The horizontal transport (figure~\ref{fig:Glyphosate-horizontal}) reflect
the location in the soil, at Silstrup we see some horizontal transport at
the top of the soil, at the bottom of the short biopores, and at the
bottom of the deep biopores.  At Estrup, we plow shortly after
application.  The plow operation as defined in Daisy distributes the
glyphosate from the surface to the bottom half of the plow layer.
Which is where we see the horizontal transport.

At Silstrup (figure~\ref{fig:Silstrup-Glyphosate-2001-vertical}) most
of the glyphosate enters the soil through the matrix, but only the
part entering the soil through biopores is transported further down.
Unlike for metamitron
(figure~\ref{fig:Silstrup-Metamitron-2000-vertical}), less glyphosate
enter the soil above the drain pipes, indicating that the glyphosate
spend more time on the surface.  For Estrup
(figure~\ref{fig:Estrup-Glyphosate-2000}) there is no horizontal
variation in how much glyphosate enter the soil, none of it does so
through the biopores.  There is some matrix transport 25 cm below surface
(the plowing operation put most glyphosate 22 cm below surface),
further down there is some biopore facilitated transport above the
drains.

\begin{figure}[htbp]\centering
  \begin{tabular}{ccc}
    \figsilstrupl{Silstrup-M-Glyphosate-2001-5} & 
    \figsilstrup{Silstrup-M-Glyphosate-2001-6} & 
    \figsilstrup{Silstrup-M-Glyphosate-2001-7} \\
    \figsilstrupl{Silstrup-M-Glyphosate-2001-8} & 
    \figsilstrup{Silstrup-M-Glyphosate-2001-9} & 
    \figsilstrup{Silstrup-M-Glyphosate-2001-10} \\
    \figsilstrupl{Silstrup-M-Glyphosate-2001-11} & 
    \figsilstrup{Silstrup-M-Glyphosate-2001-12} & 
    \figsilstrup{Silstrup-M-Glyphosate-2002-1} \\
    \figsilstrupl{Silstrup-M-Glyphosate-2002-2} & & 
  \end{tabular}
  
  \caption{Silstrup glyphosate soil content at the end of each month
    since one year after the first application of bromide.  The y-axis
    denotes depth, the x-axis distance from drain.  There are tick
    marks for every meter. The color scale is white<10 pg/l, yellow=1
    ng/l, orange=0.1 $\mu$g/l, red=10 $\mu$g/l, and black>1 mg/l}
\label{fig:Silstrup-M-Glyphosate-2001}
\end{figure}

\begin{figure}[htbp]\centering
  \begin{tabular}{ccc}
    \figsilstrupl{Silstrup-C-Glyphosate-2001-5} & 
    \figsilstrup{Silstrup-C-Glyphosate-2001-6} & 
    \figsilstrup{Silstrup-C-Glyphosate-2001-7} \\
    \figsilstrupl{Silstrup-C-Glyphosate-2001-8} & 
    \figsilstrup{Silstrup-C-Glyphosate-2001-9} & 
    \figsilstrup{Silstrup-C-Glyphosate-2001-10} \\
    \figsilstrupl{Silstrup-C-Glyphosate-2001-11} & 
    \figsilstrup{Silstrup-C-Glyphosate-2001-12} & 
    \figsilstrup{Silstrup-C-Glyphosate-2002-1} \\
    \figsilstrupl{Silstrup-C-Glyphosate-2002-2} &  & 
  \end{tabular}
  
  \caption{Silstrup glyphosate soil water concentration at the end of
    each month since one year after first application of bromide.  The
    y-axis denotes depth, the x-axis distance from drain.  There are
    tick marks for every meter. The color scale is white<10 pg/l,
    yellow=1 ng/l, orange=0.1 $\mu$g/l, red=10 $\mu$g/l, and black>1
    mg/l}
\label{fig:Silstrup-C-Glyphosate-2001}
\end{figure}

\begin{figure}[htbp]\centering
  \begin{tabular}{ccc}
    \figestrupl{Estrup-M-Glyphosate-2000-5} & 
    \figestrup{Estrup-M-Glyphosate-2000-6} & 
    \figestrup{Estrup-M-Glyphosate-2000-7} \\
    \figestrupl{Estrup-M-Glyphosate-2000-8} & 
    \figestrup{Estrup-M-Glyphosate-2000-9} & 
    \figestrup{Estrup-M-Glyphosate-2000-10} \\
    \figestrupl{Estrup-M-Glyphosate-2000-11} & 
    \figestrup{Estrup-M-Glyphosate-2000-12} & 
    \figestrup{Estrup-M-Glyphosate-2001-1} \\
    \figestrupl{Estrup-M-Glyphosate-2001-2} & 
    \figestrup{Estrup-M-Glyphosate-2001-3} & 
    \figestrup{Estrup-M-Glyphosate-2001-4}
  \end{tabular}
  
  \caption{Estrup glyphosate soil content at the end of each month
    since first application of bromide.  The y-axis denotes depth, the
    x-axis distance from drain.  There are tick marks for every
    meter. The color scale is white<10 pg/l, yellow=1 ng/l, orange=0.1
    $\mu$g/l, red=10 $\mu$g/l, and black>1 mg/l}
\label{fig:Estrup-M-Glyphosate-2000}
\end{figure}

\begin{figure}[htbp]\centering
  \begin{tabular}{ccc}
    \figestrupl{Estrup-C-Glyphosate-2000-5} & 
    \figestrup{Estrup-C-Glyphosate-2000-6} & 
    \figestrup{Estrup-C-Glyphosate-2000-7} \\
    \figestrupl{Estrup-C-Glyphosate-2000-8} & 
    \figestrup{Estrup-C-Glyphosate-2000-9} & 
    \figestrup{Estrup-C-Glyphosate-2000-10} \\
    \figestrupl{Estrup-C-Glyphosate-2000-11} & 
    \figestrup{Estrup-C-Glyphosate-2000-12} & 
    \figestrup{Estrup-C-Glyphosate-2001-1} \\
    \figestrupl{Estrup-C-Glyphosate-2001-2} & 
    \figestrup{Estrup-C-Glyphosate-2001-3} & 
    \figestrup{Estrup-C-Glyphosate-2001-4}
  \end{tabular}
  
  \caption{Estrup glyphosate soil water concentration at the end of
    each month since first application of bromide.  The y-axis denotes
    depth, the x-axis distance from drain.  There are tick marks for
    every meter. The color scale is white<10 pg/l, yellow=1 ng/l, orange=0.1
    $\mu$g/l, red=10 $\mu$g/l, and black>1 mg/l}
\label{fig:Estrup-C-Glyphosate-2000}
\end{figure}

\begin{figure}[htbp]\centering
  \begin{tabular}{ccc}
    \figestrupl{Estrup-C-Glyphosate-2001-5} & 
    \figestrup{Estrup-C-Glyphosate-2001-6} & 
    \figestrup{Estrup-C-Glyphosate-2001-7} \\
    \figestrupl{Estrup-C-Glyphosate-2001-8} & 
    \figestrup{Estrup-C-Glyphosate-2001-9} & 
    \figestrup{Estrup-C-Glyphosate-2001-10} \\
    \figestrupl{Estrup-C-Glyphosate-2001-11} & 
    \figestrup{Estrup-C-Glyphosate-2001-12} & 
    \figestrup{Estrup-C-Glyphosate-2002-1} \\
    \figestrupl{Estrup-C-Glyphosate-2002-2} & 
    \figestrup{Estrup-C-Glyphosate-2002-3} & 
    \figestrup{Estrup-C-Glyphosate-2002-4}
  \end{tabular}
  
  \caption{Estrup glyphosate soil water concentration at the end of
    each month since first application of bromide.  The y-axis denotes
    depth, the x-axis distance from drain.  There are tick marks for
    every meter. The color scale is white<10 pg/l, yellow=1 ng/l, orange=0.1
    $\mu$g/l, red=10 $\mu$g/l, and black>1 mg/l}
\label{fig:Estrup-C-Glyphosate-2001}
\end{figure}

\begin{figure}[htbp]
  \centering
  \fig{Silstrup-Glyphosate-horizontal-2001}
  \fig{Estrup-Glyphosate-horizontal-2000}
  
  \caption{Silstrup total horizontal glyphosate transport between 2001-5-1
    and 2002-3-1 and Estrup total horizontal glyphosate transport between
    2000-5-1 and 2001-5-1. The transport is shown on the x-axis (positive
    away from drain) as a function of depth shown on the y-axis.  The
    graph labels are the distance from drain in centimeters.}
  \label{fig:Glyphosate-horizontal}
\end{figure}

\begin{figure}[htbp]
  \centering
  \figtop{Silstrup-Glyphosate-2001}
  \fig{Silstrup-Glyphosate-biopore-2001}
  
  \caption{Silstrup total (top) and biopore (bottom) vertical
    glyphosate transport between 2001-5-1 and 2002-3-1.  The transport is shown
    on the y-axis (positive up) as a function of distance from drain
    shown on the x-axis.  The graph labels are depths in centimeters
    above surface.}
  \label{fig:Silstrup-Glyphosate-2001-vertical}
\end{figure}

\begin{figure}[htbp]
  \centering
  \figtop{Estrup-Glyphosate-2000}
  \fig{Estrup-Glyphosate-biopore-2000}
  
  \caption{Estrup total (top) and biopore (bottom) vertical glyphosate
    transport between 2000-5-1 and 2001-5-1.  The transport is shown on the
    y-axis (positive up) as a function of distance from drain shown on
    the x-axis.  The graph labels are depths in centimeters above
    surface.}
  \label{fig:Estrup-Glyphosate-2000}
\end{figure}
