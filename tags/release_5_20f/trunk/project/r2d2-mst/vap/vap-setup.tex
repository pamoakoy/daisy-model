\chapter{Model setup}
\label{cha:setup}

\section{Weather}

Hourly weather data for Silstrup, Tylstrup and Askov (near Estrup) was
provided by Finn Plauborg from the Faculty of Agricultural Sciences,
Aarhus University.  The idea was that Tylstrup data could be used to
fill in gaps in the Silstrup data.  Both the Silstrup and Askov data
sets contained several short gaps.  We filled those by using values
from the preceding or following hours.  The Silstrup data set ended at
2002-03-12.  The drain season ended 2002-03-20, with 0.6 mm water
collected from the drains the last 8 days.  Bearing this in mind, we
chose to end the simulation 2002-03-11, rather than continue with data
from another station.

The weather data used by Daisy consist of air temperature, wind speed,
relative humidity, precipitation, and global radiation.  Based on
these data, Daisy can use the FAO version of the Penman-Monteith
equation \citep{FAO-PM} to calculate reference evapotranspiration
(ET0).  From that, Daisy will calculate potential evapotranspiration
(ETc) by using the crop leaf area index (LAI) with Beer's law to
divide the surface into a canopy covered fraction and bare soil
fraction, and using different factors for each.  Based partly on
\citet{kjaersgaard2008crop}, a canopy factor of 1.2 and a bare soil
factor of 0.6 was chosen, resulting in a combined factor around 1.15
with full canopy for typical crops.  Actual evapotranspiration (ETa)
is further limited by the capability of the root system and the soil
surface to extract water from the soil.

\begin{figure}[htbp]
  \begin{center}
    \figtop{Silstrup-weather}
    \fig{Estrup-weather}
  \end{center}
  \caption{Accumulated precipitation and hourly values for temperature
    measured at Silstrup (top) and the Askov (bottom) station located
    near Estrup.  Calculated accumulated potential and simulated
    actual evapotranspiration are also shown.}
  \label{fig:weather}
\end{figure}
Precipitation, air temperature, ETc and ETa can be seen on
figure~\ref{fig:weather}.

\section{Management}

All management data were provided by Preben Olsen from the Faculty of
Agricultural Sciences, Aarhus University provided via Annette E.\@
Rosenbom from GEUS.

\subsection{Tillage}

Date, type, and depth were specified for all tillage operations.  All
three were entered into Daisy.  In Daisy, the main effects of tillage
are to incorporate some of the surface material into the soil
(depending on the type of tillage operation), and to mix the content
of the soil to the specified depth.  This is not expected to affect
pesticide leaching much.  In the real world, the main effect of
tillage applicable to pesticide leaching is likely to be a change in
the hydraulic properties for the top soil resulting from the tillage
operation.  Since we chose to implement dynamic hydraulic soil
properties for the Silstrup soil surface (see
section~\ref{sec:surface-plow-pan} and~\ref{sec:cal-crust}), the
tillage information were useful there as well.

\subsection{Fertilization}

Date, type, and amount were specified for each fertilization event, as
well as N, P and K content of fertilizer.  Of these nutrients, Daisy
can normally only handle N, and N has been disabled for these
simulations to save time.  All fertilization event have been added to
Daisy management description, but with N disabled the mineral
fertilizer will not affect the simulation.  The organic fertilizer
will have a minimal effect, the water content of the fertilizer will
be added, and the dry matter content will be added to the litter where
it can catch water and pesticides (see section~\ref{sec:cal-litter}),
until it becomes incorporated into the soil by either tillage
operations or earthworm activity.  However, due to the timing of the
applications, both effects are likely to have negligible effect on
pesticide leaching.

\subsection{Crop management}
\label{sec:crop-man}

Information about date, crop and sowing density were provided.  The
default Daisy crop model does not rely on sowing density, but instead
assumes that ``standard practise'' is used.  Daisy has experimental
support for a crop model that includes sowing density, but given that
the PLAP sites are expected to follow standard management practise, we
found it safer to use the better tested parametrizations of the
default model.

Information about the crow growth was given phenologically (BBCH
stage) and in terms of above ground biomass. No attempt was made to
calibrate the crop in order to match this with the simulated
development stage and biomass.  Information about the two important
parameters for the water balance, namely leaf area and root density,
were not provided.  However, those can often be estimated from the
development stage and dry mass. 

Harvest data included date, stubble height, as well as grain and straw
yields.  Date and stubble height can be directly used by Daisy.  Grain
yield can be used for calibration, however as crop production is not
the focus of this project, we merely noted that both measured and
simulated yields where within the normal range.  The ratio between
grain and straw yield was used for a coarse estimation of the fraction
of the crop left on the field as residuals after harvest.  The
residuals play a crucial role in the simulation for the Silstrup
glyphosate leaching, see section~\ref{sec:cal-litter}.

\begin{table}[htbp]
  \caption{Crop management.}
  \label{tab:crop-man}
  \centering
  \begin{tabular}{c||c|c|c||c|c|c}\hline
    & \multicolumn{3}{c||}{Silstrup} & \multicolumn{3}{c}{Estrup}\\
    Year & Crop          & Sow   & Harvest & Crop          & Sow   & Harvest\\
\hline
    2000 & Fodder Beet   & 4/5   & 15/11   & Spring Barley & 12/4  & 28/8 \\
    2001 & Spring Barley & 9/5   & 5/9     & Peas          & 5/2   & 22/8 \\
    2001 &               &       &         & Winter Wheat  & 19/10 & --- \\
  \end{tabular}
\end{table}

Two calibrations were done on crop management.  The first was to
replace the fodder beet (see table~\ref{tab:crop-man}) with spring
barley.  The Silstrup soil water measurements indicated that we
underestimated the ability of the summer 2000 crop to extract water
from the plow layer (see figure~\ref{fig:Silstrup-theta}).  The spring
barley parametrization did a better job than the less tested fodder
beet.  The second calibration served a similar purpose, an ad hoc root
density function that preserved almost the entire root mass in the
plow layer, was added to the two Silstrup crops.  The plow pan was
assumed to be so dense that only a few roots could penetrate through
earthworm channels.  This also matches the lack of seasonal variation
found with the 60 cm TDR probe (figure~\ref{fig:Silstrup-theta}).
Apart from the soil water measurements, both calibrations also served
to concentrate the uptake from the zone with most bromide, in order to
explain the low amount of bromide found in the drain water.  See also
section~\ref{sec:cal-bromide} and~\ref{sec:cal-silstrup-surface}.

\subsection{Pesticide and bromide application}

The data for pesticide application consisted of date, amount, and
trade name.  Trade name was translated to active ingredient using
``Middeldatabasen'' from \textsc{dlbr} Landbrugsinfo
(\url{http://www.landbrugsinfo.dk/}).  For potassium bromide, bromide
content was calculated from molar mass.

The applications are summarized in table~\ref{tab:man-pest}.
\begin{table}[htbp]
  \caption{Pesticide and bromide application.  Only those active 
    ingredients we track are listed.  }
  \label{tab:man-pest}
  \centering
  \begin{tabular}{c|crl|crl}\hline
    \emph{Silstrup} & Trade name & \multicolumn{2}{c|}{Amount} 
               & Active ingredient & \multicolumn{2}{c}{Amount} \\
    \hline
    2000-05-22 & Potassium bromide & 30 & kg/ha & Bromide    & 20.14 & kg/ha\\
    2000-05-22 & Goltix WG      & 1   & kg/ha & Metamitron    & 700   & g/ha\\
    2000-06-15 & Goltix WG      & 1   & kg/ha & Metamitron    & 700   & g/ha\\
    2000-07-12 & Goltix WG      & 1   & kg/ha & Metamitron    & 700   & g/ha\\
    2001-06-21 & Tilt Top       & 0.5 & L/ha & Fenpropimorph & 187.5 & g/ha\\
    2001-07-04 & Tilt Top       & 0.5 & L/ha & Fenpropimorph & 187.5 & g/ha\\
    2001-07-16 & Perfektion 500 & 0.6 & L/ha & Dimethoate    & 300   & g/ha\\
    2001-10-25 & Roundup Bio    & 4.0 & L/ha & Glyphosate    & 1440  & g/ha\\
    \hline\hline
    \emph{Estrup} & Trade name & \multicolumn{2}{c|}{Amount} 
               & Active ingredient & \multicolumn{2}{c}{Amount} \\
    \hline
    2000-05-15 & Potassium bromide & 30 & kg/ha & Bromide & 20.14 & kg/ha\\
    2000-06-15 & Tilt Top       & 0.5 & L/ha & Fenpropimorph & 187.5 & g/ha\\
    2000-06-15 & Perfektion 500 & 0.4 & L/ha & Dimethoate    & 200   & g/ha\\
    2000-07-05 & Tilt Top       & 0.5 & L/ha & Fenpropimorph & 187.5 & g/ha\\
    2000-06-15 & Perfektion 500 & 0.4 & L/ha & Dimethoate    & 200   & g/ha\\
    2000-10-13 & Roundup Bio    & 4.0 & L/ha & Glyphosate    & 1440  & g/ha\\
  \end{tabular}
\end{table}

\section{Pesticide and bromide properties}

Of the four pesticides examined, only metamitron and glyphosate were
measured in concentrations above the detection limit in the examined
data set.  A single sample at the detection limit were found for
dimethoate, and none were found for fenpropimorph.  No calibration has
therefore been performed on those two pesticides.

\subsection{Soil sorption and degradation}

Sorption and degradation parameters for pesticides are primarily taken
from \citet{ppdb20100517}, with values as shown in
table~\ref{tab:ppdb}.  The database specify a \koc{} value
independently of whether the pesticide is actually sorbed to organic
matter.  We chose to use a \kd{} values measured in Denmark for the
two main pesticides.  For metamitron, \citet{madsen2000pesticide}
specify \kd{} values together with soil properties for a number of
Danish sites.  Section~\ref{sec:cal-metamitron} describes how we
selected a \kd{} based on those.  For glyphosate, the \kd{} value is
from \citet{gjettermann2009}.  The adsorption is not instantaneous, an
adsorption rate of 0.05 h$^{-1}$ was used as an reasonable initial
guess for all pesticides.  We found no reason to change the value
during calibration of metamitron, but did for glyphosate as detailed
in section~\ref{sec:cal-glyphosate}.

The effect of depth on degradation is taken from
\citet{focus2000,focus2002}.  The effect of temperature and humidity
for turnover of organic matter in Daisy is also used for pesticides.
The default diffusion coefficient used by Daisy for pesticides of
4.6e-6 cm$^2$/s is used unchanged for all pesticides as well as for
colloids.  A value of 2.0e-5 cm$^2$/s is used instead for the smaller
bromide molecules.  We assume that the pesticide molecules are all
reflected by the roots, so there is no crop uptake of pesticides.

\begin{table}[htbp]
  \centering
  \caption{Pesticide properties from  \citet{ppdb20100517}. 
    The DT50 value is the degradation halftime in days.  For both DT50
    and \koc{} we put the value marked  `field' in \citet{ppdb20100517}
    in the center, surrounded by the lower and upper limit found in 
    field studies, as marked in a note in the database. For fenpropimorph
    the \koc{} field value did not fall within the specified interval.  
    The \kd{} value for glyphosate is from \citet{gjettermann2009}, and
    the \kd{} range for metamitron is from \citet{madsen2000pesticide}.  
    The values used in the simulation are in \textbf{bold}.}
  \label{tab:ppdb}

  \begin{tabular}{l|ccc}\hline
    Name & DT50 [d] & \koc{} [ml/g] & \kd{} [ml/g] \\\hline
    Dimethoate & 4.6 -- \textbf{7.2} -- 9.8 & 16.25 -- \textbf{30} -- 51.88 & \\
    Fenpropimorph & 8.8 -- \textbf{25.5} -- 50.6 & 2771 -- \textbf{2401} -- 5943 &\\
    Glyphosate & 5 -- \textbf{12} -- 21 & 884 -- 21699 -- 60000  & \textbf{503} \\
    Metamitron & 6.6 -- \textbf{11.1} -- 22.0 & 77.1 -- 80.7 -- 132.5 & 0.14 -- \textbf{4.0}\\
  \end{tabular}
\end{table}

\subsection{Surface degradation}

None of our sources had specific information on above ground
degradation.  As the glyphosate calibration (see
section~\ref{sec:cal-glyphosate}) depends on keeping part of the
glyphosate in the litter pack for several days, surface degradation
potentially becomes a factor.  The default value in Daisy of
DT50~=~3.5 for surface degradation of pesticides is used.

\subsection{Colloids and colloid facilitated transport}
\label{sec:coltrans}

We have no data for colloids, so the parameters for colloid generation
and filtering calibrated for R�rrendeg�rd have been reused for both
sites.  The model itself will adjust to the different clay contents.
Simulated colloid leaching is shown in
appendix~\ref{app:col-biopores}.  The pesticides are assumed to be
able to sorb to and be transported with colloids, meaning that the
colloids will be in competition with the soil matrix as potential
sorption sites for the solute form of the pesticides.  This is
difficult to measure directly, and is therefore sometimes used as a
calibration parameter, see e.g.~\citet{sef1000} where a value of 1000
was used.  As a starting point we chose a factor 10 higher than that
(that is, a soil enrichment factor of 10000) to be certain this part
of the model would be tested.  During calibration, we found no reason
to change this initial value.  See also~\citet{mst-agrovand}.

\subsection{Glyphosate calibration}
\label{sec:cal-glyphosate}

The highest glyphosate concentrations in drains were seen at both
sites right after application, or during the first large rain event
after application.  This is unlikely to be a function of the pesticide
properties as such, but rather of the transport pathways to the drain.
See section~\ref{sec:cal-silstrup-surface} for how this was
calibrated.

The initial simulations showed practically no further glyphosate
movement once the glyphosate entered the soil matrix.  The
measurements, however, did show some late findings of glyphosate in
the drain water.  In order to give the glyphosate a chance to move, we
divided the sorption into a weak but fast and a strong but slow form.
The strong but slow form represent 90\% of the \kd{} value, the weak
but fast form the remaining 10\%.  This was a pure calibration
measure, and may not necessarily reflect the chemical properties of
the pesticide.  Two phase kinetics were also observed in
\citet{glyphosate-kinetics}, but at a shorter time-scale.  The effect
is that the glyphosate is relatively mobile in the beginning, but
becomes less so as more glyphosate becomes sorbed in the slow form,
resulting in a better overall match with drain measurements.

\subsection{Metamitron calibration}
\label{sec:cal-metamitron}

Adjusting the degradation rate for metamitron had little effect on the
simulation results.  Figure~\ref{fig:Silstrup-C-Metamitron-2000}
and~\ref{fig:Silstrup-Metamitron-2000-horizontal} shows why. The
metamitron we find in the drains is the same metamitron that was first
transported vertically to the end of the biopores, and then
horizontally towards the drain.  Since the biopores in the simulation
ends 1.2 meter below surface, this means the metamitron is located
below the 1 meter depth limit for degradation specified by \focus{}.

In \citet{madsen2000pesticide} sorption parameters are measured for
several Danish sites.  The best correlation for sorption to soil
parameters is for total iron oxide (FeO$_{\mbox{total}}$), the
correlation to organic matter is weak, and no correlation was found to
the easily extracted iron oxide (FeO$_{\mbox{oxalate}}$) which was
measured at Silstrup.  The largest measured \kd{} is 3.1 $\pm$ 0,9
L/kg at Drengsted, and the lowest 0.16 $\pm$ 0.02 L/kg at Vejen.  We therefore
decided \kd{} should be within the interval 0.14 -- 4.0 L/kg.  A \kd{}
value at the high end of the interval, 4.0 L/kg, gave the best match.

\subsection{Bromide calibration}
\label{sec:cal-bromide}

As discussed in section~\ref{sec:crop-man} we wanted the crop to take
up as much bromide at possible, the parameter controlling this is
called the crop uptake reflection factor.  Setting it to zero would
give the best results for Silstrup, however at Estrup we had the
opposite problem, high amounts of bromide was observed in the drain
water, indicating a high value for the crop uptake reflection factor.
It would be possible to justify different values for the two sites, as
there were grown different crops the first year.  However, without any
direct measurements of bromide crop uptake, we found it better to use
a single value.  With a reflection factor of 0.25 we got a good match
for total amount in Silstrup (figure~\ref{fig:Silstrup-weekly}
and~\ref{fig:Silstrup-bromide-acc}).  In Estrup, this resulted in too
little total drain leaching, but still good leaching dynamics
(figure~\ref{fig:Estrup-bromide-drain}).

\section{Soil}

The soil setup is based on multiple sources, which will be described
in this section.

\subsection{The soil matrix domain}

\subsubsection{The primary domain (micropores)}
\label{sec:cal-primary}

GEUS had already calibrated the model \macro{}
\citep{jarvis1994simulation,larsbo2003macro} for both sites.  The
\macro{} setup was provided by Annette E.\@ Rosenbom from GEUS\@.  As
\macro{}, like Daisy, solves Richard's Equation, we chose to use the
\macro{} calibration of the hydraulic properties (retention and
conductivity curves) as a basis.  \Macro{} uses a bimodal description
of the hydraulic properties, where the micropore part is identical to
van Genuchten retention curve with Mualem theory for conductivity.
This also happens to be one of the models supported by Daisy, so that
part could be used directly.  We made two changes to the micropore
setup: We increased the hydraulic conductivity for the plow layer at
both sites based on the measurements depicted on figure~A4.4 and~A4.5
in \citet{vap2005}.  For Silstrup the boundary hydraulic conductivity
($K_b$) was raised from 0.1 to 1 mm/h, and for Estrup from 0.1 to 0.5
mm/h.  However, the low values used by GEUS are far from unreasonable,
as the conductivity of unprotected soil surface tend to decrease
rapidly after heavy rain.  For the Daisy setup, we added a special
surface layer with dynamic hydraulic properties to address this issue
(see section~\ref{sec:surface-plow-pan}).  The other change was the
introduction of 8\% residual water in the B horizon of Silstrup, based
on the relative lack of drying during the summer, as seen on
figure~\ref{fig:Silstrup-theta}.

\subsubsection{Soil cracks and anisotropy}

Unlike \macro{}, Daisy distinguish sharply between macropores small
enough that the capillary forces are still dominating, and macropores
so large that the capillary forces are no longer a factor.  In Daisy
terminology, these are called the secondary and tertiary domain,
respectively.  The primary domain is the micropores.  The model user
is responsible for specifying both domains, and thus for specifying
for which macropores Daisy should consider the capillary forces
dominating.  Daisy does not use Richard's Equation for calculating
transport in the tertiary domain.  Richard's Equation is used for both
the primary and secondary domain, and in fact does not distinguish
between the two.  They are (again in Daisy terminology) together
referred to as the matrix domain.  The tertiary domain is described in
section~\ref{sec:biopores}.

In the present setup, soil cracks as those described in
\citet{lindhardt2001} have been specified as part of the secondary
domain.  Daisy will use Poiseuille's law for calculating how these
cracks affect the conductivity based on aperture and density.  In
\citet{habekost1} an aperture of 50 to 150 $\mu$m is estimated.  In
\citet{jorgensen1998} a value of 78 $\mu$m is used after calibration.
Both sources specifies a density of 10 per meter.

In \citet{lindhardt2001} the cracks in the depth interval 75 -- 180 cm
in Silstrup are primarily horizontal.  As the secondary domain model
of cracks in Daisy doesn't include direction (they are assumed to be
equally distributed in all directions), we have decided not to use
that model in this interval, and instead specify an anisotropy of 100.
This means the horizontal conductivity is 100 times higher than the
vertical, which fit well with the \macro{} parametrization.  For dry
soil this is wrong, but we don't expect large horizontal hydraulic
gradients in that situation anyway.

For the plow layer at both sites (see table~\ref{tab:profile}), we
also chose an anisotropy of 100 rather than a general modification of
the hydraulic conductivity.  They idea behind this is to model how the
surface slope affect horizontal movement.  The simulation results
shows the effect of this anisotropy is negligible on Silstrup
(figure~\ref{fig:Silstrup-water-horizontal}) but quite significant on
Estrup (figure~\ref{fig:Estrup-water-horizontal}), likely due to
differences in groundwater level.

\citet{lindhardt2001} specifies no cracks at Silstrup below 3.5 m. For
Estrup, the high groundwater level could indicate a poor horizontal
conductivity.  Furthermore, the shape of the bromide drain leaching
curve (figure~\ref{fig:Estrup-bromide-drain}) where the high values
are early also indicate that the drain water are extracted from the
top soil layers.  We therefore assumed that the cracks found in
\citet{lindhardt2001} below 2m are not hydraulically connected, and
thus doesn't influence the conductivity.  Note that the higher
saturated conductivity used in the \macro{} simulation for the B and
C1 horizons are still reflected in Daisy through the biopores.  It is
only the horizontal conductivity (as Daisy biopores are vertical) that
is low.

\subsubsection{Figures and tables}

Figures~\ref{fig:Silstrup-hor} and~\ref{fig:Estrup-hor} show the
original \macro{} parametrization and the modified parametrization for
Daisy.  Only the vertical conductivity is shown, and as the
conductivity in the tertiary domain in Daisy is infinite, that domain
is not included.  For comparison, we have show the effect of the
parameters estimated from soil texture by the \hypres{} pedotransfer
function.  Table~\ref{tab:profile} summarizes the two profiles.

\newcommand{\dd}[2]{#1\hspace{-3mm} &--&\hspace{-3.5mm}#2}
\begin{table}[htbp]
  \caption{Soil profile for the two sites.  Depth is in cm below soil surface.  
    The Note column specifies \emph{Dynamic} conductivity for the soil
    surface layer, \emph{Dense} (low conductivity) for the plow pan, 
    \emph{Anisotropy} for layers with high horizontal hydraulic conductivity,
    and \emph{Cracks} for layers with high near saturated hydraulic 
    conductivity.}
  \label{tab:profile}
  \centering
  \begin{tabular}{rclll||rclll}\hline
    \multicolumn{5}{l||}{\emph{Silstrup}} & \multicolumn{5}{l}{\emph{Estrup}} \\
    \multicolumn{3}{c}{Depth} & Horizon & Note& \multicolumn{3}{c}{Depth} & Horizon & Note \\\hline
    \dd{0}{3}     & Ap (surface) & Dynamic    & \dd{0}{3}     & Ap (surface)    & Dynamic    \\
    \dd{3}{31}    & Ap           & Anisotropy & \dd{3}{27}    & Ap              & Anisotropy \\
    \dd{31}{39}   & B (plow pan) & Dense      & \dd{27}{35}   & B (plow pan)    & Dense \\
    \dd{39}{75}   & B            & Cracks     & \dd{35}{55}   & B               & \\
    \dd{75}{113}  & B            & Anisotropy & \dd{55}{105}  & C1              & \\
    \dd{113}{180} & C            & Anisotropy & \dd{105}{500} & C2              & \\
    \dd{180}{350} & C            & Cracks\\
    \dd{350}{500} & C            & \\
  \end{tabular}
\end{table}
\begin{figure}[htbp] 
  \fig{Silstrup-Ap-Theta}\figright{Silstrup-Ap-K}\\
  \fig{Silstrup-B-Theta}\figright{Silstrup-B-K}\\
  \fig{Silstrup-C-Theta}\figright{Silstrup-C-K}
  \caption{Silstrup soil hydraulic properties.  MACRO denotes the
    original parametrization, Daisy the modified parametrization
    (ignoring anisotropy and biopores), and HYPRES refers to
    parameters estimated according to \citet{hypres}.}
  \label{fig:Silstrup-hor}
\end{figure}
\begin{figure}[htbp] 
  \fig{Estrup-Ap-Theta}\figright{Estrup-Ap-K}\\
  \fig{Estrup-B-Theta}\figright{Estrup-B-K}\\
  \fig{Estrup-C1-Theta}\figright{Estrup-C1-K}\\
  \fig{Estrup-C2-Theta}\figright{Estrup-C2-K}
  \caption{Estrup soil hydraulic properties.  \Macro{} denotes the
    original parametrization, Daisy the modified parametrization
    (ignoring anisotropy and biopores), and \hypres{} refers to
    parameters estimated according to \citet{hypres}.}
  \label{fig:Estrup-hor}
\end{figure}

\subsection{Soil surface and plow pan}
\label{sec:surface-plow-pan}

Danish agricultural soils may feature both a plow pan, and highly
variable conductivity near the soil surface.  These can create layers
of near saturated soil, which is needed for activating the biopores
module in Daisy.  Hence, such layers was added to the soil
description.  The plow pan is defined as the top of the B horizon, but
with different hydraulic properties.  The cracks are removed from the
plow pan, and the hydraulic conductivity in the micropores is reduced
to 10\% \citep{petersen2008spatio}.  The surface layer constitute the
top of the Ap horizon.  Changing the parameters has not been necessary
for Estrup.  For Silstrup, the hydraulic conductivity is temporarily
decreased to 0.1\% of the original value \citep{soilseal}, see the
description in section~\ref{sec:cal-silstrup-surface}.

\subsection{Fast and slow water}
\label{sec:fast-slow}

Water movement in the matrix is calculated by Richard's Equation.
However, for pesticide transport the water is later divided into a
slow moving primary domain consisting of the smaller pores, and a fast
moving secondary domain consisting of the larger pores.  If the
horizon has cracks, the secondary domain water will consist of the
water in the cracks.  If not, the secondary domain water will consist
of the water retained above pF 2.  We have used pF 1.2 as the limit in
other simulations, but since the retention curves in the setup are
relatively flat near saturation, that value represented very little
water.  Pesticides are tracked independent in the two domains, with an
exchange factor ($\alpha$) at its default value of 0.01 h$^-1$.

The initial values were all set as part of the R{\o}rrendeg{\aa}rd
calibration, see \citet{mst-agrovand} for a more detailed discussion.
No further calibration was done on these parameters.

\subsection{Biopores}
\label{sec:biopores}

Biopores are activated once the soil is near saturation, and they
extract water from the matrix down to -30 cm pressure, at which point
the biopores will deactivate
\citep{tofteng2002film,gjettermann2004transport}.  The capability of
the biopores to extract water is further limited by the storage
capacity of the biopores themselves, and the ability to pass the water
back to the soil matrix in a deeper layer.

The biopores a divided into a number of user specified classes, each
defined by density, diameter, where they start and end (including
ending directly in drain).  \citet{lindhardt2001} contain some
information about biopores, but not enough for use by Daisy.  We have
therefore chosen to use a biopore setup based on data measured at
R�rrende specifically for use by
Daisy~\citep{habekost1,habekost2,habekost3}.

One calibration has been applied. The original setup for R�rrende had
all biopores near the drain ended in the drain.  In order to get more
tailing on the simulated leaching curves, half the biopores near the
drain now ends in the soil matrix.  Neither setup is perfect match for
the observations in~\citet{habekost3}, which show that the earth worm
tunnels are generally well connected to the drain pipes, even if they
don't end in the drain pipes.

\subsection{Groundwater table and drain pipes}
\label{sec:gwt}

Depth (1.1 m below ground level) and distance between drain pipes (18
m for Silstrup and 13 m for Estrup)) are taken from
\citet{lindhardt2001}, and can be used directly by Daisy.  Automatic
measurements of groundwater pressure near the bottom of the part of
the soil we have described in Daisy are being used as a lower boundary
condition, just like the net precipitation is used for the upper
boundary condition.  A constant offset has been added to the measured
values in order to get the drain flow right.  The offset has been
varying depending on the soil description during calibration (between
-40 and 30 cm), for the final setup it ended up being -4 cm for
Silstrup and -5 cm for Estrup.  This is less than the spatial
variation shown by the multiple measurement points,
see~\citet{vap2009}.

The simulated groundwater table is not uniquely defined, given that
the model is two dimensional and there can be multiple layers of
saturated soil.  We have chosen to show two values, a low value based
on the pressure in the lowest located unsaturated numeric cell
(usually near the drain), and a high value based on the pressure in
the highest located saturated numeric cell (usually in the center
between drains).  Measured and simulated groundwater table can be seem
on figure~\ref{fig:gw}.  The frequent zeros for the high value at
Silstrup corresponds to ponding.
\begin{figure}[htbp]
  \begin{center}
    \figtop{Silstrup-gw}
    \fig{Estrup-gw}
  \end{center}
  \caption{Groundwater table at Silstrup (top) and Estrup (bottom).
    Automatic daily measurements at Silstrup are from P3.  Manual
    monthly measurement at Estrup until 2000-09-19 are from P3,
    automatic daily measurements from 2000-09-22 are from P1.
    Simulated low value is calculated from pressure in lowest
    unsaturated numeric cell, typically located near drain.  Simulated
    high value is calculated from pressure in highest saturated cell,
    typically farthest away from the drain. See \citet{lindhardt2001}
    for location of P1 and P3.}
  \label{fig:gw}
\end{figure}

\subsection{Organic matter and nitrogen}

Inorganic nitrogen has been disabled in order to save simulation time.
Initially the organic matter turnover was also disabled.  However,
since bioincorporation of litter into the soil is part of that module,
we had to re-enable it as the litter layer appeared to be significant.
No calibration has been done apart from the bioincorporation speed, as
described in section~\ref{sec:cal-silstrup-surface}.

\FloatBarrier
\section{Silstrup surface}
\label{sec:cal-silstrup-surface}

The Silstrup simulations presented two challenges that both were
resolved through calibration of the system surface.  The first
challenge was the measurements of glyphosate in the drain, shown on
figure~\ref{fig:Silstrup-weekly2}, the first week after application of
glyphosate.  The glyphosate is applied 2001-10-25.  The drain
measurements cover the period from the 24'th to 30'th of October.

\subsection{Soil surface crust}
\label{sec:cal-crust}

Figure~\ref{fig:Silstrup-weather-glyphosate} was created to examine
what happened that week.  The two upper graphs concern the water,
which is needed to bring down the glyphosate (shown on the bottom
graph).  All the graphs are from the final simulation.  Precipitation
(top graph) was obviously measured.  Let us start with that.  What we
see is three small precipitation events in the early hours the 27'th,
28'th and 29'th ($<$ 1 mm), followed by a larger event starting at
noon the 29'th.

\begin{figure}[htbp]
  \begin{center}
    \figtop{Silstrup-weather-glyphosate}\\
    \figtop{Silstrup-water-glyphosate}\\
    \fig{Silstrup-first-glyphosate}
  \end{center}
  \caption{Silstrup surface water and glyphosate in the first week
    after application.  Top graph shows fluxes affecting surface
    water.  Middle graph shows water storage on surface, as well as
    the water holding capacity of the litter pack.  Bottom graph track
    the fate of glyphosate on the surface.}
  \label{fig:Silstrup-weather-glyphosate}
\end{figure}

What happened initially was that none of the events would initiate the
biopores, thus no glyphosate in the drains.  As glyphosate \emph{was}
found in the drains, some biopores must have been activated.  The
alternative, that a strongly sorbing pesticide would be able to move
one meter down through the soul matrix in less than a week, was not
considered realistic.

At this point in the simulation, we are more than five month after the
last soil tillage treatment, and nearly two months after harvest.  It
seems likely that the soil surface at this point would have formed a
crust with low hydraulic property.  By calibration, we found that we
generated biopore flow at the large event if we decreased the
hydraulic conductivity of the soil surface to 0.1\%.  As we don't have
a crust formation model implemented, we chose to make this change in
conductivity right after harvest.

\subsection{Litter pack}
\label{sec:cal-litter}

We now got water, but no significant amount of glyphosate, in the
simulated drains.  The explanation for this can also be found in top
graph on figure~\ref{fig:Silstrup-weather-glyphosate}.  The three
first events are too small to activate the biopores, instead the water
would infiltrate through the matrix, bringing with it all the
glyphosate.  Heavy rain and ponding may -- also in Daisy -- release
the glyphosate (possibly colloid bound) from the top soil.  But the
event wasn't that large or violent, so very little glyphosate would be
released that way.  What was needed was a mechanism to protect the
glyphosate on the surface.

The harvest data provided such a mechanism.  The yield was over 7 tons
grains per hectare, and less than 3 ton straws was removed.  This made
it likely that significant amounts of residuals was left on the field.
Furthermore, \citet{gjettermann2009} demonstrated that glyphosate did
not sorb to straws.  Thus, it seemed likely that some of the
glyphosate was kept in the litter pack together with the water from
the small events, and only washed out with the large event.  Using
this mechanism, depending on the size of the litter pack and the
actual precipitation, between 0 and 100\% of the glyphosate might be
stored in the litter pack.  To implement this in Daisy, we needed a
pre-existing and pre-calibrated model, as there were no time for new
model development at this point.

Luckily, this was not hard to find.  Mulching is a known technique to
conserve water, so other people had been interested in the water
dynamics of the litter pack before us.  The model described in
\citet{scopel2004} was a good conceptual fit with Daisy.  In this
model the plant residuals will cover a fraction (calculated by Beer's
law) of the soil based on the amount and type, where they prevent soil
evaporation for the covered area, and catch a corresponding amount of
the precipitation.  The water holding capacity is based on amount and
type of residuals.  In Daisy this was extended to also catch a
fraction of the applied pesticides.  A parametrization based on
millet from \citet{macena2003} was selected.

This left the incorporation of crop residuals from the surface by
earthworms as the remaining calibration parameter.  By decreasing the
maximum speed of incorporation from 0.5 to 0.35 g DM/m$^2$/h we were
able to get a good fit.  As can be seen on the bottom graph of
figure~\ref{fig:Silstrup-weather-glyphosate}, most of the glyphosate
still enters the soil matrix through the three small events, but more
than enough remain to be transported with the biopores at the large
event to match measured data (figure~\ref{fig:Silstrup-weekly2}).

\subsection{Surface water flow}
\label{sec:cal-silstrup-bromide}

The second problem at Silstrup was due to the bromide.  More than one
third of the net precipitation end up in the drains, yet less than 10
percent of the bromide is found there.  Despite our best efforts, we
were not able to make the crop uptake large enough to compensate for
the difference.  The division between fast and slow water we had
inherited from preliminary calibration of Agrovand data
\citep{mst-agrovand} were also inadequate to protect the
bromide.\footnote{This calibration were later changed, unfortunately
  too late to apply on the PLAP data.}  The explanation that gave the
best results was that a significant amount of water fully bypassed the
soil matrix on its way to the drain pipes, thus diluting the drain
water. The bromide was applied 2000-05-22.  The last tillage operation
was 2000-05-03, with no large precipitation events in between (total
precipitation 9.6 mm, highest intensity 1.4 mm/h).  It is therefore
likely that the hydraulic conductivity is still high at that point.
We chose to add crust 2000-06-01 (after 48.4 mm rain, max intensity
4.9 mm/h), setting the hydraulic conductivity down to one percent of
the original.  At this point, the bromide was safely in the soil
matrix

The crust would generate biopore activity, but not necessarily to the
drain (only the biopores at 20 cm to either side of the drain pipes
are assumed to be connected to them).  The biopores not connected to
the drains were not able to take all the water, resulting in ponding
at the rest of the field.  In order to lead some of this water to the
drains, a simple surface water movement was implemented.  When the
ponding is higher than the local detention capacity in any part of the
field, the surplus water is redistributed evenly to the remaining part
of the field.  Using this as a calibration parameter, we found that a
local detention capacity of 2 mm would result in 10\% of the total water
bypassing the drain pipes, and the right amount bromide in the drains
seen over a whole season.

One other observation that points to surface flow possibly being a
real factor is the response time in the drains to precipitation
events.  As the bottom graph on figure~\ref{fig:Silstrup-drain} shows,
the observed drain flow is almost identical the the net precipitation
at the beginning of the drain season.  This suggest a very fast
connection to the drains, which even with the surface flow module we
could not quite match,

