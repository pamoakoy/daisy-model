\documentclass[a4paper,10pt,titlepage]{article}

\usepackage{verbatim}
\newcommand{\daisy}{Daisy}
\newcommand{\Daisy}{Daisy}
\newcommand{\cplusplus}%
{{\leavevmode{\rm{\hbox{C\hskip -0.1ex\raise 0.5ex\hbox{\tiny ++}}}}}}
\newcommand{\Cplusplus}{\cplusplus}
\newcommand{\mshe}{Mike/\textsc{she}}
\newcommand{\wintel}{\texttt{win32}}
\newcommand{\dll}{\textsc{dll}}
\newcommand{\Dll}{\textsc{Dll}}
\newcommand{\gui}{\textsc{gui}}
\newcommand{\Gui}{\textsc{Gui}}
\newcommand{\unix}{Unix}
\newcommand{\dhi}{\textsc{dhi}}
\newcommand{\Dhi}{\textsc{Dhi}}
\newcommand{\api}{\textsc{api}}
\newcommand{\Api}{\textsc{Api}}
\newcommand{\lai}{\textsc{lai}}
\newcommand{\Lai}{\textsc{Lai}}
%\newcommand{\url}[1]{\linebreak[4]\texttt{<URL:#1>}}

%%% Local Variables: 
%%% mode: latex
%%% TeX-master: t
%%% End: 

\usepackage[latin1]{inputenc}

\begin{document}
\title{The \Daisy{} C \Api{}}
\author{Per Abrahamsen\\
The Royal Veterinary and Agricultural University\\
Department of Agricultural Sciences\\
Laboratory for Bioclimatology and Agrohydrology}
\date{1998-5-15}
\maketitle

\section*{The \Daisy{} C \Api{}}

The C~\Api{} allows the \daisy{}~\cite{daisy-def} crop/soil
simulation model to be used as a component in other programs.  It
provides a subset of the functionality of the \cplusplus{}~\api{}
described in~\cite{daisy-guide}.

The \api{} can roughly be divided in three parts, 1) running the daisy
simulation, 2) testing and setting the state during the run, and 3)
reading and writing \daisy{} setup files.  You will have to understand
the \daisy{} model itself to use the C~\api{}, and the section on
reading and writing setup files will also require knowledge of the
\cplusplus{} data structures described in ~\cite{daisy-guide}.

To use the functions in C~\api{} from your code, you will have to
include the file \texttt{cdaisy.h}.  This file is well commented, and
will always be up to date.  To link \daisy{} into your code, you can
use the \texttt{daisy.dll} file under \wintel{}.  On \unix{}, you will
usually have to link directly with the relevant object files.  The C
interface itself is implemented in the \texttt{daisy.C} file, which is
included as part of \texttt{daisy.dll}.

\section{Types}

The following opaque types are used when running the simulation.  You
can declare pointers to these types, and use them as arguments to the
\api{} functions.  However, you not use them for other purposes,
i.e. don't attempt to dereference them.

\begin{itemize}
\item \texttt{daisy\_syntax}: The syntax for a component in a \daisy{}
  setup file.  This is used by the parser.
\item \texttt{daisy\_alist}: Parsed representation of a daisy component.
\item \texttt{daisy\_library}: A collection of different parsed
  representations of a daisy component.
\item \texttt{daisy\_parser}: The runtime representation of the
  parser. 
\item \texttt{daisy\_printer}: The runtime representation of the
  pretty printer.
\item \texttt{daisy\_daisy}: The runtime representation of the
  \daisy{} simulation as a whole.
\item \texttt{daisy\_time}: The runtime representation of time.
\item \texttt{daisy\_column}: The runtime representation of a single
  column in the \daisy{} simulation.
\item \texttt{daisy\_weather}: The runtime representation of the
  \daisy{} weather module.
\end{itemize}

There is also a special boolean type in \daisy{}.  Its purpose is to
make the prototypes more self documenting.

\begin{verbatim}
  typedef int daisy_bool;
\end{verbatim}

\section{Running \Daisy{}}

Running the daisy simulation involves parsing the setup file, checking
that it is consistent, and running the simulation.  This is best
described by showing an example; section~\ref{cmain} contains a main
program using the C \api{} and offering the same functionality as the
standard \texttt{daisy} executable.  The individual functions are
documented in the \texttt{cdaisy.h} file, described in
appendix~\ref{cdaisy}. 

\section{Manipulating the State}

The `daisy' object created by `daisy\_daisy\_create' (see
appendix~\ref{cmain}) can be used to extract the runtime
representation of the `time', `weather', `column' components.  For
each of these, there are a number of functions to get and sometimes to
set specific simulation variables.  These functions only represent a
small fraction of the total state.  The example in
appendix~\ref{cmain} demonstrates how to query the `time' object.
All the functions for manipulating the state are listed in
appendix~\ref{cdaisy}. 

\section{Manipulating Setup Files}

As demonstrated in appendix~\ref{cmain}, you need a `syntax' and an
`alist' object to create the simulation.  However, these types are
used to describe all the \daisy{} components.  You can parse them from
files with the `parser' object, and pretty-print them with the
`printer' object.  The can also be stored in a number of predefined
`library' objects.  All of these concepts are described
in~\cite{daisy-guide}.  In appendix~\ref{cdaisy} is listed all the C
bindings. 

\appendix{}

\section{\texttt{cmain.C}}
\label{cmain}

\verbatiminput{../cmain.c}

\section{\texttt{cdaisy.h}}
\label{cdaisy}

\verbatiminput{../cdaisy.h}

\bibliographystyle{plain}
\bibliography{daisy}

\end{document}

%%% Local Variables: 
%%% mode: latex
%%% TeX-master: t
%%% End: 
