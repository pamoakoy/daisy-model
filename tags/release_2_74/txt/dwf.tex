%%% dwf.tex --- The Daisy Weather File Format.

\chapter{Daisy Weather File Format}
\label{cha:dwf}

The Daisy Weather File format (\emph{dwf} for short) is a flexible
format for specifying weather data.  The easiest way to create a dwf
file is to edit an existing file.  Here, we will describe the format
for reference purposes.

First some general syntax.  Empty lines and lines beginning with
`\texttt{\#}' are ignored, and can occur anywhere except at the very
first line.  Line starting with `\texttt{\#}' are called \emph{comment
  lines}.  All words are case-sensitive, and must be written exactly
as specified.

The first line \emph{must} begin with the string `\texttt{dwf-0.0}',
followed by whitespace.  The rest of the line is ignored.

After the first line, the keyword section follows.  Each line (except
blank lines and comment lines) in the keyword section have the general
format:\\
\begin{tt}
  \emph{keyword}: \emph{value}
\end{tt}\\
or\\ 
\begin{tt}
  \emph{keyword}: \emph{value} \emph{dimension}
\end{tt}\\
All keywords, with the exception of `\texttt{Note}' and
`\texttt{Timestep}', must occur exactly once.  The sequence doesn't
matter. 

Here is a list of the recognized keywords, and the dimensions allowed
for them:
\begin{description}
\item[Station] The rest of the line (after the colon) contains the
  name of the weather station.
\item[Elevation] Height above sea level of station, given in
  \texttt{[m]}. 
\item[Longitude] Longitude of station, given in either \texttt{[dgEast]} or
  \texttt{[dgWest]}. 
\item[Latitude] Latitude of station, given in either \texttt{[dgNorth]} or
  \texttt{[dgSouth]}.
\item[TimeZone] Time zone used for the data, given in either
  \texttt{[dgEast]} or \texttt{[dgWest]}.
\item[Surface] Measurement conditions at surface, the value should
  either be `\texttt{reference}' (short grass) or `\texttt{field}' if
  the data have been measured at the simulated field.  
\item[ScreenHeight] Measurement height above earth, given in
  \texttt{[m]}. 
\item[PrecipCorrect] Factors to multiply to the precipitation for each
  month.  Should have the form of twelve numbers, seperated by spaces,
  the first number representing the factor for Janurary, and the last
  for December.  So if the first number is 1.20 and the weather data
  for a given day in Janurary specifies 10 mm precipitation, Daisy
  will calculate with 12 mm precipitation for that day.
\item[Begin] First data point, given in the format
  \texttt{yyyy-mm-dd} or \texttt{yyyy-mm-dd:hh}.
\item[End] Last data point, given in the format \texttt{yyyy-mm-dd} or
  \texttt{yyyy-mm-dd:hh}.  Note that the date specified by
  \texttt{End} should be a few days before the actual end of the
  weather data, to prevent \daisy{} from reading past the end of the
  file.  Also, make sure the simulation ends before the weather data,
  or the results may be lost.
\item[Timestep] Time between data points, given in \texttt{[hours]}.
  You can leave out this keyword, in that case the time for each data
  point must be specified.  Leaving it out allows for varying
  timesteps. 
\item[Note] You can have any number of these, but they must come in
  sequence.  That is, no other keywords between two notes.  You can
  have any text after the keyword. 
\item[NH4WetDep] NH$_4$ deposition in precipitation, given in
  \texttt{[ppm]}.
\item[NH4DryDep] NH$_4$ deposition from air, given in
  \texttt{[kgN/year]}. 
\item[NO3WetDep] NO$_3$ deposition in precipitation, given in
  \texttt{[ppm]}. 
\item[NO3DryDep] NO$_3$ deposition from air, given in
  \texttt{[kgN/year]}. 
\item[TAverage] Average temperature for the location, given in
  \texttt{[dgC]}. 
\item[TAmplitude] Amplitude of yearly temperature variation, given in
  \texttt{[dgC]}. 
\item[MaxTDay] The Julian day with the highest temperature (on
  average), given in \texttt{[yday]}.
\end{description}

After the keywords there should be a line consisting solely of
hyphens.  It marks the beginning of the data section.  The lines in
the data section are divided into columns by whitespace.  All lines
must have the same number of columns.  The first line contains the
name of the data type specified in each column, the second line the
dimension of that data, and the following lines the actual measurement
data in chronological order.  The following is a list of the possible
data type names, as well as the recognized dimensions.
\begin{description}
\item[Year] Given in \texttt{[year]}.
\item[Month] Given in \texttt{[month]}.
\item[Day] Given in \texttt{[mday]}.
\item[Hour] Given in \texttt{[hour]}.
\item[GlobRad] Global radiation, given in \verb|[W/m^2]|.
\item[AirTemp] Air temperature, given in \texttt{[dgC]}.
\item[Precip] Precipitation, given in \texttt{[mm/h]} or \texttt{[mm/d]}.
\item[RefEvap] Reference evapotranspiration, given in \texttt{[mm/h]}
  or \texttt{[mm/d]}.
\item[VapPres] Vapor pressure, given in \texttt{[Pa]}.
\item[RelHum] Relative humidity, given in \texttt{[fraction]} or
  \texttt{[\%]}. 
\item[Wind] Wind speed, given in \texttt{[m/s]}
\end{description}
The columns can be arranged in any sequence.  GlobRad is mandatory,
Year, Month, and Day are mandatory if no timestep have been
specified.  You cannot specify both VapPres and relHum.  All other
columns are optional. 

%%% Local Variables: 
%%% mode: latex
%%% TeX-master: "reference"
%%% End: 

%%% dwf.tex ends here
