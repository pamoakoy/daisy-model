\chapter{Deep leaching, colloids and biopores}
\label{app:col-biopores}

Figure~\ref{fig:Silstrup-leak150} and~\ref{fig:Estrup-leak150} show
simulated leaching at 150 cm, 30 cm below the end of the biopores at
the two sites.  For bromide, about 20\% of the applied amount is lost
that way.  For Silstrup we see a slow, but steady leaching of
pesticides, in the order of 0.1\% of the applied amount.  For Estrup,
the only leaching we see is glyphosate, all apparently comming from a
single event.

\begin{figure}[htbp]
  \begin{center}
    \figtop{Silstrup-leak150bromide}\\
    \figtop{Silstrup-leak150}\\
    \fig{Silstrup-leak150acc}
  \end{center}
  \caption{Silstrup simuleret leaching at 1.5 meter, 30 cm under bioporers.}
  \label{fig:Silstrup-leak150}
\end{figure}

\begin{figure}[htbp]
  \begin{center}
    \figtop{Estrup-leak150bromide}\\
    \figtop{Estrup-leak150}\\
    \fig{Estrup-leak150acc}
  \end{center}
  \caption{Estrup simuleret leaching at 1.5 meter, 30 cm under bioporers.}
  \label{fig:Estrup-leak150}
\end{figure}

Colloid simulation is based on R{\o}rrendeg{\aa}rd data, automatically
adjusted for clay content, as discussed in section~\ref{sec:coltrans}.
Figure~\ref{fig:colloids} shows how Siltrup (with the highest clay
content in the plow layer) has the highest colloid leaching, and the
values for Estrup are somewhat higher than what have been measured at
R�rrendeg�rd (which has the lowest clay content).

\begin{figure}[htbp]
  \begin{center}
    \figtop{Silstrup-colloid}
    \fig{Estrup-colloid}
  \end{center}
  \caption{Colloids in drain water in Silstrup (top graph) and Estrup
    (bottom).}
  \label{fig:colloids}
\end{figure}

Figure~\ref{fig:Silstrup-biopore} shows all biopore activity at the
top of the Silstrup soil, while
figure~\ref{fig:Silstrup-biopore-drain} shows only the activity in the
biopores directly connected with the drain pipes.  The effect of the
crust added to the simulation 2001-06-01 is clearly visible, instead
of being activated in the plow layer, biopores are now activated on
the surface.  For Estrup, where no crust has been added, events with
biopore activity from the soil surface are rare, and the biopores are
dominated by the plow layer and plow pan.
Figure~\ref{fig:Estrup-biopore} and~\ref{fig:Estrup-biopore-drain}.

\begin{figure}[htbp]
  \begin{center}
    \figtop{Silstrup-biopore}\\
    \fig{Silstrup-biopore-acc}\\
  \end{center}
  \caption{Biopore activity in different soil layers.  The layers are
    ponded water, soil surface (top 3 cm), the rest of the plow layer,
    the plow pan, and the the B horizon below plow pan down to 50 cm.}
  \label{fig:Silstrup-biopore}
\end{figure}

\begin{figure}[htbp]
  \begin{center}
    \figtop{Silstrup-biopore-drain}\\
    \fig{Silstrup-biopore-drain-acc}
  \end{center}
  \caption{Drain contribution through biopores from different soil
    layers.  The layers are ponded water, soil surface (top 3 cm), the
    rest of the plow layer, the plow pan, and the the B horizon
    below plow pan down to 50 cm.}
  \label{fig:Silstrup-biopore-drain}
\end{figure}

\begin{figure}[htbp]
  \begin{center}
    \figtop{Estrup-biopore}\\
    \fig{Estrup-biopore-acc}\\
  \end{center}
  \caption{Biopore activity in different soil layers.  The layers are
    ponded water, soil surface (top 3 cm), the rest of the plow layer,
    the plow pan, and the the B horizon below plow pan down to 50 cm.}
  \label{fig:Estrup-biopore}
\end{figure}

\begin{figure}[htbp]
  \begin{center}
    \figtop{Estrup-biopore-drain}\\
    \fig{Estrup-biopore-drain-acc}
  \end{center}
  \caption{Drain contribution through biopores from different soil
    layers.  The layers are ponded water, soil surface (top 3 cm), the
    rest of the plow layer, the plow pan, and the the B horizon
    below plow pan down to 50 cm.}
  \label{fig:Estrup-biopore-drain}
\end{figure}


