\mchapter{Finite Element Method}

\section{Why FEM?}\index{FEM}\index{Finite Element Method}

A large number of numerical hydrological models has been developed to
solve both Richard's equation and the advection-dispersion
equation. The most commonly used types of numerical models are the
Finite Difference Method\index{Finite Difference Method}
(FDM)\index{FDM} and the Finite Element Method (FEM). The FEM has some
advantages over the FDM. One such advantage is its capability to
handle complex geometries which for example can occur with material
(here the soil) properties with a high level of heterogeneity. An
example is the heterogeneity of the hydraulic properties in a ridged
soil. This case will be considered later. \\ 
\\
Another advantage is the possibility of building in an adaptive
mesh generator if required. An adaptive mesh generator makes it
possible to regenerate the mesh on the way, for example, to make the
net finer where the uncertainties are large. \\
\\
However, there are also some disadvantages associated with the FEM. The
calculation costs are normally higher, in comparison with the FDM, for
simple geometries. The FEM is also much more mathematically
complicated as well as less intuitive than the finite difference
technique.\\
\\
This chapter describes the solving of the 2-dimensionally, vertical and
time dependent equation of Richard's and the advection-dispersion
equation using the FEM. 


\section{General description of the PDEs}

The physical explanations for Richard�s equation and the
advection-dispersion equation, respectively, are very different, whereas
the two PDEs look almost alike. Both equations are covered by the more
general expression: 


\begin{equation}
a\frac{\partial \phi}{\partial t}-\nabla \cdot 
\left(\mathbf{b}\nabla \phi - \mathbf{c}\phi+\mathbf{d} \right) +
e\phi+f=0 
\label{eq:general}
\end{equation}

where $\phi$ is the dependent variable. $a$, $e$ and $f$ are
scalars. $\mathbf{b}$ is a 2-by-2 matrix or a scalar. $\mathbf{c}$ and
$\mathbf{d}$ are vectors. Both $a$, $\mathbf{b}$, $\mathbf{c}$,
$\mathbf{d}$, $e$ and $f$ can be dependent on $\phi$.\\
\\
The working process in the development of TopFEM has been first
to develop a water movement model without thinking (too much) of the
later developed solute movement model. The solute movement model is
later built on the same skeleton as the water movement model with some
changes and expansions. \\ 
\\
The mathematical theories behind the solution of water and solute
movement equations have a lot of similarities. The solution of the
more general equation, equation \ref{eq:general} is treated
first. Later the actual specialties for the solution process of
the water movement and solute movement problems, respectively, are
discussed in their own sections. \\
\\
Actually, equation \ref{eq:general} covers a lot of different areas in
both physics\index{physics} and chemistry\index{chemistry}, but can
also be used to describe the pricing of stock options, as done in the
Black-Scholes model, \cite{Black-Scholes}\index{Black-Scholes} (which
only is 1D in space). Work behind the model gave in 1997 the Nobel
Price\index{Nobel Price} in economics\index{economics} to
Scholes\index{Scholes} and Merton\index{Merton}, \cite{Nobel}. \\ 
\\
Table \ref{tab:difcoeff} shows the coefficients of equation
\ref{eq:general} shown for Richard's equation and the
advection-dispersion equation, respectively. The coefficients are
obtained if $\phi$ is replaced with $\psi$ and $C$ in the water and
solute movement equations, respectively. According to classification
rules for PDEs in two variables (space and time),
\cite{Asmar,Trottenberg} both Richard�s equation and the
advection-dispersion equation are of parabolic type.  The PDEs are
\textit{quasi-linear} since the coefficients in general depend on
$\phi$. It should be noticed that it is assumed that the solution for
the water movement simulations, i.e. $\mathbf q$ and $\theta$ is known
when solving the solute movement equation.\\
\\


\begin{table}[h]
\tabcap{Coefficients for water and solute movement}
\label{tab:difcoeff}
\begin{tabular}{p{2.5cm}|p{4.95cm}|p{4.95cm}} \hline
\textbf{Coefficient} & \textbf{Value, water movement} & \textbf{Value,
  solute movement} \\ \hline
$a$ & $C_w$ & $\theta R$ \\ \hline
$\mathbf{b}$ & $K$\ or\ $\begin{bmatrix} K & 0 \\ 0 & K
  \end{bmatrix}$ & $\theta \begin{bmatrix} D_{xx} & D_{xy} \\ D_{yx}
  & D_{yy} \end{bmatrix}$ \\ \hline 
$\mathbf{c}$ & $\begin{bmatrix}0 \\ 0 \end{bmatrix}$ &
$\begin{bmatrix} q_x \\ q_y \end{bmatrix}$ 
\\ \hline
$\mathbf{d}$ & $\begin{bmatrix} 0 \\ K \end{bmatrix}$ &
$\begin{bmatrix} 0 \\ 0 \end{bmatrix}$ \\ \hline
$e$ & $0$ & $\frac{\partial \theta}{\partial t}$ \\ \hline
$f$ & $\Gamma$ &
$C_{sink}\Gamma-\theta\mu_l-\rho_b\mu_s$ \\ \hline
\end{tabular}
\end{table}


To solve the equation \ref{eq:general}, it is necessary to specify
initial\index{initial condition} and boundary
conditions\index{boundary condition}. The boundary conditions specify
a combination of $\phi$ and its derivative on the boundary:

%\begin{equation}
%\begin{cases}
%& \mathbf{\bar{n}} \cdot (\mathbf{b}\nabla \phi -
%\mathbf{c}\phi+\mathbf{d}) + r\phi =h \\
%& \phi=\phi_0 
%\end{cases}
%\end{equation}

\begin{equation}
\mathbf{\bar{n}} \cdot (\mathbf{b}\nabla \phi -
\mathbf{c}\phi+\mathbf{d}) + r\phi =h
\end{equation}

\begin{equation}
\phi=\phi_0
\end{equation}


where $\mathbf{\bar{n}}$ is the outward unit normal. The first type is
known as the \textit{Robin boundary condition}\index{Robin boundary
  condition}\index{boundary condition!Robin} and the second is known
as the \textit{Dirichlet boundary condition}\index{boundary
  condition!Dirichlet}\index{Dirichlet boundary condition}. The
Dirichlet conditions can be approximated by Robin boundary conditions
by setting $h=rd$ and then letting $r\rightarrow \infty$. Division
with a large $r$ cancels the derivative term. The \textit{Neumann
  boundary condition}\index{Neumann boundary condition}\index{boundary
  condition!Neumann} is obtained by setting $r=0$:

\begin{equation}
\mathbf{\bar{n}} \cdot (\mathbf{b}\nabla \phi - \mathbf{c}\phi+\mathbf{d}) = 
\mathbf{\bar{n}}\cdot(-\mathbf{q})=-q
\label{eq:Neumann}
\end{equation}

where $\mathbf{q}$ is the flux vector and $q$ is the size of the
outward flux. Only Dirichlet and the Neumann boundary conditions
are considered here.


\section{Method of weighted residual}\index{weighted residual}
 
The solution to \ref{eq:general}, $\phi$ is approximated with a
trial function. The residual is then defined as:

\begin{equation}
R=a\frac{\partial \phi}{\partial t}-\nabla \cdot 
\left(\mathbf{b}\nabla \phi -
  \mathbf{c}\phi+\mathbf{d} \right) + e\phi+f\neq 0
\end{equation}

The residual cannot generally be expected to be equal to zero in the
domain. A weight\index{weight function} or simply test
function\index{test function}, $v$ is chosen and the weighted average
over the domain is set to zero:


\begin{equation}
\begin{split}
& R_W=\int_{\Omega} R v \, dxdy = \\ & \int_{\Omega} \left( a\frac{\partial
  \phi}{\partial t}-\nabla \cdot  \left(\mathbf{b}\nabla
  \phi - \mathbf{c}\phi+\mathbf{d} \right) + e\phi+f
  \right)v\, dxdy =0
\label{eq:resweight}
\end{split}
\end{equation}


The approximation function for $\phi$ can contain one or more unknown
constants, $a_i$ that have to be determined so equation
\ref{eq:resweight} is fulfilled. In order to find the constants
unambiguous, the number of independent equations should be the same as
the number of constants. The different equations can simply be obtained
using different testfunctions. Commonly used methods are:




%Many different test functions can be chosen - for not to mention the
%approximation function\index{approximation function} that approximates
%$\phi$). The approximation functio n should contain one or more unknown
%constants, $a_i$ that have to be determined so equation
%\ref{eq:resweight} is fullfilled. 
%
%
%But the number of test functions should be
%the same as the number of unknown constants in the approximating
%function in order to find the coefficients. Commonly used methods are:


\begin{itemize}
\item \textit{Collocation Method}\index{collocation method}. For
  obtaining the unknown coefficients, $a_1, a_2, \dots a_n$ the
  residual is set equal to zeros at $n$ distinct points in the solution
  domain. Using the weighted residual equation it is the same as
  setting $v_i=\delta(x-x_i)\delta(y-y_i)$ where $\delta$ is the Dirac
  delta function. The fixed points $(x_i,y_i),  i=1,2,\dots n$ are
  called the collocation points.  
\item \textit{Least Squares Method}\index{least squares method}. The
  square of the residual is minimized with respect to the
  parameters. It can be derived that this is equal
  to setting $v_i=\frac{d R}{d a_i}$. 
\item \textit{Galerkin�s Method}\index{Galerkin�s method}. In the
  Galerkin method, the test function is  $v_i=\frac{d \phi}{d a_i}$.
\end{itemize}

Often, either the Least Squares or especially the Galerkin method are
 used. These methods give in general small errors. The Galerkin
 Method is chosen here. As approximating functions (and test
 functions) piecewise linear polynomials are used - much more about
 them later!


\section{Weak formulation}\index{weak formulation}

Since $\phi$ will be approximated by a function whose first-order
derivative has jump discontinuities at the nodes, it would be
necessary to reformulate the problem so as to remove the second-order
derivative in \ref{eq:resweight}. For that purpose Gauss divergence
theorem\index{Gauss divergence theorem}, \cite{HEJ} is excellent:


\begin{equation}
\int_{\Omega}\nabla \cdot\mathbf{V} \, d\Omega=\int_{\partial \Omega}
\mathbf{\bar{n}}\cdot \mathbf{V} \, dS
\label{eq:gauss}
\end{equation}

or in words: the total divergence of the vector field,  $\mathbf{V}$ in
the volume $\Omega$ is equal to the net flux through the surface of
$\Omega$, $\partial \Omega$. In a 2D domain, the words volume and
surface should be replaced by area and boundary, respectively. Another
useful rule from the vector calculus, \cite{HEJ} is:


\begin{equation}
\nabla \cdot (f\mathbf{V})=(\nabla f)\cdot \mathbf{V}+ (\nabla \cdot
\mathbf{V})f 
\label{eq:vectorthing}
\end{equation}

where $f$ is a $C^1$-function (continuous and 1 time
differentiable). Applying equations \ref{eq:gauss} and
\ref{eq:vectorthing} in the second term in the weighted residual
equation gives: 

\begin{equation}
\begin{split}
&\int_{\Omega} \nabla \cdot (\mathbf{b}\nabla \phi - \mathbf{c}\phi
+\mathbf{d} )v \, dxdy = \\
&\int_{\Omega} \nabla \cdot \left( (\mathbf{b}\nabla \phi - \mathbf{c}\phi
+\mathbf{d} )v \right) \, dxdy - \int_{\Omega}(\nabla v)\cdot(\mathbf{b} \nabla \phi-
  \mathbf{c}\phi+\mathbf{d})\, dxdy = \\
& \int_{\partial \Omega} \mathbf{\bar{n}}
  \cdot (\mathbf{b}\nabla \phi-\mathbf{c}\phi +  \mathbf{d})v \, dS -
\int_{\Omega}(\nabla v)\cdot(\mathbf{b} \nabla \phi-
  \mathbf{c}\phi+\mathbf{d})\, dxdy
\end{split}
\end{equation}

The weighted residual equation can now be written without second order
derivatives:

\begin{equation}
\begin{split}
& \int_{\Omega} a\frac{\partial
  \phi}{\partial t} v\,  dxdy -\int_{\partial \Omega} \mathbf{\bar{n}}
  \cdot (\mathbf{b}\nabla \phi-\mathbf{c}\phi +  \mathbf{d})v \, dS +
  \\ &\int_{\Omega}(\nabla v)\cdot(\mathbf{b} \nabla \phi-
  \mathbf{c}\phi+\mathbf{d})\, dxdy+ \int_{\Omega} e\phi v \, dxdy
  +\int_{\Omega} fv \, dxdy =0 
\end{split}
\label{eq:weak1}
\end{equation}

This equation is known as the variational, or weak form of the
differential equation. Obviously, any solution of the differential
equation is also a solution of the variational problem. If $v$ is
sufficiently smooth, one can also show the converse. A discussion of
how smooth $v$ should be is outside the scope of this report, but
piecewise linear polynomials are here considered to be smooth
enough. \\ 
\\
The boundary of $\Omega$, $\partial \Omega$ can be divided into
parts, $\partial \Omega_{1}$ and $\partial \Omega_{2}$ with a
Neumann\index{Neumann boundary condition}\index{boundary
  condition!Neumann} and a Dirichlet boundary
condition\index{Dirichlet boundary condition}\index{boundary
  condition!Dirichlet}, respectively. The terms in the parenthesis in
the boundary integral in equation \ref{eq:weak1} can be replaced with
the right hand side of \ref{eq:Neumann} on $\partial \Omega_{1}$. We
have previously seen how a Dirichlet boundary condition can be
approximated with a Robin boundary condition. But this is numerically
a bad solution because it can produce ill-conditioned
matrix-systems. The Dirichlet boundary is implemented by simply
forcing $\phi$ to be equal to the wanted value on $\partial
\Omega_{2}$. The procedure is described later. We now have to solve:


\begin{equation}
\begin{cases}
R_W=\int_{\Omega} a\frac{\partial \phi}{\partial t} v \, dxdy
 + \int_{\Omega}(\nabla
v)\cdot(\mathbf{b} \nabla \phi- \mathbf{c}\phi+\mathbf{d})\, dxdy  \\+ 
 \int_{\Omega} e\phi v \, dxdy + \int_{\Omega}fv \, dxdy +\int_{\partial \Omega_1} qv \, dS=0 &
 \text{in} \ \Omega \\ \phi=\phi_0 & \text{on} \ \partial \Omega_2 
 \end{cases}
\label{eq:weak2}
\end{equation}


\section{Finite Element basis functions}\index{basis function}

We want to solve the PDEs in a 2 -dimensional domain. 
Figure \ref{fig:mesh} shows an example of a 2D domain, $\Omega$.  A
finite element mesh of triangular elements are generated on
$\Omega$. As mentioned above it is the intention to use piecewise
linear polynomials to approximate $\phi$. To describe a linear
polynomium it is necessary to know the value in three points. The
unknowns are here defined as the vertices of the triangular mesh. This
is also called a \textit{vertex centered location of the unknowns},
\cite{Trottenberg}. \\
\\
Since the boundary, $\partial\Omega$ of $\Omega$ in the figure is
curved, the mesh cannot coincide with $\partial \Omega$. But if the
mesh is refined sufficiently an arbitrary good approximation can be
obtained. The situation is comparable with the situation for a Roman
architect who wants to build an arch, but only has simple building
blocks. If the result is not close enough to the perfect arch, he can
use smaller building blocks.\\
\\
The nodes are placed at the vertices and have the numbers
$i=1,2\ldots NP$. For every element we need to assign local node
numbers $(1,2,3)$. The local node numbers are placed in counter
clockwise order in each triangle. For each node $(x_{i},y_{i})$ we
introduce a global finite element basis function\index{basis
  function!global} $N_{i}(x,y)$ defined on $\Omega$ as follows:

\begin{itemize}
\item $N_{i}(x,y)$ is continuous on the entire mesh and linear on each
  element.
\item $N_{i}(x_{j},y_{j})=\delta_{ij}$
\end{itemize}

where $\delta_{ij}$ is Kronecker�s delta function which is equal to
one for $i=j$ and zero for $i\neq j$. Figure \ref{fig:global} shows a
picture of a global basis function. Instead of viewing the entire
mesh, a single element can be considered for many purposes. On a
triangle (or element) with the nodal points $i$, $j$, $k$, there are only
three nonzero basis functions, the local basis functions\index{basis
  function!local}, $N_{1}$, $N_{2}$ and $N_{3}$. The local basis
functions can on the triangles be expressed as:

 
\begin{figure}[h]  %here-top-bottom-page
\begin{center}
\epsffile{mesh.eps}
\figcap{Mesh fitted to a domain where part of the boundary is curved.}
\label{fig:mesh}
\end{center}
\end{figure}


\begin{equation} \begin{split}
&N_{1}=a_{1}+b_{1}x+c_{1}y\\
&N_{2}=a_{2}+b_{2}x+c_{2}y\\
&N_{3}=a_{3}+b_{3}x+c_{3}y
\end {split} \end{equation}

\begin{figure}[htbp]  %here-top-bottom-page
\begin{center}
\epsffile{global.eps}
\figcap{Global basis function, $N_{20}$ related to node no. 20.}
\label{fig:global}
\end{center}
\end{figure}

The local basis functions must fulfil:

\begin{equation} 
\begin{split} 
&N_{1}(x_{1},y_{1})=1 \\  &N_{1}(x_{2},y_{2})=0 \\  &N_{1}(x_{3},y_{3})=0 
\end{split}
\begin{split}
\quad &N_{2}(x_{1},y_{1})=0 \\ \quad &N_{2}(x_{2},y_{2})=1 \\ \quad &N_{2}(x_{3},y_{3})=0 
\end{split}
\begin{split}
\quad &N_{3}(x_{1},y_{1})=0 \\ \quad &N_{3}(x_{2},y_{2})=0 \\ \quad &N_{3}(x_{3},y_{3})=1 
\end{split} \end{equation}

This gives a linear systems of 9 equations with 9 unknowns. But it is
only $b_i$ and $c_i$ , $i=1,2,3$ which are interesting for later
purpose (they describe the slope of the basis functions). The solution
according to \cite{Segerlind} is :

\begin{equation} 
\begin{split}
& \quad b_{1}=\frac{1}{2A}(y_{2}-y_{3})\\ 
& \quad c_{1}=\frac{1}{2A}(x_{3}-x_{2})
\end{split}
\begin{split} 
& \quad b_{2}=\frac{1}{2A}(y_{3}-y_{1})\\ 
& \quad c_{2}=\frac{1}{2A}(x_{1}-x_{3}) 
\end{split}
\begin{split}
& \quad b_{3}=\frac{1}{2A}(y_{1}-y_{2})\\
& \quad c_{3}=\frac{1}{2A}(x_{2}-x_{1})
\end{split}  
\end{equation}

Where A is the area\index{element!area} of the triangles if the nodal
points on the triangles are counted counter clockwise. Since the local
nodes are placed counter clockwise, the area can be calculated as,
\cite{Segerlind}:


\begin{equation}
A=\frac{1}{2}(x_{1}y_{2}+y_{1}x_{3}+x_{2}y_{3}-x_{3}y_{2}-y_{3}x_{1}-x_{2}y_{1})
\label{eq:area}
\end{equation}

The derivatives of the local basis functions\index{basis
  function!derivative} are easily calculated to: 

\begin{equation} 
\begin{split}
&\frac{\partial N_{1}}{\partial x}=b_{1}\\ 
&\frac{\partial N_{1}}{\partial y}=c_{1}
\end{split}
\begin{split}
& \quad \frac{\partial N_{2}}{\partial x}=b_{2}\\
& \quad \frac{\partial N_{2}}{\partial y}=c_{2}
\end{split}
\begin{split}
& \quad \frac{\partial N_{3}}{\partial x}=b_{3}\\
& \quad \frac{\partial N_{3}}{\partial y}=c_{3}
\end{split} 
\label{eq:abc}
\end{equation}

If we want to integrate\index{basis function!integration} the product
of some of the basis functions over the area of a triangle, the general
formula from \cite{Segerlind} can be used.

\begin{equation}
\int_{A}N_{1}^{a}N_{2}^{b}N_{3}^{c} \, dxdy=2A\frac{a!b!c!}{(a+b+c+2)!}
\label{eq:areaint}
\end{equation}

where the powers, $a$, $b$ and $c$ are integers. A similar formula is
given for integration over a one-dimensional element with local
element basis functions $M_1$ and $M_2$:


\begin{equation}
\int_{l}M_{1}^{a}M_{2}^{b}ds=L\frac{a!b!}{(a+b+1)!}
\label{eq:boundaryint}
\end{equation}

where $L$ is the length of the edge. This is important because
integration over a one-dimensional element is the same as integration
over the border of the 2-dimensional element since a basis function
in a triangular element varies linearly along the sides between its
nodes and the other two nodes. At the side opposite its nonzero node
the 2 dimensional basis function is zero.\\
\\
It is in \cite{Barker} shown that a given function $f(x,y)$ in
$\Omega$ can be approximated with $f_{I}(x,y)$ where:

\begin{equation}
f_{I}=\sum_{j=1}^{NP}f(x_{j},y_{j})N_{j}(x,y)
\label{eq:approx}
\end{equation}

Because each $N_{i}(x,y)$ is continuous and piecewise linear, 
$f_{I}$ is also continuous and piecewise linear. It is also easy to
see that $f_{I}(x_{i},y_{i})=f(x_{i},y_{i})$. Figure \ref{fig:approx}
shows an approximation of a function on the same domain as in
figures \ref{fig:mesh} and \ref{fig:global}. The mesh in figure
\ref{fig:approx} is just refined. Under the assumption that the second
order derivatives of $f$ are bound and that the mesh coincides with
$\partial \Omega$, a constant, $C$ in each element, $e_N$ exists
according to \cite{Barker}:

\begin{equation}
|f_I(x,y)-f(x,y)|\leq Ch_n^2\ \ , \ \ \ (x,y)\in e_n
\end{equation}

where $h_n$ is the largest element side, i.e a constant, $C$ exists so
the size of the largest error in $\Omega$ is below $Ch_{max}^2$ where
$h_{max}$ is the largest element side in the entire mesh.

\begin{figure}[h]  %here-top-bottom-page
\begin{center}
\epsffile{approx.eps}
\figcap{Approximation of a function on a triangular mesh.}
\label{fig:approx}
\end{center}
\end{figure}


\section{Mesh
  generation}\index{grid generation}\index{mesh!generation}


Today meshes used for FEM analysis can consist of thousands or even
millions of elements. Previously, the much smaller meshes were often
generated by hand. Today the meshes are normally generated by meshing
algorithms for example included in the CAE tool or as independent
software. \\ 
\\
The fineness of the meshes is of course just a question of sufficient
computational power. But as the speed of computers has increased, also
the tasks and complexity of the problems have increased
correspondingly. \\
\\
The necessary size of the elements depends naturally of the physical
problems, but also of where the solution is of interest. Sharp
gradients of the solution demand fine meshes. Also boundaries with
high curvature small elements have to be described properly. The size of
the elements in a single mesh can often vary strongly. \\
\\
Some finite element codes have even adaptive meshing algorithms where
the mesh dynamically during the computation process is adopted. If
the mesh refinement is controlled by an \textit{adaption
  criterion}\index{adaption criterion} the adaption approach
is \textit{self-adaptive}, \cite{Trottenberg}. When the solution changes
rapidly due to for example shocks, adaptive meshes can be an advantage. In
hydrology for example, sharp wetting fronts that occur in connection
with infiltration or the movement of contaminants. \\
\\
Another possibility is using multigrid\index{multigrid} solvers,
\cite{Silvester,Trottenberg}. Multigrid solvers use at least 2
levels of grid size. When solving the PDE, we end up with a linear
system of the form $\mathbf{A}\mathbf{x}=\mathbf{b}$ which has to be solved at
least 1 time for each timestep (see later). Multigrid works together
with iterative solvers for the (sparse) linear equation system, typically
\textit{Richardson} \textit{Gauss-Seidel} and \textit{Jacobi}
iterations. The multigrid approach is very efficient to PDEs of elliptic
type, where the mathematical theory also is quite well known,
\cite{Trottenberg}. Using a multigrid approach for the governing
equation for either the water or the solute movement (or
for equation \ref{eq:general}) would be a very comprehensive task. But
for example a multigrid program exists for solving compressible
RANS (Reynolds-Averaged Navier-Stokes equations), \cite{Trottenberg}.
Neither the rather complicated adaptive grids nor the multigrid
approach (or combinations) are used in TopFEM. \\
\\
The algorithms for generation of a non-regular triangular mesh are not
discussed in this report. But many 2D grid generators produce triangular
grids which fulfil the \textit{Delaunay criterion}. All the unregular
meshes used for the simulations described in the thesis are generated
with such an algorithm.


\subsection{Delaunay criterion}\index{Delaunay}

In a 3D space, the \textit{Delaunay
  criterion}\index{Delaunay!criterion} according to \cite{Owen} is:
  \textit{``Any node must not be contained within the circumsphere of
  any tetrahedra within the mesh''}. A circumsphere can be defined as
  the sphere passing through all four vertices of a tetrahedron. The
  similar two-dimensional rule is:


\begin{itemize}
\item Any node must not be contained within the circumcircle of any
  triangle within the mesh.
\end{itemize}

In figure \ref{fig:Delaunay} the criterion is illustrated with 2
simple two-element meshes representing the same domain. It should be
noted that the Delaunay criterion is not an algorithm, just a
rule!. The advanced algorithm, which FEMLAB uses, is only roughly
described in \cite{FEMLAB}. All meshes presented in this dissertation
are produced using FEMLAB\index{FEMLAB}.


\begin{figure}[H]  %here-top-bottom-page
\begin{center}
\epsfig{file=delaunayfig.eps,width=9cm}
%\epsffile{delaunayfig.eps}
\figcap{Illustration of the Delaunay criterion. (A) fulfills the
  criterion, (B) does not.}
\label{fig:Delaunay}
\end{center}
\end{figure}


\subsection{Mesh quality}\index{mesh!quality}

\cite{Segerlind} has analyzed the heat transfer equation with linear
coefficients and found the statement: \textit{No interior angle of the
  triangle should exceed 90 degrees}. Similar statements that no
angel in the elements must be too large or too small (which is
the same) are given in many other places as for example in
\cite{Bank}.\\
\\
In order to describe the quality of the meshes, especially the
quality indicator by Bank\index{Bank}, \cite{Bank} is often used:

\begin{equation}
q=\frac{\sqrt{3}A}{l_1^2+l_2^2+l_3^2}
\end{equation}

where $l_1$, $l_2$ and $l_3$ are the side lengths of
the triangle. $A$ is the area. $q$ is a number between 0 and 1. The
quality is good when $q$ is high and  $q$ is equal to one for an
equilateral triangle. The indicator is independent of the
element size. Another indicator is the indicator by
Berzins\index{Berzins}, \cite{Berzins}:

\begin{equation}
q=\frac{l_1^2+l_2^2+l_3^2+(l_1+l_2+l_3)^2}{16\sqrt{3}A}
\end{equation}

The fundamental assumption made here, is that the real solution is
quadratic and that the approximation is based on piecewise linear basis
functions. $q$ has the value one for an equilateral triangle and
tends $\infty$ if one of the side lengths are constant and the area
tends to zero. Other mesh quality indicators also exist as for
example the one made by Weatherill as described in \cite{Berzins}.  


\subsection{Element size}\index{element!size}

In the description of the solute movement the mesh dependent P\'{e}clet
$P_e$\index{P\'{e}clet Number} and Courant, $C_r$ numbers\index{Courant
  number} are introduced, and it is discussed how the size of the
elements can influence the computed solution. Techniques to avoid some
of the problems are also presented there.


\newpage

\section{Galerkin�s method}\index{Galerkin�s method}

Here it is explained how the Galerkins method works on the general
equation, \ref{eq:general}. Some details are omitted but explained
in detail in subsections for the water and the solute movement,
respectively. The solution, $\phi$ is approximated with piecewise
linear functions: 


\begin{equation} 
\phi(x,y,t)\approx
\sum_{j=1}^{NP}\phi_{j}(t)N_{j}(x,y)=\sum_{j=1}^{NP}\phi_{j}N_{j}
\label{eq:sum}
\end{equation}

where $\phi_{j}$ is the value for $\phi$ in the nodal point
no. $j$. $NP$ is the total number of nodal points. The task is for a
given time to find the vector $\boldsymbol{\phi}$ containing the
values for $\phi$ in the nodal points
($\phi_1,\phi_2,\ldots,\phi_{NP})$. \\
\\
The idea in the Galerkin�s method is to use the same functions, for
the interpolation functions as for the test function and apply it in
the weak form of the differential equation, \ref{eq:weak2}. If we set
$v=v_{i}=N_{i}$ for $i=1,2,\ldots ,NP$ in equation \ref{eq:weak2}, we
will get $NP$ equations (and unknowns). In other words, a weighted
residual\index{weighted residual} equation for each nodal point:

\begin{equation}
\begin{bmatrix}
R_{W,1} \\
R_{W,2} \\
\vdots  \\
R_{W,NP} 
\end{bmatrix}=\begin{bmatrix}
0 \\
0 \\ 
\vdots \\
0 \end{bmatrix}
\label{eq:residualvec}
\end{equation}

where $R_{W,i}$ is the weighted residual equation as defined in equation
\ref{eq:weak2}. By inserting the approximation for $\phi$, equation
\ref{eq:sum} and the test function it can be written as:

\begin{equation}
\begin{split}
& R_{W,i}=\int_{\Omega} a \sum_{j=1}^{NP}\dot{\phi_{j}}N_{j} N_{i} \, dxdy
 + \\
& \int_{\Omega}(\nabla N_{i})\cdot \left( \mathbf{b} \nabla \left(
 \sum_{j=1}^{NP}\phi_{j}N_{j}\right) - \mathbf{c}
 \sum_{j=1}^{NP}\phi_{j}N_{j} +\mathbf{d} \right) \, dxdy  + \\ & \int_{\Omega}
 e\sum_{j=1}^{NP}\phi_{j}N_{j} N_{i} \, dxdy + \int_{\Omega}f
 N_{i} \, dxdy  +\int_{\partial  \Omega_1} q N_{i} \, dS=0  
\end{split}
\label{eq:residuali}
\end{equation}

The only unknown in equation \ref{eq:residuali} are $\phi$ and
$\dot{\phi}$ in all the nodal points. Equation \ref{eq:residualvec}
and \ref{eq:residuali} can in combination be written on matrix form:

\begin{equation}
\mathbf{A}\dot{\boldsymbol{\phi}}+(\mathbf{B}-\mathbf{C}+\mathbf{E})\boldsymbol{\phi}+\mathbf{D}+\mathbf{F}+\mathbf{Q}=0
\label{eq:matrix}
\end{equation}

Where each term in equation \ref{eq:matrix} corresponds to a term in
equation \ref{eq:residuali}. $\boldsymbol{\phi}$ and
$\boldsymbol{\dot{\phi}}$ are vectors that contain $\phi$ and the time
derivatives in the nodal points. This equation is a first order
\textit{ordinary differential equation} (ODE)\index{ODE}.\\ 
\\
It shall be noted that $N_{i}$ only is nonzero in the triangles with
nodal point $i$, i.e. there are only few elements witch have nonzero
contributions to equation \ref{eq:residuali} for each test
function. This fact has the positive effect that the matrices in
\ref{eq:matrix} are sparse. \\ 
\\
Instead of looking at node $i$ (test function $N_{i}$) and then
calculating the contributions to equation \ref{eq:weak2} from all the
neighboring triangles, it is of computational reasons more convenient first
to look at a given triangle in the mesh and then evaluate the
integrals for the 3 nonzero test functions in the element. The
contributions to the system of $NP$ equations from each triangle is
collected in a system of so called local matrices\index{local
  matrices} and vectors\index{local vectors}. \\
\\
In an element, the local nodes 1, 2 and 3 (with basis functions $N_1$,
$N_2$ and $N_3$) correspond to the global node numbers $i$, $j$ and
$k$. In each element, there are only three nonzero test functions, $v_1$, $
v_2$ and $v_3$ (local numbers). The contributions to the rows $i$, $j$
and $k$ in \ref{eq:matrix} from each element are calculated and placed
in the local matrices:
$\mathbf{(A}\boldsymbol{\dot{\phi}}\mathbf{)_e}$ ,
$\mathbf{(B}\boldsymbol{\phi}\mathbf{)_e}$,
$\mathbf{(C}\boldsymbol{\phi}\mathbf{)_e}$,
$\mathbf{(E}\boldsymbol{\phi}\mathbf{)_e}$,
$\mathbf{D_e}$,
$\mathbf{F_e}$ and 
$\mathbf{Q_e}$. Subscript $\mathbf{_e}$ means that it is the
contribution from a single element. To calculate the matrices the
integration formula from \cite{Segerlind}, equations \ref{eq:areaint}
and \ref{eq:boundaryint} are used. \\
\\
After calculation of the local matrices/vectors, the
assembling\index{assembling} process starts where the contributions
from the local matrices are assembled in the global matrices of
equation \ref{eq:matrix}. 



\subsection{Local matrices and vectors }\index{local
  matrices}\index{local vectors}

The element contributions to equation \ref{eq:matrix} are collected in
the local matrices and vectors. The local matrices and vectors are
simply calculated by inserting the approximations and test functions
on element base into the parts of equation \ref{eq:weak2}. The three
nonzero test functions (basis functions) are placed in a vector,
i.e. $\mathbf{v}=[v_1 \ v_2 \ v_3]^T=[N_1 \ N_2 \
N_3]^T=\mathbf{N}^T$. A given function, $f$ which is linear over an
element, can on an element be written as:


\begin{equation}
f_{e}=\begin{bmatrix} N_{1}\ N_{2}\ N_{3} \end{bmatrix} \begin{bmatrix}
f_{1} \\
f_{2} \\
f_{3} \end{bmatrix}=\mathbf{N}\boldsymbol{f_{e}}
\label{eq:linapprox}
\end{equation}

where $\mathbf{f_{e}}$ is a vector containing the values of $f$ in the
nodal points. In an element the approximation for $\phi$, expressed by
\ref{eq:sum} can be written as: 

\begin{equation}
\phi_{e}=\mathbf{N}\boldsymbol{\phi_{e}}
\end{equation}

where $\boldsymbol{\phi_{e}}$ is a vector containing the
values of $\phi$ in the nodal points (vertices of the triangles). The
time-derivative of $\phi$ can on an element be described analogically. \\
\\
The contribution to $\mathbf{A}$ from an element can be calculated as:


\begin{equation} 
\begin{split}
&\boldsymbol{(A\dot{\psi})_e}=\int_{A}a\frac{\partial \phi}{\partial
  t}\mathbf{N}^T \,dxdy \approx \int_{A} a \mathbf{N}
\boldsymbol{\dot{\phi}_e} \mathbf{N}^T \,dxdy
  =\mathbf{A_e}\boldsymbol{\dot{\phi}_e}, \\
&\mathbf{A_e}=\int_{A} a \begin{tiny}
  \begin{bmatrix}
 N_1^2 & N_1N_2 & N_1N_3 \\
N_2N_1 & N_2^2  & N_2N_3 \\
N_3N_1 & N_3N_2 & N_3^2 \end{bmatrix} \end{tiny} \, dxdy
\end{split}
\label{eq:localA}
\end{equation}

$(\nabla V)\cdot \mathbf{B}\nabla \phi $ appearing in equation
\ref{eq:weak2} can be rewritten as:

\begin{equation}
(\nabla v)\cdot (\mathbf{B}\nabla \phi) =\frac{\partial v}{\partial
  x}\left(b_{xx}\frac{\partial \phi}{\partial x}+b_{xy}\frac{\partial
  \phi}{\partial y}\right)+\frac{\partial v}{\partial
  y}\left(b_{yx}\frac{\partial \phi}{\partial
  x}+b_{yy}\frac{\partial \phi}{\partial y}\right)
\end{equation}

The contribution to $\mathbf{B}\boldsymbol{\phi}$ from a single
element is:

\begin{equation}
\begin{split}
&\boldsymbol{(B\phi)_e}=\int_{A}\frac{\partial \mathbf{N}^T}{\partial
  x}\left(b_{xx}\frac{\partial
  \mathbf{N}}{\partial x}\mathbf{\phi_e}+b_{xy}\frac{\partial
  \mathbf{N}}{\partial y}\mathbf{\phi_e}\right) \, dxdy + \\
& \int_{A} \frac{\partial \mathbf{N}^T}{\partial
  y}\left(b_{yx}\frac{\partial \mathbf{N}}{\partial
  x}\mathbf{\phi_e}+b_{yy}\frac{\partial \mathbf{N}}{\partial
  y}\mathbf{\phi_e}\right)\,dxdy =\\
& \int_{A} b_{xx}\frac{\partial \mathbf{N}^T}{\partial
  x}\frac{\partial \mathbf{N}}{\partial x}+b_{xy}\frac{\partial
  \mathbf{N}^T}{\partial x}\frac{\partial \mathbf{N}}{\partial y}
  \,dxdy \boldsymbol{\phi_e}  + \\ 
& \int_{A} b_{yx} \frac{\partial \mathbf{N}^T}{\partial
  y}\frac{\partial \mathbf{N}}{\partial x}+ b_{yy}\frac{\partial
  \mathbf{N}^T}{\partial y}\frac{\partial \mathbf{N}}{\partial y} \,
  dxdy \boldsymbol{\phi_e}=\mathbf{B_e}\boldsymbol{\phi_e}
\end{split}
\end{equation}

where:

\begin{equation} \begin{split}
& \mathbf{B_e}= \begin{tiny}
\begin{bmatrix} (\frac{\partial
  N_{1}}{\partial x})^{2}
 &\frac{\partial N_{1}}{\partial x}\frac{\partial N_{2}}{\partial x}
&\frac{\partial N_{1}}{\partial x}\frac{\partial N_{3}}{\partial x}\\ 
\frac{\partial N_{2}}{\partial x}\frac{\partial N_{1}}{\partial x}
&(\frac{\partial N_{2}}{\partial x})^{2}
&\frac{\partial N_{2}}{\partial x}\frac{\partial N_{3}}{\partial x}\\ 
\frac{\partial N_{3}}{\partial x}\frac{\partial N_{1}}{\partial x}
&\frac{\partial N_{3}}{\partial x}\frac{\partial N_{2}}{\partial x}  
&(\frac{\partial N_{3}}{\partial x})^{2}
\end{bmatrix} \end{tiny} \int_{A}b_{xx} \, dxdy + \\
&\begin{tiny} \begin{bmatrix} 
\frac{\partial N_{1}}{\partial y}\frac{\partial N_{1}}{\partial x}
&\frac{\partial N_{1}}{\partial y}\frac{\partial N_{2}}{\partial x}
&\frac{\partial N_{1}}{\partial y}\frac{\partial N_{3}}{\partial x}\\ 
\frac{\partial N_{2}}{\partial y}\frac{\partial N_{1}}{\partial x}
&\frac{\partial N_{2}}{\partial y}\frac{\partial N_{2}}{\partial x}
&\frac{\partial N_{2}}{\partial y}\frac{\partial N_{3}}{\partial x}\\ 
\frac{\partial N_{3}}{\partial y}\frac{\partial N_{1}}{\partial x}
&\frac{\partial N_{3}}{\partial y}\frac{\partial N_{2}}{\partial x}  
&\frac{\partial N_{3}}{\partial y}\frac{\partial N_{3}}{\partial x}
\end{bmatrix} \end{tiny} \int_{A}b_{xy} \, dxdy + \\
& \begin{tiny} \begin{bmatrix} 
\frac{\partial N_{1}}{\partial x}\frac{\partial N_{1}}{\partial y}
&\frac{\partial N_{1}}{\partial x}\frac{\partial N_{2}}{\partial y}
&\frac{\partial N_{1}}{\partial x}\frac{\partial N_{3}}{\partial y}\\ 
\frac{\partial N_{2}}{\partial x}\frac{\partial N_{1}}{\partial y}
&\frac{\partial N_{2}}{\partial x}\frac{\partial N_{2}}{\partial y}
&\frac{\partial N_{2}}{\partial x}\frac{\partial N_{3}}{\partial y}\\ 
\frac{\partial N_{3}}{\partial x}\frac{\partial N_{1}}{\partial y}
&\frac{\partial N_{3}}{\partial x}\frac{\partial N_{2}}{\partial y}  
&\frac{\partial N_{3}}{\partial x}\frac{\partial N_{3}}{\partial y}
\end{bmatrix} \end{tiny}  \int_{A}b_{yx} \, dxdy + \\
& \begin{tiny} 
\begin{bmatrix} (\frac{\partial N_{1}}{\partial y})^{2}
 &\frac{\partial N_{1}}{\partial y}\frac{\partial N_{2}}{\partial y}
&\frac{\partial N_{1}}{\partial y}\frac{\partial N_{3}}{\partial y}\\ 
\frac{\partial N_{2}}{\partial y}\frac{\partial N_{1}}{\partial y}
&(\frac{\partial N_{2}}{\partial y})^{2}
&\frac{\partial N_{2}}{\partial y}\frac{\partial N_{3}}{\partial y}\\ 
\frac{\partial N_{3}}{\partial y}\frac{\partial N_{1}}{\partial y}
&\frac{\partial N_{3}}{\partial y}\frac{\partial N_{2}}{\partial y}  
&(\frac{\partial N_{3}}{\partial y})^{2}
\end{bmatrix} \end{tiny} \int_{A}b_{yy} \, dxdy  
 \end{split} 
\label{eq:localB}
\end{equation}

$(\nabla V)\cdot \mathbf{C} \phi $ from equation \ref{eq:weak2} can be
written as:

\begin{equation}
(\nabla v) \cdot (\mathbf{C} \phi) =\frac{\partial v}{\partial
  x} c_{x}\phi  +\frac{\partial v}{\partial
  y} c_{y}\phi 
\end{equation}

$\boldsymbol{(C\phi)_e}$ can then on local element basis be written
as:

\begin{equation}
\begin{split}
& \boldsymbol{(C\phi)_e}=\int_{A} c_x
\frac{\mathbf{N}^T}{\partial x}(\mathbf{N}\boldsymbol{\phi_e})+c_y
\frac{\mathbf{N}^T}{\partial y}(\mathbf{N}\boldsymbol{\phi_e}) \, dxdy
\\ & \int_{A} c_x \frac{\mathbf{N}^T}{\partial
  x}\mathbf{N}+c_y \frac{\mathbf{N}^T}{\partial
  y}\mathbf{N} \, dxdy
\boldsymbol{\phi_e}=\mathbf{C_e}\boldsymbol{\phi_e}
\end{split}
\end{equation}

where:

\begin{equation}
\begin{split} 
& \mathbf{C_e}=\int_{A} c_{x}  \begin{tiny}
\begin{bmatrix}
N_1\frac{\partial N_1}{\partial x} &
N_2\frac{\partial N_1}{\partial x} & 
N_3\frac{\partial N_1}{\partial x} \\
N_1\frac{\partial N_2}{\partial x} &
N_2\frac{\partial N_2}{\partial x} & 
N_3\frac{\partial N_2}{\partial x} \\
N_1\frac{\partial N_3}{\partial x} &
N_2\frac{\partial N_3}{\partial x} & 
N_3\frac{\partial N_3}{\partial x} \end{bmatrix}
\end{tiny} \, dxdy + \\
&\int_{A} c_{y}  \begin{tiny}
\begin{bmatrix}
N_1\frac{\partial N_1}{\partial y} &
N_2\frac{\partial N_1}{\partial y} & 
N_3\frac{\partial N_1}{\partial y} \\
N_1\frac{\partial N_2}{\partial y} &
N_2\frac{\partial N_2}{\partial y} & 
N_3\frac{\partial N_2}{\partial y} \\
N_1\frac{\partial N_3}{\partial y} &
N_2\frac{\partial N_3}{\partial y} & 
N_3\frac{\partial N_3}{\partial y} \end{bmatrix}
\end{tiny} \, dxdy 
\end{split}
\label{eq:localC}
\end{equation}

$(\nabla v) \cdot \mathbf{D}$ can be expressed as:

\begin{equation}
(\nabla v) \cdot \mathbf{D} =\frac{\partial v}{\partial
  x} d_{x}  +\frac{\partial v}{\partial
  y} d_{y} 
\end{equation}

The $\mathbf{D}$-vector is on local element base:

\begin{equation}
\begin{split} 
&\boldsymbol{D_e}=\int_A  \frac{\partial \mathbf{N}^T}{\partial
  x} d_{x}  +\frac{\partial \mathbf{N}^T}{\partial y} d_{y} \, dxdy =
  \\ & \begin{tiny}\begin{bmatrix}\frac{\partial N_{1}}{\partial x}\\
  \frac{\partial N_{2}}{\partial x}\\ \frac{\partial N_{3}}{\partial
    x}\end{bmatrix}\end{tiny}\int_{A} D_x \, dxdy + \begin{tiny}\begin{bmatrix}\frac{\partial N_{1}}{\partial y}\\ \frac{\partial
    N_{2}}{\partial y}\\ \frac{\partial N_{3}}{\partial
    y}\end{bmatrix} \end{tiny} \int_{A} D_y \, dxdy
\end{split}
\label{eq:localD}
\end{equation}

The expression for $\boldsymbol{(E\phi)_e}$ yields:

\begin{equation} \begin{split}
& \boldsymbol{(E\phi)_e} = \int_{A} e \phi \mathbf{N}^T \,dxdy 
\approx \int_{A} e  \mathbf{N} \boldsymbol{\phi_e} \mathbf{N}^T \,
dxdy, = \int_{A} e \mathbf{N}^T  \mathbf{N} \, dxdy
\boldsymbol{\phi_e} = \\
& \mathbf{E_e}\boldsymbol{\phi_e}, \ \ \ \ \ \ \
\boldsymbol{E_e}=\int_{A} e  \begin{tiny}
\begin{bmatrix}
 N_1^2 & N_1N_2 & N_1N_3 \\
N_2N_1 & N_2^2  & N_2N_3 \\
N_3N_1 & N_3N_2 & N_3^2 \end{bmatrix}\end{tiny} \, dxdy
\end{split}
\label{eq:localE}
\end{equation}

The element contribution to the $\mathbf{F}$-vector is:

\begin{equation}
\mathbf{F_{e}}=\int_{A} f \mathbf{N}^T \, dxdy = \int_{A} f
\begin{tiny} \begin{bmatrix} N_1 \\ N_2 \\ N_3 \end{bmatrix}\end{tiny}
\, dxdy
\label{eq:localF}
\end{equation}

For fulfilment of the necessary integration in equations
\ref{eq:localA}, \ref{eq:localB}, \ref{eq:localC}, \ref{eq:localD},
\ref{eq:localE} and \ref{eq:localF}, the analytical
integration formula, equation \ref{eq:areaint} is used. The
integration round the element of boundary flux term gives:

\begin{equation}
\begin{split}
&\mathbf{Q_{e}}=\int_{l}q\mathbf{N}^{T}ds= \int_{l}q  \begin{tiny}
  \begin{bmatrix} N_1 \\ N_2 \\ N_3 \end{bmatrix}\end{tiny} \, ds = \\
& \begin{tiny} \frac{q_{1-2}l_{1-2}}{2}\begin{bmatrix}
    1 \\ 1 \\ 0 \end{bmatrix} +\frac{q_{1-3}l_{1-3}}{2}\begin{bmatrix}
    1 \\ 0 \\ 1 \end{bmatrix} +\frac{q_{2-3}l_{2-3}}{2}\begin{bmatrix}
    0 \\ 1 \\ 1 \end{bmatrix} \end{tiny}
\label{eq:localQ}
\end{split}
\end{equation}

where the flow was approximated to be constant through an element side
The integration shall only be done on element edges with Neumann
boundary conditions. For fulfilling the integration equation
\ref{eq:boundaryint} has been used. $l_{1-2}$, $l_{1-3}$ and $l_{2-3}$
are the side lengths and $q_{1-2}$, $q_{1-3}$ and $q_{2-3}$ are the
fluxes through sides with Neumann boundary conditions. \\
\\
After calculation of all the local element matrices, all the element
contributions are added (assembled) to equation \ref{eq:matrix}. The
task is now to solve the ODE.


\subsubsection{Example}

An example of construction of the $\mathbf{B}$-matrix occurring in
equation \ref{eq:matrix} and originating from the term that includes
$\mathbf{b}$ in the general PDE, equation \ref{eq:general}, is given
here. The geometry, consisting of 2 elements is shown in figure
\ref{fig:assembling}. \\
\\
For constructing $\mathbf{B}$ it is necessary first to calculate the
local $\mathbf{B_e}$-matrices as given by equation
\ref{eq:localB}. For the simplicity the value of $\mathbf{b}$ is 1 in
the whole domain, i.e. an isotropic situation equivalent with:

\begin{equation}
\mathbf{b}=\begin{bmatrix} 1 & 0 \\ 0 & 1 \end{bmatrix}
\end{equation}


\begin{figure}[h]  %here-top-bottom-page
\begin{center}
\epsffile{assembling.eps}
%\epsfig{file=assembling.eps,width=9cm}
\figcap{Geometry used for assembling example. Nodes and elements are
  numbered. Local node numbering in ().}
\label{fig:assembling}
\end{center}
\end{figure}


Inserting the actual $\mathbf{b}$ in equation \ref{eq:localB} followed
by integration gives:

\begin{equation}
\mathbf{B_e}= A \begin{tiny}
\begin{bmatrix} (\frac{\partial
  N_{1}}{\partial x})^{2}+(\frac{\partial N_{1}}{\partial y})^{2}
 &\frac{\partial N_{1}}{\partial x}\frac{\partial N_{2}}{\partial x}+ \frac{\partial N_{1}}{\partial y}\frac{\partial N_{2}}{\partial y}
&\frac{\partial N_{1}}{\partial x}\frac{\partial N_{3}}{\partial x}+
 \frac{\partial N_{1}}{\partial y}\frac{\partial N_{3}}{\partial y} \\ 
\frac{\partial N_{2}}{\partial x}\frac{\partial N_{1}}{\partial
 x}+\frac{\partial N_{2}}{\partial y}\frac{\partial N_{1}}{\partial y}
&(\frac{\partial N_{2}}{\partial x})^{2}+ (\frac{\partial
 N_{2}}{\partial y})^{2} &\frac{\partial N_{2}}{\partial
 x}\frac{\partial N_{3}}{\partial x}+\frac{\partial N_{2}}{\partial
 y}\frac{\partial N_{3}}{\partial y}\\ \frac{\partial N_{3}}{\partial
 x}\frac{\partial N_{1}}{\partial x}+\frac{\partial N_{3}}{\partial
 y}\frac{\partial N_{1}}{\partial y} 
&\frac{\partial N_{3}}{\partial x}\frac{\partial N_{2}}{\partial x}  +
 \frac{\partial N_{3}}{\partial y}\frac{\partial N_{2}}{\partial y}
&(\frac{\partial N_{3}}{\partial x})^{2}+ (\frac{\partial
 N_{3}}{\partial y})^{2}
\end{bmatrix} \end{tiny}
\end{equation}

The area of each of the elements, $A$ is 1. The partial derivatives of
the basis functions are shown in table \ref{tab:partderiv}. 


\begin{table}[H]
\centering
\sbox{\mybox}{\begin{tabular}{p{2cm}|p{0.9cm}|p{0.9cm}|p{0.9cm}|p{0.9cm}|p{0.9cm}|p{0.9cm}}\hline
\textbf{Element} & $\frac{\partial N_1}{\partial x}$ & $\frac{\partial
  N_2}{\partial x}$ & $\frac{\partial N_3}{\partial x}$ &
$\frac{\partial N_1}{\partial y}$ & $\frac{\partial N_2}{\partial y}$
& $\frac{\partial N_3}{\partial y}$ \\ \hline
1 & -0.5 & 0.5 & 0 & 0 & -1 & 1 \\ \hline
2 & 0 & 0.5 & -0.5 & -1 & 0 & 1 \\ \hline 
\end{tabular}}
\settowidth{\mylength}{\usebox{\mybox}}
\setcaptionwidth{\mylength}
\tabcap{Partial derivatives of the basis functions}
\label{tab:partderiv}
\usebox{\mybox}
\end{table}

The local matrices are then:

\begin{equation}
\mathbf{B_1}=
\begin{bmatrix} 0.25 & -0.25 & 0 \\ 
-0.25 & 1.25 & -1 \\
0 & -1 & 1 
\end{bmatrix}, \quad 
\mathbf{B_2}= 
\begin{bmatrix}
1 & 0 & -1 \\
0 & 0.25 & -0.25 \\
-1 & -0.25 & 1.25 
\end{bmatrix}
\end{equation}

The global $\mathbf{B}$-matrix is a 4-by-4 matric since the mesh have
4 nodal points. The connectivity between the elements retrieved from figure
\ref{fig:assembling} is summarized in table \ref{tab:connectivity}.

\begin{table}[H]
\centering 
\sbox{\mybox}{\begin{tabular}{p{2.0cm}|p{1.6cm}|p{1.6cm}|p{1.6cm}} \hline
\textbf{Element} & local 1 & local 2 & local 3 \\ \hline
1 & 1 & 2 & 4 \\ \hline 
2 & 1 & 4 & 3 \\ \hline 
\end{tabular}}
\settowidth{\mylength}{\usebox{\mybox}}
\setcaptionwidth{\mylength}
\tabcap{Global node numbering}
\label{tab:connectivity}
\usebox{\mybox}
\end{table}

Using the table it is possible to calculate the contribution to
$\mathbf{B}$ from element 1 and 2:


\begin{equation}
\begin{bmatrix}
0.25 & -0.25 & 0 & 0 \\
-0.25 & 1.25 & 0 & -1 \\
0 & 0 & 0 & 0 \\
0 & -1 & 0 & 1 
\end{bmatrix} \quad \text{and} \quad
\begin{bmatrix}
1 & 0 & -1 & 0 \\
0 & 0 & 0 & 0 \\
-1 & 0 & 1.25 & -0.25 \\
0 & 0 & -0.25 & 0.25 
\end{bmatrix}
\end{equation}

The $\mathbf{B}$ is the sum of the contributions from each element:

\begin{equation}
\mathbf{B}=
\begin{bmatrix}
1.25 & -0.25 & -1 & 0 \\
-0.25 & 1.25 & 0 & -1 \\
-1 & 0 & 1.25 & -0.25 \\
0 & -1 & -0.25 & 1.25
\end{bmatrix}
\end{equation}

Node 1 and 4 have contributions from both elements, whereas the
remaining elements have contribution from only one element.

\section{Time stepping procedure}\index{time stepping}

There are several more or less sophisticated methods to solve
ODEs\index{ODE}. Methods as Euler and the trapezoidal rule are widely
used. Both the trapezoidal rule\index{trapezoidal rule} and the Euler
method are included in the $\theta$-method as described by
\cite{Iserles}. Instead of using $\theta$ as the parameter, $\omega$
is here used to prevent confusing misunderstandings with the water
content. If the \textit{initial value problem} (IVP)\index{IVP} can be
expressed as:


\begin{equation}
\boldsymbol{\dot{\phi}}=\mathbf{f}(t,\boldsymbol{\phi}),\ \ \ \ \ \ t
\geq t_{0}, \ \ \ \ \ \boldsymbol{\phi}(t_{0})=\boldsymbol{\phi}_{0}
\label{eq:IVP}
\end{equation}

where $\boldsymbol{\phi}$ is a vector and
$\mathbf{f}(t,\boldsymbol{\phi})$ is a vector function, a numerical
procedure to solve the IVP can be written as: 

\begin{equation} \begin{split}
\boldsymbol{\phi}(t_{n+1}) &=\boldsymbol{\phi}(t_{n})+ \omega \Delta t
\mathbf{f}(t_{n},\boldsymbol{\phi}_{n})+(1-\omega) \Delta t\mathbf{f}(t_{n+1}, 
\boldsymbol{\phi}(t_{n+1}))  
 \\ &+(\omega-\frac{1}{2})\Delta
t^2\boldsymbol{\phi}''(t_{n})+(\frac{1}{2}\omega-\frac{1}{3}) \Delta
t^3\boldsymbol{\phi}'''(t_{n})+O(\Delta t^4) \\
& \approx \boldsymbol{\phi}(t_{n})+ \omega \Delta t
\mathbf{f}(t_{n},\boldsymbol{\phi}_{n})+(1-\omega) \Delta t\mathbf{f}(t_{n+1}, 
\boldsymbol{\phi}(t_{n+1})) 
\label{eq:diffsol}
\end{split} \end{equation}

where $n$ and $n+1$ are numbers of the time levels and $\Delta t$ is
the length of the timestep. The timeweighting parameter\footnote{Many
  places, for instance \cite{Quarteroni} the timeweighting parameter,
  $\omega$ is associated with timestep $n+1$ and $(\omega-1)$ with
  timestep $n$, i.e the reverse.}, $\omega$ is restricted to the interval,
$0 \leq \omega \leq 1$. $\omega$ decides how the weighting in time
shall be for $\mathbf{f}$. For $\omega=\frac{1}{2}$ it can be seen
that the method is of order 2. For other values it is order 1. For
$\omega=\frac{2}{3}$ we see the $O(\Delta t^3)$ term in
\ref{eq:diffsol} vanishes while the $O(\Delta t^2)$ remains. In very
special cases this can be an advantage, \cite{Iserles}. The method is
according to definition in \cite{Quarteroni} a \textit{one-step
method}\index{one-step method}\index{method!one-step}. For
$\omega=1$ is it \textit{explicit}\index{explicit
  method}\index{method!explicit}, oherwise method is
\textit{implicit}\index{implicit method}\index{method!implicit}.



The method has
names for some special values of $\omega$: 

\begin{itemize}
\item $\omega=0$ \textit{backward difference}\index{backward
    difference} or \textit{backward Euler}\index{backward Euler}
\item $\omega=\frac{1}{2}$ \textit{central difference}\index{central
    difference}, \textit{trapezoidal rule}\index{trapezoidal rule} or
  \textit{Crank-Nicolson}\index{Crank-Nicolson}
\item $\omega=1$ \textit{forward difference}\index{forward difference}
  or \textit{forward Euler}\index{forward Euler}
\end{itemize}

The backward Euler method is widely used in models of unsaturated
flow. For the advection-dispersion equation the Crank-Nicolson 
scheme is often used.\\
\\

The actual IVP can by applying equation \ref{eq:matrix} be written as:

\begin{equation} 
\boldsymbol{\dot{\phi}}=\mathbf{A^{-1}}(\mathbf{G}-\mathbf{H}\boldsymbol{\phi})
, \ \ \ \ \ t\geq t_0, \ \ \ \ \
\boldsymbol{\phi}(t_0)=\boldsymbol{\phi}_0
\label{eq:IVPdef}
\end{equation}

where $\mathbf{G}=-(\mathbf{D}+\mathbf{F}+\mathbf{Q})$ and
$\mathbf{H}=\mathbf{B}-\mathbf{C}+\mathbf{E}$.  It is very important
to note that $\mathbf{A}$ shall be regular. How this is fulfilled
will be described later. By applying \ref{eq:diffsol} we get:

\begin{equation} \begin{split}
\boldsymbol{\phi}_{n+1}=& \boldsymbol{\phi}_{n}+ \omega \Delta t
\mathbf{A}_{n}^{-1}(\mathbf{G}_{n}-\mathbf{H}_{n}\boldsymbol{\phi}_n)
+\\ &
(1-\omega) \Delta t
\mathbf{A}_{n+1}^{-1}(\mathbf{G}_{n+1}-\mathbf{H}_{n+1}\boldsymbol{\phi}_{n+1})
\label{eq:gentheta}
\end{split} \end{equation}

Equation \ref{eq:gentheta} can be written so we get the unknown,
$\boldsymbol{\phi}_{n+1}$ on the left hand side:

\begin{equation} \begin{split} 
(\mathbf{A}_{n+1} + &\Delta t(1-\omega)\mathbf{H}_{n+1})\boldsymbol{\phi}_{n+1} =\mathbf{A}_{n+1}\boldsymbol{\phi}_{n} +\\ &\Delta t \omega
\mathbf{A}_{n+1}\mathbf{A}^{-1}_{n}(\mathbf{G}_{n}-\mathbf{H}_{n}\boldsymbol{\phi}_{n}) + \\  &\Delta
t(1-\omega)\mathbf{G}_{n+1}
\end{split} \end{equation}

But both $\mathbf{A}$, $\mathbf{H}$ and $\mathbf{G}$ are 
  functions of $\boldsymbol{\phi}$. For solving the equations, the
  \textit{Picard iterations}\index{Picard iterations} can be used. The unknown
  $\boldsymbol{\phi}_{n+1}$ is estimated by using the latest estimate
  of $\mathbf{A}_{n+1}$, $\mathbf{G}_{n+1}$ and
  $\mathbf{H}_{n+1}$. The iteration scheme can be written as:


\begin{equation} \begin{split}
(\mathbf{A}_{n+1,m}+&\Delta
t(1-\omega)\mathbf{H}_{n+1,m})\boldsymbol{\phi}_{n+1,m+1} = \\
&\mathbf{A}_{n+1,m}\boldsymbol{\phi}_{n}+\Delta t \omega
\mathbf{A}_{n+1}(\mathbf{A}_{n})^{-1}(\mathbf{G}_{n}-\mathbf{H}_{n}\boldsymbol{\phi}_{n})+
\\ &\Delta t(1-\omega)\mathbf{G}_{n+1,m}  
\end{split}
\label{eq:Picard}
\end{equation}

where $m$ and $m+1$ denote the iteration levels. The iteration
procedure stops when a chosen norm, for example the $\infty$-norm of
the change in $\boldsymbol{\phi}$ between to iterations is below a
certain chosen value, $\epsilon$.\\
\\
It can bee seen that many advantages are obtained by choosing $\omega$
to be $0$ because no time is spent on the rather time consuming
process to calculate $\mathbf{A}^{-1}$.  But in the general case where
the $\mathbf{A}$ has to be inverted, it can save a lot of time if
$\mathbf{A}$ is a diagonal, i.e. if it is lumped. The choice of
$\omega=0$ is also for stability reasons a good choice,
\cite{Iserles}. Later it is discussed how the size of $\Delta t$ is
controlled.

 


\subsection{Lumping}\index{lumping}

The former expressed form of the local $\mathbf{A}$-matrix is called
the \textit{consistent}\index{consistent form} or
\textit{distributed}\index{distributed} form. The matrix is not a diagonal matrix since the local matrices are non-diagonal. Except for
$\omega=0$ the matrix has to be inverted for each timestep. The
invertion\index{matrix!invertion} is a quite time consuming
task. The invertion could be much faster if $A$ was a diagonal matrix
- often called a \textit{lumped} matrix. The lumped matrices are often
produced by using different basis functions for the derivative
compared to the other terms in the weak formulation - i.e there is not
consistence. \\
\\
Also in structural mechanics\index{structural mechanics} it is normal
to use lumped mass matrices, see for example \cite{Bang,Bathe}. Here the unknown is often the displacement and the
differential equation of second order according to Newton�s second
law. The term before the second order time derivative is then the
mass. The physical explanation of a lumped mass matrix is that all the
mass is placed in the nodal points. The similar physical explanation
in for example the water movement equation is that all water capacity
is placed in the nodal points, which may not seem to be physically
reasonable at first. \\
\\
\cite{Segerlind} has theoretically examined the (linear) heat
flow\index{heat flow equation} equation (which is not so different
from Richards equation) and came to the conclusion that lumping also
prevents oscillations. \cite{Celia} came to the conclusion that a
lumped capacity matrix ($\mathbf{A}$) can prevent numerical
oscillations\index{numerical oscillations} and slow
convergence\index{convergence} for the solution of Richard's
equation. Both \cite{Neuman} and \cite{SWMS} use a lumped formulation
of the time derivative in the water movement models. \cite{SWMS} also
use lumped formulation of the time derivative in the
advection-dispersion equation. The conclusions by \cite{Wood} is
somewhat different. \cite{Wood} examined the flow in a hill slope with
rainfall and recommend the use of a distributed matrix instead of a
lumped one. The discharge predictions were more precise and the
calculation times only slightly higher.\\
\\
TopFEM makes it up to the user to decide whether the consistent or a
lumped formulation shall be used for the time derivatives.\\
\\
There are different ways to construct the lumped matrix. But the basic 
idea is as explained before, that the $\mathbf{A}$ matrix and then also
the  $\mathbf{A_e}$-matrices are diagonal matrices. \cite{Segerlind}
divides each element into three equally large areas as shown in figure
\ref{fig:Segerlindlumping}. The local element basis functions are
defined so that $N_{i}^{*}=1$ for $(x,y) \in A_i$ and $N_{i}^{*}=0$
for $(x,y)\notin A_i$. This leads to:

\begin{equation}
\begin{split}
&\boldsymbol{(A\dot{\phi})_e}=\int_{A} a
\boldsymbol{\dot{\phi}_e}\mathbf{N^*}^{T}dA=\mathbf{A^*_e}\boldsymbol{\dot{\phi}_{e}}, \\
& \mathbf{A^{*}_{e}}=\int_A a 
\begin{tiny} \begin{bmatrix}
(N_1^*)^2 & 0 & 0 \\
0 & (N_2^*)^2 & 0 \\ 
0 & 0 & (N_3^*)^2 
\end{bmatrix} \end{tiny} \,dxdy
\end{split}
\end{equation}

where $\mathbf{N^*}=[N_1^* \ N_2^*\ N_3^*]$.

\begin{figure}[h]  %here-top-bottom-page
\begin{center}
\begin{picture}(160,120)(-20,-20)
\thicklines
\put(0,0) {\line(1,0){120}}
\put(120,0){\line(-2,3){60}}
\put(60,90){\line(-2,-3){60}}
\thinlines
\put(60,0){\line(0,1){30}}
\put(30,45){\line(2,-1){30}}
\put(90,45){\line(-2,-1){30}}
\put(0,0){\circle*{3}}
\put(120,0){\circle*{3}}
\put(60,90){\circle*{3}}
\put(-15,-15){P1}
\put(130,-10){P2}
\put(60,95){P3}
\put(20,15){A1}
\put(75,15){A2}
\put(60,50){A3}
\end{picture}
\figcap{Division of an element into three equally large areas for
  definition of $N^*$ used by \cite{Segerlind}.}
\label{fig:Segerlindlumping}
\end{center}
\end{figure}

A second way to diagonalize $\mathbf{A}$ is to do as
\cite{Neuman,SWMS}\index{Neuman}\index{SWMS2D@{SWMS\_2D}} where the
time derivative of $\phi$ in nodal point $n$, $\partial \phi_n
/\partial t$ is defined as a weighted average of $\partial \phi/
\partial t$ over the flow region (or simply just where the test
function is nonzero):



\begin{equation}
\frac{\partial \phi_n}{\partial t} \equiv \frac{\int_{ \Omega}a
  \frac{\partial \phi}{\partial t}v_n \, d\Omega} {\int_{
  \Omega}av_{n}\, d\Omega} \Leftrightarrow \int_{ \Omega}a
  \frac{\partial \phi}{\partial t}v_n \, d\Omega=\frac{\partial
  \phi_n}{\partial t}\int_{\Omega}av_{n}\, d\Omega
\end{equation}

This is equivalent to moving the off-diagonal of the consistent
matrix onto the diagonal. This can on local element basis be written
as:

\begin{equation}
\begin{split}
&\boldsymbol{(A\dot{\phi})_e}=\int_{A}a \begin{tiny}
\begin{bmatrix}
  N_1 & 0 & 0 \\ 
0 & N_2 & 0 \\
0 & 0 & N_3 \end{bmatrix}\end{tiny} \,dxdy \begin{tiny} \begin{bmatrix} \dot{\phi}_1 \\ \dot{\phi}_2 \\
\dot{\phi}_3 \end{bmatrix}\end{tiny}
=\mathbf{A_{e}}\boldsymbol{\dot{\phi}}_{e},  \\
& \mathbf{A_e}=\int_{A}a \begin{tiny} \begin{bmatrix}
  N_1 & 0 & 0 \\ 
0 & N_2 & 0 \\
0 & 0 & N_3 \end{bmatrix}\end{tiny}\,dxdy
\end{split}
\end{equation}

\cite{Istok} uses the following interpolation functions for the time
derivative of $\phi$ at each node and for the weighting
function:

\begin{equation}
 N_{i}^{*}N_{j}^{*}=\begin{cases} \frac{1}{3} & \text{if
        $i=j$}\\
0 & \text{otherwise} \end{cases}
\end{equation}

On local element basis it is:


\begin{equation}\begin{split}
& \boldsymbol{(A\dot{\phi})_e}=\int_{A} a
\phi_{e}\mathbf{N^*}^{T} \, dA=\mathbf{A_e} \boldsymbol{\dot{\phi}_e}\\ 
& \mathbf{A_e}=\frac{1}{3}\int_{A} a \begin{tiny}\begin{bmatrix} 1 &
    0 &  0 \\ 0 & 1 & 0 \\ 0 & 0 & 1 \end{bmatrix}\end{tiny} \,dxdy
\end{split} \end{equation}

We can see that generally the different lumping techniques lead to
different numerical schemes. The following question is then: Does the
different schemes have significant influence of the obtained solution
and rate of convergence?  


\section{Dirichlet boundary conditions}\index{boundary
  condition!Dirichlet}


Both in the timestepping iterations for the general PDE according to
 equation \ref{eq:Picard} and for the mixed formulation used for water
 movement simulations, the iteration from step $m$ to $m+1$ forms a
 matrix equation of the form: 

\begin{equation}
\mathbf{A}_m\boldsymbol{\phi}_{m+1}=\mathbf{b}_{m}
\label{eq:Axb}
\end{equation}

Not to be confused with $\mathbf{A}$ used earlier in for example in
equation \ref{eq:matrix}. The Dirichlet boundary conditions are directly
implemented in equation \ref{eq:Axb}. If node no. $i$ is placed at a
Dirichlet boundary, the value of element $a(i,i)$ is set to 1 and the
values of $a(i,j), \ j=1,2 \dots NP,  j \neq i$ are replaced
with zeros. $b(i)$ is set to the specified $\phi$ (pressure
or concentration). \\ 
\\
$\mathbf{A}$ from water movement equations is symmetric (see later), i.e
$\mathbf{A}=\mathbf{A}^T$. Solution of symmetric matrix equations is
often cheaper (less CPU-costs) than the solution of non-symmetric
matrices. To keep a symmetric $\mathbf{A}$ matrix symmetric after
implementation of the Dirichlet boundary conditions, some extra row
operations are done. Columns $i$ are replaced with zeros except at the
diagonal, by the row operation: $Row \ j = Row\ j -a(j,i) *Row\ i$ for
$j=1\dots NP, \ j\neq i$. The here described operations have to be
done for every Dirichlet node.  \\ 
\\
Figure \ref{fig:sparsea} is an example of the $\mathbf{A}$-matrix from
water movement simulations before implementation of the Dirichlet
boundary conditions. Figure \ref{fig:sparsechange} shows the elements in
$\mathbf{A}$, which after implementation of a (little) boundary with
Dirichlet conditions have been changed to zero. \\
\\

\begin{figure}[h]  %htbp=here-top-bottom-page
\begin{center}
\epsffile{sparsea.eps}
%\epsfig{file=sparsea.eps,width=9cm}
\figcap{Example of the $\mathbf{A}$-matrix before implementation of
  the Dirichlet boundary conditions. The 3443 dots are the nonzero
  elements in $\mathbf{A}$. $\mathbf{A}$ is symmetric.}
\label{fig:sparsea}
\end{center}
\end{figure}

\begin{figure}[h]  %htbp=here-top-bottom-page
\begin{center}
\epsffile{sparsechange.eps}
%\epsfig{file=sparsechange.eps,width=9cm}
\figcap{Example of the change in the $\mathbf{A}$-matrix after
  implementation of the Dirichlet boundary conditions. The 84 dots are
  elements which have changed to zero in $\mathbf{A}$ after
  implementation of the Dirichlet boundary conditions. The matrix is
  symmetric so $\mathbf{A}$ is still symmetric.}
\label{fig:sparsechange}
\end{center}
\end{figure}

If $\mathbf{A}_{m,dir}$ and $\mathbf{b}_{m,dir}$ are the $\mathbf{A}_m$
and $\mathbf{b}_m$-matrices after implementation of the Dirichlet
boundary conditions, the new equation which have to be solved is: 

\begin{equation}
\mathbf{A}_{m,dir}\boldsymbol{\phi}_{m+1}=\mathbf{b}_{m,dir}
\label{eq:Axbdir}
\end{equation}

The method described here is relatively easy to implement. But another
method to implement the Dirichlet boundary conditions is using
\textit{Lagrange multipliers} as for example in FEMLAB,
\cite{FEMLAB}. \\ 
\\
After implementation of the Dirichlet boundary conditions, the natural
subsequent question for the hydrologist is, what is the size of the
flow through these boundaries?


\subsection{Boundary fluxes}\index{boundary flux}

During the simulations, the flow through the boundaries with
Dirichlet boundaries are evaluated by calculating backwards after the
implementation of the Dirichlet boundary conditions to see what
Neumann boundaries it will correspond to. \\
\\
The procedure is simply to calculate a new right side using
$\mathbf{A}$ before the implementation and the solution,
$\boldsymbol{\phi}$ obtained after implementation of the Dirichlet
boundary conditions. The difference between the new right side and the
old one is simply due to the changes from the Dirichlet boundary
conditions:

\begin{equation}
\Delta \mathbf{b}=\mathbf{b}_{m}-\mathbf{A}_m\boldsymbol{\phi_{m+1}}
\end{equation}

The change in $\mathbf{b}$ is because the Dirichlet boundaries now are
expressed as Neumann boundaries. Using equation \ref{eq:Picard} it can
be realized that:

\begin{equation}
\mathbf{Q_{dir}}=-\frac{1}{\Delta t}\Delta \mathbf{b}
\end{equation}

where $\mathbf{Q_{dir}}$ is a vector similar to $\mathbf{Q}$ - but here
  due the flow out of Dirichlet boundaries. The task is now simply to
  calculate backwards to evaluate the flux through the element sides,
  according to equation \ref{eq:localQ}. It is of course also possible to
  calculate boundary fluxes using the solution and the equation for
  the flux (for example Darcys equation to determine the
  water flow). Tests have shown that this method gives occasion to
  rather bad mass balances, even if the solution is good. It is in
  general a hard job to evaluate the fluxes at the nodal points (or
  along element edges) as the derivative of the approximation to the
  solution has jump-discontinuities here. In, the Solute movement
  section, it is described how water fluxes are evaluated at nodal
  points, but the information is not used for evaluating the boundary
  fluxes.




\section{Matrix solution technique}\index{matrix!solution}

When solving the ODE defined by equation \ref{eq:matrix} it is when
$\omega\neq 0$ also necessary to use the inverse of a matrix, for that
purpose the built-in function \textsf{inv} is used. The computational
costs for a non-diagonal matrix is very high so the lumped (diagonal)
versions can be used with advantage. \\
\\
For solving the large matrix system of the type
$\mathbf{Ax}=\mathbf{b}$, the MATLAB backslash
operator\index{MATLAB!backslash operator} (also called
leftdivision) is used. By using the backslash operator, MATLAB makes
some tests and finds an appropriate direct method for solving the
equation. MATLAB (and then also TopFEM) gives warning messages for
badly scaled (or singular) matrices (where the solution maybe have
large errors). It is an (often made) mistake to use \textsf{inv} to
solve the system of equations by
$\mathbf{x}=\mathbf{A}^{-1}\mathbf{b}$, the computational costs are 2-3
times larger than using the backslash operator and the accuracy is much
smaller according to MATLAB. \\ 
\\
The calculation costs of physical entities, such as water capacity
(water movement simulations) and water fluxes (solute movement
simulations) all more or less proportional with the number of nodal
points, $\text{NP}$, whereas the solution of the matrix is strongly
dependent on the dimension of $\mathbf{A}$. When using the backslash
operator (also denoted leftdivision) MATLAB uses a direct method for
solving the linear equations. For large systems, the number of
floating point operations in Gauss elimination are proportional to
$\text{NP}^3$. The actual number of floating point operations are
probably lower since $\mathbf{A}$ has low density and MATLAB has
special algorithms for sparse matrices.\\
\\
If $\mathbf{A}$ is symmetric, MATLAB attempts to use a Cholesky
factorization\index{Cholesky factorization} of $\mathbf{A}$:

\begin{equation}
\mathbf{A}=\mathbf{L}^T\mathbf{L}
\end{equation}

Where $\mathbf{L}$ is a lower triangular matrix. The solution is
simply first to solve $\mathbf{L}^T\mathbf{z}=\mathbf{b}$ and then solve
$\mathbf{L}\mathbf{x}=\mathbf{z}$. The Cholesky factorization can be
used with success if $\mathbf{A}$ is positive definite. The Cholesky
factorization requires according to MATLAB less than half the
computational time of a general factorization. Similar results are
given in \cite{Atkinson,Numerisk}. If $\mathbf{A}$ is not
symmetric or the Cholesky factorization fails, other factorization
methods as for example LU-factorization\index{LU factorization} with
pivoting (a kind of Gauss elimination) are used:


\begin{equation}
\mathbf{A}=\mathbf{L}\mathbf{U}
\end{equation}

where $\mathbf{L}$ is a lower triangular matrix and $\mathbf{U}$ is an 
upper triangular matrix. The solution method is first to solve
$\mathbf{L}\mathbf{z}=\mathbf{b}$ and
$\mathbf{U}\mathbf{x}=\mathbf{z}$ by using forward and backward
substitution. \\
\\
For larger especially sparse systems it is often an advantage to use
iterative methods. Jacobi\index{Jacobi} and
Gauss-Seidel\index{Gauss-Seidel} and SOR\index{SOR} (Successive Over
Relaxation) are often used. MATLAB has build-in solvers for GMRES and
PCG (only for symmetric matrices). All the here mentioned iterative
methods are discussed in \cite{DCAM}.


\section{Water movement}

In this section special features of the solution of Richard's
equation are discussed. One of these is the mixed formulation where
the solution strategy is different from the general one described by
equation \ref{eq:matrix}. The local matrices and the assumptions for
calculating them are also explained.



\subsection{Mixed formulation}\index{mixed
  formulation}\index{formulation!mixed}


The $\psi$-based formulation\index{pressure
  formulation}\index{formulation!pressure} has the disadvantage that
  it does not conserve the mass and can give erroneous estimates of
  infiltration depths, \cite{Celia}. In the mixed
  formulation the time derivative part of Richard�s equation is
  approximated in another way as in the $\psi$-based
  formulation. Picard iterations are still applied. In the mixed
  formulation proposed by \cite{Celia}, the water content is at time
  step $n+1$ and iteration step $m+1$ approximated by:

\begin{equation}\begin{split}
\theta_{n+1,m+1}&=\theta_{n+1,m}+\frac{d\theta}{d\psi}\mid_{n+1,m}(\psi_{n+1,m+
  1}-\psi_{n+1,m})\\
&=\theta_{n+1,m} +C_{w,n+1,m}(\psi_{n+1,m+1}-\psi_{n+1,m}) 
\label{eq:taylor}
\end{split}
\end{equation}

This is simply a Taylor expansion. The time derivative of $\theta$ can
then be approximated as:

\begin{equation}\begin{split}
\frac{\partial \theta}{\partial t}&\approx
\frac{\theta_{n+1,m+1}-\theta_{n}}{\Delta
  t}=\frac{\theta_{n+1,m+1}-\theta_{n+1,m}}{\Delta
  t}+\frac{\theta_{n+1,m}-\theta_{n}}{\Delta t}\\ & \approx C_{w,n+1,m}
\frac{\psi_{n+1,m+1}-\psi_{n+1,m}}{\Delta
  t}+\frac{\theta_{n+1,m}-\theta_{n}}{\Delta t}
\end{split}\end{equation}

The Taylor expansion cancels the unknown water content at step $m+1$
and replaces it with the pressure. So the terms with $\theta$ are at
previous iteration steps, i.e they are known. \\
\\
The time derivative term in the weak formulation, equation
\ref{eq:weak2} can on local element basis now be evaluated as: 

\begin{equation}
\begin{split}
& \left(\int_A \frac{\partial \theta}{\partial t}
  \mathbf{N}^T\,dxdy\right)_{\mathbf{e}} \approx  \frac{1}{\Delta t} \int_A
(\theta_{n+1,m}-\theta_n)\mathbf{N}^T \,dxdy
 + \\ & \frac{1}{\Delta t} \int_A C_{w,n+1,m}
  \mathbf{N}(\boldsymbol{\psi_e}_{n+1,m+1}-\boldsymbol{\psi_e}_{n+1,m})\mathbf{N}^T\,dxdy = \\ &  \frac{1}{\Delta
  t}(\mathbf{W_e}_{n+1,m}-\mathbf{W_e}_n)+ \frac{1}{\Delta t}\mathbf{A_e}_{n+1,m}(\boldsymbol{\psi_e}_{n+1,m+1}-\boldsymbol{\psi_e}_{n+1,m})
\end{split}
\label{eq:mixedmass}
\end{equation}

where:

\begin{equation}
\mathbf{W_e}=\int_A \theta \mathbf{N}^T \, dxdy = \int_A \theta
\begin{tiny} \begin{bmatrix} N_1 \\ N_2 \\ N_3 \end{bmatrix}
\end{tiny} \, dxdy
\end{equation}

Instead of using the consistent formulation, both $\mathbf{A}$ and
$\mathbf{W}$ can be lumped. Different lumping procedures for $\mathbf{A}$
have been described in the previous chapter. Similar ones can be used
for $\mathbf{W}$. \\
\\
The local terms $\mathbf{W_e}$ are assembled to the vector
$\mathbf{W}$. Almost the same numerical scheme as for solving
\ref{eq:IVPdef} can be used:

\begin{equation}
\frac{1}{\Delta
  t}\mathbf{A}_{n+1,m}(\boldsymbol{\psi}_{n+1,m+1}-\boldsymbol{\psi}_{n+1,m})+\frac{1}{\Delta
  t}(\mathbf{W}_{n+1,m}-\mathbf{W}_n)
  =\mathbf{G}-\mathbf{H}\boldsymbol{\psi} 
\end{equation}

$\mathbf{G}$ and $\boldsymbol{H\psi}$ kan still be weighted between
timestep $n$ and $n+1$. The unknown $\boldsymbol{\psi}_{n+1}$ and
$\boldsymbol{\theta}$ are estimated using the latest estimate of
$\mathbf{A}_{n+1}$, $\mathbf{G}_{n+1}$ and $\mathbf{H}_{n+1}$:

\begin{equation}
\begin{split}
(\mathbf{A}_{n+1,m}&+\Delta
t(1-\omega)\mathbf{H}_{n+1,m})\boldsymbol{\psi}_{n+1,m+1}\\ &=
\omega \Delta t(\mathbf{G}_{n}-\mathbf{H}_{n}\boldsymbol{\psi}_{n}) +
(1-\omega)\Delta t \mathbf{G}_{n+1,m} \\ &+\mathbf{W}_n -\mathbf{W}_{n+1,m+1} + \mathbf{A}_{n+1,m}\boldsymbol{\psi}_{n+1,m}
\end{split}
\label{eq:Picard_mixed}
\end{equation}

It is worth to mention that $\mathbf{A}$ not have to be inverted in
this model, independent of the chosen $\omega$. 


\subsection{Local Matrices}\index{local matrices}

Both the hydraulic conductivity\index{hydraulic conductivity}, the
specific water capacity\index{specific water capacity} function
and for the mixed formulation also the water content\index{water
  content} are approximated by linear functions in each element, i.e.:


\begin{equation}
\begin{split}
& C_w(\psi)_e \approx \mathbf{N}\mathbf{C_{w,e}} \\
& K(\psi)_e \approx \mathbf{N}\mathbf{K_e} \\
& \theta(\psi)_e \approx \mathbf{N}\boldsymbol{\theta_e} 
\end{split}
\label{eq:approxwat}
\end{equation}

It is worth to mention that $C_w$, $K$ and $\theta$ in contrast to
$\psi$ are not generally assumed to have continuity. Possible 
discontinuities exist between subdomains with different soil
properties. The subdomain borders are here always coincident with
element edges. \\
\\
The local element matrices and vectors are calculated by using
equations \ref{eq:areaint}, \ref{eq:boundaryint}, and the approximation
by equation \ref{eq:approxwat} applied in equations \ref{eq:localA} to
\ref{eq:localQ}. The results are given in table
\ref{tab:localmatwat}. $\bar{K}_e$ means the mean value of the hydraulic
conductivity in the element, i.e. $\frac{1}{3}(K_1+K_2+K_3)$. \\
\\
 
\begin{table}[h]
\tabcap{Local matrices, water movement}
\begin{tabular}{p{1.5cm}|p{11.4cm}} \hline
\textbf{Matrix} & \textbf{Value} \\ \hline
$\mathbf{A_e}$ & $\begin{tiny}
\frac{A}{60}\begin{bmatrix}
6C_{w,1}+2C_{w,2}+2C_{w,3} & 2C_{w,1}+2C_{w,2}+C_{w,3}  &
2C_{w,1}+C_{w,2}+2C_{w,3} \\
2C_{w,1}+2C_{w,2}+C_{w,3}  & 2C_{w,1}+6C_{w,2}+2C_{w,3} &
C_{w,1}+2C_{w,2}+2C_{w,3} \\ 
2C_{w,1}+C_{w,2}+2C_{w,3}  & C_{w,1}+2C_{w,2}+2C_{w,3}  &
2C_{w,1}+2C_{w,2}+6C_{w,3} \end{bmatrix} \end{tiny}$ \\ \hline
$\mathbf{B_e}$ & $\begin{tiny} A \bar{K}_{e} \begin{bmatrix} (\frac{\partial
  N_{1}}{\partial  x})^{2}+(\frac{\partial N_{1}}{\partial y})^{2}
  &\frac{\partial  N_{1}}{\partial x}\frac{\partial N_{2}}{\partial
  x}+ \frac{\partial N_{1}}{\partial y}\frac{\partial N_{2}}{\partial
  y} &\frac{\partial N_{1}}{\partial x}\frac{\partial N_{3}}{\partial
  x}+ \frac{\partial N_{1}}{\partial y}\frac{\partial N_{3}}{\partial
  y}\\  \frac{\partial N_{2}}{\partial x}\frac{\partial
  N_{1}}{\partial x}+ \frac{\partial N_{2}}{\partial y}\frac{\partial
  N_{1}}{\partial y}  &(\frac{\partial N_{2}}{\partial
  x})^{2}+(\frac{\partial N_{2}}{\partial y})^{2}  &\frac{\partial
  N_{2}}{\partial x}\frac{\partial N_{3}}{\partial x}+  \frac{\partial
  N_{2}}{\partial y}\frac{\partial N_{3}}{\partial y}\\
  \frac{\partial N_{3}}{\partial x}\frac{\partial N_{1}}{\partial x}+
  \frac{\partial N_{3}}{\partial y}\frac{\partial N_{1}}{\partial y}
  &\frac{\partial N_{3}}{\partial x}\frac{\partial N_{2}}{\partial x}+
  \frac{\partial N_{3}}{\partial y}\frac{\partial N_{2}}{\partial y}
  &(\frac{\partial N_{3}}{\partial x})^{3}+(\frac{\partial 
N_{3}}{\partial y})^{3} \end{bmatrix} \end{tiny}$  \\ \hline 
$\mathbf{C_e}$ & $\begin{tiny}\begin{bmatrix} 0 & 0 \\ 0 & 0
  \end{bmatrix}\end{tiny}$ \\ \hline
$\mathbf{D_e}$ & $A\bar{K}_{e}\begin{tiny}\begin{bmatrix}\frac{\partial
 N_{1}}{\partial y}\\ \frac{\partial N_{2}}{\partial y}\\
  \frac{\partial N_{3}}{\partial y}\end{bmatrix}\end{tiny}$  \\ \hline 
$\mathbf{E_e}$ & $\begin{tiny}\begin{bmatrix} 0 & 0 \\ 0 & 0
  \end{bmatrix}\end{tiny}$ \\ \hline
$\mathbf{F_e}$ & $ \frac{A}{12} \begin{tiny} \begin{bmatrix}
2\Gamma_{1}+\Gamma_{2}+\Gamma_{3} \\
\Gamma_{1}+2\Gamma_{2}+\Gamma_{3} \\
\Gamma_{1}+\Gamma_{2}+2\Gamma_{3}\end{bmatrix}\end{tiny}$\\ \hline
$\mathbf{Q_e}$ & $\begin{tiny} \frac{q_{1-2}l_{1-2}}{2}\begin{bmatrix}
    1 \\ 1 \\ 0 \end{bmatrix} +\frac{q_{1-3}l_{1-3}}{2}\begin{bmatrix}
    1 \\ 0 \\ 1 \end{bmatrix} +\frac{q_{2-3}l_{2-3}}{2}\begin{bmatrix}
    0 \\ 1 \\ 1 \end{bmatrix} \end{tiny} $ \\ \hline
$\mathbf{W_e}$ & $\frac{A}{12}\begin{tiny}\begin{bmatrix} 2\theta_1 &
    \theta_2 & \theta_3 \\ \theta_1 & 2\theta_2 & \theta_3 \\ \theta_1
    & \theta_2 & 2\theta_3 \end{bmatrix}\end{tiny}$ \\ \hline
\end{tabular}
\label{tab:localmatwat}
\end{table}

Table \ref{tab:watlump} gives the values of both the consistent and the
different lumped versions of $\mathbf{A}$ and $\mathbf{W}$. The
construction of the lumped matrices with the methods proposed by
Neuman\index{Neuman} and Istok\index{Istok} is straightforward. For
approximation of the specific water capacity function and the water
content in the element are used linear polynomials, $C_e \approx
\mathbf{NC_e}$ and $\theta_e \approx \boldsymbol{N
  \theta_e}$. \cite{Segerlind} only considers a situation where the
capacity function (here for heatflux) is constant in the domain. This
led us to the problem of how to express the specific water capacity
function. Here, the water capacity and the water content are
approximated with the basisfunction $\mathbf{N^{*}}$. I.e. $C_e
\approx \mathbf{N^{*}C_e}$ and $\theta_e \approx
\boldsymbol{N^*\theta_e}$.


\begin{table}[h]
\tabcap{Lumping, water movement}
\begin{tabular}{p{2cm}|p{10.9cm}} \hline
\textbf{Method} & \textbf{Value} \\ \hline
Consistent & $\mathbf{A_e}=\begin{tiny}
\frac{A}{60}\begin{bmatrix}
6C_{w,1}+2C_{w,2}+2C_{w,3} & 2C_{w,1}+2C_{w,2}+C_{w,3}  &
2C_{w,1}+C_{w,2}+2C_{w,3} \\
2C_{w,1}+2C_{w,2}+C_{w,3}  & 2C_{w,1}+6C_{w,2}+2C_{w,3} &
C_{w,1}+2C_{w,2}+2C_{w,3} \\ 
2C_{w,1}+C_{w,2}+2C_{w,3}  & C_{w,1}+2C_{w,2}+2C_{w,3}  &
2C_{w,1}+2C_{w,2}+6C_{w,3}  \end{bmatrix} \end{tiny}$ \\ 
& $\mathbf{W_e}=\frac{A}{12}\begin{tiny}\begin{bmatrix} 2\theta_1 &
    \theta_2 & \theta_3 \\ \theta_1 & 2\theta_2 & \theta_3 \\ \theta_1
    & \theta_2 & 2\theta_3 \end{bmatrix}\end{tiny}$ \\ \hline
Segerlind & $\mathbf{A_{e}}=\frac{A}{3}\begin{tiny} \begin{bmatrix}
C_{w,1} & 0 & 0 \\
0 & C_{w,2} & 0 \\ 
0 & 0 & C_{w,3} 
\end{bmatrix} \end{tiny}$ \\ 
& $\mathbf{W_e}=\frac{A}{3}\begin{tiny}\begin{bmatrix} \theta_1 &
    0 & 0 \\ 0 & \theta_2 & 0 \\ 0
    & 0 & \theta_3 \end{bmatrix}\end{tiny}$ \\ \hline
Istok & $\mathbf{A_{e}}=\frac{A \bar{C}_e}{3} \begin{tiny}\begin{bmatrix}1 & 0 & 0 \\ 0 & 1 & 0 \\ 0
  & 0 & 1 \end{bmatrix} \end{tiny}$ \\ 
& $\mathbf{W_e}=\frac{A}{3}\begin{tiny}\begin{bmatrix} \theta_1 &
    0 & 0 \\ 0 & \theta_2 & 0 \\ 0
    & 0 & \theta_3 \end{bmatrix}\end{tiny}$ \\ \hline
Neuman & $\mathbf{A_{e}}=\frac{A}{12}\begin{tiny}\begin{bmatrix}
2C_{w,1}+C_{w,2}+C_{w,3} & 0  & 0 \\
0 & C_{w,1}+2C_{w,2}+C_{w,3} & 0 \\ 
0 & 0 & C_{w,1}+C_{w,2}+2C_{w,3} 
\end{bmatrix}\end{tiny}$ \\ 
& $\mathbf{W_e}=\frac{A}{12}\begin{tiny}\begin{bmatrix} 2\theta_1 +
    \theta_2 + \theta_3 & 0 & 0\\ 0 & \theta_1 + 2\theta_2 + \theta_3
    & 0 \\ 0 & 0 & \theta_1  + \theta_2 + 2\theta_3
    \end{bmatrix}\end{tiny}$ \\ \hline
\end{tabular}
\label{tab:watlump}
\end{table}


\subsection{Size of the timesteps}\index{timesteps, size of}

For water movement simulations it is possible either to choose a
constant size of the time steps, $\Delta t$ or to choose timesteps
which dynamically changes the size, dependent on how easy a
sufficiently good solution is obtained in the Picard
iterations\index{Picard iterations}. \\ 
\\
For the constant size of the timesteps a new timestep starts if either
the iteration criterion is fulfilled or if the number of Picard
iterations have reached a chosen maximum, $m_{max}$.\\
\\
For dynamically changing size of the timesteps the maximum number of
Picard iterations $m_{max}$ should also be chosen. Also a minimum and
maximum size to the timesteps, $\Delta t_{min}$ and $\Delta t_{max}$
must be specified. The procedure is:


\begin{enumerate}
\item if $m\leq 4$ then $\Delta t_{n+1} = 1.1\Delta t_n$ but not larger than
  $\Delta t_{max}$
\item if $m=5$ then $\Delta t_{n+1}=\Delta t_n$ 
\item if $6 \leq m\leq 7$ then $\Delta t_{n+1}=0.8 \Delta t_n$ but not lower
  than $\Delta t_{min}$
\item if $m>7$ then $\Delta t_{n+1}=0.3\Delta t_n$ but not lower than $\Delta
   t_{min}$
\item if $m=m_{max}$ then the time is only updated if $\Delta t_{n+1}=\Delta
  t_{min}$ else it tries again with smaller timesteps  $\Delta t_n=
  0.3\Delta t_n$ but not smaller than $\Delta t_{min}$
\end{enumerate}

\newpage

\subsection{Mass balance}\index{mass balance!water}

For a fast validation of simulations, a water balance index is
calculated. It is here defined as:

\begin{equation}
\text{water balance index}=\frac{\Delta W+Q+S}{\frac{1}{3}(|\Delta
  W|+|Q|+|S|)} 
\label{eq:watbal}
\end{equation}

where $\Delta W$ is total change in the water storage. $Q$ is the
total flux out of the domain, $S$ is the total amount of water removed
from $\Omega$ by sinks (sources are negative sinks). \\
\\
The water balance index shall for most simulations ideally be
zero. But if it is 0, it is not necessarily a correct solution. The
water balance for $\Omega$ can be fulfilled in many ways. The
water balance for the domain does not say anything about the internal
distribution of the water. There can for example be oscillations
(wiggles) around the real solution. The absolute value of the water
balance index can rise (up to 3) for redistribution cases without
water interchanging with the surrounding environment even if the
calculations are acceptable.




\section{Solute movement}

In this section special features of the solution of the
advection-dispersion equation are discussed. First of all, it is
discussed how information from the water movement module is treated. The
assumptions for the development of the local matrices are
discussed. The numerical instabilities which occur in advection
dominated problems and how to decide the size of the next timestep are
finally discussed.


\subsection{Coupling water and solute movement models}

The water and solute movement represented by equations
\ref{eq:watermovement} and \ref{eq:solutemovement} are coupled in a
sense that the solute movement are dependent of the water movement,
but not the reverse. The used solution process is to calculate the
water movement first. The results (the matric pressure potential and
the boundary fluxes) are stored at pre-described times, subsequently the
concentration is calculated. Meanwhile the water content, the time
derivative of the water content and fluxes are estimated using the
stored values from the water movement simulations. Another possibility
is to calculate both the water and solute movement in one
run. The advantage is that time is not spent on unnecessary
recalculations. The disadvantage is that maybe the timesteps
necessary for fast convergence in the water and the solute
calculations, respectively, are of different magnitude, so that
unnecessary small timesteps are used in one of the models. By dividing
the models into two modules, it is also possible to simulate
different solute movement scenarios with the same water flow as
background without repeating the calculation for solving Richard's
equation.\\
\\
Estimating the water flux\index{water flow!estimation} in the nodal
points is connected with problems since both the conductivities and
the derivative of the pressure, $\psi$ can have discontinuities
between elements edges. The spatial derivatives of $\psi$ can in each
element be calculated as:


\begin{equation}
\begin{split}
& \left(\frac{\partial \psi}{\partial x} \right)_e=\begin{bmatrix} \partial N_1/\partial
  x & \partial N_2/\partial x & \partial N_3/\partial x \end{bmatrix}
  \begin{bmatrix} \psi_1 \\ \psi_2 \\ \psi_3 \end{bmatrix} \\
& \left(\frac{\partial \psi}{\partial y} \right)_e=\begin{bmatrix} \partial N_1/\partial
  y & \partial N_2/\partial y & \partial N_3/\partial y \end{bmatrix}
  \begin{bmatrix} \psi_1 \\ \psi_2 \\ \psi_3 \end{bmatrix}
\end{split}
\end{equation}

The fluxes in the nodal points are estimated using mean values of
estimated fluxes in the surrounding elements. Two different methods
can be used. In the first, the mean hydraulic conductivity,
$\bar{K}_e$ in each of the surrounding elements is used:

\begin{equation}
\begin{split}
& q_x=-\frac{1}{\text{NEP}}\sum_{e=1}^{\text{NEP}} \bar{K_e} \left(\frac{\partial \psi}{\partial x}\right)_e
\\
& q_y=-\frac{1}{\text{NEP}}\sum_{e=1}^{\text{NEP}} \bar{K_e}\left(\frac{\partial \psi}{\partial y}+1\right)_e
\end{split}
\end{equation}

where $\text{NEP}$ is the number of elements surrounding the current
nodal point. In the second method, only the conductivities, $K_{e,p}$
in the current point are used, subscript $e$ and $p$ refer to the 
element and nodal number, respectively.

\begin{equation}
\begin{split}
& q_x=-\frac{1}{\text{NEP}}\sum_{e=1}^{\text{NEP}} K_{e,p} \left(
  \frac{\partial \psi}{\partial x} \right)_e
\\
& q_y=-\frac{1}{\text{NEP}}\sum_{e=1}^{\text{NEP}}
  K_{e,p}\left(\frac{\partial \psi}{\partial y}+1\right)_e
\end{split}
\end{equation}

Method number 1 is preferred as there is more consistence between the
points for evaluation of conductivity and the derivative. 


\subsubsection{Example}

As an example are the Darcy velocity calculated for nodal point no. 4
in a Finite Element mesh. The nodal point and its surrounding points
are shown in figure \ref{fig:velotri}. The situation is that $\psi$ is
constant in the area considered i.e $\partial \psi/\partial
x=\partial \psi/\partial y=0$ a situation that ideally occurs if the
water movement only is driven by gravity in a simulation. The node
numbering refers here to the global numbering.


\begin{figure}[h]  %here-top-bottom-page
\epsfig{file=velonode.eps,width=6.5cm} \hspace{.5cm}
\epsfig{file=velotri.eps,width=6.5cm}
\figcap{Node and triangle numbering for calculation of Darcy flux}
\label{fig:velotri}
\end{figure}

Method 1:

\begin{equation}
\begin{split}
& q_x=0 \\
& q_y=- \frac{1}{6} ( 
  \frac{1}{3}(K_{1,4}+K_{1,6}+K_{1,3})+\frac{1}{3}
    (K_{2,4}+K_{2,1}+K_{2,2})+ \\ 
& \ \ \ \ \ \ \ \ \ \ \ \ \ \ \ \cdots +
    \frac{1}{3}(K_{6,4})+K_{6,7}+K_{6,6}) )   
\end{split} 
\end{equation}


Method 2:

\begin{equation}
\begin{split}
& q_x=0 \\
& q_y=- \frac{1}{6}\left(
  K_{1,4}+K_{2,4}+K_{3,4}+K_{4,4}+K_{5,4}+K_{6,4} \right)
\end{split}
\end{equation}

The difference between the two methods is obvious. 


\subsection{Local matrices}\index{local matrices}\index{matrix!local}

Before calculating the local matrices given by equation
\ref{eq:localA}-\ref{eq:localQ}, it is necessary to decide how the
entities occurring in the weak form of the PDE, shall be approximated
in the elements. Recapitulated, the weak form of the
advection-dispersion is:


\begin{equation}
\begin{split}
& R_W=\int_{\Omega} \theta R \frac{\partial C}{\partial t} v \, dxdy
 +  \\ & \int_{\Omega} (\nabla
v)\cdot \left(\theta \begin{bmatrix} D_{xx} & D_{xy} \\ D_{yx}
  & D_{yy} \end{bmatrix} \nabla C- \begin{bmatrix} q_x \\ q_y
 \end{bmatrix} C \right) \, dxdy + 
 \int_{\Omega} \frac{\partial \theta}{\partial t}  C v \, dxdy + \\ &
 \int_{\Omega}(C_{sink}\Gamma-\theta\mu_l-\rho_b\mu_s) v \, dxdy
 +\int_{\partial \Omega_1} j v \, dS=0 
\label{eq:weaksolute}
\end{split}
\end{equation}


The approximations of the water flow, $q_x$ and $q_y$ and the
concentration are piecewise linear and continuous in $\Omega$. Many of
the other terms in equation \ref{eq:weaksolute} can not be expected to
be continuous between element sides which also are subdomain
borders, but can be approximated to variate linearly in each
element. The linear approximations can be expressed by equation
\ref{eq:linapprox}. The made linear approximations for the
coefficients in equation \ref{eq:weaksolute} are placed in table
\ref{tab:advdisplin}.

\begin{table}[H]
\tabcap{Approximations for solute movement}
\label{tab:advdisplin}
\begin{tabular}{p{4cm}|p{8.9cm}} \hline
\textbf{Coefficient} & \textbf{Approximation} \\ \hline
$\theta R$ &  
$\theta_e \approx \boldsymbol{N\theta_e}$ \newline $R_e \approx
\mathbf{NR_e}$  \\ \hline
$\theta \begin{bmatrix} D_{xx} & D_{xy} \\ D_{yx} & D_{yy}
\end{bmatrix}$ &  $\begin{matrix} 
(\theta D_{xx})_e \approx \boldsymbol{N(\theta
  D_{xx})_e} \\ (\theta D_{yy})_e \approx \boldsymbol{N(\theta
  D_{yy})_e} \\ (\theta D_{xy})_e \approx \boldsymbol{N(\theta
  D_{xy})_e} \\ (\theta D_{yx})_e \approx \boldsymbol{N(\theta
  D_{yx})_e}  
\end{matrix}$
\\ \hline 
$\begin{bmatrix} q_x \\ q_y \end{bmatrix}$ & $\begin{matrix} q_{x,e} \approx \mathbf{Nq_{x,e}} \\ q_{y,e} \approx \mathbf{Nq_{y,e}}
  \end{matrix}$ 
\\ \hline  
$\frac{\partial \theta}{\partial t}$ & $\dot{\theta}_e \approx
\boldsymbol{N\dot{\theta}_e}$ \\ \hline
$C_{sink}\Gamma$ & $C_{sink,e} \approx
  \mathbf{NC_{sink,e}}$ \newline $\Gamma_e \approx
  \boldsymbol{N\Gamma_e}$ \\ \hline
$\theta\mu_l$ & $\boldsymbol{N\theta_e}$ \newline $\mu_{l,e} \approx
\boldsymbol{N\mu_{l,e}}$  \\ \hline
$\rho_b\mu_s$ & $\boldsymbol{N\mu_{c,e}}$ \\ \hline
\end{tabular}
\end{table}

The bulk density, $\rho_b$ is treated as constant in each subdomain.
For calculating $\mathbf{Q_e}$, the flux through the specified Neumann
boundaries  $j_{1-2}$, $j_{1-3}$ and $j_{2-3}$ is regarded as constant
on the sides of a single element. The hereby obtained local matrices are
shown i table \ref{tab:localmatsol}. ${\bar{\theta D_{xx}}}$, $\bar{\theta
  D_{xy}}$ and $\bar{\theta D_{yy}}$ denotes mean values which can be
calculated as one third of the sum of the values in the vertices.\\
\\
It is also here possible to use the lumped formulation using
the techniques described before. The resulting matrices are given in
table \ref{tab:sollump}. Only the time derivatives of the solution
component, $C$ and not the part containing the time derivative of
$\theta$ are lumped. For the lumping method by
Segerlind\index{Segerlind}, $\mathbf{N^*}$ is used to describe both
the retardation factor and the water content.





\begin{table}[H]
\tabcap{Local matrices, solute movement}
\begin{tabular}{p{1.5cm}|p{11.4cm}} \hline
\textbf{Matrix} &  \textbf{Value} \\ \hline
$\mathbf{A_e}$ & $\frac{A}{180}\begin{bmatrix}
\mathbf{A_{e,column1}} & \mathbf{A_{e,column2}} & \mathbf{A_{e,column3}}
\end{bmatrix},$ \\ 
& $\mathbf{A_{e,column1}}= \begin{tiny} \begin{bmatrix}
  R_1(12\theta_{1}+3\theta_{2}+3\theta_{3}) +
  R_2(3\theta_{1}+2\theta_{2}+\theta_{3}) +
  R_3(3\theta_{1}+\theta_{2}+2\theta_{3}) \\
  R_1(3\theta_{1}+2\theta_{2}+\theta_{3}) +
  R_2(2\theta_{1}+3\theta_{2}+\theta_{3}) +
  R_3(\theta_{1}+\theta_{2}+\theta_{3}) \\
  R_1(3\theta_{1}+\theta_{2}+2\theta_{3}) +
  R_2(\theta_{1}+\theta_{2}+\theta_{3}) +
  R_3(2\theta_{1}+\theta_{2}+3\theta_{3}) \end{bmatrix} \end{tiny}$ \\
& $\mathbf{A_{e,column2}}= \begin{tiny} \begin{bmatrix}
  R_1(3\theta_{1}+2\theta_{2}+\theta_{3}) +
  R_2(2\theta_{1}+3\theta_{2}+\theta_{3}) +
  R_3(\theta_{1}+\theta_{2}+\theta_{3}) \\
  R_1(2\theta_{1}+3\theta_{2}+\theta_{3}) +
  R_2(3\theta_{1}+12\theta_{2}+3\theta_{3}) +
  R_3(\theta_{1}+3\theta_{2}+2\theta_{3}) \\
  R_1(\theta_{1}+\theta_{2}+\theta_{3}) +
  R_2(\theta_{1}+3\theta_{2}+2\theta_{3}) +
  R_3(\theta_{1}+2\theta_{2}+3\theta_{3}) \end{bmatrix} \end{tiny}$ \\ 
& $\mathbf{A_{e,column3}}= \begin{tiny} \begin{bmatrix}
  R_1(3\theta_{1}+\theta_{2}+2\theta_{3}) +
  R_2(\theta_{1}+\theta_{2}+\theta_{3}) +
  R_3(2\theta_{1}+\theta_{2}+3\theta_{3}) \\
  R_1(\theta_{1}+\theta_{2}+\theta_{3}) +
  R_2(\theta_{1}+3\theta_{2}+2\theta_{3}) +
  R_3(\theta_{1}+2\theta_{2}+3\theta_{3}) \\
  R_1(2\theta_{1}+\theta_{2}+3\theta_{3}) +
  R_2(\theta_{1}+2\theta_{2}+3\theta_{3}) +
  R_3(3\theta_{1}+3\theta_{2}+12\theta_{3}) 
\end{bmatrix} \end{tiny}$ \\ \hline
$\mathbf{B_e}$ &$\begin{tiny}
  A \overline{\theta D}_{xx} \begin{bmatrix} (\frac{\partial
  N_{1}}{\partial x})^{2}
 &\frac{\partial N_{1}}{\partial x}\frac{\partial N_{2}}{\partial x}
&\frac{\partial N_{1}}{\partial x}\frac{\partial N_{3}}{\partial x}\\ 
\frac{\partial N_{2}}{\partial x}\frac{\partial N_{1}}{\partial x}
&(\frac{\partial N_{2}}{\partial x})^{2}
&\frac{\partial N_{2}}{\partial x}\frac{\partial N_{3}}{\partial x}\\ 
\frac{\partial N_{3}}{\partial x}\frac{\partial N_{1}}{\partial x}
&\frac{\partial N_{3}}{\partial x}\frac{\partial N_{2}}{\partial x}  
&(\frac{\partial N_{3}}{\partial x})^{2}
\end{bmatrix} \end{tiny} +$ \\
& $\begin{tiny}
  A\overline{\theta D}_{xy} \begin{bmatrix} 
\frac{\partial N_{1}}{\partial y}\frac{\partial N_{1}}{\partial x}
&\frac{\partial N_{1}}{\partial y}\frac{\partial N_{2}}{\partial x}
&\frac{\partial N_{1}}{\partial y}\frac{\partial N_{3}}{\partial x}\\ 
\frac{\partial N_{2}}{\partial y}\frac{\partial N_{1}}{\partial x}
&\frac{\partial N_{2}}{\partial y}\frac{\partial N_{2}}{\partial x}
&\frac{\partial N_{2}}{\partial y}\frac{\partial N_{3}}{\partial x}\\ 
\frac{\partial N_{3}}{\partial y}\frac{\partial N_{1}}{\partial x}
&\frac{\partial N_{3}}{\partial y}\frac{\partial N_{2}}{\partial x}  
&\frac{\partial N_{3}}{\partial y}\frac{\partial N_{3}}{\partial x}
\end{bmatrix} \end{tiny} +$ \\
& $\begin{tiny}
  A \overline{\theta D}_{xy} \begin{bmatrix} 
\frac{\partial N_{1}}{\partial x}\frac{\partial N_{1}}{\partial y}
&\frac{\partial N_{1}}{\partial x}\frac{\partial N_{2}}{\partial y}
&\frac{\partial N_{1}}{\partial x}\frac{\partial N_{3}}{\partial y}\\ 
\frac{\partial N_{2}}{\partial x}\frac{\partial N_{1}}{\partial y}
&\frac{\partial N_{2}}{\partial x}\frac{\partial N_{2}}{\partial y}
&\frac{\partial N_{2}}{\partial x}\frac{\partial N_{3}}{\partial y}\\ 
\frac{\partial N_{3}}{\partial x}\frac{\partial N_{1}}{\partial y}
&\frac{\partial N_{3}}{\partial x}\frac{\partial N_{2}}{\partial y}  
&\frac{\partial N_{3}}{\partial x}\frac{\partial N_{3}}{\partial y}
\end{bmatrix} \end{tiny} +$ \\
& $\begin{tiny}
A \overline{\theta D}_{yy}  
\begin{bmatrix} (\frac{\partial N_{1}}{\partial y})^{2}
 &\frac{\partial N_{1}}{\partial y}\frac{\partial N_{2}}{\partial y}
&\frac{\partial N_{1}}{\partial y}\frac{\partial N_{3}}{\partial y}\\ 
\frac{\partial N_{2}}{\partial y}\frac{\partial N_{1}}{\partial y}
&(\frac{\partial N_{2}}{\partial y})^{2}
&\frac{\partial N_{2}}{\partial y}\frac{\partial N_{3}}{\partial y}\\ 
\frac{\partial N_{3}}{\partial y}\frac{\partial N_{1}}{\partial y}
&\frac{\partial N_{3}}{\partial y}\frac{\partial N_{2}}{\partial y}  
&(\frac{\partial N_{3}}{\partial y})^{2}
\end{bmatrix} \end{tiny}$ \\ \hline 
$\mathbf{C_e}$ & $\begin{tiny}
\frac{A}{12}
\begin{bmatrix}
(2q_{x,1}+q_{x,2}+q_{x,3})\frac{\partial N_1}{\partial x} &
(q_{x,1}+2q_{x,2}+q_{x,3})\frac{\partial N_1}{\partial x} & 
(q_{x,1}+q_{x,2}+2q_{x,3})\frac{\partial N_1}{\partial x} \\
(2q_{x,1}+q_{x,2}+q_{x,3})\frac{\partial N_2}{\partial x} &
(q_{x,1}+2q_{x,2}+q_{x,3})\frac{\partial N_2}{\partial x} & 
(q_{x,1}+q_{x,2}+2q_{x,3})\frac{\partial N_2}{\partial x} \\
(2q_{x,1}+q_{x,2}+q_{x,3})\frac{\partial N_3}{\partial x} &
(q_{x,1}+2q_{x,2}+q_{x,3})\frac{\partial N_3}{\partial x} & 
(q_{x,1}+q_{x,2}+2q_{x,3})\frac{\partial N_3}{\partial x} 
\end{bmatrix}
\end{tiny}+$ \\ 
& $\begin{tiny}
\frac{A}{12}
\begin{bmatrix}
(2q_{y,1}+q_{y,2}+q_{y,3})\frac{\partial N_1}{\partial y} &
(q_{y,1}+2q_{y,2}+q_{y,3})\frac{\partial N_1}{\partial y} & 
(q_{y,1}+q_{y,2}+2q_{y,3})\frac{\partial N_1}{\partial y} \\
(2q_{y,1}+q_{y,2}+q_{y,3})\frac{\partial N_2}{\partial y} &
(q_{y,1}+2q_{y,2}+q_{y,3})\frac{\partial N_2}{\partial y} & 
(q_{y,1}+q_{y,2}+2q_{y,3})\frac{\partial N_2}{\partial y} \\
(2q_{y,1}+q_{y,2}+q_{y,3})\frac{\partial N_3}{\partial y} &
(q_{y,1}+2q_{y,2}+q_{y,3})\frac{\partial N_3}{\partial y} & 
(q_{y,1}+q_{y,2}+2q_{y,3})\frac{\partial N_3}{\partial y} 
\end{bmatrix} \end{tiny}$ \\ \hline
$\mathbf{D_e}$ &  $\begin{tiny}\begin{bmatrix} 0 \\ 0 
  \end{bmatrix}\end{tiny}$ \\ \hline 
$\mathbf{E_e}$ &  $\frac{A}{60}\begin{tiny}\begin{bmatrix}
  6\dot{\theta}_{1}+2\dot{\theta}_{2}+2\dot{\theta}_{3} &
  2\dot{\theta}_{1}+2\dot{\theta}_{2}+\dot{\theta}_{3} &
  2\dot{\theta}_{1}+\dot{\theta}_{2}+2\dot{\theta}_{3} \\
  2\dot{\theta}_{1}+2\dot{\theta}_{2}+\dot{\theta}_{3} &
  2\dot{\theta}_{1}+6\dot{\theta}_{2}+2\dot{\theta}_{3} &
  \dot{\theta}_{1}+2\dot{\theta}_{2}+2\dot{\theta}_{3} \\
  2\dot{\theta}_{1}+\dot{\theta}_{2}+2\dot{\theta}_{3} &
  \dot{\theta}_{1}+2\dot{\theta}_{2}+2\dot{\theta}_{3} &
  2\dot{\theta}_{1}+2\dot{\theta}_{2}+6\dot{\theta}_{3} 
\end{bmatrix}\end{tiny}$ \\ \hline
$\mathbf{F_e}$ &  $\begin{tiny}
\frac{A}{60}\begin{bmatrix}
C_{sink,1}(6\Gamma_1 + 2\Gamma_2 + 2\Gamma_3) + 
C_{sink,2}(2\Gamma_1 + 2\Gamma_2 + \Gamma_3) + 
C_{sink,3}(2\Gamma_1 + \Gamma_2 + 2\Gamma_3)\\
C_{sink,1}(2\Gamma_1 + 2\Gamma_2 + \Gamma_3) + 
C_{sink,2}(2\Gamma_1 + 6\Gamma_2 + 2\Gamma_3) + 
C_{sink,3}(\Gamma_1 + 2\Gamma_2 + 2\Gamma_3)\\
C_{sink,1}(2\Gamma_1 + \Gamma_2 + 2\Gamma_3) + 
C_{sink,2}(\Gamma_1 + 2\Gamma_2 + 2\Gamma_3) + 
C_{sink,3}(2\Gamma_1 + 2\Gamma_2 + 6\Gamma_3)
\end{bmatrix} \end{tiny}  +$\\
& $\begin{tiny}
\frac{A}{60}\begin{bmatrix}
\mu_{l,1}(6\theta_1 + 2\theta_2 + 2\theta_3) + 
\mu_{l,2}(2\theta_1 + 2\theta_2 + \theta_3) + 
\mu_{l,3}(2\theta_1 + \theta_2 + 2\theta_3)\\
\mu_{l,1}(2\theta_1 + 2\theta_2 + \theta_3) + 
\mu_{l,2}(2\theta_1 + 6\theta_2 + 2\theta_3) + 
\mu_{l,3}(\theta_1 + 2\theta_2 + 2\theta_3)\\
\mu_{l,1}(2\theta_1 + \theta_2 + 2\theta_3) + 
\mu_{l,2}(\theta_1 + 2\theta_2 + 2\theta_3) + 
\mu_{l,3}(2\theta_1 + 2\theta_2 + 6\theta_3)
\end{bmatrix} \end{tiny} +$
\\ & $\frac{\rho_b A}{12}
\begin{tiny}
\begin{bmatrix}
2\mu_{s,1}+\mu_{s,2} +\mu_{s,3} \\
\mu_{s,1} +2\mu_{s,2} +\mu_{s,3} \\
\mu_{s,1} +\mu_{s,2} +2\mu_{s,3} 
 \end{bmatrix}
\end{tiny}$ \\ \hline
$\mathbf{Q_e}$ & $\begin{tiny}\frac{j_{1-2}l_{1-2}}{2}\begin{bmatrix} 1 \\ 1 \\ 0
\end{bmatrix} +\frac{j_{1-3}l_{1-3}}{2}\begin{bmatrix} 1 \\ 0 \\ 1
\end{bmatrix} +\frac{j_{2-3}l_{2-3}}{2}\begin{bmatrix} 0 \\ 1 \\ 1
\end{bmatrix} \end{tiny}$  \\ \hline
\end{tabular}
\label{tab:localmatsol}
%\end{center}
%\end{sidewaystable}
\end{table}


%It is also here possible to use the lumped formulation using
%the techniques described before. The resulting matrices are given in
%table \ref{tab:sollump}. Only the time derivatives of the solution
%component, $C$ and not the part containing the time derivative of
%$\theta$ are lumped. For the lumping method by
%Segerlind\index{Segerlind}, $\mathbf{N^*}$ is used to describe both
%the retardation factor and the water content.
 

\begin{table}[H]
\tabcap{Lumping, solute movement}
\begin{tabular}{p{2cm}|p{10.9cm}} \hline
\textbf{Method} & \textbf{Value} \\ \hline
Consistent &   $\mathbf{A_e}=\frac{A}{180}\begin{bmatrix}
\mathbf{A_{e,column1}} & \mathbf{A_{e,column2}} & \mathbf{A_{e,column3}}
\end{bmatrix},$ \\ 
& $\mathbf{A_{e,column1}}= \begin{tiny} \begin{bmatrix}
  R_1(12\theta_{1}+3\theta_{2}+3\theta_{3}) +
  R_2(3\theta_{1}+2\theta_{2}+\theta_{3}) +
  R_3(3\theta_{1}+\theta_{2}+2\theta_{3}) \\
  R_1(3\theta_{1}+2\theta_{2}+\theta_{3}) +
  R_2(2\theta_{1}+3\theta_{2}+\theta_{3}) +
  R_3(\theta_{1}+\theta_{2}+\theta_{3}) \\
  R_1(3\theta_{1}+\theta_{2}+2\theta_{3}) +
  R_2(\theta_{1}+\theta_{2}+\theta_{3}) +
  R_3(2\theta_{1}+\theta_{2}+3\theta_{3}) \end{bmatrix} \end{tiny}$ \\
& $\mathbf{A_{e,column2}}= \begin{tiny} \begin{bmatrix}
  R_1(3\theta_{1}+2\theta_{2}+\theta_{3}) +
  R_2(2\theta_{1}+3\theta_{2}+\theta_{3}) +
  R_3(\theta_{1}+\theta_{2}+\theta_{3}) \\
  R_1(2\theta_{1}+3\theta_{2}+\theta_{3}) +
  R_2(3\theta_{1}+12\theta_{2}+3\theta_{3}) +
  R_3(\theta_{1}+3\theta_{2}+2\theta_{3}) \\
  R_1(\theta_{1}+\theta_{2}+\theta_{3}) +
  R_2(\theta_{1}+3\theta_{2}+2\theta_{3}) +
  R_3(\theta_{1}+2\theta_{2}+3\theta_{3}) \end{bmatrix} \end{tiny}$ \\ 
& $\mathbf{A_{e,column3}}= \begin{tiny} \begin{bmatrix}
  R_1(3\theta_{1}+\theta_{2}+2\theta_{3}) +
  R_2(\theta_{1}+\theta_{2}+\theta_{3}) +
  R_3(2\theta_{1}+\theta_{2}+3\theta_{3}) \\
  R_1(\theta_{1}+\theta_{2}+\theta_{3}) +
  R_2(\theta_{1}+3\theta_{2}+2\theta_{3}) +
  R_3(\theta_{1}+2\theta_{2}+3\theta_{3}) \\
  R_1(2\theta_{1}+\theta_{2}+3\theta_{3}) +
  R_2(\theta_{1}+2\theta_{2}+3\theta_{3}) +
  R_3(3\theta_{1}+3\theta_{2}+12\theta_{3}) 
\end{bmatrix} \end{tiny}$ \\ \hline
Segerlind  &  $\mathbf{A_e}=\begin{tiny}
\frac{A}{3}\begin{bmatrix}
R_1\theta_1 & 0 & \\
0 & R_2\theta_2 & 0 \\
0 & 0 & R_3\theta_3 \\
\end{bmatrix} \end{tiny}$ \\ \hline
Istok      &  $\mathbf{A_e}=
\frac{A}{36}\begin{small}(R_1(2\theta_1+\theta_2+\theta_3)+R_2(\theta_1+2\theta_2+\theta_3)+R_3(\theta_1+\theta_2+2\theta_3))\end{small}\mathbf{I}$
\\ \hline
Neuman     &  $\mathbf{A_e}=\frac{A}{60}\mathbf{I}
\begin{tiny}\begin{bmatrix}
R_1(6\theta_1+2\theta_2+2\theta_3) +
R_2(2\theta_1 2\theta_2+ \theta_3) +
R_3(2\theta_1 +\theta_2 +2\theta_3)\\
R_1(2\theta_1+2\theta_2+\theta_3) +
R_2(2\theta_1 6\theta_2+ 2\theta_3) +
R_3(\theta_1 +2\theta_2 +2\theta_3)\\
R_1(2\theta_1+\theta_2+2\theta_3) +
R_2(\theta_1 2\theta_2+ 2\theta_3) +
R_3(2\theta_1 +2\theta_2 +6\theta_3) \end{bmatrix} \end{tiny}$ \\ \hline 
\end{tabular}
\label{tab:sollump}
\end{table}


\subsection{Point source}\index{point source!implementation}


In \cite{Segerlind} it is described how a point source (or sink) can
be implemented. The point source can simply be described using Diracs
delta function:

\begin{equation}
\begin{split}
& \mathbf{F_e,p}=\int_A q_p \delta(x-x_p)\delta(y-y_p)\mathbf{N}^T \,dxdy
= \\ & q_p \int_A  \delta(x-x_p)\delta(y-y_p) \begin{tiny}
  \begin{bmatrix} N_1 \\ N_2 \\ N_3 \end{bmatrix} \end{tiny} \,dxdy  
\end{split}
\end{equation}

where $x_p,y_p$ are the coordinates of the point sink. The best
location of a sink is at a node. In the program it is only
possible to place the sink at a node. The sink is not implemented by
first calculating the contribution from each element for subsequently
assembling the element contributions to the $\mathbf{F}$ in
\ref{eq:matrix}. Instead, the point sources are implemented more
directly. It is easy to realize that integration of a point source
inside $\Omega$ is equal $q_p$. The point source is then simply
implemented by adding $q_p$ to the row $p$ in $\mathbf{F}$:

\begin{equation}
\mathbf{F_p}=q_p
\label{eq:fpoint}
\end{equation}

It is important to note that the node shall be placed in the interior
of $\Omega$ for using equation \ref{eq:fpoint}, if this is not the
case then the integration of the Dirac delta function is not equal to
1!



\subsection{Advection dominated problems}\index{stability}


Several numerical problems can be involved with the solving of
the advection-diffusion problem, especially when the problems are
dominated by advection. The numerical solutions have often unexpected
oscillations in that situation. A lot of more or less complicated
methods to reduce the problems have been developed. Two of the methods
are \textit{upstream weighting}\index{upstream weighting} and
\textit{streamline diffusion}\index{streamline diffusion} - both in
many variants. 


\subsection{$P_e$ and $C_r$ numbers}\index{P\'{e}clet
  number}\index{Courant number}

There are two different dimensionless numbers which are important for
the stability. The \textit{P\'{e}clet number}, which in 1D is:

\begin{equation}
P_e=v\Delta x/D 
\end{equation}

where $v$ is the velocity, $\Delta x$ is the space increment and
$D$ is the dispersion. In other words a ratio between the convective
and the dispersive terms. The \textit{Courant number} is here defined
as:

\begin{equation}
C_r=v\Delta t/(R\Delta x)
\end{equation}
 
where $R$ is the retardation factor. In other places, even for models
with adsorption it may be defined as $C_r=v\Delta t/(\Delta x)$. The
Courant number describes the ratio between the movement of a particle
by advection in one time increment and the grid spacing. Theoretical
stability investigations are rather complicated, especially in a two
or three dimensional space with heterogenous soils. Most of the
theoretical considerations for stability are done for one dimensional
flow with uniform velocity. \cite{Perrochet} investigated the
advection-dispersion equation without any chemical processes and found
that The classical Crank-Nicolson-Galerkin scheme is stable for
$P_e\leq 2$ and $C_r \leq 1$, \cite{Perrochet}. The analysis was done
without considering  adsorption. By using the same theory, it can
easily be shown that stability for the advection-dispersion model is
insured for the same constraints with the Courant number which is used
here. \\
\\
It can be concluded that keeping the Courant number lower than one
is just a question of sufficiently small timesteps. But is it possible
to make a mesh which under all circumstances prevents that the P\'{e}clet
number exceeds 2?.  The sidelengths of the elements that coincide
with the boundaries are not calculated in TopFEM. Instead all triangle
areas are computed. The characteristic length of the elements,
$\Delta x$  can be approximated as $\Delta x=2\sqrt{A}$,
where $A$ is the area of the element (triangle). $\sqrt{2} \cdot
\sqrt{2A} = 2 \sqrt{A}$  is the length of the diagonal in a square
which is made of two isoscele right-angled triangles (elements) with
the area $A$. This is a good measure of the sidelength in the real
elements, where the 3 side lengths in an element for stability reasons
should have almost equal lengths, \cite{FEMLAB,Segerlind}. 
\\
\\
The P\'{e}clet number for the flow in the x-direction can be calculated as:

\begin{equation}
\begin{split}
P_{e,x}&=\frac{q_x 2\sqrt{A}}{\theta D_{xx}}=\frac{2q_x\sqrt{A}}{\alpha_L
  \frac{q_xq_x}{|\mathbf{q}|}+\alpha_T\frac{q_yq_y}{|\mathbf{q}|}+D_0\frac{\theta^{10/3}}{\theta_s}}\\
& <\frac{2q_x\sqrt{A}}{\alpha_T\frac{q_xq_x+q_yq_y}{|\mathbf{q}|}}
  = \frac{2q_x\sqrt{A}}{\alpha_T |\mathbf{q}|} \leq \frac{2\sqrt{A}}{\alpha_T}
\end{split}
\label{eq:pex}
\end{equation}

where it is assumed that $\alpha_L \geq  \alpha_T$. The same procedure
can of course be used to evaluate $P_{e,y}$. It can then be concluded
that the maximum P\'{e}clet number in the x and y-direction is under
$2\sqrt{A}/\alpha_T$. If the longitudinal dispersivity is 5 cm and the
transversal is 1/10 of the longitudinal dispersion and the maximum
allowed $P_e$ is 2 it can be concluded that the maximum sidelength of
the elements will be approximately 1/2 cm. This will result in a very
fine mesh. In practise the judgments made in equation \ref{eq:pex} are
so rough that somewhat larger elements probably can be used without
stability problems.\\
\\
In the present code it is possible to choose between 2 stabilizing
methods\index{stabilizing methods}:

\begin{enumerate}
\item Varying the size of $\Delta t$ so $P_eC_r \leq \gamma$.
\item Introducing extra diffusion in the streamline direction so
  $P_eC_r\leq\gamma$ is fulfilled. $\gamma$ is called the
  \textit{performance index}\index{performance index}.  
\end{enumerate}

It is of course also possible to not choosing any stabilizing
methods. Put into practice there is often stability as long as
$P_eC_r\leq \gamma$  where $2\leq \gamma \leq 10$, \cite{Perrochet}
which under any circumstances is less restrictive than keeping both
$P_e\leq 2$ and $C_r\leq 1$. 



\subsection{Varying the size of the timesteps}

The calculation of the P\'{e}clet and Courant numbers are of computational
reasons a little different in the two stabilizing methods. For the
method where the size of the timesteps are varied in order to fulfill
the stability criterion, requirements are 


\begin{equation}
\begin{split}
& (P_eC_r)_x=\frac{|v_x|\Delta x}{D}\cdot \frac{|v_x|\Delta t}{R \Delta
  x}=\frac{\theta v_x^2\Delta t}{R(\theta D_{xx})} \\
& (P_eC_r)_y=\frac{|v_y|\Delta x}{D} \cdot \frac{|v_y|\Delta t}{R\Delta
  x}=\frac{\theta v_y^2\Delta t}{R(\theta D_{yy})}
\end{split}
\end{equation}

$\Delta t$ is chosen so both $(P_eC_r)_x$ and $(P_eC_r)_y$ are less
than $\gamma$:


\begin{equation}
\Delta t=min \left( \frac{R(\theta D_{xx})\gamma}{\theta
    v_x^2},\frac{R(\theta D_{yy})\gamma}{\theta v_y^2} \right) 
\end{equation}

In the program, it is possible to set a minimum and maximum value of
$\Delta t$, $\Delta t_{min}$ and $\Delta t_{max}$. The minimum value, to
prevent the timesteps to get too small (and the CPU-time too large). The
maximum value can be chosen to take into account possible changes in
time dependent boundary conditions or sink terms.



\subsection{Streamline diffusion}\index{streamline diffusion}

$P_eC_r$ are here evaluated as:

\begin{equation}
P_eC_r=\frac{|\mathbf{v}|\Delta
  x}{D^*+\alpha_L|\mathbf{v}|}\frac{|\mathbf{v}| \Delta t}{R \Delta
  x}=\frac{|\mathbf{v}|^2\Delta
  t}{R(D^*+\alpha_L|\mathbf{v}|)}=\frac{|\mathbf{q}|^2\Delta t}{\theta
  R(\theta D^*+\alpha_L|\mathbf{q}|)}
\end{equation}

In the streamline diffusion method, according to \cite{Perrochet}
is some additional longitudinal dispersion added to prevent that
$P_eC_r$ exceeds the chosen performance index. The additional
longitudinal dispersion, $\bar{\alpha_L}$ can be calculated as:


\begin{equation}
\bar{\alpha_L}=\begin{cases} \frac{|\mathbf{q}|\Delta
    t}{\theta R\gamma}-\alpha_L-\frac{\theta D^*}{|\mathbf{q}|}, & \text{for}
    \ \alpha_L + \frac{\theta D^*}{|\mathbf{q}|} < \frac{|\mathbf{q}|\Delta
    t}{\theta R\gamma}
 \\ 0, & \text{for} \ \alpha_L +
    \frac{\theta D^*}{|\mathbf{q}|} \geq \frac{|\mathbf{q}|\Delta
    t}{\theta R\gamma}\end{cases}
\end{equation}

\subsection{Stability tests}

To investigate the stability of the numerical model a simple system
 has been modeled. The situation here is a one dimensional column,
 horizontal column with steady-state water flow with pore velocity
 $v$. And a given diffusion, $D$ (both molecular diffusion and
 hydrodynamic dispersion) and retardation factor, $R$. The
 advection-dispersion equation in one dimension can be written as:


\begin{equation}
R\frac{\partial C}{\partial t}=D\frac{\partial^2C}{\partial
  x^2}-v\frac{\partial C}{\partial x}
\end{equation}

where $v=q/\theta$. The initial condition is that the concentration is
uniformly distributed in the column:

\begin{equation}
C(x,0)=C_i
\end{equation}

At the left boundary the solute flux is:

\begin{equation}
(-D\frac{\partial C}{\partial x}+vC)|_{x=0}=\begin{cases} vC_0 &
  0<t\leq t_0 \\ 0 & t>t_0 \end{cases}
\end{equation}

The solution can then according to \cite{Genuchtenanalytical} be
written as:

\begin{equation}
C(x,t)=\begin{cases}C_i+(C_0-C_i)A(x,t) & 0<t\leq t_0 \\
  C_i+(C_0-C_i)A(x,t)-C_0A(x,t) & t>t_0 \end{cases}
\end{equation}

where:

\begin{equation}
\begin{split}
A(x,t)=&\frac{1}{2} \erfc\left
  [\frac{Rx-vt}{2(DRt)^{1/2}}\right]+\left(\frac{v^2t}{\pi
  DR}\right)\exp\left[-\frac{(Rx-vt)^2}{4DRt}\right ] \\
&-\frac{1}{2}(1+\frac{vx}{D}+\frac{v^2t}{DR})\exp(vx/D)\erfc\left[\frac{Rx+vt}{2(DRt)^{1/2}}\right]
\end{split}
\end{equation}

For the simulations, a waterflow situation is made with steady state
flow with the chosen porewater velocity, $v$=10 cm/day. The solute is
injected at the left border from $t=0$ to $t=t_0$. $t_0$ is chosen to
be 2 days. $C_0$ is for the simplicity chosen to 1. For the FEM
simulations the virtual soil column is 1/2 cm high and 50 cm wide. On
the domain a regular mesh is generated with 100 equally large
elements, each  with characteristic lengths of $\Delta x$=1 cm. The
numerical parameter, $\omega$ is set to 1/2, i.e. a Crank-Nicolson
scheme.\\ 
\\
Figure \ref{fig:soltest1_sub} shows simulation results for 
$C_r=1$ and different $P_e$-numbers. The low courant number insures
according to \cite{Perrochet} that the time increments are sufficiently
low. The different P\'{e}clet numbers are obtained by varying the
diffusion. i.e the simulations are not representing the same physical
situation. As it can be seen, there are  saw-tooth instabilities also
called wiggles\index{wiggles}, \cite{Abbott} for the high P\'{e}clet
numbers. As expected, no instabilities are observed for P\'{e}clet
numbers below 2. For $P_e=5$ and $P_e=10$ the wiggles are
significant.

\begin{figure}[H]  %here-top-bottom-page
\begin{center}
%\epsffile{soltest1_sub.eps}
\epsfig{file=soltest1_sub.eps,width=12cm,height=7.2cm}
\figcap{FEM solutions shown as concentration as function of x after
3 days(upper) and as function of time for x=10 cm (lower). Different
$P_e$-numbers have been used. The Courant number,
$C_r$ is 1. There are wiggles for large $P_e$-numbers}
\label{fig:soltest1_sub}
\end{center}
\end{figure}

Figure \ref{fig:soltest2_sub} provides graphs for situations with
constant P\'{e}clet number, $P_e=2$ and varying Courant number. The
simulations represent the same physical situation. For Courant numbers,
below, less, or equal to 1 practically the same solution is obtained. For
$C_r=2$ which in this case corresponds to timesteps of 1/5 of a day,
the results are somewhat different and unstable. Very large
oscillations on the graph that shows the concentration as function of
$x$ can be seen for $x \geq 43$ cm. The simulations with $C_r=2$ are
maybe also critical as the timesteps are too large compared with the
time (2 days) for the injection of solute at the left boundary.\\ 
\\
Figure \ref{fig:soltest3_sub} shows results for different combinations
of $P_e$ and $C_r$, but restricted so $P_eC_r=10$. For all the
simulations, wiggles can be observed, but there is for example no
simple relationship between the P\'{e}clet number and the size of the
saw-tooths. The wiggles stretch over longer time and space for
$C_r=2$ compared with the other simulations. The reasons for that can
simply be that it takes a number of timesteps before the wiggles are
eliminated. \\
\\
Figure \ref{fig:soltest4_sub} shows a case with a P\'{e}clet number of
20. The Courant number is 1. One of the graphs shows the simulations
without any stabilization. Here the wiggles are significant. By
comparing with the analytical solution it is  evaluated that there is
some additional dispersion (numerical
dispersion)\index{dispersion!numerical}. Another graph shows the same
simulation with streamline diffusion with $\gamma=5$. The additional
dispersion is significant but the wiggles have also
disappeared. Another graph shows simulation with the stabilizing
method where the size of $\Delta t$ is changed so
$P_eC_r\leq\gamma$. $\gamma=5$ is used which is the same as reducing
the $C_r$-number to 1/4, i.e. 4 times as many timesteps (or
approximately 4 times longer CPU-time). The last results are close to
the analytical solution. In practical use it is difficult to choose
the stabilizing method - what is good in one situation may be
applicable in another. In the 1D simulations provided here there are
no problems with increasing the space or time discretization - the
CPU-time is under all circumstances limited.


%\begin{figure}[H]  %here-top-bottom-page
%\begin{center}
%%\epsffile{soltest1_sub.eps}
%\epsfig{file=soltest1_sub.eps,width=12cm,height=7cm}
%\figcap{FEM solutions shown as concentration as function of x after
%3 days(upper) and as function of time for x=10 cm (lower). Different
%$P_e$-numbers have been used. The Courant number,
%$C_r$ is 1. There are wiggles for large $P_e$-numbers}
%\label{fig:soltest1_sub}
%\end{center}
%\end{figure}



\begin{figure}[H]  %here-top-bottom-page
\begin{center}
%\epsffile{soltest2_sub.eps}
\epsfig{file=soltest2_sub.eps,width=12cm,height=7.2cm}
\figcap{FEM solutions shown both as concentration as function of x
for time=3 days (upper) and as function of time with x=10 cm
(lower). Different $C_r$-numbers have been used. The P\'{e}clet
number, $P_e$ is 2 for all the simulations.  It can be seen that the
simulation with $C_r=2$ is unstable (very large wiggles for $x\geq
43$).} 
\label{fig:soltest2_sub}
\end{center}
\end{figure}

 

\begin{figure}[H]  %here-top-bottom-page
\begin{center}
%\epsffile{soltest3_sub.eps}
\epsfig{file=soltest3_sub.eps,width=12cm,height=7.2cm}
\figcap{FEM solutions shown both as concentration as function of x,
  time=3 days (upper) and as function of time with x=10 cm
  (lower). Different $P_e$ and $C_r$-numbers have been used, but
  $P_eC_r$ is equal to 10 for all the figures.}
\label{fig:soltest3_sub}
\end{center}
\end{figure}

\begin{figure}[H]  %here-top-bottom-page
\begin{center}
%\epsffile{soltest4_sub.eps}
\epsfig{file=soltest4_sub.eps,width=12cm,height=7.2cm}
\figcap{Analytical and FEM solutions shown both as concentration as
  function of x, time=3 days (upper) and as function of time with x=10 cm
(lower). For the FEM-simulations 3 different strategies are used. One
without stabilization. One with streamline diffusion, $\gamma=5$ and
lastly one with changing $\Delta t$ so $P_eC_r\leq \gamma$,
$\gamma=5$.}
\label{fig:soltest4_sub}
\end{center}
\end{figure}


%\subsubsection{Solution strategies}

%For some simulations of solute movement it is not necessary to use a
%Picard iteration loop as specified by equation \ref{eq:Picard}. These
%situations occur if none of the matrices  $\mathbf{A}$, $\mathbf{G}$
%and $\mathbf{H}$ are dependent of the solution (i.e the
%concentration). The ODE is then linear. \\
%\\
%In many cases $\mathbf{A}$, $\mathbf{G}$ or $\mathbf{H}$ matrices are
%dependent of the concentration. These situations occur if the sink
%term, chemical processes, retardation factor or boundary conditions
%are concentration dependent. 


%\begin{table}[H]
%\tabcap{Different solution strategies}
%\label{tab:solstrat}
%\begin{tabular}{p{5cm}|p{0.4cm}|p{0.4cm}|p{0.4cm}|p{0.4cm}|p{0.4cm}|p{0.4cm}|p{0.4cm}}\hline
%Solver & 1 & 2 & 3 & 4 & 5 & 6 & 7 \\ \hline
%Non-linear &  &  &  & $\bullet$ & $\bullet$ & $\bullet$ & $\bullet$  \\ \hline
%Linear & $\bullet$ & $\bullet$ & $\bullet$ &  &  &  &   \\ \hline
%Streamline diffusion &  & $\bullet$ &  &  & $\bullet$ &  & $\bullet$  \\ \hline
%$\Delta t$ is constant & $\bullet$ & $\bullet$ &  & $\bullet$ & $\bullet$ &
%&   \\ \hline
%$\Delta t_{m+1}=f(m_n)$ &  &  &  &  &  & $\bullet$ & $\bullet$  \\ \hline
%$\Delta t$ so $P_eC_r\leq \gamma$ &  &  & $\bullet$ &  &  &  &  \\ \hline
%\end{tabular}
%\end{table}

\subsection{Size of the timesteps}\index{timesteps, size of}

Like in the water movement model it is possible to choose between
solution strategies, but here it is combined with different methods to
avoid instabilities like the streamline diffusion method. The
different combinations which TopFEM can handle are given in table
\ref{tab:solstrat}.

\begin{table}[H]
%\tabcap{Different solution strategies}
%\label{tab:solstrat}
\centering 
\sbox{\mybox}{\begin{tabular}{p{5cm}|p{0.4cm}|p{0.4cm}|p{0.4cm}|p{0.4cm}|p{0.4cm}|p{0.4cm}|p{0.4cm}}\hline
Solver & 1 & 2 & 3 & 4 & 5 & 6 & 7 \\ \hline
Non-linear &  &  &  & $\bullet$ & $\bullet$ & $\bullet$ & $\bullet$  \\ \hline
Linear & $\bullet$ & $\bullet$ & $\bullet$ &  &  &  &   \\ \hline
Streamline diffusion &  & $\bullet$ &  &  & $\bullet$ &  & $\bullet$  \\ \hline
$\Delta t$ is constant & $\bullet$ & $\bullet$ &  & $\bullet$ & $\bullet$ &
&   \\ \hline
$\Delta t_{m+1}=f(m_n)$ &  &  &  &  &  & $\bullet$ & $\bullet$  \\ \hline
$\Delta t$ so $P_eC_r\leq \gamma$ &  &  & $\bullet$ &  &  &  &  \\ \hline
\end{tabular}}
\settowidth{\mylength}{\usebox{\mybox}}
\setcaptionwidth{\mylength}
\tabcap{Different solution strategies}
\usebox{\mybox}
\label{tab:solstrat}
\end{table}

For some simulations of solute movement it is not necessary to use a
Picard iteration loop as specified by equation \ref{eq:Picard}. These
situations occur if none of the matrices  $\mathbf{A}$, $\mathbf{G}$
and $\mathbf{H}$ are dependent of the solution (i.e the
concentration). The ODE is then linear. \\
\\
In many cases $\mathbf{A}$, $\mathbf{G}$ or $\mathbf{H}$ matrices are
dependent of the concentration. These situations occur if the sink
term, chemical processes, retardation factor or boundary conditions
are concentration dependent. For the non-linear solvers with varying
size of the timesteps, i.e. solver 6 and 7 in table
\ref{tab:solstrat}, $\Delta t_{n+1}=f(m_n)$ means that the size of the
timesteps is dependent on the number of Picard iterations in the previous
timestep. The iteration procedure is then almost similar to the one
from the water movement simulations:

\begin{enumerate}
\item if $m\leq 4$ then $\Delta t_{n+1} = 1.1\Delta t_n$ but not larger than
  $\Delta t_{max}$
\item if $m=5$ then $\Delta t_{n+1}=\Delta t_n$ 
\item if $6 \leq m\leq 7$ then $\Delta t_{n+1}=0.5 \Delta t_n$ but not lower
  than $\Delta t_{min}$
\item if $m>7$ then $\Delta t_{n+1}=0.1\Delta t_n$ but not lower than $\Delta
   t_{min}$
\item if $m=m_{max}$, the time is only updated if $\Delta t_{n+1}=\Delta
  t_{min}$ else it tries again with smaller timesteps  $\Delta t_n=
  0.1\Delta t_n$ but not smaller than $\Delta t_{min}$
\end{enumerate}


\subsection{Mass balance}\index{mass balance!solute}

The mass balance for solute can be defined as:


\begin{equation}
\begin{split}
\text{solute balance} & = \text{Change in stored amount of solute} \\
& + \text{Total flux out of borders} \\
& + \text{Solute removed by physical sinks} \\
& - \text{Solute produced by chemical processes}
\end{split}
\label{eq:solbal}
\end{equation}

With physical sinks is meant that the solute follows with the water
as for example some kinds of root extraction. Degradation processes
are in the mass balance calculations treated as negative
production. The solute balance index is defined as the right hand side
of equation \ref{eq:solbal} divided by 1/4 of the sum of the absolute
value of each of the terms. Ideally it is zero, except for redistribution
cases where both the numerator and the denominator equals zero. TopFEM
calculates the solute balance index after each simulation, and it is
easy to calculate the mass balance.

















