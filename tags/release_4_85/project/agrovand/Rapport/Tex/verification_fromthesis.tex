\mchapter{Verification}\index{verification}

The purpose of the present chapter is to verify that the numerical
schemes for solving Richard's equation and the advection-dispersion
equation solve the equations properly. The numerical solution of
Richard's equations is tested against a quasi-analytical solution by
Philip. The quasi-analytical solution describes the vertical
one-dimensional flow in a semi-infinite column with pre-described
pressure as upper boundary condition. The advection dispersion
equation is verified for 1D flow with both adsorption, zero and
first order chemical reactions. The implementation of a point source is
tested by radial (semi-2D) flow in a situation with diffusion and
no advection. Finally, the numerical model is tested for ``more
real'' 2D flow in an aquifer. The FEM-solution is compared with a
model in which the concentration is described with integrals which
have to be evaluated numerically. All the tests of the FEM solution of
the advection-dispersion equation used here, have unfortunately in
common, that the analytical models for the advection-dispersion
equation all presuppose a steady-state, uniform water movement. It
would have been desirable to make a test where the terms with the time
derivative of the water content in equation \ref{eq:solutemovement}
are nonzero. 



\section{Water movement}\index{verification!water movement}

It is common practise to verify numerical models which solve
Richard's equation by comparing with results from a quasi-analytical
infiltration model by Philip\index{Philip}, \cite{Philip}. This is for
example done by
\cite{Gregersen,Haverkamp,JensenSim}.
\cite{Gregersen,Haverkamp,JensenSim} use results given by Philip,
\cite{Philip,Philipinf} where Philip uses tabulated values
for the diffusivity and the conductivity for a Yolo light clay. Later,
\cite{Haverkamp} have fitted the soil data from Philip to describe the
soil-water retention and the hydraulic conductivity analytically. The
analytical descriptions are then used in their own models. \\ 
\\
In my comparison, the analytical description for the hydraulic
properties given by \cite{Haverkamp} is used in both the mathematical
model by Philip and in the FEM. I.e it is in mathematically sense
totally the same soil used for both the models. This is a more precise
method to compare the mathematical methods.\\
\\
The equations in the quasi-analytical model that have to be solved, are
solved by using numerical methods, also from Philip,
\cite{PhilipTrans,PhilipAus}. The quasi-analytical model is only described
shortly here. In the appendix both the quasi-analytical model and
the solution of it are described in detail.


\subsection{Philip�s quasi-analytical
  solution}\index{Philip!quasi-analytical solution}

Richard's equation for 1-dimensionally, vertical flow yields:

\begin{equation}
\frac{\partial \theta}{\partial t}=\frac{\partial}{\partial z}\left(
  K\frac{\partial \psi}{\partial z} \right) -\frac{\partial K}{\partial z}
\end{equation}

where $z$ is the vertical coordinate, positive downward. Using the
diffusivity\index{diffusivity}, $D=K\partial\psi/\partial \theta$ the
equation can be expressed as:

\begin{equation}
\frac{\partial \theta}{\partial t}=\frac{\partial}{\partial z} \left(
  D \frac{\partial \theta}{\partial z} \right) - \frac{\partial
  K}{\partial z}
\label{eq:diffusionphil}
\end{equation}

where it is assumed that $\psi \leq 0$. The equation will be solved
with the following initial and boundary conditions:

\begin{equation}
\theta=\begin{cases}
\theta_{n} &\text{for $t=0, \ z>0$}\\
\theta_{0} &\text{for $t\geq 0, \ z=0$} \end{cases}
\end{equation}

Equation \ref{eq:diffusionphil} is rewritten with $\theta$ as the
independent variable. By transformations, see \cite{PhilipAus} or
appendix for details, it is shown that the solution can be written
as a power series in $t^{1/2}$:


\begin{equation}
z=\phi t^{1/2} + \chi t + \psi t^{3/2} + \omega t^{2}+ \dots +
f_m(\theta)t^{m/2} + \dots   
\label{eq:tseries}
\end{equation}

where $\phi=\phi(\theta), \ \chi=\chi(\theta), \cdots$ can be calculated:

\begin{equation}
\begin{split}
& \int_{\theta_n}^{\theta}\phi \, d\theta =-2D\frac{d\theta}{d \phi} \\
& \int_{\theta_n}^{\theta}\chi \, d\theta =P\frac{d\chi}{d 
  \theta}+(K-K_n) \\
&\frac{3}{2}\int_{\theta_n}^{\theta}\psi
\,d\theta=P\frac{d\psi}{d\theta}-Q \\
& 2\int_{\theta_n}^{\theta}\omega \, d\theta =
P\frac{d\omega}{d\theta}-Q\left(2\frac{d\psi}{d
    \chi}-\frac{d\chi}{d\phi}\right), \dots
\end{split}
\label{eq:integro}
\end{equation}

where:

\begin{equation}
\begin{split}
& P(\theta)=D\left(\frac{d\theta}{d\phi}\right)^2 \\
& Q(\theta)=D\frac{d\theta}{d\phi}\left(\frac{d\chi}{d\phi}\right)^2
\end{split}
\end{equation}

which shall fulfill $\phi=\chi=\psi=\omega=\dots=0$ for
$\theta=\theta_0$. The equations are then solved numerically after
methods stated by \cite{PhilipTrans,PhilipAus}. The
solution described by equation \ref{eq:tseries} is according to
\cite{PhilipAdv} \textit{practical convergent}\index{practical
  convergence} as long as:

\begin{equation}
t\leq t_{grav}=\left(\frac{S}{K_0-K_n}\right)^2 
\end{equation}

where $S$ is the sorptivity\index{sorptivity}:

\begin{equation}
S=\int_{\theta_n}^{\theta_0}\phi \, d\theta
\end{equation}


\subsection{Infiltration into Yolo light clay}


The hydraulic conductivity and the volumetric water content can
according to \cite{Haverkamp} be described as:

\begin{equation}
K=K_s\frac{A}{A+|\psi|^{B}}, \ K_s=4.428\cdot10^{-2} \text{cm/h}, \
A=124.6, \ B=1.77
\label{eq:Kyolo}
\end{equation}

\begin{equation}
\theta=\theta_r+\frac{\alpha(\theta_s-\theta_r)}{\alpha+(\ln|\psi|)^{\beta}},
\ \theta_s=0.495, \ \theta_r=0.124, \ \alpha=739, \ \beta=4
\end{equation}

The equations are according to \cite{Haverkamp} valid as long as $\psi
\leq -1$ cm. The  upper boundary condition in the models are set to
$\psi_0$=-10 cm or $\theta_0=\theta(\psi_0)$. In figure
\ref{fig:4in1_hav} the coefficients $\phi$, $\chi$, $\psi$ and
$\omega$ are shown for the simulations. A discretization in $\theta$
on 1000 intervals has been used, see the appendix for the computation
of the quasi-analytical solution. The limit for practical convergence,
$t_{grav}$ is here found to be approximately 15 days. \\
\\



%\begin{figure}[h!]  %here-top-bottom-page
%\begin{center}
%\epsffile{4in1_hav.eps}
%\figcap{$\phi$, $\chi$, $\psi$ and $\omega$ for the Yolo light clay and
%  $\theta_0$ corresponding to $\psi_0$=-10 cm and $\theta_n$
%  corresponding to $\psi_n$=-600 cm.}
%\label{fig:4in1_hav}
%\end{center}
%\end{figure}


The FEM-model uses as standard the models of retention and
conductivity by van Genuchten/Mualem, but they are temporarily
changed to the model described by Haverkamp. The ``soil column'' is
200 cm heigh (much more than the infiltration depth). The lower
boundary condition can then without problems be set to no flow. On the
soil column a regular mesh with 400 equally large elements is
generated. A part of the mesh is shown in figure \ref{fig:1dmesh}. In
the simulation the mixed formulation with consistent matrices is
used. The length of the timesteps is 1 minute, and the backward Euler
iteration scheme with a iteration criteria, $\epsilon$=0.001 cm,
where the $\infty$-norm is taken of the change in the matrix pressure
potential between two iterations, is used. \\
\\
\begin{figure}[H]  %here-top-bottom-page
\begin{center}
%\epsffile{4in1_hav.eps}
\epsfig{file=4in1_hav.eps,width=12cm}
\figcap{$\phi$, $\chi$, $\psi$ and $\omega$ for the Yolo light clay and
  $\theta_0$ corresponding to $\psi_0$=-10 cm and $\theta_n$
  corresponding to $\psi_n$=-600 cm.}
\label{fig:4in1_hav}
\end{center}
\end{figure}

In figure \ref{fig:comparexd_hav}, the water content profiles
obtained by FEM and the quasi-analytical method are compared after 2, 4,
6, 8, 10, 12, 14 and 16 days. The dots are values from the nodal
points in the FEM-solution. It can be seen that the agreement between
the models is excellent. On day 14 and 16, respectively, a little
discrepancy can be observed, but we are also close to $t_{grav}$.


\begin{figure}[H]  %here-top-bottom-page
\begin{center}
%\epsffile{1dmesh.eps}
\epsfig{file=1dmesh.eps,width=9.9cm}
\figcap{A part of the ``one-dimensional'' mesh used for verification of
  infiltration into Yolo light clay}
\label{fig:1dmesh}
\end{center}
\end{figure}



 
\begin{figure}[H]  %here-top-bottom-page
\begin{center}
%\epsffile{comparexd_hav.eps}
\epsfig{file=comparexd_hav.eps,width=9.9cm}
\figcap{Comparison of water content profiles obtained with FEM and
  Philip�s quasi-analytical method. The results are obtained for the
  Yolo light clay after 2, 4, 6, 8, 10, 12, 14 and 16 days of
  infiltration.} 
\label{fig:comparexd_hav}
\end{center}
\end{figure}


\subsection{Infiltration into a sandy soil}

The two models are also tested against each other in a situation where
the van Genuchten/Mualem models of the hydraulic properties have been
used. During recent years these models are frequently used. A code for
calculating the infiltration with the van Genuchten/Mualem models has
been developed. For a case study, soil data are from Jyndevad,
Denmark, \cite{Jacobsen} used. The soil is classified as a JB1 (sandy
soil) after the Danish classification system. The soil can be
described with the hydraulic models by van Genuchten/Mualem. The
parameters are found to be $\theta_s=0.438$, $\theta_r=0.042$,
$\alpha=0.054$, $n=2.416$ and $K_s=64.1$ cm/h using a RETC-like
model. The initial condition is now $\psi_n$=-100 cm and the upper
boundary condition is $\psi_0$=0 or $\theta_0=\theta_s$. \\
\\
Figure \ref{fig:4in1_vanG} shows $\phi$, $\chi$, $\psi$ and $\omega$
for the simulations. By comparison with the similar graphs for the
Yolo light clay it can be observed that the magnitude of the
coefficients is different. Calculation with the model by Philip shows
that $t_{grav}$ here is approximately 6 min. \\
\\

\begin{figure}[h!]  %here-top-bottom-page
\begin{center}
\epsffile{4in1_vanG.eps}
\figcap{The coefficients, $\phi$, $\chi$, $\psi$ and $\omega$ for a
  sandy soil (JB1 after the Danish classification system) and
  $\theta_0=\theta_s$ corresponding to $\psi_0$=0 cm and $\theta_n$
  corresponding to $\psi_n$=-100 cm.} 
\label{fig:4in1_vanG}
\end{center}
\end{figure}

For the FEM-calculations the same numerical values are used as in the
previous example, except that the mesh now consists of 2000
elements. The width of the column has been changed to 0.2 cm for
keeping an acceptable mesh quality. The size of the timesteps is now
only 10 seconds. In figure \ref{fig:comparexd_vanG} the water content
profiles obtained by FEM and the quasi-analytical method are compared
after 1-8 minutes. It can be seen that the agreement between
the models generally is quite good even though the infiltration front
is very sharp. Larger discrepancies can be observed for 7 and 8
minutes where $t>t_{grav}$. \\
\\
Summarized, the verification has shown fine agreement between the
infiltration carried out with the FEM and the quasi-analytical
solution. The fine agreement was obtained with two very different
soil types.


\begin{figure}[h!]  %here-top-bottom-page
\begin{center}
\epsffile{comparexd_vanG.eps}
\figcap{Comparison of water content profiles obtained with FEM and
  Philip's quasi-analytical method. The results are obtained for a
  sandy soil (JB1 after the Danish classification system) for
  1, 2, 3, 4, 5, 6, 7 and 8 minutes.} 
\label{fig:comparexd_vanG}
\end{center}
\end{figure}

%\newpage

\section{Solute movement}\index{verification!solute movement}

A lot of analytical solutions for the 1D advection-dispersion equation
have been developed, see for example \cite{Genuchtenanalytical}. The
equations are developed for situations where the diffusion is constant
and the water flow is at a steady state. These conditions are seldom
fulfilled in ``real life'' where both the water content and the flux
are time-dependent. For testing the solute transport model, situations
with steady state water movement are simulated. Both for one and
two-dimensional flows.



\subsection{One-dimensional flow with retardation, production and
  degradation}

In \cite{Genuchtenanalytical} cases of degradation of both zero
and first order are considered. The governing differential equation can
then be expressed as:

\begin{equation}
R\frac{\partial C}{\partial t}=D\frac{\partial^2C}{\partial
  x^2}-v\frac{\partial C}{\partial x}-\mu C +\gamma
\end{equation}

where $\mu$ is the rate constant of first order decay in the liquid
and $\gamma$ represents the similar rate constant for zero-order
production in the liquid phase. The initial and boundary conditions
are :

\begin{equation}
C(x,0)=C_i
\end{equation}

\begin{equation}
(-D\frac{\partial c}{\partial x}+vc)|_{x=0}=\begin{cases} vC_0 &
  0<t\leq t_0 \\ 0 & t>t_0 \end{cases}
\end{equation}

The solution can be written as: 

\begin{equation}
\begin{small}
C=\begin{cases}\frac{\gamma}{\mu}+ (C_i-\frac{\gamma}{\mu})A(x,t)+(C_0-\frac{\gamma}{\mu})B(x,t), 0<t\leq t_0 \\
\frac{\gamma}{\mu} + (C_i-\frac{\gamma}{\mu})A(x,t)+(C_0-\frac{\gamma}{\mu})B(x,t)-C_0B(x,t-t_0), t>t_0 
\end{cases}
\end{small}
\end{equation}

where:


\begin{equation}
\begin{split}
A(x,t) & =\exp(-\mu t/R)\cdot \\
& \{ 1-\frac{1}{2}\erfc\left[\frac{Rx-vt}{2(DRt)^{1/2}}\right]
-\left(\frac{v^2t}{\pi
    DR}\right)^{1/2}\exp\left[-\frac{(Rx-vt)^2}{4DRt}\right] \\ &
+\frac{1}{2}\left(1+\frac{vx}{D}+\frac{v^2t}{DR}\right)\exp(vx/D)\erfc\left[\frac{Rx+vt}{2(DRt)^{1/2}}\right]
\}
\end{split}
\end{equation}

\begin{equation}
\begin{split}
B(x,t)=&\frac{v}{v+u}\exp\left[\frac{(v-u)x}{2D}\right]\erfc\left[\frac{Rx-ut}{2(DRt)^{1/2}}\right]
+\\
& 
\frac{v}{v-u}\exp\left[\frac{(v+u)x}{2D}\right]\erfc\left[\frac{Rx+ut}{2(DRt)^{1/2}}\right]
+\\ &
 \frac{v^2}{2\mu D}\exp\left[\frac{vx}{D}-\frac{\mu
     t}{R}\right]\erfc\left[\frac{Rx+vt}{2(DRt)^{1/2}}\right]
\end{split}
\end{equation}

where:

\begin{equation}
u=v \sqrt{1+\frac{4\mu D}{v^2}}
\end{equation}

For comparing the analytical solution with the FEM-solution is chosen
a scenario with $R$=2, $v$=10 $\text{cm\,day}^{-1}$, $D=5
\text{m}^2\text{day}^{-1}$, $\gamma=0.2 \, \text{day}^{-1}$ and
$\mu=0.5\, \text{day}^{-1}$. \\
\\

For the FEM-modeling a 100 cm long soil column with a width
of 1 cm is used. On the geometry a regular mesh consisting
of 200 elements is generated. For computing the FEM-solution of the
advection-dispersion equation, a Non-linear solver is used with
constant size of the time-steps, $\Delta t$ on 1/10 day. The matrices
are chosen to be consistent and $\omega=1/2$ (Crank-Nicolson) has been
used. Setting $\mu$ and $\gamma$ to the above mentioned values is
equivalent to setting $\mu_l=-0.5C+0.2$ in the FEM-model. For lower
boundary condition, the solute flux is set to be equal $qC$. In
figure \ref{fig:solanalyt}, the FEM-solution is compared with the
analytical solution. As it can be seen, the two solutions almost
coincide. 


\begin{figure}[h!]  %here-top-bottom-page
\begin{center}
\epsffile{solanalyt.eps}
%\epsfig{file=solanalyt,width=12cm,height=5.9cm}
\figcap{Comparison between analytical and FEM-simulation of a
process with adsorption, zero order production and first order
degradation.}
\label{fig:solanalyt}
\end{center}
\end{figure}

\subsection{Radial flow from continuous point
  source}\index{point source!verification}


The equation for heat flow by conduction is equivalent to the
equation for solute movement by diffusion. In \cite{Carslaw}
an analytical solution is given for the continuous line source in a
three-dimensional space where heat is liberated at the rate $J$
parallel to the $z$-axis. In a 2D space it is
equivalent to a point source. If solute is liberated at a rate of $J$
at the origin, and the concentration initially in the infinitely
large space is 0, the solution is:


\begin{equation}
C=\frac{J}{4\pi D}E_1\left(\frac{r^2}{4Dt}\right)
\end{equation}

where $r^2=x^2+y^2$ is the distance from point source squared. The
equation is only valid as long as the water flow is zero (no
advection). The exponential integral, $E_1(x)$ is in \cite{Abramowitz}
defined as:


\begin{equation}
E_1(x)=\int_{x}^{\infty}\frac{e^{-u}}{u}\, du 
\end{equation}

The situation is modeled on a circular domain with a radius of 10
m. The boundary condition is no flux. This is of course not the same
as an infinitely large circle which the equation covers, but as long
as the concentrations near the boundary are ``small'' the
approximation should be acceptable. For the simulation
$D=1\,\text{m}^2/\text{s}$ . Figure \ref{fig:pointmesh} shows the mesh
used for the simulation which consists of 1281 elements and 2464 nodal
points. 


\begin{figure}[H]  %here-top-bottom-page
\begin{center}
%\epsffile{pointmesh.eps}
\epsfig{file=pointmesh,width=10cm}
\figcap{Mesh used for test of a continuous point source}
\label{fig:pointmesh}
\end{center}
\end{figure}


In figure \ref{fig:pointsubplot_3d}, both the FEM and the
analytical solution are given for three different times ($t$=60, 120 and
180 hours). The scale for the concentration is between 0 and 1.2 even
though the concentrations near the centre in the analytical
solution are higher. The peak near the centre is very steep in the
analytical solution - the actual concentration in the centre is
infinite for $t>0$. The used plot routine cannot plot the value of
infinity, so the plot does not show the peak placed in the centre.\\
\\
In figure \ref{fig:pointerrsubplot}, both the absolute and the
relative error of the FEM-solution are given at the same three times as
used in figure \ref{fig:pointsubplot_3d}. It can be seen that the
observed absolute errors generally are very small and that the largest
absolute errors at the three times are concentrated near the
centre. The relative errors are generally low except near the centre
and the boundary. It should be mentioned that that very small insignificant
absolute errors can give large relative errors. The negative relative
errors At $t$=120 and 180 hours around the border can be explained
with the boundary condition used for the FEM-simulations which
prevents that the contaminant flows out from the domain. 


\begin{figure}[H]  %here-top-bottom-page
\begin{center}
\epsffile{pointsubplot_3d.eps}
\figcap{FEM and analytical solutions for continuous point source in
  the origin at 3 different times. FEM-solutions to the left and the
  analytical solutions to the right.}
\label{fig:pointsubplot_3d}
\end{center}
\end{figure}


%\begin{figure}[h!]  %here-top-bottom-page
%\begin{center}
%\epsffile{pointerrsubplot_3d.eps}
%\tabcap{Absolute and relative errors for FEM model at 3
%  different times.}
%\label{fig:pointerrsubplot}
%\end{center}
%\end{figure}

\begin{figure}[H]  %here-top-bottom-page
\begin{center}
\epsffile{pointerrsubplot.eps}
\figcap{Absolute and relative errors for the FEM-simulations at 3
  different times, seen from above.}
\label{fig:pointerrsubplot}
\end{center}
\end{figure}

It can be concluded that the point source is implemented in the proper
way. An instantaneous point source has also been tested by liberating a
given mass of solute in a single time-step. Analytical solutions for
this situation are for example given in \cite{Spitz}. Also here the
results are excellent. 


\subsection{Two-dimensional flow}

Numerous model for 2D solute movement in an aquifer\index{aquifer,
  flow in} has
been developed. The analytical expressions often contains parts witch in practical use must be
evaluated numerically or by use of (large) mathematical tables. Here,
the FEM-model is  compared with a rather complicated analytical
solution found by \cite{Shan}. The expressions here contains some
integrals which are evaluated numerically. \\
\\
\cite{Shan} have developed analytical solutions for modeling the solute
transport in a vertical section of a homogeneous, isotropic aquifer,
see figure \ref{fig:Shanfig1}.


\begin{figure}[H]  %here-top-bottom-page
\begin{center}
\epsffile{Shanfig1.eps}
\figcap{Schematic diagram of the section. After figure in \cite{Shan}}
\label{fig:Shanfig1}
\end{center}
\end{figure}


The flow is horizontal and uniform with an average pore velocity,
$v$. The contaminant is applied along a width of $l$ with constant
flux, $j$ at the upper boundary (the water table). The chosen
coordinate system has the $x$-axis placed at the water table in the
top of the aquifer parallel with the streamlines for the water flow,
and the $z$-axis is vertically downward, see figure \ref{fig:Shanfig1}
where the aquifer also is divided into three regions. The flow
situation described here is maybe not completely coincident with the
flow region which this thesis deals with (the topsoil). But similar
situations with nearly horizontal water movement can for example be
found near rivers. Under all circumstances, such a flow situation is a
good way to verify the numerical model. \\ 
\\
The aquifer in the expressions shown here is assumed to be infinite
both in depth and length. %The concentration gradient, $\partial
%C/\partial z$ through the water table is assumed to be zero, i.e there
%is no flux across the upper boundary.\\
%\\
\cite{Shan} have used dimensionless variables, here marked with $'$:

\begin{equation}
\begin{split}
&C'=C/C_0, \ \ x'=\frac{x}{l},\ \ z'=\frac{z}{l}, \ \ t'=\frac{vt}{Rl}\\
&\alpha'_x=\frac{\alpha_x}{l}, \ \ \alpha'_z=\frac{\alpha_z}{l}, \ \
j'=\frac{j}{v}, \ \ \lambda'=\frac{Rl\lambda}{v}
\end{split}
\end{equation}

The dimensionless equation for all the three regions can be written as: 

\begin{equation}
\frac{\partial C'}{\partial t'}+\lambda C'+\frac{\partial C'}{\partial
  x'}-\alpha'_{x}\frac{\partial^2 C'}{\partial
  {x'}^2}-\alpha'_z\frac{\partial^2 C'}{\partial {z'}^2}=0
\end{equation}

The original equation can by backward calculation be expressed as:

\begin{equation}
R \frac{\partial C}{\partial t}+R\lambda C
-\alpha_xv\frac{\partial^2 C}{\partial x^2}-\alpha_z\frac{\partial^2
  C}{\partial z^2}=0
\label{eq:advdispShan}
\end{equation}

This is a little different from the advection-dispersion equation used in
TopFEM (and other places) because the degradation term includes a $R$,
else it is the same.\\
\\

The solution to the dimensionless equation in the 3 regions is
according to \cite{Shan}: 

\begin{equation}
\begin{split}
C'_1=& C'_{\lambda}-\frac{j'}{2\sqrt{\pi\alpha'_z}} \cdot \\
& \int_{0}^{t'} e^{-z^2/(4\alpha'_z\tau)-\lambda'
  \tau} \left[\erfc\left(\frac{\tau-x'}{2\sqrt{\alpha'_x\tau}}\right)+\erfc\left(\frac{x'+1-\tau}{2\sqrt{\alpha'_x\tau}}\right)\right]\frac{d
  \tau}{\sqrt{\tau}}
\end{split}
\label{eq:Cint_1}
\end{equation}

\begin{equation}
\begin{split}
C'_2=&\frac{j'}{2\sqrt{\pi\alpha'_z}} \cdot \\ & \int_{0}^{t'}e^{{-z'}^2/(4\alpha'_z\tau)-\lambda'
  \tau}\left[\erfc\left(\frac{x'-\tau}{2\sqrt{\alpha'_x\tau}}\right)-\erfc\left(\frac{x'+1-\tau}{2\sqrt{\alpha'_x\tau}}\right)\right]\frac{d
  \tau}{\sqrt{\tau}}
\end{split}
\label{eq:Cint_2}
\end{equation}

\begin{equation}
\begin{split}
C'_3= & \frac{j'}{2\sqrt{\pi\alpha'_z}} \cdot \\ & \int_{0}^{t'}e^{{-z'}^2/(4\alpha'_z\tau)-\lambda'
  \tau}\left[\erfc\left(\frac{\tau-x'-1}{2\sqrt{\alpha'_x\tau}}\right)-\erfc\left(\frac{\tau-x}{2\sqrt{\alpha'_x\tau}}\right)\right]\frac{d
  \tau}{\sqrt{\tau}}
\end{split}
\label{eq:Cint_3}
\end{equation}

\begin{equation}
\begin{split}
C'_{\lambda}= & \frac{j'}{2\sqrt{\lambda' \alpha'_z}}\cdot \left[e^{-z'\sqrt{\lambda /\alpha'_z}} \erfc\left(\frac{z'}{2\sqrt{\alpha'_xt'}}-\sqrt{\lambda'
    t'}\right)\right] - \\ &  \frac{j'}{2\sqrt{\lambda' \alpha'_z}}\cdot \left[
e^{z'\sqrt{\lambda/\alpha'_z}}\erfc\left(\frac{z'}{2\sqrt{\alpha'_z
    t'}}+\sqrt{\lambda' t'}\right)\right]
\end{split}
\label{eq:C_lambda}
\end{equation}

for $\lambda \neq 0$. For $\lambda=0$:

\begin{equation}
C'_{\lambda}=\frac{j' }{\sqrt{\alpha'_z}}
\left(\frac{2\sqrt{t'}}{\sqrt{\pi}}e^{{-z'}^2/(4\alpha'_zt')}-\frac{z'}{\sqrt{\alpha'_z}}\erfc\left(\frac{z'}{2\sqrt{\alpha'_zt'}}\right)\right)
\label{eq:C_lambda0}
\end{equation}

For evaluation of the integrals is used the MATLAB5 built-in function,
\textsf{quad8}\index{quad8@\textsf{quad8}}. \textsf{quad8},
\cite{MATLABfuncref} uses an adaptive recursive Newton-Cotes 8 panel
rule, and is not set up to handle integrable singularities such as: 


\begin{equation}
\int_{0}^{1}\frac{1}{\sqrt{x}}\, dx
\end{equation}

In equation \ref{eq:Cint_1}-\ref{eq:Cint_3} there can be singularities
for $\tau \rightarrow 0$ since $1/\sqrt{\tau} \rightarrow \infty$. The
complementary error-function, $\erfc$ is always bounded between 0 and
2. If $z=0$, $exp(-z^2/(4\alpha_z\tau)-\lambda\tau)=exp(-\lambda
\tau)$ else the exponential function dominates $1/\sqrt{\tau}$
and then there are no singularities under the integration signs.\\
\\
For $z=0$ there are integrable singularities, and the value of the
integration in the region near the singularity cannot be computed by
\textsf{quad8}. Instead, a very rough approximation is used close to
the singularity:

\begin{equation}
\int_{0}^{\Delta}\frac{f(u)}{\sqrt{u}}\, du\approx
=\frac{1}{2}(f(0)+f(\Delta))\int_{0}^{\Delta}\frac{f(u)}{\sqrt{u}}\,
du\approx=(f(0)+f(\Delta))\sqrt{\Delta}
\end{equation}

where $\frac{1}{2}(f(0)+f(\Delta))$ is an approximation to the mean
value of $f$ in the interval from 0 to $\Delta$. \\
\\
In the simulation in TopFEM the domain, $\Omega$ must be limited, i.e
the lower horizontal boundary can not be placed at infinity. But as
long as the solute concentrations are zero or very low near the chosen
lower boundary it should not give rise to discrepancies between the
solutions.\\
\\
The situation with horizontal steady, uniform water flow in the
direction of the $x$-axis can easily be achieved by introducing a
negative horizontal pressure gradient on $\partial \psi/\partial x=-1$
and make sure that the soil is saturated in the whole domain. Actually,
it is impossible to ensure both a horizontal water table (with
$\psi=0$) and a steady state uniform vertical flow at the same
time - i.e the assumptions for using the theory by \cite{Shan} can not
be fulfilled!. Here the matric pressure potential at the upper
boundary is not equal to zero. The velocity should be (and is) equal to the
saturated hydraulic conductivity, $K_s$ which for the simulations is
chosen to be equal to 1 m/day.  \\ 
\\
The boundary conditions for the advection-dispersion equation in the
FEM simulations are no flux on all the boundaries, except the upper
boundary of region 1 where the flux is $-j$, and the vertical right
boundary where it is $qC$ (the flux is defined positive outwards in
the numerical model). The mesh generated with FEMLAB is shown in
figure \ref{fig:Shanmesh_rough}. The mesh consists of 1008 elements
and 549 nodal points. \\
\\
In the simulations the retardation factor, $R$  equal to 2 and
$\lambda=0.01$ ($\lambda$ as used in \ref{eq:advdispShan}) which
corresponds to $\mu_l=-0.02C$ in the FEM-model. An iterative
Crank-Nicolson procedure ($\omega=0.5$) without streamline
diffusion. $\Delta t$ is chosen to 1/24 day and $\epsilon=0.001$ has
been used. Consistent matrices (no lumping) have been used.

\newpage 

\begin{figure}[H]  %here-top-bottom-page
\begin{center}
\epsffile{Shanmesh_rough.eps}
\figcap{Mesh used for 2D solute movement calculations in an aquifer.}
\label{fig:Shanmesh_rough}
\end{center}
\end{figure}

In figure \ref{fig:Shansubplot_rough}, both the FEM-solutions and the
analytical solutions are shown for $t$=300, 600 and 900 hours. It can be
seen that there is no evident difference between the FEM-solution and the
analytical solution. 

\begin{figure}[H]  %here-top-bottom-page
\begin{center}
%\epsffile{Shansubplot_rough.eps}
\epsfig{file=Shansubplot_rough.eps,width=13.5cm}
\figcap{FEM-simulation of 2D solute movement (left) compared with the
  analytical solutions (right).}
\label{fig:Shansubplot_rough}
\end{center}
\end{figure}

In figure \ref{fig:Shansuberrplot_rough}, both the
absolute and the relative error is given for the same times. Here the
semi-analytical by \cite{Shan} is assumed to provide results very close
to the true solution, even numerical errors can occur
here. Under the computation of the integrals, very fine error
tolerances are used, but some errors can occur close to the singularities
since the approximation in these intervals are very rough. As it can be
seen, the absolute errors are small and have highest density (absolute
values) in region 1 (see figure \ref{fig:Shanfig1}) where also the solute
concentration is high. At $t$=900 hours some absolute errors at
the vertical right boundary can also be seen. This can be explained
by the fact that the the solute only can be transported through the
vertical boundary by advection in the FEM-model. The problem could
probably have been avoided by choosing a boundary more to the
right. By comparing figure \ref{fig:Shansubplot_rough} with figure
\ref{fig:Shansuberrplot_rough}, it can be seen that the relative
errors are only high where the concentrations are very low. 

\begin{figure}[H]  %here-top-bottom-page
\begin{center}
%\epsffile{Shansuberrplot_rough.eps}
\epsfig{file=Shansuberrplot_rough.eps,width=13.5cm}
\figcap{Absolute (left) and relative (right) errors of the
  FEM-simulation of solute movement. The errors are shown at 3
  different times.}
\label{fig:Shansuberrplot_rough}
\end{center}
\end{figure}

The simulations have also been conducted with a very fine mesh, see
figure \ref{fig:Shanmesh_fine}. The mesh consists of 2105 nodal points
and 4032 elements, i.e. 4 times as many elements as in the previous
simulations. Each element in the previous mesh is divided
into 4 new triangles (elements).

\newpage



\begin{figure}[H]  %here-top-bottom-page
\begin{center}
\epsffile{Shanmesh_fine.eps}
\figcap{Mesh used for calculation of 2D solute movement in an
  aquifer. The mesh consists of 4 times (4032) as many elements as the
  mesh in figure \ref{fig:Shanmesh_rough}.}
\label{fig:Shanmesh_fine}
\end{center}
\end{figure}

At first the concentrations look similar to the results obtained with
the rough mesh. But figure \ref{fig:Shansuberrplot_fine} shows that
both the absolute and relative errors are smaller than for the rougher
mesh. \\
\\
It can be concluded that the verification has shown that the
implemented FEM-model in all the cases has been able to calculate the
2D solute movement satisfactorily.  



\begin{figure}[H]  %here-top-bottom-page
\begin{center}
%\epsffile{Shansuberrplot_fine.eps}
\epsfig{file=Shansuberrplot_fine.eps,width=13.5cm}
\figcap{Absolute (left) and relative (right) errors of FEM-simulation
  of solute movement. A very fine mesh has been used. The figure can be
 compared with figure \ref{fig:Shansuberrplot_rough} where the errors
 for an equivalent simulation with a rougher mesh (1/4 so many
 elements) are shown. The errors are shown at 3 different times.}
\label{fig:Shansuberrplot_fine}
\end{center}
\end{figure}

















