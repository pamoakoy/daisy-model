\chapter{Results}

In this chapter, dynamic measurements are compared to simulated
results.  We have chosen to present all the measured soil and drain
data we received, even those that for some reason or another have not
been considered in the calibration process.  Data regarding dynamic
crop growth is not presented.  Static data used for the initial setup
(soil physics) and dynamic data used to drive the simulation (weather,
groundwater pressure, and crop management) are presented in
chapter~\ref{cha:setup}.  Daisy will calculate a lot of additional
information, which is useless for validation purposes, but can be
important for interpretation of the results.  We have chosen to put
what we consider the most important of such data (regarding deep
leaching, colloids, biopores and 2D movement) in
appendix~\ref{app:col-biopores} and~\ref{app:plot-2d}.  The data
presented in this chapter fall in two broad categories: measurements
of water and solutes within the soil, and measurements of water and
solutes in the drains.  The measurement points referred to throughout
this chapter can be found in \citet{vap2009}.

\section{Soil}

\subsection{TDR measurements}

Plauborg from the Faculty of Agricultural Sciences, Aarhus University,
were responsible for the TDR measurements.  The data was provided by
Annette E.\@ Rosenbom from GEUS\@.  Soil water content was measured at
both sites using horizontal TDR probes located at the lowest corner of
field.  At Estrup (figure~\ref{fig:Estrup-theta}) we only have data
for 25 cm, at Silstrup (figure~\ref{fig:Estrup-theta}) we have for 25,
60 and 110 cm below soil surface.  The 110 cm probe values show two
distinct curves when plotted as points rather than lines.  The
variation on the 60 cm probe seem to bear little relationship to the
seasons.  The 25 cm probes at both sites are a better match for our
expectations.  The ability of the crop to dry out the soil is larger
than the simulated at both sites.  Also, the simulated high (winter)
level at Silstrup is slightly above the measured high level.

In general, we didn't want to calibrate our soil physics based on
these measurements (e.g.\ by lowering the porosity of the Silstrup Ap
horizon), as the soil physics were based on distributed samples from
the field, and as such more likely to be representative of the field
as a whole, than the TDR measurements.  However, as the bromide
leaching data for Silstrup also lead us to believe that we
underestimated the crop ability to extract water from the top horizon
(containing most of the bromide during the summer), two changes were
made.  The residual water of the B horizon was set to 8\% (up from 0),
and the crop was calibrated so that most of the roots would be
concentrated in the Ap horizon.  See section~\ref{sec:cal-primary} and
section~\ref{sec:crop-man}.

\begin{figure}[htbp]
  \begin{center}
    \figtop{Silstrup-theta-SW025cm}\\
    \figtop{Silstrup-theta-SW060cm}\\
    \fig{Silstrup-theta-SW110cm}
  \end{center}
  \caption{Silstrup soil water content for measurement point S1.}
  \label{fig:Silstrup-theta}
\end{figure}

\begin{figure}[htbp]
  \begin{center}
    \fig{Estrup-theta-SW025cm}
  \end{center}
  \caption{Estrup soil water content for measurement point S1.}
  \label{fig:Estrup-theta}
\end{figure}

\FloatBarrier
\subsection{Suction cups and horizontal filters}

Bromide and pesticide concentration in soil water were measured with
small suction cells one meter below surface, in the same part of the
field as where the TDR's were installed, and 3.5 meter below surface
within large horizontal filters.  The suction cup measurements are
unlikely to be representative for the field as a whole, due to the
large heterogeneity observed.  The horizontal filters, on the other
hand, are placed downstream in the expected general direction of
groundwater flow, and should thus more likely represent the entire
field.

As Daisy keep separate track of solutes in small and large pores (see
section~\ref{sec:fast-slow}), and it is likely that the suction cups
will predominately extract water from the large pores, we have
provided simulation results for concentration in large pores alone, as
well as concentration in total soil water.  Simulated and measured
bromide in both suction cells and filters are shown for Silstrup on
figure~\ref{fig:Silstrup-bromide} and Estrup on
figure~\ref{fig:Estrup-bromide}.  The simulated values for 1 meter are
well within the variation shown by the the suction cups.  The
measurements does hint that the first bromide should arrive earlier
though, especially in Silstrup.  The concentration in the large pores
compared to average does not change this picture.  Variation between
the two is short lived at the time scale of the graphs.  For 3.5
meter, the simulation is still within the general variation, however
the filters clearly show that some bromide find its way to 3.5 meter
very fast (two months after application).

We did not get pesticide measurement data for 3.5 meter depth in time
for this report, but none were above the detection limit anyway.  This
fit well with the simulated results shown on
figure~\ref{fig:pest-horizontal}.

\begin{figure}[htbp]
  \begin{center}
    \figtop{Silstrup-sc-bromide}\\
    \fig{Silstrup-Bromide-horizontal}
  \end{center}
  \caption{Silstrup soil bromide content at 1.0 m depth (top) and 3.5
    m depth (bottom).  Sim (avg) is the average simulated
    concentration, Sim (fast) is the simulated concentration in the
    large (fast) pores.  S1 and S2 are suction cup measurements.
    H$n$.$m$ refer to measured values in different sections of
    horizontal filters.}
  \label{fig:Silstrup-bromide}
\end{figure}

\begin{figure}[htbp]
  \begin{center}
    \figtop{Estrup-sc-bromide}\\
    \fig{Estrup-Bromide-horizontal}
  \end{center}
  \caption{Estrup soil bromide content at 1.0 m depth (top) and 3.5 m
    depth (bottom).  Sim (avg) is the average simulated concentration,
    Sim (fast) is the simulated concentration in the large (fast)
    pores.  S1 and S2 are suction cup measurements.  S2 is noted by
    GEUS as unreliable.  H$1$.$m$ refer to measured values in
    different sections of horizontal filters.}
  \label{fig:Estrup-bromide}
\end{figure}

\begin{figure}[htbp]
  \begin{center}
    \figtop{Silstrup-horizontal}
    \fig{Estrup-horizontal}
  \end{center}
  \caption{Pesticide concentration in soil water at 3.5 meters depth
    for Silstrup (top) and Estrup (bottom).  The simulated values for
    Estrup are in the order of femtograms (10$^{-15}$ g) per hectare,
    and not visible on a nanogram (10$^{-9}$ g) per hectare scale.}
  \label{fig:pest-horizontal}
\end{figure}

\FloatBarrier
\section{Drain}

Drain water flow was measured continuesly, GEUS provided daily values.
The measurements of bromide and pesticides were done using a mixture
of two sampling methods.  The first is time proportional sampling
where samples are taken at specific time intervals.  The other is flow
proportional sampling, where samples are taken with intervals
proportional to the amount of water flow in the drains.  GEUS has
combined the two into a ``best estimate'' of the total weekly flow,
which is what we have used for calibration.

The water and bromide drain data was provided by Annette E.\@ Rosenbom
from GEUS, with Ruth Grant from DMU, Aarhus University as the
responsible scientist.  The pesticide data was provided by Jeanne
Kj{\ae}r from GEUS.

\subsection{Water}

Calibrating the simulated total drain flow over the two seasons is
``just'' a question of picking the right offset for the measured
ground water pressure (see section~\ref{sec:gwt}).  Getting the length
of the drain seasons right is trickier, and involves calibrating the
soil physics.  Drain flow for Silstrup is shown on
figure~\ref{fig:Silstrup-drain} and for Estrup on
figure~\ref{fig:Estrup-drain}.  For Silstrup the drain season length
is right the first year, but the distribution is more even in the
simulation, compared to the measurements where the flow almost
directly follows the precipitation.  For the second season, the
simulation underestimate water flow at the start of the season, and
compensate by overestimating at the end of the season.  For Estrup we
got an overall good match both seasons, slightly underestimating the
drain flow at the beginning of the first season, while overestimating
the drain flow at the beginning of the second season.

\begin{figure}[htbp]
  \begin{center}
    \figtop{Silstrup-drain}\\
    \fig{Silstrup-drain-acc}
  \end{center}
  \caption{Silstrup drain flow, daily values and accumulated.}
  \label{fig:Silstrup-drain}
\end{figure}

\begin{figure}[htbp]
  \begin{center}
    \figtop{Estrup-drain}\\
    \fig{Estrup-drain-acc}
  \end{center}
  \caption{Estrup drain flow, daily values and accumulated.}
  \label{fig:Estrup-drain}
\end{figure}

\FloatBarrier
\subsection{Bromide and metamitron}

Bromide was a challenge to get right, especially for Silstrup, as
described in section~\ref{sec:cal-silstrup-bromide}.  For Silstrup
(figure~\ref{fig:Silstrup-weekly} and~\ref{fig:Silstrup-bromide-acc})
we get a good match the first year, but the second year the dynamics
are off even if the total amount is right.  The poor second year
dynamics for bromide likely reflects the poor second year dynamics for
water.  For Estrup (figure~\ref{fig:Estrup-bromide-drain}), we
underestimate both the initial leaching the first season, and the
leaching the entire second season.

Metamitron is one of the two pesticides we have interesting data for,
unfortunately only for one site.  By increasing the \kd{} parameter to
the largest value we could defend by literature values (see
section~\ref{sec:cal-metamitron}) we were able to get a good match
with both weekly (figure~\ref{fig:Silstrup-weekly}) and accumulated
(figure~\ref{fig:Silstrup-bromide-acc}) measured values.  The
accumulated values may seem off, but that is only due to two weeks
where the majority of leaching in the simulation occurs, but where the
measured drain water were not analyzed for metamitron.

\begin{figure}[htbp]
  \begin{center}
    \figtop{Silstrup-Bromide-weekly}\\
    \fig{Silstrup-Metamitron-weekly}
  \end{center}
  \caption{Silstrup weekly drain transport of bromide and metamitron.}
  \label{fig:Silstrup-weekly}
\end{figure}

\begin{figure}[htbp]
  \begin{center}
    \figtop{Silstrup-Bromide-acc}\\
    \fig{Silstrup-Metamitron-acc}
  \end{center}
  \caption{Silstrup accumulated drain transport of bromide and metamitron.}
  \label{fig:Silstrup-bromide-acc}
\end{figure}

\begin{figure}[htbp]
  \begin{center}
    \figtop{Estrup-Bromide-weekly}\\
    \fig{Estrup-Bromide-acc}
  \end{center}
  \caption{Estrup weekly and accumulated drain transport of bromide.}
  \label{fig:Estrup-bromide-drain}
\end{figure}

\FloatBarrier
\subsection{Glyphosate,  fenpropimorph, and dimethoate}

The second interesting pesticide is glyphosate, here presented
together with fenpropimorph and dimethoate.  As can be seen on
figure~\ref{fig:Silstrup-acc} and figure~\ref{fig:Estrup-acc} we get
the total glyphosate amount right for both sites.  The weekly numbers
show that the dynamics is also reasonable for Silstrup
(figure~\ref{fig:Silstrup-weekly2}), but that the simulation
underestimate the later leaching at Estrup
(figure~\ref{fig:Estrup-weekly}).  The early Silstrup simulated
results required a lot of focus on surface processes (see
section~\ref{sec:cal-silstrup-surface}), while the late values are a
result of adjusting the pesticide sorption model (see
section~\ref{sec:cal-glyphosate}).  No (additional) adjustment where
made for Estrup.

There is a single measurement at the detection limit of dimethoate at
Silstrup.  The simulation has three spikes at roughly the same size,
one of them matching the detection.  There are no measurements of
dimethoate above detection limit at Estrup, and none for fenpropimorph
at either site.  The simulation results are in agreement with this, as
the two large spikes simulated at Silstrup both occur before the
measured drain water is analyzed fenpropimorph.  For both pesticides,
the accumulated simulated values continue to grow in periods where the
accumulated measured values are constant, this is because the
simulation doesn't operate with a detection limit.

\begin{figure}[htbp]
  \begin{center}
    \figtop{Silstrup-Dimethoate-weekly}\\
    \figtop{Silstrup-Fenpropimorph-weekly}\\
    \fig{Silstrup-Glyphosate-weekly}
  \end{center}
  \caption{Silstrup weekly drain transport of selected pesticides.}
  \label{fig:Silstrup-weekly2}
\end{figure}

\begin{figure}[htbp]
  \begin{center}
    \figtop{Silstrup-Dimethoate-acc}\\
    \figtop{Silstrup-Fenpropimorph-acc}\\
    \fig{Silstrup-Glyphosate-acc}
  \end{center}
  \caption{Silstrup accumulated drain transport of selected pesticides.}
  \label{fig:Silstrup-acc}
\end{figure}

\begin{figure}[htbp]
  \begin{center}
    \figtop{Estrup-Dimethoate-weekly}\\
    \figtop{Estrup-Fenpropimorph-weekly}\\
    \fig{Estrup-Glyphosate-weekly}\\
  \end{center}
  \caption{Estrup weekly drain transport of selected pesticides.}
  \label{fig:Estrup-weekly}
\end{figure}

\begin{figure}[htbp]
  \begin{center}
    \figtop{Estrup-Dimethoate-acc}\\
    \figtop{Estrup-Fenpropimorph-acc}\\
    \fig{Estrup-Glyphosate-acc}\\
  \end{center}
  \caption{Estrup accumulated drain transport of selected pesticides.}
  \label{fig:Estrup-acc}
\end{figure}


