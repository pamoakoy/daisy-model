\documentclass[a4paper,11pt,twoside]{article}
\usepackage{a4}
\usepackage[T1]{fontenc}
\usepackage[latin1]{inputenc}
\usepackage{hyperref}
%\usepackage{times}
\usepackage{palatino}

% Larger pages.
\addtolength{\textwidth}{4cm}
\addtolength{\hoffset}{-2cm}
\addtolength{\oddsidemargin}{1cm}
\addtolength{\evensidemargin}{-1cm}

\bibliographystyle{apalike}
\newcommand{\daisy}{Daisy}
\newcommand{\Daisy}{Daisy}
\newcommand{\cplusplus}%
{{\leavevmode{\rm{\hbox{C\hskip -0.1ex\raise 0.5ex\hbox{\tiny ++}}}}}}
\newcommand{\Cplusplus}{\cplusplus}
\newcommand{\mshe}{Mike/\textsc{she}}
\newcommand{\wintel}{\texttt{win32}}
\newcommand{\dll}{\textsc{dll}}
\newcommand{\Dll}{\textsc{Dll}}
\newcommand{\gui}{\textsc{gui}}
\newcommand{\Gui}{\textsc{Gui}}
\newcommand{\unix}{Unix}
\newcommand{\dhi}{\textsc{dhi}}
\newcommand{\Dhi}{\textsc{Dhi}}
\newcommand{\api}{\textsc{api}}
\newcommand{\Api}{\textsc{Api}}
\newcommand{\lai}{\textsc{lai}}
\newcommand{\Lai}{\textsc{Lai}}
%\newcommand{\url}[1]{\linebreak[4]\texttt{<URL:#1>}}

%%% Local Variables: 
%%% mode: latex
%%% TeX-master: t
%%% End: 


% External references.
\usepackage{xr}
\externaldocument[tut-]{tutorial}
\externaldocument[ref-]{reference}

% Constant references.
\newcommand{\daisytut}{\textit{\Daisy{} Tutorial}}
\newcommand{\daisyref}{\textit{\Daisy{} Program Reference Manual}}

% Easy of edit more important than ease of reading.
\sloppy

\begin{document}

\section*{Data Requirements for the \Daisy{} model}

\Daisy{} requires data about the climate, the management and the soil
in order to run a simulation.  \Daisy{} is to some degree able to
adjust to the available data, either by using simpler models
internally when less data is available, or by trying to synthesize the
missing data from what is available.  In the following, we will
describe the minimum data requirements for running \daisy{} at all,
the data we recommend for reasonable simulation results, and finally
list some of the optional data that you can feed the model for an even
more detailed simulation.

\section{Minimal amount of data needed}
\label{sec:minimum}

\subsection{Climate requirement}

\Daisy{} will as a minimum need daily values for the following item
for the entire simulation period, including the initialization period
(see section~\ref{sec:warmup}).
\begin{itemize}
\item Average air temperature.
\item Precipitation.
\item Global radiation.
\end{itemize}
\Daisy{} will also need information about the station itself, in
particular the name, elevation, location in longitude and latitude,
information about the whether the data is recorded at the field or on
a reference surface, the screen height, wet and dry deposition of
NH$_4$ and NO$_3$, average temperature for the location, the average
amplitude of the temperature variation over the year, and the average
day where the maximum temperature is reached.

All of this information should be supplied in a special climate file,
whose format is described in Appendix~\ref{ref-cha:dwf} of
\daisyref{}.

\subsubsection{Missing data}

All the above data must be supplied to the program in order for the
simulation to run.  Use the best data available, which means data from
the closest available climate station.  In some case you will have to
supplement local recordings with regional records, precipitation is
the most important to get local numbers for.

\subsubsection{Predictions}

If you are using \daisy{} for making predictions, you won't have any
climate data.  In that case, we recommend that you run the simulation
with old data from a number of different years.  Be sure to permute
the years, as the results are very sensitive to variations not only in
the weather, but in the sequence the years appear.  Do not use
``average'' weather data, as that tend to give unrealistically high
yields.  For the same reason, be sure that the years you use contain
some examples of extreme weather, not just weather from ``normal''
years.

\subsection{Management}

\Daisy{} will need to know the exact date of the following management
operations:
\begin{description}
\item[Fertilizing] In addition to the date, we need to know the amount
  and type of the fertilizer used, and whether it was incorporated or
  not.
\item[Irrigation] We need also to know the amount of water applied,
  how it was applied (surface, overhead or subsoil), as well as the
  amount of mineral fertilizer in the water, if any.
\item[Tillage operations] We need to know the type of operation, e.g.\ 
  plowing or seed bed preparation.
\item[Sowing]  We need to know the name of the crop.
\item[Harvesting] We need to know whether residuals are removed or
  left on the field.
\end{description}

All this information should be given in the \daisy{} setup files, or
\texttt{.dai} files, in the format described in \daisytut{},
especially section~\ref{tut-sec:management}.

\subsubsection{Missing data}

Again, \daisy{} need this information in order to run.  If you don't
have it, you must make estimates.  The dates can be essential, for
example, if the irrigation date does not match the time the crop need
the water the most, the yield may be much worse.  You may then
actually want to use the techniques for predictions described in the
next section instead.
\begin{description}
\item[Fertilizing] If you are interested in nitrogen harvest or
  leaching, it is essential to get the input right.  The same is true
  for yield and to a lesser degree water balance if the crop growth
  is nitrogen limited.  For organic fertilizer types, use the closest
  match from the `\texttt{fertilizer.dai}' file.
\item[Irrigation] Get it right.
\item[Tillage operations]  Use the closest match from the
  `\texttt{tillage.dai}' file.
\item[Sowing] \Daisy{} need a lot of information about the crop, this
  information have been assembled for a number of crops, which can be
  found in the `\texttt{crops.dai}' file.  If the actual crop is not
  found there, you may try to substitute with another crop with a
  similar development cycle and yield, if any such crop exists.
\item[Harvesting] Use your best estimate.
\end{description}

\subsubsection{Predictions}
\label{sec:predict}

\Daisy{} allows the manager to ``be smart'' about the state of the
simulation, so instead of using fixed dates it can harvest when the
crop is ripe and irrigate when the crop needs water.  You can use these
techniques to implement your management plan in the simulation.  As
hinted earlier, this is also useful when you don't have information
about exact dates.  This is particular useful when you analyze large
areas, where you do not has information from individual field.  From
statistical information about preferred crop types and knowledge about
common praxis in the area, you can create representative crop
rotations and run these under various climate conditions, and on the
soil types found in the area.

\subsection{Soil}

\Daisy{} will need the following information about the soil.
\begin{itemize}
\item The soil texture, i.e.\ the clay, silt and sand fractions of the
  soil.  Daisy need this information for all horizons in the profile,
  not just the top horizon.
\item The humus content in each horizon.
\item The maximum root depth of the soil.
\item The location of the groundwater and whether soil is drained.
  There are the following possibilities:
  \begin{enumerate}
  \item The soil is drained.  Then we need to know the position of the
    drains.  We also need the drainage period or the amount of drain
    water, preferably both.  This will be needed to calibrate the
    drainage information.
  \item The groundwater is located below 4 meter.  If so, we don't
    need to know more.
  \item The groundwater is located above 4 meter.  If so, we need to
    know where it is located.
  \end{enumerate}
\end{itemize}
This information is also placed in the \texttt{.dai} files, and has
been described in \daisytut{}, in particular
section~\ref{tut-sec:column}. 

\subsubsection{Missing data}

The texture is very important for water transport, if you can't measure
it, you must use your best estimate.  The B and C horizons may in fact
be even more important than the Ap horizon, and more difficult to get
information on.  Sometimes regional maps are available, or
measurements from nearby sites.

Good numbers for the humus content of the top horizon is essential for
a reasonable nitrogen balance, the lower horizons are much less
important.

The maximum root depth can usually be estimated from knowledge from
similar soil profiles.

\section{Recommended data}
\label{sec:recommended}

In section~\ref{sec:minimum} we listed the minimum data required in
order for \daisy{} to run.  In this section we will list additional
data you should acquire in order for the simulation result to be
reliable.

\subsection{Climate requirement}

You should supply your own data for reference evaporation.  If you do
not, \daisy{} will attempt to estimate it based on the conditions at
Taastrup, Denmark.  These may be incorrect for other locations.

\subsection{Management}
\label{sec:warmup}

We recommend that you run with a warmup period of at least 5 years
before the period you are interested in.  You don't need exact dates
for management operations for the warmup period (although having them
helps), but can use the techniques described in
section~\ref{sec:predict}.

For fertilization, you should specify the amount of carbon, nitrogen
and ammonium directly, rather than rely on the values from
\texttt{fertilizer.dai}

\subsection{Soil}

\begin{itemize}
\item An estimate for the average yearly carbon input to the soil in
  the decades before the simulation period.  The yearly carbon input
  is rarely available, you can use numbers from a predefined table of
  input from various farm types if someone have created such a table
  for your area, or better yet run a simulation with a ``typical''
  historical crop rotation to let \daisy{} calculate the input.
\item You should measure the C/N ratio for the humus in at least the
  Ap horizon.
\item You should measure the hydraulic properties of the soil.
\end{itemize}

\section{Extra data to be used when available}

Most \daisy{} simulation has ran with just the data recommended in
section~\ref{sec:recommended}, so that part of the model is the most
well-tested.  However, some test have been performed with more data
available, and we know this will improve the precision of the
simulation results further.  Here are some examples.

\subsection{Climate requirement}

Supplying hourly values in the climate file will enable a much more
detailed simulation.  If you furthermore supply information about
vapor pressure, relative humidity and wind speed, \daisy{} will be
able to use a more advanced bioclimate models.

If you have measured LAI during the simulation period, \daisy{} can
use this instead of the simulated LAI, giving more accurate simulation
results.  The crop development will tend to adjust to fit the measured
LAI\@.  How to do this is described in \daisytut{},
section~\ref{tut-sec:forced-lai}.

\subsection{Management}

You can supply information about application of pesticides.
Pesticides does not affect the rest of the model (as pests are not
part of the model), but \daisy{} will simulate the fate of the
pesticides.

You can get better information from most management operations by not
relying on the parameterizations in the standard library.
\begin{description}
\item[Fertilizing] You can examine the exact content of organic
  fertilizer, which tend to vary a lot between farms, and you can
  estimate the turnover rate with laboratory experiments.
\item[Tillage operations] You can fit the tillage operations to the
  machinery used at the actual farm, and make sure e.g.\ the plowing
  depth is right.
\item[Sowing] You can parameterize new crop sorts by running fields
  experiments with various levels of irrigation and fertilization, and
  measure nitrogen and dry matter content of the individual crop parts
  (leafs, stem, storage organ, and if you are ambitious, roots), as
  well as measuring the leaf area index.  All of these measurements
  must be done throughout the growth season, just having harvest data
  is not enough.
\item[Harvesting] Measure or estimate for how large a fraction of the
  various crop parts are left on the field, rather than assume all or
  nothing.
\end{description}

\subsection{Soil}

If possible, you should measure the nitrogen yield from a low
fertilized crop, and use this data to calibrate the background
mineralization as described in \daisytut{}~\ref{tut-sec:background}.

\end{document}

%%% Local Variables: 
%%% mode: latex
%%% TeX-master: t
%%% End: 
