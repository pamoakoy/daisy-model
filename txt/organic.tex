\documentclass[a4paper]{article}
\usepackage{verbatim}
\newcommand{\daisy}{Daisy}
\newcommand{\Daisy}{Daisy}
\newcommand{\cplusplus}%
{{\leavevmode{\rm{\hbox{C\hskip -0.1ex\raise 0.5ex\hbox{\tiny ++}}}}}}
\newcommand{\Cplusplus}{\cplusplus}
\newcommand{\mshe}{Mike/\textsc{she}}
\newcommand{\wintel}{\texttt{win32}}
\newcommand{\dll}{\textsc{dll}}
\newcommand{\Dll}{\textsc{Dll}}
\newcommand{\gui}{\textsc{gui}}
\newcommand{\Gui}{\textsc{Gui}}
\newcommand{\unix}{Unix}
\newcommand{\dhi}{\textsc{dhi}}
\newcommand{\Dhi}{\textsc{Dhi}}
\newcommand{\api}{\textsc{api}}
\newcommand{\Api}{\textsc{Api}}
\newcommand{\lai}{\textsc{lai}}
\newcommand{\Lai}{\textsc{Lai}}
%\newcommand{\url}[1]{\linebreak[4]\texttt{<URL:#1>}}

%%% Local Variables: 
%%% mode: latex
%%% TeX-master: t
%%% End: 

\usepackage[latin1]{inputenc}

\begin{document}

\title{\textbf{Initializing organic matter pools}}
\author{
    Per Abrahamsen, S�ren Hansen and Henrik Svendsen\\
    The Royal Veterinary and Agricultural University\\
    Department of Agricultural Sciences\\
    Laboratory for Agrohydrology and Bioclimatology
}
\date{\today}
\maketitle

\begin{abstract}
Organic matter models using multiple pools for soil biomass and soil
organic matter have proved able to simulate both short term and long
term change in humus content of agricultural soil.  However, these
pools do not correspond to measurable physical quantities, and are
therefore difficult for a non-expert to use.  An alternative sometimes
used is to assume the organic matter is in a state of equilibrium.
Unfortunately, the time to reach equilibrium is best measured in
centuries, so this assumptions is unlikely to hold true.  For the same
reason, the use of a warmup period cannot replace the need for a good
initial partitioning.  In this paper we propose a milder assumption,
namely a quasi-equilibrium where all but the slowest pool is in
equilibrium with the amount of carbon input.  This assumption allows
the non-expert user to initialize the model.
\end{abstract}

\tableofcontents
\pagebreak

\section{Soil organic matter modelling}

Daisy og andre SOM modeller, validering

Daisy software generel

Thorsten m�llers �ndring 

Sanders �ndring 

Fremtidige �ndringer DOM, Biomod, P, forrest

konklusion: model �ndrer sig

\section{Expanding user base}

sanders equilibrium

tove og kresten

the standardization project

konklusion: behov for forst�elige og model-uafh�ngige begreber,
equilibrium assumption uholbar.

\section{Equations}

The the sytem of equations.

Assume quasi-equilibrium with input.

Attempt to provoke background mineralization.

Assumptions and limitations.

-- clay, T og h

\section{Results}

Validation.  Lange tidsserier af Per og S�ren.

Brug i standardisering.

\end{document}

%%% Local Variables: 
%%% mode: latex
%%% TeX-master: t
%%% End: 
