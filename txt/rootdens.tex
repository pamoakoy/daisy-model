\documentclass[a4paper,11pt,twoside]{article}
\usepackage{a4}
\usepackage[T1]{fontenc}
\usepackage[latin1]{inputenc}
\usepackage{hyperref}
\usepackage{natbib}
\newcommand{\daisy}{Daisy}
\newcommand{\Daisy}{Daisy}
\newcommand{\cplusplus}%
{{\leavevmode{\rm{\hbox{C\hskip -0.1ex\raise 0.5ex\hbox{\tiny ++}}}}}}
\newcommand{\Cplusplus}{\cplusplus}
\newcommand{\mshe}{Mike/\textsc{she}}
\newcommand{\wintel}{\texttt{win32}}
\newcommand{\dll}{\textsc{dll}}
\newcommand{\Dll}{\textsc{Dll}}
\newcommand{\gui}{\textsc{gui}}
\newcommand{\Gui}{\textsc{Gui}}
\newcommand{\unix}{Unix}
\newcommand{\dhi}{\textsc{dhi}}
\newcommand{\Dhi}{\textsc{Dhi}}
\newcommand{\api}{\textsc{api}}
\newcommand{\Api}{\textsc{Api}}
\newcommand{\lai}{\textsc{lai}}
\newcommand{\Lai}{\textsc{Lai}}
%\newcommand{\url}[1]{\linebreak[4]\texttt{<URL:#1>}}

%%% Local Variables: 
%%% mode: latex
%%% TeX-master: t
%%% End: 


\begin{document}

\section*{Estimating the root density distribution from root dry matter and 
  depth}

\section{One dimension}

\subsection{Unlimited growth}

In accordance with \cite{gp74}, the root density distribution $L_z$
for a crop can be described by
\begin{equation}
  L_z = L_0\, e^{-a z}
  \label{eq:g+p}
\end{equation}
where $L_0$ is the root density at the soil surface, $a$ is a
distribution parameter, and $z$ is the depth below soil surface.

We here assume that the density is uniformly distributed on the
horizontal plane, an assumption that fails with e.g.\ row crops.

The parameters $a$ and $L_0$ will both vary with time.  For a
production oriented simulation model like
Daisy~\citep{daisy-def,daisy-ems}, it can be more convenient to
specify the density in terms of accumulated root dry matter $M_r$ and
total root depth $d$.  To do this, we define the root depth at the
lowest depth where the root density is at above specified threshold
$L_m$.  By inserting this in (\ref{eq:g+p}), we get
\begin{equation}
  L_m = L_0\, e^{-a d}
  \label{eq:root-depth}
\end{equation}

We convert the root mass to root length $l_r$ by assuming the specific
root length $S_r$ is a known constant (rather than varying with depth)
\begin{equation}
  l_r = S_r \, M_r
  \label{eq:root-length}
\end{equation}

The total root length is also the integral of the root density over
the profile
\begin{equation}
  l_r = \int_0^{\infty} L_z \: dz = \int_0^{\infty} L_0\, e^{-a z} \, dz
  \label{eq:root-integral}
\end{equation}

An indefinite integral to (\ref{eq:g+p}) is 
\begin{equation}
  G_z = -\frac{L_0}{a}\, e^{-a z}
  \label{eq:indefinite}
\end{equation}

Using (\ref{eq:indefinite}) in (\ref{eq:root-integral}) gives us
\begin{equation}
    l_r = G_{\infty} - G_0 = \frac{L_0}{a}
  \label{eq:root-integrated}
\end{equation}

By inserting the expressin we get for $L_0$ from
(\ref{eq:root-integrated}) in (\ref{eq:root-depth}) we get
\begin{equation}
  L_m = L_r \, a \, e^{-a d}
  \label{eq:a-only}
\end{equation}

If we substitute $W = -a d$ (and thus $a = -W / d$) we get
\begin{equation}
   L_m = L_r \frac{-W}{d} e^W
  \label{eq:sub-W}
\end{equation}

By isolating the known values on the right side this gives us
\begin{equation}
   W e^W = - L_m \frac{d}{L_r}
  \label{eq:Lambert}
\end{equation}

The solution to this equation with regard to $W$ happens to be the
definition of the Lambert-W function \citep{euler83,lambert58}.  An
existing implementation such as \citep{lambertwcode} can then be used
to find a by substituting $a$ back
\begin{equation}
   a = -\frac{W (-\frac{L_m d}{L_r})}{d}
  \label{eq:a-solved}
\end{equation}

Having found $a$, $L_0$ can be determined by rewriting (\ref{eq:root-depth}) to
\begin{equation}
  L_0 = \frac{L_m}{e^{-a d}}
  \label{eq:L0-found}
\end{equation}

\section{Limited growth}

The distribution described in (\ref{eq:g+p}) assumes a gradual
decrease of roots.  For some soil we see !!!REF!! a sharp decrease at
a specific depth, as the roots are unable to penetrate further down.
The way we model this is to is to divide the root depth into a crop
specific and soil independent potential root depth $d_c$, and soil
specific and crop independent maximum root depth $d_s$.

When the crop potential root depth haven't reached the soil maximum
root depth (that is $d_s > d_c$), we calculate the root density
distribution parameters as previously described for unlimited growth.

When the potential root depth is below the soil maximum root depth

\begin{equation}
  \begin{array}{rcll}
           &=& L_z     &; d_s >= d_c \\
     L_z^* &=& 0       &; z > d_s \wedge d_s < d_c \\
           &=& k^* L_z &; z < d_s \wedge d_s < d_c 
  \end{array}
  \label{eq:limited-depth}
\end{equation}

\begin{equation}
  k^* = \frac{\int_0^{\infty} L_z \, dz}{\int_0^{d_s}, L_z \, dz}
\end{equation}

\section*{List of symbols}
\addcontentsline{toc}{section}{\numberline{}List of symbols}

\begin{tabular}{lll}
Symbol & Unit & Description\\\hline
$a$     & m$^{-1}$ & Root density distribution parameter\\
$d$     & m       & Root depth\\
$d_c$   & m       & Crop portential root depth\\
$d_s$   & m       & Soil maximum root depth\\
$k^*$   &         & Soil root limit factor\\
$l_r$   & m/m$^2$ & Total root length\\
$L_0$   & m/m$^3$ & Root density at soil surface\\
$L_d$   & m/m$^3$ & Minimal root density\\
$L_z$   & m/m$^3$ & Root density at soil depth $z$\\
$L_z^*$ & m/m$^3$ & Soil modified root density at soil depth $z$\\
$M_r$   & kg/m$^2$& Total root dry matter\\
$S_r$   & m/kg    & Specific root length\\
$W$     &         & Lambert W function\\
$z$     & m       & Soil depth. \\
\end{tabular}

\addcontentsline{toc}{section}{\numberline{}Bibliography}
\bibliographystyle{elsart-harv}
\bibliography{daisy}

\end{document}
