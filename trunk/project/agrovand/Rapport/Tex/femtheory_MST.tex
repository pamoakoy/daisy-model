
\section{General description of the PDEs}

The physical explanations for Richard�s equation and the
advection-dispersion equation, respectively, are very different, whereas
the two PDEs look almost alike. Both equations are covered by the more
general expression:


\begin{equation}
a\frac{\partial \phi}{\partial t}-\nabla \cdot
\left(\mathbf{b}\nabla \phi - \mathbf{c}\phi+\mathbf{d} \right) +
e\phi+f=0
\label{eq:general}
\end{equation}

where $\phi$ is the dependent variable. $a$, $e$ and $f$ are
scalars. $\mathbf{b}$ is a 2-by-2 matrix or a scalar. $\mathbf{c}$ and
$\mathbf{d}$ are vectors. Both $a$, $\mathbf{b}$, $\mathbf{c}$,
$\mathbf{d}$, $e$ and $f$ can be dependent on $\phi$.\\
\\
The working process in the development of TopFEM has been first
to develop a water movement model without thinking (too much) of the
later developed solute movement model. The solute movement model is
later built on the same skeleton as the water movement model with some
changes and expansions. \\
\\
The mathematical theories behind the solution of water and solute
movement equations have a lot of similarities. The solution of the
more general equation, equation \ref{eq:general} is treated
first. Later the actual specialties for the solution process of
the water movement and solute movement problems, respectively, are
discussed in their own sections. \\
\\
Actually, equation \ref{eq:general} covers a lot of different areas in
both physics\index{physics} and chemistry\index{chemistry}, but can
also be used to describe the pricing of stock options, as done in the
Black-Scholes model, \cite{Black-Scholes}\index{Black-Scholes} (which
only is 1D in space). Work behind the model gave in 1997 the Nobel
Price\index{Nobel Price} in economics\index{economics} to
Scholes\index{Scholes} and Merton\index{Merton}, \cite{Nobel}. \\
\\
Table \ref{tab:difcoeff} shows the coefficients of equation
\ref{eq:general} shown for Richard's equation and the
advection-dispersion equation, respectively. The coefficients are
obtained if $\phi$ is replaced with $\psi$ and $C$ in the water and
solute movement equations, respectively. According to classification
rules for PDEs in two variables (space and time),
\cite{Asmar,Trottenberg} both Richard�s equation and the
advection-dispersion equation are of parabolic type.  The PDEs are
\textit{quasi-linear} since the coefficients in general depend on
$\phi$. It should be noticed that it is assumed that the solution for
the water movement simulations, i.e. $\mathbf q$ and $\theta$ is known
when solving the solute movement equation.\\
\\


\begin{table}[h]
\caption{Coefficients for water and solute movement}
\label{tab:difcoeff}
\begin{tabular}{p{2.5cm}|p{4.95cm}|p{4.95cm}} \hline
\textbf{Coefficient} & \textbf{Value, water movement} & \textbf{Value,
  solute movement} \\ \hline
$a$ & $C_w$ & $\theta R$ \\ \hline
$\mathbf{b}$ & $K$\ or\ $\begin{bmatrix} K & 0 \\ 0 & K
  \end{bmatrix}$ & $\theta \begin{bmatrix} D_{xx} & D_{xy} \\ D_{yx}
  & D_{yy} \end{bmatrix}$ \\ \hline
$\mathbf{c}$ & $\begin{bmatrix}0 \\ 0 \end{bmatrix}$ &
$\begin{bmatrix} q_x \\ q_y \end{bmatrix}$
\\ \hline
$\mathbf{d}$ & $\begin{bmatrix} 0 \\ K \end{bmatrix}$ &
$\begin{bmatrix} 0 \\ 0 \end{bmatrix}$ \\ \hline
$e$ & $0$ & $\frac{\partial \theta}{\partial t}$ \\ \hline
$f$ & $\Gamma$ &
$C_{sink}\Gamma-\theta\mu_l-\rho_b\mu_s$ \\ \hline
\end{tabular}
\end{table}


To solve the equation \ref{eq:general}, it is necessary to specify
initial\index{initial condition} and boundary
conditions\index{boundary condition}. The boundary conditions specify
a combination of $\phi$ and its derivative on the boundary:

\begin{equation}
\mathbf{\bar{n}} \cdot (\mathbf{b}\nabla \phi -
\mathbf{c}\phi+\mathbf{d}) + r\phi =h
\end{equation}

\begin{equation}
\phi=\phi_0
\end{equation}


where $\mathbf{\bar{n}}$ is the outward unit normal. The first type is
known as the \textit{Robin boundary condition}\index{Robin boundary
  condition}\index{boundary condition!Robin} and the second is known
as the \textit{Dirichlet boundary condition}\index{boundary
  condition!Dirichlet}\index{Dirichlet boundary condition}. The
Dirichlet conditions can be approximated by Robin boundary conditions
by setting $h=rd$ and then letting $r\rightarrow \infty$. Division
with a large $r$ cancels the derivative term. The \textit{Neumann
  boundary condition}\index{Neumann boundary condition}\index{boundary
  condition!Neumann} is obtained by setting $r=0$:

\begin{equation}
\mathbf{\bar{n}} \cdot (\mathbf{b}\nabla \phi - \mathbf{c}\phi+\mathbf{d}) =
\mathbf{\bar{n}}\cdot(-\mathbf{q})=-q
\label{eq:Neumann}
\end{equation}

where $\mathbf{q}$ is the flux vector and $q$ is the size of the
outward flux. Only Dirichlet and the Neumann boundary conditions
are considered here.



\section{Weak formulation}\index{weak formulation}

Since $\phi$ will be approximated by a function whose first-order
derivative has jump discontinuities at the nodes, it would be
necessary to reformulate the problem so as to remove the second-order
derivative in \ref{eq:resweight}. For that purpose Gauss divergence
theorem\index{Gauss divergence theorem}, \cite{HEJ} is excellent:


\begin{equation}
\int_{\Omega}\nabla \cdot\mathbf{V} \, d\Omega=\int_{\partial \Omega}
\mathbf{\bar{n}}\cdot \mathbf{V} \, dS
\label{eq:gauss}
\end{equation}

or in words: the total divergence of the vector field,  $\mathbf{V}$ in
the volume $\Omega$ is equal to the net flux through the surface of
$\Omega$, $\partial \Omega$. In a 2D domain, the words volume and
surface should be replaced by area and boundary, respectively. Another
useful rule from the vector calculus, \cite{HEJ} is:


\begin{equation}
\nabla \cdot (f\mathbf{V})=(\nabla f)\cdot \mathbf{V}+ (\nabla \cdot
\mathbf{V})f
\label{eq:vectorthing}
\end{equation}

where $f$ is a $C^1$-function (continuous and 1 time
differentiable). Applying equations \ref{eq:gauss} and
\ref{eq:vectorthing} in the second term in the weighted residual
equation gives:

\begin{equation}
\begin{split}
&\int_{\Omega} \nabla \cdot (\mathbf{b}\nabla \phi - \mathbf{c}\phi
+\mathbf{d} )v \, dxdy = \\
&\int_{\Omega} \nabla \cdot \left( (\mathbf{b}\nabla \phi - \mathbf{c}\phi
+\mathbf{d} )v \right) \, dxdy - \int_{\Omega}(\nabla v)\cdot(\mathbf{b} \nabla \phi-
  \mathbf{c}\phi+\mathbf{d})\, dxdy = \\
& \int_{\partial \Omega} \mathbf{\bar{n}}
  \cdot (\mathbf{b}\nabla \phi-\mathbf{c}\phi +  \mathbf{d})v \, dS -
\int_{\Omega}(\nabla v)\cdot(\mathbf{b} \nabla \phi-
  \mathbf{c}\phi+\mathbf{d})\, dxdy
\end{split}
\end{equation}

The weighted residual equation can now be written without second order
derivatives:

\begin{equation}
\begin{split}
& \int_{\Omega} a\frac{\partial
  \phi}{\partial t} v\,  dxdy -\int_{\partial \Omega} \mathbf{\bar{n}}
  \cdot (\mathbf{b}\nabla \phi-\mathbf{c}\phi +  \mathbf{d})v \, dS +
  \\ &\int_{\Omega}(\nabla v)\cdot(\mathbf{b} \nabla \phi-
  \mathbf{c}\phi+\mathbf{d})\, dxdy+ \int_{\Omega} e\phi v \, dxdy
  +\int_{\Omega} fv \, dxdy =0
\end{split}
\label{eq:weak1}
\end{equation}

This equation is known as the variational, or weak form of the
differential equation. Obviously, any solution of the differential
equation is also a solution of the variational problem. If $v$ is
sufficiently smooth, one can also show the converse. A discussion of
how smooth $v$ should be is outside the scope of this report, but
piecewise linear polynomials are here considered to be smooth
enough. \\
\\
The boundary of $\Omega$, $\partial \Omega$ can be divided into
parts, $\partial \Omega_{1}$ and $\partial \Omega_{2}$ with a
Neumann\index{Neumann boundary condition}\index{boundary
  condition!Neumann} and a Dirichlet boundary
condition\index{Dirichlet boundary condition}\index{boundary
  condition!Dirichlet}, respectively. The terms in the parenthesis in
the boundary integral in equation \ref{eq:weak1} can be replaced with
the right hand side of \ref{eq:Neumann} on $\partial \Omega_{1}$. We
have previously seen how a Dirichlet boundary condition can be
approximated with a Robin boundary condition. But this is numerically
a bad solution because it can produce ill-conditioned
matrix-systems. The Dirichlet boundary is implemented by simply
forcing $\phi$ to be equal to the wanted value on $\partial
\Omega_{2}$. The procedure is described later. We now have to solve:


\begin{equation}
\begin{cases}
R_W=\int_{\Omega} a\frac{\partial \phi}{\partial t} v \, dxdy
 + \int_{\Omega}(\nabla
v)\cdot(\mathbf{b} \nabla \phi- \mathbf{c}\phi+\mathbf{d})\, dxdy  \\+
 \int_{\Omega} e\phi v \, dxdy + \int_{\Omega}fv \, dxdy +\int_{\partial \Omega_1} qv \, dS=0 &
 \text{in} \ \Omega \\ \phi=\phi_0 & \text{on} \ \partial \Omega_2
 \end{cases}
\label{eq:weak2}
\end{equation}






\section{Time stepping procedure}\index{time stepping}

There are several more or less sophisticated methods to solve
ODEs\index{ODE}. Methods as Euler and the trapezoidal rule are widely
used. Both the trapezoidal rule\index{trapezoidal rule} and the Euler
method are included in the $\theta$-method as described by
\cite{Iserles}. Instead of using $\theta$ as the parameter, $\omega$
is here used to prevent confusing misunderstandings with the water
content. If the \textit{initial value problem} (IVP)\index{IVP} can be
expressed as:


\begin{equation}
\boldsymbol{\dot{\phi}}=\mathbf{f}(t,\boldsymbol{\phi}),\ \ \ \ \ \ t
\geq t_{0}, \ \ \ \ \ \boldsymbol{\phi}(t_{0})=\boldsymbol{\phi}_{0}
\label{eq:IVP}
\end{equation}

where $\boldsymbol{\phi}$ is a vector and
$\mathbf{f}(t,\boldsymbol{\phi})$ is a vector function, a numerical
procedure to solve the IVP can be written as:

\begin{equation}
\begin{split}
\boldsymbol{\phi}(t_{n+1}) &=\boldsymbol{\phi}(t_{n})+ \omega \Delta t
\mathbf{f}(t_{n},\boldsymbol{\phi}_{n})+(1-\omega) \Delta t\mathbf{f}(t_{n+1},
\boldsymbol{\phi}(t_{n+1}))
 \\ &+(\omega-\frac{1}{2})\Delta
t^2\boldsymbol{\phi}''(t_{n})+(\frac{1}{2}\omega-\frac{1}{3}) \Delta
t^3\boldsymbol{\phi}'''(t_{n})+O(\Delta t^4) \\
& \approx \boldsymbol{\phi}(t_{n})+ \omega \Delta t
\mathbf{f}(t_{n},\boldsymbol{\phi}_{n})+(1-\omega) \Delta t\mathbf{f}(t_{n+1},
\boldsymbol{\phi}(t_{n+1}))
\label{eq:diffsol}
\end{split}
\end{equation}

where $n$ and $n+1$ are numbers of the time levels and $\Delta t$ is
the length of the timestep. The timeweighting parameter\footnote{Many
  places, for instance \cite{Quarteroni} the timeweighting parameter,
  $\omega$ is associated with timestep $n+1$ and $(\omega-1)$ with
  timestep $n$, i.e the reverse.}, $\omega$ is restricted to the interval,
$0 \leq \omega \leq 1$. $\omega$ decides how the weighting in time
shall be for $\mathbf{f}$. For $\omega=\frac{1}{2}$ it can be seen
that the method is of order 2. For other values it is order 1. For
$\omega=\frac{2}{3}$ we see the $O(\Delta t^3)$ term in
\ref{eq:diffsol} vanishes while the $O(\Delta t^2)$ remains. In very
special cases this can be an advantage, \cite{Iserles}. The method is
according to definition in \cite{Quarteroni} a \textit{one-step
method}\index{one-step method}\index{method!one-step}. For
$\omega=1$ is it \textit{explicit}\index{explicit
  method}\index{method!explicit}, oherwise method is
\textit{implicit}\index{implicit method}\index{method!implicit}.



The method has names for some special values of $\omega$:

\begin{itemize}
\item $\omega=0$ \textit{backward difference}\index{backward
    difference} or \textit{backward Euler}\index{backward Euler}
\item $\omega=\frac{1}{2}$ \textit{central difference}\index{central
    difference}, \textit{trapezoidal rule}\index{trapezoidal rule} or
  \textit{Crank-Nicolson}\index{Crank-Nicolson}
\item $\omega=1$ \textit{forward difference}\index{forward difference}
  or \textit{forward Euler}\index{forward Euler}
\end{itemize}

The backward Euler method is widely used in models of unsaturated
flow. For the advection-dispersion equation the Crank-Nicolson
scheme is often used.\\
\\

The actual IVP can by applying equation \ref{eq:matrix} be written as:

\begin{equation}
\boldsymbol{\dot{\phi}}=\mathbf{A^{-1}}(\mathbf{G}-\mathbf{H}\boldsymbol{\phi})
, \ \ \ \ \ t\geq t_0, \ \ \ \ \
\boldsymbol{\phi}(t_0)=\boldsymbol{\phi}_0
\label{eq:IVPdef}
\end{equation}

where $\mathbf{G}=-(\mathbf{D}+\mathbf{F}+\mathbf{Q})$ and
$\mathbf{H}=\mathbf{B}-\mathbf{C}+\mathbf{E}$.  It is very important
to note that $\mathbf{A}$ shall be regular. How this is fulfilled
will be described later. By applying \ref{eq:diffsol} we get:

\begin{equation}
\begin{split}
\boldsymbol{\phi}_{n+1}=& \boldsymbol{\phi}_{n}+ \omega \Delta t
\mathbf{A}_{n}^{-1}(\mathbf{G}_{n}-\mathbf{H}_{n}\boldsymbol{\phi}_n)
+\\ &
(1-\omega) \Delta t
\mathbf{A}_{n+1}^{-1}(\mathbf{G}_{n+1}-\mathbf{H}_{n+1}\boldsymbol{\phi}_{n+1})
\label{eq:gentheta}
\end{split}
\end{equation}

Equation \ref{eq:gentheta} can be written so we get the unknown,
$\boldsymbol{\phi}_{n+1}$ on the left hand side:

\begin{equation}
\begin{split}
(\mathbf{A}_{n+1} + &\Delta t(1-\omega)\mathbf{H}_{n+1})\boldsymbol{\phi}_{n+1} =\mathbf{A}_{n+1}\boldsymbol{\phi}_{n} +\\ &\Delta t \omega
\mathbf{A}_{n+1}\mathbf{A}^{-1}_{n}(\mathbf{G}_{n}-\mathbf{H}_{n}\boldsymbol{\phi}_{n}) + \\  &\Delta
t(1-\omega)\mathbf{G}_{n+1}
\end{split}
\end{equation}

But both $\mathbf{A}$, $\mathbf{H}$ and $\mathbf{G}$ are
  functions of $\boldsymbol{\phi}$. For solving the equations, the
  \textit{Picard iterations}\index{Picard iterations} can be used. The unknown
  $\boldsymbol{\phi}_{n+1}$ is estimated by using the latest estimate
  of $\mathbf{A}_{n+1}$, $\mathbf{G}_{n+1}$ and
  $\mathbf{H}_{n+1}$. The iteration scheme can be written as:


\begin{equation}
\begin{split}
(\mathbf{A}_{n+1,m}+&\Delta
t(1-\omega)\mathbf{H}_{n+1,m})\boldsymbol{\phi}_{n+1,m+1} = \\
&\mathbf{A}_{n+1,m}\boldsymbol{\phi}_{n}+\Delta t \omega
\mathbf{A}_{n+1}(\mathbf{A}_{n})^{-1}(\mathbf{G}_{n}-\mathbf{H}_{n}\boldsymbol{\phi}_{n})+
\\ &\Delta t(1-\omega)\mathbf{G}_{n+1,m}
\end{split}
\label{eq:Picard}
\end{equation}

where $m$ and $m+1$ denote the iteration levels. The iteration
procedure stops when a chosen norm, for example the $\infty$-norm of
the change in $\boldsymbol{\phi}$ between to iterations is below a
certain chosen value, $\epsilon$.\\
\\
It can bee seen that many advantages are obtained by choosing $\omega$
to be $0$ because no time is spent on the rather time consuming
process to calculate $\mathbf{A}^{-1}$.  But in the general case where
the $\mathbf{A}$ has to be inverted, it can save a lot of time if
$\mathbf{A}$ is a diagonal, i.e. if it is lumped. The choice of
$\omega=0$ is also for stability reasons a good choice,
\cite{Iserles}. Later it is discussed how the size of $\Delta t$ is
controlled.





\section{Matrix solution technique}\index{matrix!solution}

When solving the ODE defined by equation \ref{eq:matrix} it is when
$\omega\neq 0$ also necessary to use the inverse of a matrix, for that
purpose the built-in function \textsf{inv} is used. The computational
costs for a non-diagonal matrix is very high so the lumped (diagonal)
versions can be used with advantage. \\
\\
For solving the large matrix system of the type
$\mathbf{Ax}=\mathbf{b}$, the MATLAB backslash
operator\index{MATLAB!backslash operator} (also called
leftdivision) is used. By using the backslash operator, MATLAB makes
some tests and finds an appropriate direct method for solving the
equation. MATLAB (and then also TopFEM) gives warning messages for
badly scaled (or singular) matrices (where the solution maybe have
large errors). It is an (often made) mistake to use \textsf{inv} to
solve the system of equations by
$\mathbf{x}=\mathbf{A}^{-1}\mathbf{b}$, the computational costs are 2-3
times larger than using the backslash operator and the accuracy is much
smaller according to MATLAB. \\
\\
The calculation costs of physical entities, such as water capacity
(water movement simulations) and water fluxes (solute movement
simulations) all more or less proportional with the number of nodal
points, $\text{NP}$, whereas the solution of the matrix is strongly
dependent on the dimension of $\mathbf{A}$. When using the backslash
operator (also denoted leftdivision) MATLAB uses a direct method for
solving the linear equations. For large systems, the number of
floating point operations in Gauss elimination are proportional to
$\text{NP}^3$. The actual number of floating point operations are
probably lower since $\mathbf{A}$ has low density and MATLAB has
special algorithms for sparse matrices.\\
\\
If $\mathbf{A}$ is symmetric, MATLAB attempts to use a Cholesky
factorization\index{Cholesky factorization} of $\mathbf{A}$:

\begin{equation}
\mathbf{A}=\mathbf{L}^T\mathbf{L}
\end{equation}

Where $\mathbf{L}$ is a lower triangular matrix. The solution is
simply first to solve $\mathbf{L}^T\mathbf{z}=\mathbf{b}$ and then solve
$\mathbf{L}\mathbf{x}=\mathbf{z}$. The Cholesky factorization can be
used with success if $\mathbf{A}$ is positive definite. The Cholesky
factorization requires according to MATLAB less than half the
computational time of a general factorization. Similar results are
given in \cite{Atkinson,Numerisk}. If $\mathbf{A}$ is not
symmetric or the Cholesky factorization fails, other factorization
methods as for example LU-factorization\index{LU factorization} with
pivoting (a kind of Gauss elimination) are used:


\begin{equation}
\mathbf{A}=\mathbf{L}\mathbf{U}
\end{equation}

where $\mathbf{L}$ is a lower triangular matrix and $\mathbf{U}$ is an
upper triangular matrix. The solution method is first to solve
$\mathbf{L}\mathbf{z}=\mathbf{b}$ and
$\mathbf{U}\mathbf{x}=\mathbf{z}$ by using forward and backward
substitution. \\
\\
For larger especially sparse systems it is often an advantage to use
iterative methods. Jacobi\index{Jacobi} and
Gauss-Seidel\index{Gauss-Seidel} and SOR\index{SOR} (Successive Over
Relaxation) are often used. MATLAB has build-in solvers for GMRES and
PCG (only for symmetric matrices). All the here mentioned iterative
methods are discussed in \cite{DCAM}.




\subsection{Size of the timesteps}\index{timesteps, size of}

For water movement simulations it is possible either to choose a
constant size of the time steps, $\Delta t$ or to choose timesteps
which dynamically changes the size, dependent on how easy a
sufficiently good solution is obtained in the Picard
iterations\index{Picard iterations}. \\
\\
For the constant size of the timesteps a new timestep starts if either
the iteration criterion is fulfilled or if the number of Picard
iterations have reached a chosen maximum, $m_{max}$.\\
\\
For dynamically changing size of the timesteps the maximum number of
Picard iterations $m_{max}$ should also be chosen. Also a minimum and
maximum size to the timesteps, $\Delta t_{min}$ and $\Delta t_{max}$
must be specified. The procedure is:


\begin{enumerate}
\item if $m\leq 4$ then $\Delta t_{n+1} = 1.1\Delta t_n$ but not larger than
  $\Delta t_{max}$
\item if $m=5$ then $\Delta t_{n+1}=\Delta t_n$
\item if $6 \leq m\leq 7$ then $\Delta t_{n+1}=0.8 \Delta t_n$ but not lower
  than $\Delta t_{min}$
\item if $m>7$ then $\Delta t_{n+1}=0.3\Delta t_n$ but not lower than $\Delta
   t_{min}$
\item if $m=m_{max}$ then the time is only updated if $\Delta t_{n+1}=\Delta
  t_{min}$ else it tries again with smaller timesteps  $\Delta t_n=
  0.3\Delta t_n$ but not smaller than $\Delta t_{min}$
\end{enumerate}

\newpage

\subsection{Mass balance}\index{mass balance!water}

For a fast validation of simulations, a water balance index is
calculated. It is here defined as:

\begin{equation}
\text{water balance index}=\frac{\Delta W+Q+S}{\frac{1}{3}(|\Delta
  W|+|Q|+|S|)}
\label{eq:watbal}
\end{equation}

where $\Delta W$ is total change in the water storage. $Q$ is the
total flux out of the domain, $S$ is the total amount of water removed
from $\Omega$ by sinks (sources are negative sinks). \\
\\
The water balance index shall for most simulations ideally be
zero. But if it is 0, it is not necessarily a correct solution. The
water balance for $\Omega$ can be fulfilled in many ways. The
water balance for the domain does not say anything about the internal
distribution of the water. There can for example be oscillations
(wiggles) around the real solution. The absolute value of the water
balance index can rise (up to 3) for redistribution cases without
water interchanging with the surrounding environment even if the
calculations are acceptable.


\section{Solute movement}

In this section special features of the solution of the
advection-dispersion equation are discussed. First of all, it is
discussed how information from the water movement module is treated.
The assumptions for the development of the local matrices are
discussed. The numerical instabilities which occur in advection
dominated problems and how to decide the size of the next timestep
are finally discussed.


\subsection{Coupling water and solute movement models}

The water and solute movement represented by equations
\ref{eq:watermovement} and \ref{eq:solutemovement} are coupled in a
sense that the solute movement are dependent of the water movement,
but not the reverse. The used solution process is to calculate the
water movement first. The results (the matric pressure potential and
the boundary fluxes) are stored at pre-described times, subsequently the
concentration is calculated. Meanwhile the water content, the time
derivative of the water content and fluxes are estimated using the
stored values from the water movement simulations. Another possibility
is to calculate both the water and solute movement in one
run. The advantage is that time is not spent on unnecessary
recalculations. The disadvantage is that maybe the timesteps
necessary for fast convergence in the water and the solute
calculations, respectively, are of different magnitude, so that
unnecessary small timesteps are used in one of the models. By dividing
the models into two modules, it is also possible to simulate
different solute movement scenarios with the same water flow as
background without repeating the calculation for solving Richard's
equation.\\
\\
Estimating the water flux\index{water flow!estimation} in the nodal
points is connected with problems since both the conductivities and
the derivative of the pressure, $\psi$ can have discontinuities
between elements edges. The spatial derivatives of $\psi$ can in each
element be calculated as:


\begin{equation}
\begin{split}
& \left(\frac{\partial \psi}{\partial x} \right)_e=\begin{bmatrix} \partial N_1/\partial
  x & \partial N_2/\partial x & \partial N_3/\partial x \end{bmatrix}
  \begin{bmatrix} \psi_1 \\ \psi_2 \\ \psi_3 \end{bmatrix} \\
& \left(\frac{\partial \psi}{\partial y} \right)_e=\begin{bmatrix} \partial N_1/\partial
  y & \partial N_2/\partial y & \partial N_3/\partial y \end{bmatrix}
  \begin{bmatrix} \psi_1 \\ \psi_2 \\ \psi_3 \end{bmatrix}
\end{split}
\end{equation}

The fluxes in the nodal points are estimated using mean values of
estimated fluxes in the surrounding elements. Two different methods
can be used. In the first, the mean hydraulic conductivity,
$\bar{K}_e$ in each of the surrounding elements is used:

\begin{equation}
\begin{split}
& q_x=-\frac{1}{\text{NEP}}\sum_{e=1}^{\text{NEP}} \bar{K_e} \left(\frac{\partial \psi}{\partial x}\right)_e
\\
& q_y=-\frac{1}{\text{NEP}}\sum_{e=1}^{\text{NEP}} \bar{K_e}\left(\frac{\partial \psi}{\partial y}+1\right)_e
\end{split}
\end{equation}

where $\text{NEP}$ is the number of elements surrounding the current
nodal point. In the second method, only the conductivities, $K_{e,p}$
in the current point are used, subscript $e$ and $p$ refer to the
element and nodal number, respectively.

\begin{equation}
\begin{split}
& q_x=-\frac{1}{\text{NEP}}\sum_{e=1}^{\text{NEP}} K_{e,p} \left(
  \frac{\partial \psi}{\partial x} \right)_e
\\
& q_y=-\frac{1}{\text{NEP}}\sum_{e=1}^{\text{NEP}}
  K_{e,p}\left(\frac{\partial \psi}{\partial y}+1\right)_e
\end{split}
\end{equation}

Method number 1 is preferred as there is more consistence between the
points for evaluation of conductivity and the derivative.


\subsubsection{Example}

As an example are the Darcy velocity calculated for nodal point no. 4
in a Finite Element mesh. The nodal point and its surrounding points
are shown in figure \ref{fig:velotri}. The situation is that $\psi$ is
constant in the area considered i.e $\partial \psi/\partial
x=\partial \psi/\partial y=0$ a situation that ideally occurs if the
water movement only is driven by gravity in a simulation. The node
numbering refers here to the global numbering.


%\begin{figure}[h]  %here-top-bottom-page
%\epsfig{file=velonode.eps,width=6.5cm} \hspace{.5cm}
%\epsfig{file=velotri.eps,width=6.5cm}
%\figcap{Node and triangle numbering for calculation of Darcy flux}
%\label{fig:velotri}
%\end{figure}

Method 1:

\begin{equation}
\begin{split}
& q_x=0 \\
& q_y=- \frac{1}{6} (
  \frac{1}{3}(K_{1,4}+K_{1,6}+K_{1,3})+\frac{1}{3}
    (K_{2,4}+K_{2,1}+K_{2,2})+ \\
& \ \ \ \ \ \ \ \ \ \ \ \ \ \ \ \cdots +
    \frac{1}{3}(K_{6,4})+K_{6,7}+K_{6,6}) )
\end{split}
\end{equation}


Method 2:

\begin{equation}
\begin{split}
& q_x=0 \\
& q_y=- \frac{1}{6}\left(
  K_{1,4}+K_{2,4}+K_{3,4}+K_{4,4}+K_{5,4}+K_{6,4} \right)
\end{split}
\end{equation}

The difference between the two methods is obvious.





\subsection{Advection dominated problems}\index{stability}


Several numerical problems can be involved with the solving of
the advection-diffusion problem, especially when the problems are
dominated by advection. The numerical solutions have often unexpected
oscillations in that situation. A lot of more or less complicated
methods to reduce the problems have been developed. Two of the methods
are \textit{upstream weighting}\index{upstream weighting} and
\textit{streamline diffusion}\index{streamline diffusion} - both in
many variants.


\subsection{$P_e$ and $C_r$ numbers}\index{P\'{e}clet
  number}\index{Courant number}

There are two different dimensionless numbers which are important for
the stability. The \textit{P\'{e}clet number}, which in 1D is:

\begin{equation}
P_e=v\Delta x/D
\end{equation}

where $v$ is the velocity, $\Delta x$ is the space increment and
$D$ is the dispersion. In other words a ratio between the convective
and the dispersive terms. The \textit{Courant number} is here defined
as:

\begin{equation}
C_r=v\Delta t/(R\Delta x)
\end{equation}

where $R$ is the retardation factor. In other places, even for models
with adsorption it may be defined as $C_r=v\Delta t/(\Delta x)$. The
Courant number describes the ratio between the movement of a particle
by advection in one time increment and the grid spacing. Theoretical
stability investigations are rather complicated, especially in a two
or three dimensional space with heterogenous soils. Most of the
theoretical considerations for stability are done for one dimensional
flow with uniform velocity. \cite{Perrochet} investigated the
advection-dispersion equation without any chemical processes and found
that The classical Crank-Nicolson-Galerkin scheme is stable for
$P_e\leq 2$ and $C_r \leq 1$, \cite{Perrochet}. The analysis was done
without considering  adsorption. By using the same theory, it can
easily be shown that stability for the advection-dispersion model is
insured for the same constraints with the Courant number which is used
here. \\
\\
It can be concluded that keeping the Courant number lower than one
is just a question of sufficiently small timesteps. But is it possible
to make a mesh which under all circumstances prevents that the P\'{e}clet
number exceeds 2?.  The sidelengths of the elements that coincide
with the boundaries are not calculated in TopFEM. Instead all triangle
areas are computed. The characteristic length of the elements,
$\Delta x$  can be approximated as $\Delta x=2\sqrt{A}$,
where $A$ is the area of the element (triangle). $\sqrt{2} \cdot
\sqrt{2A} = 2 \sqrt{A}$  is the length of the diagonal in a square
which is made of two isoscele right-angled triangles (elements) with
the area $A$. This is a good measure of the sidelength in the real
elements, where the 3 side lengths in an element for stability reasons
should have almost equal lengths, \cite{FEMLAB,Segerlind}.
\\
\\
The P\'{e}clet number for the flow in the x-direction can be calculated as:

\begin{equation}
\begin{split}
P_{e,x}&=\frac{q_x 2\sqrt{A}}{\theta D_{xx}}=\frac{2q_x\sqrt{A}}{\alpha_L
  \frac{q_xq_x}{|\mathbf{q}|}+\alpha_T\frac{q_yq_y}{|\mathbf{q}|}+D_0\frac{\theta^{10/3}}{\theta_s}}\\
& <\frac{2q_x\sqrt{A}}{\alpha_T\frac{q_xq_x+q_yq_y}{|\mathbf{q}|}}
  = \frac{2q_x\sqrt{A}}{\alpha_T |\mathbf{q}|} \leq \frac{2\sqrt{A}}{\alpha_T}
\end{split}
\label{eq:pex}
\end{equation}

where it is assumed that $\alpha_L \geq  \alpha_T$. The same procedure
can of course be used to evaluate $P_{e,y}$. It can then be concluded
that the maximum P\'{e}clet number in the x and y-direction is under
$2\sqrt{A}/\alpha_T$. If the longitudinal dispersivity is 5 cm and the
transversal is 1/10 of the longitudinal dispersion and the maximum
allowed $P_e$ is 2 it can be concluded that the maximum sidelength of
the elements will be approximately 1/2 cm. This will result in a very
fine mesh. In practise the judgments made in equation \ref{eq:pex} are
so rough that somewhat larger elements probably can be used without
stability problems.\\
\\
In the present code it is possible to choose between 2 stabilizing
methods\index{stabilizing methods}:

\begin{enumerate}
\item Varying the size of $\Delta t$ so $P_eC_r \leq \gamma$.
\item Introducing extra diffusion in the streamline direction so
  $P_eC_r\leq\gamma$ is fulfilled. $\gamma$ is called the
  \textit{performance index}\index{performance index}.
\end{enumerate}

It is of course also possible to not choosing any stabilizing
methods. Put into practice there is often stability as long as
$P_eC_r\leq \gamma$  where $2\leq \gamma \leq 10$, \cite{Perrochet}
which under any circumstances is less restrictive than keeping both
$P_e\leq 2$ and $C_r\leq 1$.



\subsection{Varying the size of the timesteps}

The calculation of the P\'{e}clet and Courant numbers are of computational
reasons a little different in the two stabilizing methods. For the
method where the size of the timesteps are varied in order to fulfill
the stability criterion, requirements are


\begin{equation}
\begin{split} & (P_eC_r)_x=\frac{|v_x|\Delta x}{D}\cdot
\frac{|v_x|\Delta t}{R \Delta
  x}=\frac{\theta v_x^2\Delta t}{R(\theta D_{xx})} \\
& (P_eC_r)_y=\frac{|v_y|\Delta x}{D} \cdot \frac{|v_y|\Delta t}{R\Delta
  x}=\frac{\theta v_y^2\Delta t}{R(\theta D_{yy})}
\end{split}
\end{equation}

$\Delta t$ is chosen so both $(P_eC_r)_x$ and $(P_eC_r)_y$ are less
than $\gamma$:


\begin{equation}
\Delta t=min \left( \frac{R(\theta D_{xx})\gamma}{\theta
    v_x^2},\frac{R(\theta D_{yy})\gamma}{\theta v_y^2} \right)
\end{equation}

In the program, it is possible to set a minimum and maximum value of
$\Delta t$, $\Delta t_{min}$ and $\Delta t_{max}$. The minimum value, to
prevent the timesteps to get too small (and the CPU-time too large). The
maximum value can be chosen to take into account possible changes in
time dependent boundary conditions or sink terms.



\subsection{Streamline diffusion}\index{streamline diffusion}

$P_eC_r$ are here evaluated as:

\begin{equation}
P_eC_r=\frac{|\mathbf{v}|\Delta
  x}{D^*+\alpha_L|\mathbf{v}|}\frac{|\mathbf{v}| \Delta t}{R \Delta
  x}=\frac{|\mathbf{v}|^2\Delta
  t}{R(D^*+\alpha_L|\mathbf{v}|)}=\frac{|\mathbf{q}|^2\Delta t}{\theta
  R(\theta D^*+\alpha_L|\mathbf{q}|)}
\end{equation}

In the streamline diffusion method, according to \cite{Perrochet}
is some additional longitudinal dispersion added to prevent that
$P_eC_r$ exceeds the chosen performance index. The additional
longitudinal dispersion, $\bar{\alpha_L}$ can be calculated as:


\begin{equation}
\bar{\alpha_L}=\begin{cases} \frac{|\mathbf{q}|\Delta
    t}{\theta R\gamma}-\alpha_L-\frac{\theta D^*}{|\mathbf{q}|}, & \text{for}
    \ \alpha_L + \frac{\theta D^*}{|\mathbf{q}|} < \frac{|\mathbf{q}|\Delta
    t}{\theta R\gamma}
 \\ 0, & \text{for} \ \alpha_L +
    \frac{\theta D^*}{|\mathbf{q}|} \geq \frac{|\mathbf{q}|\Delta
    t}{\theta R\gamma}\end{cases}
\end{equation}

\subsection{Stability tests}

To investigate the stability of the numerical model a simple system
 has been modeled. The situation here is a one dimensional column,
 horizontal column with steady-state water flow with pore velocity
 $v$. And a given diffusion, $D$ (both molecular diffusion and
 hydrodynamic dispersion) and retardation factor, $R$. The
 advection-dispersion equation in one dimension can be written as:


\begin{equation}
R\frac{\partial C}{\partial t}=D\frac{\partial^2C}{\partial
  x^2}-v\frac{\partial C}{\partial x}
\end{equation}

where $v=q/\theta$. The initial condition is that the concentration is
uniformly distributed in the column:

\begin{equation}
C(x,0)=C_i
\end{equation}

At the left boundary the solute flux is:

\begin{equation}
(-D\frac{\partial C}{\partial x}+vC)|_{x=0}=\begin{cases} vC_0 &
  0<t\leq t_0 \\ 0 & t>t_0 \end{cases}
\end{equation}

The solution can then according to \cite{Genuchtenanalytical} be
written as:

\begin{equation}
C(x,t)=\begin{cases}C_i+(C_0-C_i)A(x,t) & 0<t\leq t_0 \\
  C_i+(C_0-C_i)A(x,t)-C_0A(x,t) & t>t_0 \end{cases}
\end{equation}

where:

\begin{equation}
\begin{split}
A(x,t)=&\frac{1}{2} \erfc\left
  [\frac{Rx-vt}{2(DRt)^{1/2}}\right]+\left(\frac{v^2t}{\pi
  DR}\right)\exp\left[-\frac{(Rx-vt)^2}{4DRt}\right ] \\
&-\frac{1}{2}(1+\frac{vx}{D}+\frac{v^2t}{DR})\exp(vx/D)\erfc\left[\frac{Rx+vt}{2(DRt)^{1/2}}\right]
\end{split}
\end{equation}

For the simulations, a waterflow situation is made with steady state
flow with the chosen porewater velocity, $v$=10 cm/day. The solute is
injected at the left border from $t=0$ to $t=t_0$. $t_0$ is chosen to
be 2 days. $C_0$ is for the simplicity chosen to 1. For the FEM
simulations the virtual soil column is 1/2 cm high and 50 cm wide. On
the domain a regular mesh is generated with 100 equally large
elements, each  with characteristic lengths of $\Delta x$=1 cm. The
numerical parameter, $\omega$ is set to 1/2, i.e. a Crank-Nicolson
scheme.\\
\\
Figure \ref{fig:soltest1_sub} shows simulation results for
$C_r=1$ and different $P_e$-numbers. The low courant number insures
according to \cite{Perrochet} that the time increments are sufficiently
low. The different P\'{e}clet numbers are obtained by varying the
diffusion. i.e the simulations are not representing the same physical
situation. As it can be seen, there are  saw-tooth instabilities also
called wiggles\index{wiggles}, \cite{Abbott} for the high P\'{e}clet
numbers. As expected, no instabilities are observed for P\'{e}clet
numbers below 2. For $P_e=5$ and $P_e=10$ the wiggles are
significant.

%\begin{figure}[H]  %here-top-bottom-page
%\begin{center}
%%\epsffile{soltest1_sub.eps}
%\epsfig{file=soltest1_sub.eps,width=12cm,height=7.2cm}
%\figcap{FEM solutions shown as concentration as function of x after
%3 days(upper) and as function of time for x=10 cm (lower). Different
%$P_e$-numbers have been used. The Courant number,
%$C_r$ is 1. There are wiggles for large $P_e$-numbers}
%\label{fig:soltest1_sub}
%\end{center}
%\end{figure}

Figure \ref{fig:soltest2_sub} provides graphs for situations with
constant P\'{e}clet number, $P_e=2$ and varying Courant number. The
simulations represent the same physical situation. For Courant numbers,
below, less, or equal to 1 practically the same solution is obtained. For
$C_r=2$ which in this case corresponds to timesteps of 1/5 of a day,
the results are somewhat different and unstable. Very large
oscillations on the graph that shows the concentration as function of
$x$ can be seen for $x \geq 43$ cm. The simulations with $C_r=2$ are
maybe also critical as the timesteps are too large compared with the
time (2 days) for the injection of solute at the left boundary.\\
\\
Figure \ref{fig:soltest3_sub} shows results for different combinations
of $P_e$ and $C_r$, but restricted so $P_eC_r=10$. For all the
simulations, wiggles can be observed, but there is for example no
simple relationship between the P\'{e}clet number and the size of the
saw-tooths. The wiggles stretch over longer time and space for
$C_r=2$ compared with the other simulations. The reasons for that can
simply be that it takes a number of timesteps before the wiggles are
eliminated. \\
\\
Figure \ref{fig:soltest4_sub} shows a case with a P\'{e}clet number of
20. The Courant number is 1. One of the graphs shows the simulations
without any stabilization. Here the wiggles are significant. By
comparing with the analytical solution it is  evaluated that there is
some additional dispersion (numerical
dispersion)\index{dispersion!numerical}. Another graph shows the same
simulation with streamline diffusion with $\gamma=5$. The additional
dispersion is significant but the wiggles have also
disappeared. Another graph shows simulation with the stabilizing
method where the size of $\Delta t$ is changed so
$P_eC_r\leq\gamma$. $\gamma=5$ is used which is the same as reducing
the $C_r$-number to 1/4, i.e. 4 times as many timesteps (or
approximately 4 times longer CPU-time). The last results are close to
the analytical solution. In practical use it is difficult to choose
the stabilizing method - what is good in one situation may be
applicable in another. In the 1D simulations provided here there are
no problems with increasing the space or time discretization - the
CPU-time is under all circumstances limited.


%\begin{figure}[H]  %here-top-bottom-page
%\begin{center}
%%\epsffile{soltest2_sub.eps}
%\epsfig{file=soltest2_sub.eps,width=12cm,height=7.2cm}
%\figcap{FEM solutions shown both as concentration as function of x
%for time=3 days (upper) and as function of time with x=10 cm
%(lower). Different $C_r$-numbers have been used. The P\'{e}clet
%number, $P_e$ is 2 for all the simulations.  It can be seen that the
%simulation with $C_r=2$ is unstable (very large wiggles for $x\geq
%43$).}
%\label{fig:soltest2_sub}
%\end{center}
%\end{figure}



%\begin{figure}[H]  %here-top-bottom-page
%begin{center}
%%\epsffile{soltest3_sub.eps}
%\epsfig{file=soltest3_sub.eps,width=12cm,height=7.2cm}
%\figcap{FEM solutions shown both as concentration as function of x,
%  time=3 days (upper) and as function of time with x=10 cm
%  (lower). Different $P_e$ and $C_r$-numbers have been used, but
%  $P_eC_r$ is equal to 10 for all the figures.}
%\label{fig:soltest3_sub}
%\end{center}
%\end{figure}

%\begin{figure}[H]  %here-top-bottom-page
%\begin{center}
%%\epsffile{soltest4_sub.eps}
%\epsfig{file=soltest4_sub.eps,width=12cm,height=7.2cm}
%\figcap{Analytical and FEM solutions shown both as concentration as
%  function of x, time=3 days (upper) and as function of time with x=10 cm
%(lower). For the FEM-simulations 3 different strategies are used. One
%without stabilization. One with streamline diffusion, $\gamma=5$ and
%lastly one with changing $\Delta t$ so $P_eC_r\leq \gamma$,
%$\gamma=5$.}
%\label{fig:soltest4_sub}
%\end{center}
%\end{figure}




\subsection{Size of the timesteps}\index{timesteps, size of}

Like in the water movement model it is possible to choose between
solution strategies, but here it is combined with different methods to
avoid instabilities like the streamline diffusion method. The
different combinations which TopFEM can handle are given in table
\ref{tab:solstrat}.

%\begin{table}[H]
%%\tabcap{Different solution strategies}
%%\label{tab:solstrat}
%\centering
%\sbox{\mybox}{\begin{tabular}{p{5cm}|p{0.4cm}|p{0.4cm}|p{0.4cm}|p{0.4cm}|p{0.4cm}|p{0.4cm}|p{0.4cm}}\hline
%Solver & 1 & 2 & 3 & 4 & 5 & 6 & 7 \\ \hline Non-linear &  &  &  &
%$\bullet$ & $\bullet$ & $\bullet$ & $\bullet$  \\ \hline Linear &
%$\bullet$ & $\bullet$ & $\bullet$ &  &  &  &   \\ \hline Streamline
%diffusion &  & $\bullet$ &  &  & $\bullet$ &  & $\bullet$  \\ \hline
%$\Delta t$ is constant & $\bullet$ & $\bullet$ &  & $\bullet$ &
%$\bullet$ & &   \\ \hline $\Delta t_{m+1}=f(m_n)$ &  &  &  &  &  &
%$\bullet$ & $\bullet$  \\ \hline $\Delta t$ so $P_eC_r\leq \gamma$ &
%&  & $\bullet$ &  &  &  &  \\ \hline
%\end{tabular}
%\settowidth{\mylength}{\usebox{\mybox}} \setcaptionwidth{\mylength}
%\caption{Different solution strategies}
%\usebox{\mybox}
%\label{tab:solstrat}
%\end{table}

For some simulations of solute movement it is not necessary to use a
Picard iteration loop as specified by equation \ref{eq:Picard}. These
situations occur if none of the matrices  $\mathbf{A}$, $\mathbf{G}$
and $\mathbf{H}$ are dependent of the solution (i.e the
concentration). The ODE is then linear. \\
\\
In many cases $\mathbf{A}$, $\mathbf{G}$ or $\mathbf{H}$ matrices are
dependent of the concentration. These situations occur if the sink
term, chemical processes, retardation factor or boundary conditions
are concentration dependent. For the non-linear solvers with varying
size of the timesteps, i.e. solver 6 and 7 in table
\ref{tab:solstrat}, $\Delta t_{n+1}=f(m_n)$ means that the size of the
timesteps is dependent on the number of Picard iterations in the previous
timestep. The iteration procedure is then almost similar to the one
from the water movement simulations:

\begin{enumerate}
\item if $m\leq 4$ then $\Delta t_{n+1} = 1.1\Delta t_n$ but not larger than
  $\Delta t_{max}$
\item if $m=5$ then $\Delta t_{n+1}=\Delta t_n$
\item if $6 \leq m\leq 7$ then $\Delta t_{n+1}=0.5 \Delta t_n$ but not lower
  than $\Delta t_{min}$
\item if $m>7$ then $\Delta t_{n+1}=0.1\Delta t_n$ but not lower than $\Delta
   t_{min}$
\item if $m=m_{max}$, the time is only updated if $\Delta t_{n+1}=\Delta
  t_{min}$ else it tries again with smaller timesteps  $\Delta t_n=
  0.1\Delta t_n$ but not smaller than $\Delta t_{min}$
\end{enumerate}


\subsection{Mass balance}\index{mass balance!solute}

The mass balance for solute can be defined as:


\begin{equation}
\begin{split}
\text{solute balance} & = \text{Change in stored amount of solute} \\
& + \text{Total flux out of borders} \\
& + \text{Solute removed by physical sinks} \\
& - \text{Solute produced by chemical processes}
\end{split}
\label{eq:solbal}
\end{equation}

With physical sinks is meant that the solute follows with the water
as for example some kinds of root extraction. Degradation processes
are in the mass balance calculations treated as negative
production. The solute balance index is defined as the right hand side
of equation \ref{eq:solbal} divided by 1/4 of the sum of the absolute
value of each of the terms. Ideally it is zero, except for redistribution
cases where both the numerator and the denominator equals zero. TopFEM
calculates the solute balance index after each simulation, and it is
easy to calculate the mass balance.
