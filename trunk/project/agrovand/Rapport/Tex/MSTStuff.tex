%%January 20- 2002
%\documentclass[wrr]{agu2001}
\documentclass{report}


\usepackage{natbibDK}
\usepackage{amsmath}
\usepackage{amssymb}
\usepackage[dvips]{graphicx}



\begin{document}

%% ------------------------------------------------------------------------ %%
%
%  TITLE
%
%% ------------------------------------------------------------------------ %%


\title{Formler til Milj\ostyrrelsen}

%% ------------------------------------------------------------------------ %%
%
%  AUTHORS AND AFFILIATIONS
%
%% ------------------------------------------------------------------------ %%

\author{Mikkel Mollerup}





\chapter{Introduction}



\chapter{Theory}


\section{Richard's Equation}

The water flow in porous media can be described with the formula
of Richard. The equation is here derived. The water flux density
vector, $\mathbf{q}$ can be calculated by the Darcy�s law. For a
two-dimensional vertical transect it yields:
\begin{equation}
\mathbf{q}=-\mathbf{K}(\psi)\nabla(\psi + z) \label{eq:darcy}
\end{equation}
where $\mathbf{K}(\psi)$ is the hydraulic conductivity tensor,
$\psi$ is the potential head. The x-axis is chosen in horizontal
direction and the z-axis is positive upwards. The conductivity
tensor can be expressed as:
\begin{equation}
\mathbf{K}=\begin{bmatrix} K_{xx} & K_{xz} \\ K_{zx} & K_{zz}
 \end{bmatrix}
\end{equation}
\begin{equation}
\mathbf{K}=\begin{bmatrix} K_{xx} & 0 \\ 0 & K_{zz}
 \end{bmatrix}
\end{equation}
The mass balance for the system gives
\begin{equation}
\frac{\partial \theta}{\partial t}=-\nabla \cdot \mathbf{q}
-\Gamma \label{eq:continuity}
\end{equation}
where $\theta$ is the water content and $\Gamma$ is the sink term.
The partial differential equation can be developed by combining
Darcy�s law, equation \ref{eq:darcy} and the mass balance,
equation \ref{eq:continuity}, thus
\begin{equation}
\frac{\partial \theta}{\partial t}=\nabla \cdot
\left(\mathbf{K}(\psi)\nabla (\psi + z)\right) - \Gamma
\label{eq:richards}
\end{equation}
This is known as Richard's equation. For the modelling it is
assumed that the soil-water retention is without hysteresis, i.e.
there is a unique relation between the matric pressure potential
and the water content.

In order to describe Richard equation in $\psi$ the specific water
capacity, $C=\partial \theta/\partial\psi$ has applied. To solve
Richard's equation it is necessary to specify initial and boundary
conditions. The boundary conditions specify a combination of
$\psi$ and its derivative on the boundary. In TopFEM it is
possible to use different forms of flux (Neumann) and predescribed
pressure (Dirichlet) boundary conditions. The problem to be solved
for determining the water movement can be summarized to
\begin{equation}
\begin{cases}
C \cdot \frac{\partial \psi}{\partial t}=\nabla \cdot
\left(\mathbf{K}(\psi)\nabla (\psi + z) \right)-\Gamma & \text{in}\  \Omega \\
\mathbf{\bar{n}} \cdot \left(\mathbf{K}(\psi)\nabla (\psi + z)
\right)=
-q & \text{on}\ \partial \Omega_1 \\
\psi=\psi_0 & \text{on}\ \partial \Omega_{2}
\end{cases}
\label{eq:watermovement}
\end{equation}
where $\mathbf{\bar{n}}$ is the outward unit normal, and $q$ is
the size of the size of the outward flow from the domain. $\psi_0$
is the predescribed pressure at the boundary. $\partial\Omega_1$
and $\partial\Omega_2$ are part of the boundary with Neumann and
Dirichlet boundaries, respectively. Each of $\partial\Omega_1$ and
$\partial\Omega_2$ are not necessarily one continuous curve piece.
For lower boundary condition is a special case of the Neumann
boundary conditions often applied. It here assumed that the flow
it is only driven by the gravity (gravity boundary condition),
i.e. $\partial \psi/
\partial x= \partial \psi/ \partial y=0$ which gives
\begin{equation}
q=\mathbf{\bar{n}} \cdot \begin{bmatrix}0 \\ K_{zz}
\end{bmatrix}
\end{equation}
Another often used boundary condition is the seepage boundary
  condition where we for $\psi>0$ have specified pressure
  corresponding to the depth of the overlaying water. For $\psi \leq0$
  have specified flow to be equal to 0, i.e. a Dirichlet boundary
  condition for $\psi>0$ and a Neumann boundary condition for $\psi
  \leq 0$.


\chapter{2 Dimensional Model}

\section{FVM Model}


Richards ligning, integration over kontrol-volumen (celle) og Gauss-Green divergens teorem:


\begin{equation}
\int_{Q_i} \frac{\partial \theta}{\partial t} d\mathbf{x} =
\int_{\partial Q_i} \left(\mathbf{K}(\psi)\nabla (\psi + z) \right)\cdot \mathbf{\bar{n}} dl
- \int_{Q_i} \Gamma d\mathbf{x}
\label{eq:integratet}
\end{equation}


Sammenfattes til:

\begin{equation}
|Q_{i}|\frac{d \theta_i}{dt} = \sum_{j \in \sigma_i} D_{ij}(\boldsymbol{\psi})
 + \sum_{j \in \sigma_i} G_{ij}(\boldsymbol{\psi})
 + \sum_{j' \in \sigma_i} B_{ij'}(\boldsymbol{\psi})
 + S_{i}(\boldsymbol{\psi})
\end{equation}

hvor:

$D_{ij}(\boldsymbol{\psi})$ diffusiv transport mellem interne kanter\\
$G_{ij}(\boldsymbol{\psi})$ gravitation transport mellem interne kanter  \\
$B_{ij'}(\boldsymbol{\psi})$ rand kanter $j'\in \sigma_{i}'$ \\
$S_{i}(\boldsymbol{\psi})$ er kildeledet, der b�de kan inkludere punkt- og fladekilder i cellen. \\


\begin{equation}
D_{ij}(\boldsymbol{\psi})=|e_{ij}|(\mathbf{K}(\boldsymbol{\psi})\cdot (\nabla \psi)_{ij})\cdot \mathbf{\bar{n}}_{ij}
\end{equation}

\begin{equation}
G_{ij}(\boldsymbol{\psi})=|e_{ij}|(\mathbf{K}(\boldsymbol{\psi})\cdot([0\ 1]^T))\cdot \mathbf{\bar{n}}_{ij}
\end{equation}




\subsection{Rektangul�re celler}

Taylorudvikling...

\begin{equation}
\psi_E = \psi(x+\delta x^+)=\sum_{k=0}^{m}  \frac{1}{k!} \left(\frac{d^k \psi}{dx^k}\right)_f (\delta x^+)^k  + R^+
\end{equation}

hvor $m$ er Taylorudviklingens orden og $R^+$ er Lagranges restled. Tilsvarende f�s

\begin{equation}
\psi_i = \psi(x-\delta x^-)=\sum_{k=0}^{m}  \frac{1}{k!} \left(\frac{d^k \psi}{dx^k}\right)_f (-\delta x^-)^k  + R^-
\end{equation}

$R^+ - (-1)^{m+1}R^- \approx 0$ Hvis der v�lges en f�rste ordens Taylorudvikling ($m=1$) f�s

\begin{equation}
\left( \frac{d \psi}{dx} \right)_f (\delta x^+ +\delta x^-)\approx \psi_E-\psi_i
\end{equation}

S�fremt der v�lges en h�jere ordens Taylorudvikling f�s:

\begin{equation}
\left( \frac{d \psi}{dx} \right)_f (\delta x^+ +\delta x^-)\approx \psi_E-\psi_i - \epsilon_{Ei}
\end{equation}

hvor korrektionsledet kan beregnes som

\begin{equation}
\epsilon_{Ei} \approx \sum_{k=2}^{m}  \frac{1}{k!} \left(\frac{d^n \psi}{dx^n}\right)_f
\left[ (\delta x^+)^n - (-\delta x^-)^n \right]
\end{equation}


Det kan indses at der opn�s anden ordens pr�cision med $m=1$ s�fremt $\delta x^+=\delta x^-$


\begin{equation}
\sum_{j \in \sigma_i} D_{ij}(\boldsymbol{\psi})=
\end{equation}


\begin{equation}
\sum_{j \in \sigma_i} D_{ij}(\boldsymbol{\psi})=
\end{equation}

\subsubsection{Dirichlet randbetingelser}

Simplest ville det v�re......  Her benyttes dog




\subsection{Trapezoide celler}


\subsubsection{Line{\ae}r rekonstruktion}


\begin{equation}
\hat{\psi}(\mathbf{x},t)=\psi_i(t)+\eta_i(\boldsymbol{\psi})\cdot(\mathbf{x}-\mathbf{x}_i), \ \ \mathbf{x} \in Q_i, \ t>0
\end{equation}

Divergens teoremet:

Trekanter:
\begin{equation}
\overline{\nabla \psi} \approx \sum \psi_j \mathbf{n}_j A_j \approx
\frac{1}{2|T_i|}\mathbf{R}\left[\psi_{\alpha}(\mathbf{x}_{\beta}-\mathbf{x}_{\gamma})
+\psi_{\beta}(\mathbf{x}_{\gamma}-\mathbf{x}_{\alpha})
+\psi_{\gamma}(\mathbf{x}_{\alpha}-\mathbf{x}_{\beta})\right]
\end{equation}

Firkanter:
\begin{equation}
\overline{\nabla \psi} \approx \sum \psi_j \mathbf{n}_j A_j \approx
\frac{1}{2|Q_i|}\mathbf{R}\left[(\psi_{\alpha}-\psi_{\gamma})(\mathbf{x}_{\beta}-\mathbf{x}_{\delta})
+(\psi_{\beta}-\psi_{\delta})(\mathbf{x}_{\gamma}-\mathbf{x}_{\alpha})\right]
\end{equation}

hvor

\begin{equation}
\mathbf{R}=\begin{bmatrix} 0 & 1 \\ -1 & 0 \end{bmatrix}
\end{equation}



Harmonisk gennemsnit:

\begin{equation}
\frac{1}{K_{ij}} = \frac{1}{2}\left[
\frac{1}{K(\psi_i)}+\frac{1}{K(\psi_j)}\right]
\end{equation}




\subsection{Iterations skema}


\begin{equation}
\mathbf{Q}\frac{d\boldsymbol{\theta}}{dt}=\mathbf{E}(\boldsymbol{\psi})\boldsymbol{\psi}+\mathbf{F}(\boldsymbol{\psi})
\end{equation}

hvor $\mathbf{Q}$ er en diagonalmatrix med $Q(i,i)=|Q_i|$

Backward Euler:

\begin{equation}
\mathbf{Q}\frac{\boldsymbol{\theta}^{n+1,m+1}-\boldsymbol{\theta}^{n}}{\Delta t}
=\mathbf{E}(\boldsymbol{\psi}^{n+1,m})\boldsymbol{\psi}^{n+1,m}+\mathbf{F}(\boldsymbol{\psi}^{n+1,m})
\end{equation}


massconservative \citet{Celia}


The $\psi$-based formulation have the disadvantage that it doesn't
conserve the mass and can give erroneous estimate of infiltration
depths, \linebreak  \cite{Celia}. In the mixed formulation is the
water content at time step $n+1$ and iteration step $m+1$
appromximated by a Taylor expansion:

\begin{equation}\begin{split}
\theta^{n+1,m+1}&=\theta^{n+1,m}
+\frac{d\theta}{d\psi}\mid^{n+1,m}(\psi^{n+1,m+1}-\psi^{n+1,m})\\
&=\theta^{n+1,m} +C^{n+1,m}(\psi^{n+1,m+1}-\psi^{n+1,m})
\label{eq:taylor}
\end{split}\end{equation}

The time derivative of $\theta$ can then be approximated as:

\begin{equation}\begin{split}
\frac{\partial \theta}{\partial t}&\approx
\frac{\theta^{n+1,m+1}-\theta_{n}}{\Delta
  t}=\frac{\theta^{n+1,m+1}-\theta^{n+1,m}}{\Delta
  t}+\frac{\theta^{n+1,m}-\theta_{n}}{\Delta t}\\ & \approx C^{n+1,m}
\frac{\psi^{n+1,m+1}-\psi^{n+1,m}}{\Delta
  t}+\frac{\theta^{n+1,m}-\theta^{n}}{\Delta t}
\end{split}\end{equation}


Hermed f�s f�lgende iterative skema...

\begin{eqnarray}
\left( \frac{1}{\Delta t} \mathbf{QC}(\boldsymbol{\psi}^{n+1,m})-\mathbf{E}(\boldsymbol{\psi}^{n+1,m}) \right)
\boldsymbol{\psi}^{n+1,m+1} = && \nonumber \\
\mathbf{F}(\boldsymbol{\psi}^{n+1,m}) + \frac{1}{\Delta t} \mathbf{QC}(\boldsymbol{\psi}^{n+1,m}) \boldsymbol{\psi}^{n+1,m}
+\frac{1}{\Delta t} \mathbf{Q}\left( \boldsymbol{\theta}^{n}-\boldsymbol{\theta}^{n+1,m} \right)
\end{eqnarray}


hvor $\mathbf{C}$ er en diagonalmatrix med $C(i,i)=C_i$




\subsection{L�sning af matrixligning}

blabre blabre

\subsection{Hydrauliske modeller}

In numerical models for the unsaturated zone the soil-water model
by \citet{vanGenuchten} is widely used:
\begin{equation}
\theta=\begin{cases} \theta_{r} +
\frac{\theta_s-\theta_r}{[1+|\alpha \psi|^n]^m} & \text{for
  $\psi<0$}\\
\theta_{s} &\text{for $\psi \geq 0 $} \end{cases}
\end{equation}
where $\alpha$, $n$ and $m$ are empirical parameters, $\theta_s$
and $\theta_r$ are the saturated and the residual water content,
respectively. By combination with the hydraulic conductivity model
by \citet{Mualem} and choosing $m=1-1/n$, the hydraulic
conductivity can be calculated as
\begin{equation}
K=K_sS_{e}^{1/2}[1-(1-S_{e}^{1/m})^m]^2
\end{equation}
where $K_s$ is the hydraulic conductivity at saturation and $S_e$
is the effective saturation defined as
\begin{equation}
S_e=\frac{\theta-\theta_r}{\theta_s-\theta_r}
\end{equation}
The retention model by van Genuchten have been adopted to a large
class of soils.



\subsection{Ridge}

For describing the geometry and producing the finite element mesh
is the general FEM-code, \citet{FEMLAB} used. In the actual case
the two-dimensional geometry described using a so called geometry
m-file. Of geometrical reasons only the half of a ridge is
described. The soil profile is divided into 7 strata or subdomains
with different soil properties which are described elsewhere in
the paper. The ridge with the different subdomains is plotted in
figure XXX. The ridge height can be described with a sine
function:
\begin{equation}
f(x)=A \left[ 1+sin\left(-\frac{\pi}{2}+2\frac{\pi x}{W}\right)
\right], \ \ \ 0 \leq x \leq W/2
\end{equation}
where $W$ is the width of the ridge and $A$ is the amplitude of
the sine wave which is the same as half of the ridge height. The
curve only describes half a ridge that will be used for the
modelling.







\section{Verifikation}

Her skal der st� lidt Philipssjov



\subsection{Philips Infiltration Model}


\citet{Philip} showed that the infiltration depth as function of
time and saturation can be written as a power series in
$t^{\frac{1}{2}}$. The coefficients are then functions of soil
water content, $\theta$. From the expression for the infiltration
depth, as function of water content and time it is relatively easy
to derive the cumulative infiltration, also written as a power
series in $t^{\frac{1}{2}}$. The assumptions for the theory is an
1-dimensional vertical flow into a homogenous soil semi-infinite
soil column, initially with uniform water content. The upper
boundary condition is a constantly held water content or pressure
(Neumann boundary condition).

Results from infiltration experiments
\citep[e.g.,][]{Austin,Mbagwu1,Mbagwu2,Maheshwari,Valiantzas} are
often fitted to a infiltrations equation only consisting of the
first 2 terms of the power series solution

\begin{equation}
I=S\sqrt{t}+At \label{eq:Philip}
\end{equation}

where $S$ is the often refereed sorptivity as defined in
\citet{PhilipAdv}. For some of the experiments, the geometry,
initial and boundary condition does not corresponds to the
assumptions Philip made for using the power series as a solution
to Richards equation. The appliance of the Philip infiltration
model is here more of empirical than mechanistic nature.




\section{Referencer}



Liste med referencer:\\
\\

\begin{itemize}
\item \citep{Mollerupphd}\\
\item \cite{Philip} \\
\item \cite{PhilipTrans,PhilipAus}
\end{itemize}




% ------------------------------------------------------------------------ %%
%
%  REFERENCE LIST AND TEXT CITATIONS
%
%% ------------------------------------------------------------------------ %%


\bibliographystyle{natbibDK}
\bibliography{MST}


\end{document}
