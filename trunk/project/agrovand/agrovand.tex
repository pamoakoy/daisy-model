\documentclass[a4paper]{report}

%%\usepackage[left=1cm,top=2cm,right=1cm]{geometry}
\usepackage[top=3cm,bottom=2cm]{geometry}
\usepackage[latin1]{inputenc}
\usepackage[T1]{fontenc}
\usepackage[danish,english]{babel}
\usepackage{natbib}
\bibliographystyle{apalike}
\usepackage{graphicx}
\usepackage{hyperref}
\usepackage{fancyhdr}
\usepackage{placeins}
\pagestyle{fancy}
\lhead{\today}
\usepackage{multirow}

\newcommand{\koc}{$\mbox{K}_{\mbox{\textsc{oc}}}$}
\newcommand{\kclay}{$\mbox{K}_{\mbox{clay}}$}
\newcommand{\kd}{$\mbox{K}_{\mbox{d}}$}
\newcommand{\rhob}{$\rho_{\mbox{b}}$}
\newcommand{\hlim}{\mbox{$h_{\mbox{lim}}$}}

\newcommand{\focus}{\textsc{focus}}
\newcommand{\hypres}{\textsc{hypres}}
\newcommand{\Hypres}{\textsc{Hypres}}
\newcommand{\macro}{\textsc{macro}}
\newcommand{\Macro}{\textsc{Macro}}

\newcommand{\figl}{\hspace*{-2cm}}
\newcommand{\figright}[1]{\includegraphics{fig/#1}}
\newcommand{\fig}[1]{\figl\figright{#1}}
\newcommand{\figtop}[1]{\figl\includegraphics[trim=0mm 5mm 0mm 0mm,clip]{fig/#1}}

\newcommand{\figctop}[1]{\hspace*{-1cm}\figright{#1}} 
\newcommand{\figc}[1]{\vspace*{-1.5cm}\figctop{#1}}

\newcommand{\MyID}{}

\begin{document}

\chapter*{Daisy 2D simulation of R�rrendeg�rd}

Part of project
\begin{otherlanguage}{danish}
  \begin{it}
    Flerdimensional modelling of vandstr�mning og stoftransport i de
    �verste 1-2 m af jorden i systemer med markdr�n
  \end{it}
\end{otherlanguage}
for the Danish Environmental Protection Agency.
\vspace{1cm}

\begin{bf}
  \begin{large}
    \noindent
    S�ren Hansen \texttt{$<$sha@life.ku.dk$>$}\\
    Per Abrahamsen \texttt{$<$abraham@dina.kvl.dk$>$}\\
    Carsten Pedersen \texttt{$<$cpe@life.ku.dk$>$}\\
    Marie Habekost Nielsen \texttt{$<$maha@life.ku.dk$>$}\\
    Mikkel Mollerup \texttt{$<$mmo@geus.dk$>$}\\
    \\
    \today{}\\
  \end{large}
\end{bf}
\vfill\noindent
University of Copenhagen\\
Department of Basic Sciences and Environment\\
Environmental Chemistry and Physics\\
Thorvaldsensvej 40\\
DK-1871 Frederiksberg C\\
Tel: \texttt{$+$45 353 32300}\\
Fax: \texttt{$+$45 353 32398}

\tableofcontents

\chapter{Introduction}

The R�rrendeg�rd site is part of the Copenhagen University
experimental station in T�strup.  It was selected for for the present
project mainly because because of the high resolution flow
proportional drain data collected as part of the Agrovand project in
the four drain season between between 1998 and 2002, which included
soil particles, a likely transport path for strongly sorbing
pesticides.  The results of the Agrovand project have been partly
documented in~\citet{petersen200181} (biopores),
\citet{petersen2002movement} (particles and pesticides),
\citet{petersen2004movement} (particles and bromide), and
\citet{petersen2008spatio} (anisotropy).

The main focus of the Agrovand project was the influence of tillage on
the soil as a transport medium, so four plots with different tillage
strategies were followed.  In this project we have only studied the
data from plot 4, representing conventional tillage, and only the
first three season where the best data is available.  Apart from
particles, the drain water has been analysed for bromide (one
application), pendimethalin (two applications) and ioxynil (one
application), which we have chosen to include in our simulations
simulate.  Soil water have been followed with piezometers,
tensiometers, and TDR probes.  The most stable results are from the
TDR probes, and the only one we have used for our calibration.
Furthermore, transport pathways have been explored using dye tracer,
and biopores have been counted both in the original project, and with
more detail forming the basis for the new Daisy biopore model in the
present project~\citep{habekost1,habekost2,habekost3}.

The goal for the simulations presented in this paper is to test the
two subcomponents of the the newly developed 2D Daisy against real
data: The first is the particle generation and filtration modules; the
second is the slow/fast water movement.  For the later, we will use
the drain bromide data which are available at a high resolution, and
where we have reliable soil measurements to back them up with.  The
pesticide data is presented together with uncalibrated simulation
results as the PLAP sites have more detailed pesticide data
available~\citep{lindhardt2001,vap2009,vap2d}.  The Agrovand data is a
useful suplement though, as we don't have particle data for the PLAP
sites, and the PLAP bromide suffer from the fact that the application
was in spring, meaning an unknown amount have been uptaken by the
crop.  In Agrovand the bromide application was in the autumn,
minimizing the potential plant uptake.

\chapter{Setup}

The Agrovand data has been used from the beginning of the current
project for developing the new model, giving the final setup a rich
history.
\begin{enumerate}
\item An initial setup was developed for water and bromide using the
  original Daisy model by Tilde Hellsten, as part of her Master
  Thesis~\citep{tilde-agrovand}.
\item A setup for water using the new 2D model was developed by Nanna
  Gudmand-H�yer, as part of her Master Thesis~\citep{nanna-agrovand}.
\item This 2D setup was extended for bromide, particles, and
  pesticides by Mikkel Mollerup, and used as a basis for the PLAP site
  calibration~\citep{vap2d}.
\item Based on the model changes and experience gained made during the
  PLAP site calibration, the setup was recalibrated by Per Abrahamsen.
\end{enumerate}

This history does mean that the setup likely contain parameter choices
that no longer are applicable due to changes in the model, and that a
new setup made from scratch could be simpler or give better results,
had time permitted.

\section{Weather}

All weather data with the exception of precipitation were collected at
a station located at H�jbakkeg�rd.  Three sources were considered for
precipitation.  Hourly measurements 1.2 meter above ground at the
field in the drain seasons, hourly measurements at H�jbakkeg�rd also
1.2 meter above ground, and daily measurements at ground level.

As a starting point, we used the hourly field measurements for the
drain season, supplemented with the hourly measurements from
H�jbakkeg�rd for the rest of the season.  These were compared with the
daily measurements.  Where the daily measurements showed precipitan
but the hourly measurements didn't we examined the TDR measurements
near the surface.  If they indicated precipitation, the daily were
used to supplement the hourly measurements.  Comparison of monthly
sums between the hourly and daily precipitation data indicated no
systematic bias, thus the hourly data were used without correction for
possible effect of wind and snowfall.  

Whether the precipitation falls as snow or rain will obviusly affect
the drain flow, especially at short time scale.  Unfortunately, we did
not have direct measurements of the type of precipitation.  A build-in
model of Daisy will let an increasing amount of the precipitation fall
as snow when the air temperature drops below 2$\,^\circ$C.  This works
reasonable well for long time simulations, but not when we as here are
inteersted in the individual events.  For simplicity, we chose to full
disable this snow model, so all precipitation in the simulation will
fall as rain.

\section{Management}

All seasons had winter wheat with mineral fertilizer, with one plowing
operation between harvest and sowing.  For Daisy, the dates of the
plowing, sowing and harvest is used (table~\ref{tab:crop-man}).
Furthermore, Daisy use information about the seed bed preperation.  As
we have not enabled nitrogen in the simulation, the fertilization
operations are irrelevant.  We use default parameters for the tillage
operations.  For the harvest, we specify 8 cm stub and that stems and
leaves are left on the field.  However, since we have not enabled a
model for above ground litter, and we are not interested in soil
organic matter, that information is not used in the simulation.

In the 2004 and 2005 seasons, the potential evapotranspiration for a
winter wheat on the experimental field was measured using an eddy
covariance system, and from this a dynamic crop factor was calculated
(\citet{kjaersgaard2008crop}).  The the default parameterization was
adjusted based on this, and furthermore as part of calibration of soil
water the max penetration depth was increased to 1.5 meter, and the
interception coeeficient were lowered to 0.05 mm per LAI.

\begin{table}[htbp]
  \caption{Dates for crop management operations.  The initial crop was
    sowed 1997-9-23.}
  \label{tab:crop-man}
  \centering
  \begin{tabular}{l|lll}\hline
    Operation & 1998 & 1999 & 2000 \\
\hline
    Harvest & 8-20 & 8-20 & 8-20  \\
    Plowing & 9-15 & 9-15 & 9-15 \\
    Sow & 9-23 & 9-27 & 10-18 
  \end{tabular}
\end{table}

Date and amount is specified for pesticide and bromide application.
The model setup described in \citet{vap2d} were duplicated here, with
field values for DT50 and \koc{} taken from \citet{ppdb20100517}.  No
calibration was done on the persticides.  See
table~\ref{tab:pest-man}.

\begin{table}[htbp]
  \caption{Pesticide and bromide application.}
  \label{tab:pest-man}
  \centering
  \begin{tabular}{l|l|l|r|r}\hline
    Date       & Name          & Amount [g/ha] & DT50 [d] & \koc{} [ml/g] \\
\hline
    1998-11-24 & Bromide       & 34000 & & \\
    1999-11-16 & Pendimethalin &  2000 & \multirow{2}{*}{90}
                                       & \multirow{2}{*}{15744} \\
    \multirow{2}{*}{2000-11-10} 
               & Pendimethalin &  2000 & & \\
               & Ioxynil       &   200 & 5 & 276
  \end{tabular}
\end{table}

All management operation are assumed to be performed at noon.

\section{Soil profile and biopores}
\label{sec:soil-profile}

The soil profile and the description of the drain canyon is based on
the work presented in \citet{habekost3}, where \textsc{isss4} testure
classification was used.  \citet{petersen200181} presents testure and
dry bulk density (\rhob{}) analyses for four depths, which have been
used as basis for the main horizons.  Unfortunately, no measurements
for the C horizon is presented, instead we use the measurement from
the bottom of the B horizon (85--90 cm).  The Ap measurements (10--15
cm) differers between treatments and between spring and autumn, we
have used the spring values for T4 (conventional tillage).  The soil
humus data are from plot A in \citet{petersen2002movement}.  The
values used are summarized in table~\ref{tab:texture}.

\begin{table}[htbp]
  \caption{Soil properties. Depth is specified in cm below soil surface, 
    and the dry bulk density (\rhob{}) specified in g/cm$^3$.
    Humus is given as a percentage of total weight.  For the drain canying, 
    where the \textsc{isss4} texture classification system was used,
    the mineral mineral soil particle distribution is also given as
    fraction of total weight.  For the other horizons the \textsc{usda3} 
    system was used, and the mineral soil particle distribution is given 
    as percentage of total mineral weight.}
  \label{tab:texture}
  \centering
  \begin{tabular}{rrrrrrrr}\hline
    Horizon & Depth & Clay & Silt & \multicolumn{2}{r}{Sand} & Humus
            & \rhob{} \\
    & & $< 2\;\mu$m & $< 50\;\mu$m & \multicolumn{2}{r}{$< 2\;$mm} & & \\\hline
    Ap & 0--25 & 10.7 & 22.2 & \multicolumn{2}{r}{67.1} & 3.0 & 1.49 \\
    Plow pan & 25--33 &  14.8 & 21.4 & \multicolumn{2}{r}{63.8} & 1.6 & 1.70 \\
    Bt & 33--120 & 22.2 & 19.5 & \multicolumn{2}{r}{58.3} & 1.6 & 1.65 \\
    C & 120--200 & 20.7 & 23.5 & \multicolumn{2}{r}{55.8} & 1.0 &  1.69 \\
    \\
    Area & Depth & Clay & Silt & Fine Sand & Coarse sand & Humus
            & \rhob{} \\
    &  & $< 2\;\mu$m & $< 20\;\mu$m & $< 200\;\mu$m & $< 2\;$mm & & \\\hline
    Drain canyon & 33--120 & 21.3 & 19.0 & 24.4 & 33.9 & 1.4 & 1.65 
  \end{tabular}
\end{table}

Initially, three classes of biopores were used in the simulation based
on \citet{habekost1}, where we focused on the biopores that had
potentially had connection to the drain pipes.  Initially we assumed
that all the deep biopores (the two classes ending in 120 cm) in the
drain canyon would be directly connected to the drain pipes.  Based on
pesticide measurements in drains in the PLAP sites, we decided to
change this so only half the deep biopores in the drain canyon would
be directly connected to the drain pipes \citep{vap2d}.  Compared to
the PLAP simulations, we were had additional soil bromide measurements
(section~\ref{sec:brom-cal}), so we decided to add an extra class
ending halfway down.  The meaurements of \citet{petersen200181}
indicated a roughly linear decrease of biopore densisty with depth, so
we chose to use the same density as for the full length biopores.  The
classes are summarized in table~\ref{tab:biopores}.

\begin{table}[htbp]
  \caption{Biopore classes.}
  \label{tab:biopores}
  \centering
  \begin{tabular}{ll|rrrr}\hline
    Depth    & cm      & 0--25 & 0--120 & 30--120 & 0--60\\
    Diameter & mm      & 2    & 4       & 4       & 4 \\
    Density  & m$^{-2}$& 100  & 23      & 7       & 23
  \end{tabular}
\end{table}

The organic matter and nitrogen modules were disabled in order to save
time.

\section{TDR and hydraulic properties}

\Hypres{} was used initially to estimate hydraulic properties for all
horizons. The TDR measurements (see figure~\ref{fig:tdr}) have been
used for calibrating.  The only parameter that has been changes is
$K_{\mbox{sat}}$ (saturated conductivity).  For the surface layer (top
3 cm), this has been decreased to 10\% of the value suggested by
\hypres{}.  In the Bt horizon conductivity has been decreased to 50\%,
and in the C horizon is has been trippled.  The result is shown on
figure~\ref{fig:Rorrende-hor}.

\begin{figure}[htbp] 
  \fig{Rorrende-Ap-Theta}\figright{Rorrende-Ap-K}\\
  \fig{Rorrende-Bt-Theta}\figright{Rorrende-Bt-K}\\
  \fig{Rorrende-C-Theta}\figright{Rorrende-C-K}\\
  \fig{Rorrende-DC-Theta}\figright{Rorrende-DC-K}
  \caption{R{\o}rrende soil hydraulic properties.  \Hypres{} refers to
    parameters estimated according to \citet{hypres}, Daisy to the
    final parametrization (ignoring anisotropy and biopores), and
    surface and plow pan to the conditions at the top of the A and Bt
    horizons.}
  \label{fig:Rorrende-hor}
\end{figure}

Based on \citet{petersen2008spatio} we chose to add an
anisotropy of 12 (meaning horizontal flow is 12 times faster than
vertical) to the plow pan.

\section{Groundwater table and drain water}


An EM38 map of the field indicated large areas to have a sandy
underground, and the piezometers showed that these areas had a
significantly lower groundwater level.  As a rough cut, we estimated
that two thirds of the field did not contribute to the drainage
through the groundwater level.  In Daisy we modelled this by dividing
the field into two columns.  The first column had a free drainage
lower boundary, and represented twice the area of the other column,
with an aquitard bottom.  The aquitard layer was described with a size
(2 meters), a conductivity (0.5 mm/h) and the pressure table of the
underlying aquifer.  The presure table was made variable in order to
match drain flow, and is shown on figure~\ref{fig:gwt} together with
the simulated ground water table.  The free drainage column would
still contribute to the drain water through directly connected
biopores.  The TDR and soil bromide measurements were all performed in
the part of the field with clay underground, and the comparisons are
therefore done to the column with an aquitard.

\begin{figure}[htbp]
  \begin{center}
    \fig{gwt}
  \end{center}
  \caption{Groundwater table.  Simulated low value is calculated from
    pressure in lowest unsaturated numeric cell, typically located
    near drain.  Simulated high value is calculated from pressure in
    highest saturated cell, typically farthest away from the drain.
    The sudden jumps of the high value represents situations with
    surface ponding, where the top numeric cell becomes saturated.
    The pressure table is a calibration parameter.}
  \label{fig:gwt}
\end{figure}

\section{Soil bromide and the secondary domain}
\label{sec:brom-cal}

We have not included cracks in the description of the conductivity
curve, but still divide water into two domains for the sake of solute
transport.  This division was esxclusively calibrated based on bromide
soil measurements shown on the top graph on
figure~\ref{fig:bromide-acc}.  The simulated dynamics shown on
figure~\ref{fig:bromide} were used as a help.  The two figures are
explained in section~\ref{sec:soil-bromide}.  

The division between water in to domain (the primary ``slow'' domain
and the secondary ``fast'' domain) is controlled by single horizon
specific parameter, \hlim{}, a pressure head.  If the actual pressure
head ($h_a$) is below \hlim{}, all matrix water will be part of the
primary domain where.  Othewise, the water in the soil corresponding
to \hlim{} is considered part of the primary domain, and any
additional matrix water is considered part of the secondary
domain. The water flux calculated by Richard's equation ($q$) will be
divided so the primary domain water flux ($q_1$) is
\begin{displaymath}
  q_1 = \frac{K (\hlim)}{K (h_a)}
\end{displaymath}
where $K(h)$ is the hydraulic conductivity at pressure head $h$, and
the secondary domain water flux ($q_2$) is $q_2 = q - q_1$.
Solute transport in the primary domain is calculated with the
convection-dispertion equation, while solute movement in the secondary
domain is handled as pure convection.  A second parameter, $\alpha{}$
determine the speed of exchange between the two domains.

The bromide were measured in for intervals of 25 cm, starting from the
soil surface.  The results show the highest bromide concentrations
below 50 cm.  The results were based on 16 random samples of each
plot, and the patern were similar in the three other
plots.~\citep{petersen2004movement}.  Using a plain one domain
convection-dispertion equation, our simulations showed that most
bromide should still be in the top 50 cm.  In other word, this was a
classic case where the convection-dispertion equation, which assumes
full equilibrium between solute content in different pore classes, was
inadequate.  The idea was that by dividing the pore classes in two
domains, and calculating transport seperately for each domain, the
bromide could stay in the secondary domain and move down faster.

As an initial guess, we used $\hlim{} = 2\; \mbox{pF}$ and $\alpha =
0.01\; \mbox{h}^{-1}$, the later taken from \citet{jaynes1995field}.
Using these values, our initial results were far worse than with the
pure convection-dispertion equation.  In these simulations, the bromid
would stay in the top 25 cm.  There were two problems: The soil
surface was dry enough that much of solute would enter the primary
domain, and stay relatively protected there.  Lowering \hlim{} to 3 pF
in the soil surface would ensure that all the water (and solute) would
enter the secondary domain.  The second problem was the long period,
over a month, before two large events caused significant leakage out
of the plow layer.  With an $\alpha$ of $0.01\; \mbox{h}^{-1}$ this
was plenty of time to reach equilibrium, again causing some of the
bromide to be protected in the primary domain.  We got the best
results by lowering $\alpha$ to $0.00003\; \mbox{h}^{-1}$ in the top
soil (to the bottom of the plow pan), decreasing it further had little
effect.  

As the biopores were the main transport mechanism through the plow
pan, we added a new biopore class that ended in 60 cm in order not to
bypass the 50--100 cm area entirely, see
section~\ref{sec:soil-profile}.  This gave a problem for estimation of
$\alpha$ below 33 cm.  A too high value would cause some bromide to
stick just below the plow pan, where it would count as part of the
25-50 cm interval.  A too low value would cause the bromide that were
transported down to 60 cm through the biopores to move too fast below
100 cm.  We never found a good value.  The values used are listed in
table~\ref{tab:secondary}.

\begin{table}[htbp]
  \caption{Two domain solute transport parameters.}
  \label{tab:secondary}
  \centering
  \begin{tabular}{lll}\hline
    Depth [cm] & \hlim{} [pF] & $\alpha$ [h$^{-1}$] \\\hline
    0-33 & 3.0 & 0.00003 \\
    33-  & 2.0 & 0.0001
  \end{tabular}
\end{table}

\section{Particles}

Particles in Daisy is generated on the soil surface as a result of
rainfall, and then transported down through the soil matrix or
biopores.  We use the filter function from~\citet{macro-colloid} for
the matrix domain.  As the matrix domain in Daisy is divided into a
primary and secondary domain, we use difference filter cooeficients
for the two domains.  We choose values of 80 and 40 m$^-1$ for the
primary and secondary domain respictively, in order to stay near the
50 m$^-1$ used in~\citet{macro-colloid}. Daisy will (unlike \macro{})
not filter particles in the biopores, only in the matrix.

For the particle generation we tried multiple models
\citep{Styczen88,EUROSEM,macro-colloid}, but
only~\citet{macro-colloid} gave anything near the desired dynamics.
It was also the only of the models designed to match drain
measurements, and the only model with a pool of readily available
particles.  We use the values from~\citet{macro-colloid} as a starting
point, except for the maximum particle storage ($M_{\mbox{max}}$)
which is estimated based on the clay content as described
in~\citet{mmax} (method 1).  From calibration, we would initially
conclude that the detachment rate coefficient ($k_d$) should be
decreased to 7.5 g/J, the replenishment rate ($k_r$) to 0.1 g/m$^2$/h,
and the depth of the soil affected by detachment and dispersion
($z_i$) to 0.5 mm.  These values were used for PLAP simulations.
Later we found that reverting to the values from~\citet{macro-colloid}
gave better results, and those values are used for the present
simulations.

The results are discussed in section~\ref{sec:drains}.

\chapter{Results}

The simulation results are presented together with measured data in
figures~\ref{fig:first} to~\ref{fig:last}, found at the end of the
report.  The figures have a high information density, and have
therefore been allowed to fill most of the page.  Each figure contains
multiple graphs, all of which share the same x-axis.  This structure
is intended to facilitate comparison.  The same figures were used for
calibration.  All figures have precipitation and air temperature for
the period in the top graph.

\section{Soil water}

Figure~\ref{fig:tdr} shows horizontal TDR measurements for different
depths.  The two authomn gaps are after plowing, when the TDR probes
are removed.  At the third season the TDR probes had drifted, and were
left out.  The TDR probes does not work on ice, which explain the
noise in the measurements during periods with frost.  The simulation
overstimate the water level near the soil surface, which could
possibly be a problem with the TDR measuring some air.  We may
overstimate the dynamics near the bottom of the plow layer.  The
measurements for the bottom TDR show fast variations during the winter
which looks mostly like noise, something not duplicated in the
simulation.

Figure~\ref{fig:tdr-zoom} shows the same data, but only for the first
summer after installation, where the TDR measurements are likely to be
most reliable.  The general water level seem to be slightly
overestimated at the end of the period, except in the 60 cm TDR where
it is underestimated.  Also, the effect of the two first large events
(based on daily precipitation) goes deeper in the simulation than in
the measurements.

\section{Soil bromide}
\label{sec:soil-bromide}

The measured and simulated bromide content in the top four 25 cm
intervals is shown on the top graph in figure~\ref{fig:bromide-acc}.
The period is from right before application, to right after the soil
measurement.  

As can be seen, the content of 25-50 cm is overestimated in the
simulation, while the content of 50-75 and 75-100 cm are both
underestimated.  The same pattern (most bromide found in the deeper
intervals) were found for the three other treatments, not shown here.
The two next graphs below that divide the content in the same
intervals into the primary domain (small pores, slow water movement)
and the secondary domain (large pores, fast water movement).  The
remaining graphs shows bromide transport through the borders between
the soil intervals.  As can be seen, the secondary domain dominate at
0 and 25 cm, while the tertiary domain (biopores) below.  There is no
significant transport in the primary domain.  However, at the end of
the period the primary domain dominate storage.

Figure~\ref{fig:bromide} show the usual weather graph at the top.
Next is a graph showing how the water enter the system.  We see that
the first rain after application enter the soil through the secondary
domain.  So does most of the remaining rain, the exception being four
events with ponding above the thresshold for activating surface
biopores, as shown on the third graph.  The bottom five graph
correspond to the bottom five graphs of figure~\ref{fig:bromide-acc},
except the values are not accumulated.

\section{Drains}
\label{sec:drains}

The full drain seasons are depicted on
figure~\ref{fig:season9899},~\ref{fig:season9900},
and~\ref{fig:season0001}, while
figure~\ref{fig:season9899zoom},~\ref{fig:season9900zoom},
and~\ref{fig:season0001zoom} focus on a single event within each drain
season.

\chapter{Discussion}

\section{Soil and drain water}

The simulation match the TDR measurements very well, indicating that
our description of the upper boundary (precipitation) and top 60 cm of
soil is likely to be good.

Getting the model to simulate the measured responsiveness to events
has been a problem from the start, for example~\citet{nanna-agrovand}
dedicate a sections to the issue.  The situation as improved a lot
since then, but the figures showing drain flow for individual events
still show too litte response.  We had more luck with the PLAP sites
\citep{vap2d}.  A difference between the two is that we used measured
pressure 5 meter below surface there, where in the present simulation
we use a mostly static pressure table.  As the dynamics of the
individual events is likely very important for pestice leaching in
drain pipes, we should probably start using the piezometers for
estimating the lower boundary, if we are going to return to the
dataset.  Part of the problem is also the heterogenity of the lower
layers of soil, as indicated by the EM38 map.  In any case, the lower
boundary definitely need more attention.

The piezometer measurements for plot 4 are shown on
figure~\ref{fig:piezo-9899} and~\ref{fig:piezo-9900}.  They show large
variation within the field, but also a temporal variance much larger
than what we see on figure~\ref{fig:gwt}.

\begin{figure}[htbp]
  \begin{center}
    \figtop{piezo-70-9899}
    \figtop{piezo-400-9899}
    \fig{piezo-800-9899}
  \end{center}
  \caption{Pressure at 230 cm below surface, 70 cm (top), 4 m (middle)
    and 8 m (bottom) from drain.  First drain season.  The labels
    indicate distance from drain well (in meters) and whether the
    piezometer is located \textbf{N}orth or \textbf{S}outh of the
    drain pipe.}
  \label{fig:piezo-9899}
\end{figure}

\begin{figure}[htbp]
  \begin{center}
    \figtop{piezo-70-9900}
    \figtop{piezo-400-9900}
    \fig{piezo-800-9900}
  \end{center}
  \caption{Pressure at 230 cm below surface, 70 cm (top), 4 m
    (middle) and 8 m (bottom) from drain.  Second drain season.  The labels
    indicate distance from drain well (in meters) and whether the
    piezometer is located \textbf{N}orth or \textbf{S}outh of the
    drain pipe.}
  \label{fig:piezo-9900}
\end{figure}

\begin{figure}[htbp]
  \begin{center}
    \fig{piezo-drain-9899}
    \fig{piezo-drain-9900}
  \end{center}
  \caption{Median measured pressure level at 230 cm below surface for
    different distances from drain pipes.}
  \label{fig:piezo-drain}
\end{figure}

\begin{figure}[htbp]
  \begin{center}
    \fig{piezo-well-9899}
    \fig{piezo-well-9900}
  \end{center}
  \caption{Median measured pressure level at 230 cm below surface for
    different distances from drain wells.}
  \label{fig:piezo-well}
\end{figure}

\begin{figure}[htbp]
  \begin{center}
    \fig{piezo-plot-9899}
    \fig{piezo-plot-9900}
  \end{center}
  \caption{Median measured pressure level at 230 cm below surface for
    different plots.}
  \label{fig:piezo-plot}
\end{figure}

\section{Secondary domain}

We do not believe in the calibration of the secondary domain in
section~\ref{sec:soil-bromide}.  The bromide was applied when the soil
was frozen, and figure~\ref{fig:bromide} shows that the temperature
fluctuates around the freezing point up to the events where the
bromide is moved out of the plowing layer.  The present version of the
Daisy model does not handle ice, and we know that ice will force water
out of the small pores which will obviously have a large effect on the
primary/secondary domain dynamics.    

However, the work is far from wasted.  The simulation now reacts as
expected on the parameters, so it has been useful as a test of the
implementation of our model.

\section{Colloid generation}

The most significant part of the dataset is the colloid leaching for
different soil treatment regimes.  We have only looked at the
conventional tillage regime, and used that for calibrating the colloid
generation model described in \citet{macro-colloid}.  By using data
from the three other regimes, it should be possible to improve the
model to takes into account the timing of tillage operations, and help
predict the possible effect of low tillage regimes on pesticide
leaching.

\section{Pesticides}

Too much -> May indicate we overstimate fast pathways


\addcontentsline{toc}{chapter}{\numberline{}References}
\bibliography{../../txt/daisy}

\appendix{}

%% Drain figures

\newgeometry{left=1cm,top=1cm,right=1cm,bottom=1cm,nohead,nofoot}
\pagestyle{empty}
\begin{figure}[htbp]
  \begin{center}
    \figctop{weather} \\
    \figc{theta4cm} \\
    \figc{theta8cm} \\
    \figc{theta12cm} \\
    \figc{theta16cm} \\
    \figc{theta20cm} \\
    \figc{theta24cm} \\
    \figc{theta36cm} \\
    \figc{theta60cm}
  \end{center}
  \caption{\MyID{}TDR measurements.}
  \label{fig:tdr}
  \label{fig:first}
\end{figure}

\begin{figure}[htbp]
  \begin{center}
    \figctop{weather_short} \\
    \figc{theta_short4cm} \\
    \figc{theta_short8cm} \\
    \figc{theta_short12cm} \\
    \figc{theta_short16cm} \\
    \figc{theta_short20cm} \\
    \figc{theta_short24cm} \\
    \figc{theta_short36cm} \\
    \figc{theta_short60cm}
  \end{center}
  \caption{\MyID{}Early TDR measurements.}
  \label{fig:tdr-zoom}
\end{figure}

\begin{figure}[htbp]
  \begin{center}
    \figctop{weather_brominf} \\
    \figc{infiltration}\\
    \figc{pondingdepth}\\
    \figc{brom-input} \\
    \figc{brom-0-25-output} \\
    \figc{brom-25-50-output} \\
    \figc{brom-50-75-output} \\
    \figc{brom-75-100-output}
  \end{center}
  \caption{\MyID{}Bromide dynamics.}
  \label{fig:bromide}
\end{figure}

\begin{figure}[htbp]
  \begin{center}
    \figctop{brom-total} \\
    \figc{brom-primary} \\
    \figc{brom-secondary} \\
    \figc{brom-input-acc} \\
    \figc{brom-0-25-acc} \\
    \figc{brom-25-50-acc} \\
    \figc{brom-50-75-acc} \\
    \figc{brom-75-100-acc}
  \end{center}
  \caption{\MyID{}Accumulated bromide.}
  \label{fig:bromide-acc}
\end{figure}

\begin{figure}[htbp]
  \begin{center}
    \figctop{weather-98-99} \\
    \figc{drainflow-98-99} \\
    \figc{drainflowacc-98-99} \\
    \figc{particles-98-99} \\
    \figc{particlesacc-98-99} \\
    \figc{bromide-98-99} \\
    \figc{brommass-98-99}
  \end{center}
  \caption{\MyID{}Drain season 1998 -- 1999.}
  \label{fig:season9899}
\end{figure}

\begin{figure}[htbp]
  \begin{center}
    \figctop{weather-98-99-zoom} \\
    \figc{drainflow-98-99-zoom} \\
    \figc{drainflowacc-98-99-zoom} \\
    \figc{particles-98-99-zoom} \\
    \figc{particlesacc-98-99-zoom} \\
    \figc{bromide-98-99-zoom} \\
    \figc{brommass-98-99-zoom}
  \end{center}
  \caption{\MyID{}Drain season 1998 --- 1999, single event.}
  \label{fig:season9899zoom}
\end{figure}

\begin{figure}[htbp]
  \begin{center}
    \figctop{weather-99-00} \\
    \figc{drainflow-99-00} \\
    \figc{drainflowacc-99-00} \\
    \figc{particles-99-00} \\
    \figc{particlesacc-99-00} \\
    \figc{bromide-99-00} \\
    \figc{brommass-99-00} \\
    \figc{pendconc-99-00} \\
    \figc{pendmass-99-00}
  \end{center}
  \caption{\MyID{}Drain season 1999 --- 2000.}
  \label{fig:season9900}
\end{figure}

\begin{figure}[htbp]
  \begin{center}
    \figctop{weather-99-00-zoom} \\
    \figc{drainflow-99-00-zoom} \\
    \figc{drainflowacc-99-00-zoom} \\
    \figc{particles-99-00-zoom} \\
    \figc{particlesacc-99-00-zoom} \\
    \figc{bromide-99-00-zoom} \\
    \figc{brommass-99-00-zoom} \\
    \figc{pendconc-99-00-zoom} \\
    \figc{pendmass-99-00-zoom}
  \end{center}
  \caption{\MyID{}Drain season 1999 --- 2000, single event.}
  \label{fig:season9900zoom}
\end{figure}

\begin{figure}[htbp]
  \begin{center}
    \figctop{weather-00-01} \\
    \figc{drainflow-00-01} \\
    \figc{drainflowacc-00-01} \\
    \figc{particles-00-01} \\
    \figc{particlesacc-00-01} \\
    \figc{ioxconc-00-01} \\
    \figc{ioxmass-00-01} \\
    \figc{pendconc-00-01} \\
    \figc{pendmass-00-01}
  \end{center}
  \caption{\MyID{}Drain season 2000 --- 2001.}
  \label{fig:season0001}
\end{figure}

\begin{figure}[htbp]
  \begin{center}
    \figctop{weather-00-01-zoom} \\
    \figc{drainflow-00-01-zoom} \\
    \figc{drainflowacc-00-01-zoom} \\
    \figc{particles-00-01-zoom} \\
    \figc{particlesacc-00-01-zoom} \\
    \figc{ioxconc-00-01-zoom} \\
    \figc{ioxmass-00-01-zoom} \\
    \figc{pendconc-00-01-zoom} \\
    \figc{pendmass-00-01-zoom}
  \end{center}
  \caption{\MyID{}Drain season 2000 --- 2001, single event.}
  \label{fig:season0001zoom}
  \label{fig:last}
\end{figure}

%%% Local Variables: 
%%% mode: latex
%%% TeX-master: nil
%%% End: 


\end{document}

%%% Local Variables: 
%%% mode: latex
%%% TeX-master: t
%%% End: 
