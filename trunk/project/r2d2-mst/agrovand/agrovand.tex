\documentclass[a4paper]{report}

%%\usepackage[left=1cm,top=2cm,right=1cm]{geometry}
\usepackage[top=3cm,bottom=2cm]{geometry}
\usepackage[latin1]{inputenc}
\usepackage[T1]{fontenc}
\usepackage[danish,english]{babel}
\usepackage{natbib}
\bibliographystyle{apalike}
\usepackage{graphicx}
\usepackage{hyperref}
\usepackage{fancyhdr}
\usepackage{placeins}
\pagestyle{fancy}
\lhead{\today}
\usepackage{multirow}

\newcommand{\koc}{$\mbox{K}_{\mbox{\textsc{oc}}}$}
\newcommand{\kclay}{$\mbox{K}_{\mbox{clay}}$}
\newcommand{\kd}{$\mbox{K}_{\mbox{d}}$}
\newcommand{\rhob}{$\rho_{\mbox{b}}$}
\newcommand{\hlim}{\mbox{$h_{\mbox{lim}}$}}

\newcommand{\focus}{\textsc{focus}}
\newcommand{\hypres}{\textsc{hypres}}
\newcommand{\Hypres}{\textsc{Hypres}}
\newcommand{\macro}{\textsc{macro}}
\newcommand{\Macro}{\textsc{Macro}}

\newcommand{\figl}{\hspace*{-2cm}}
\newcommand{\figright}[1]{\includegraphics{fig/#1}}
\newcommand{\fig}[1]{\figl\figright{#1}}
\newcommand{\figtop}[1]{\figl\includegraphics[trim=0mm 5mm 0mm 0mm,clip]{fig/#1}}

\newcommand{\figctop}[1]{\hspace*{-1cm}\figright{#1}} 
\newcommand{\figc}[1]{\vspace*{-1.5cm}\figctop{#1}}

\newcommand{\MyID}{}

\begin{document}

\chapter*{Daisy 2D simulation of R�rrendeg�rd}

Part of project
\begin{otherlanguage}{danish}
  \begin{it}
    Flerdimensional modellering of vandstr�mning og stoftransport i de
    �verste 1-2 m af jorden i systemer med markdr�n
  \end{it}
\end{otherlanguage}
for the Danish Environmental Protection Agency.
\vspace{1cm}

\begin{bf}
  \begin{large}
    \noindent
    S�ren Hansen \texttt{$<$sha@life.ku.dk$>$}\\
    Carsten Petersen \texttt{$<$cpe@life.ku.dk$>$}\\
    Per Abrahamsen \texttt{$<$abraham@dina.kvl.dk$>$}\\
    Marie Habekost Nielsen \texttt{$<$maha@life.ku.dk$>$}\\
    Mikkel Mollerup \texttt{$<$mmo@geus.dk$>$}\\
    \\
    \today{}\\
  \end{large}
\end{bf}
\vfill\noindent
University of Copenhagen\\
Department of Basic Sciences and Environment\\
Environmental Chemistry and Physics\\
Thorvaldsensvej 40\\
DK-1871 Frederiksberg C\\
Tel: \texttt{$+$45 353 32300}\\
Fax: \texttt{$+$45 353 32398}

\tableofcontents

\chapter{Introduction}

The R�rrendeg�rd site is part of the Copenhagen University
experimental station at T�strup.  It was selected for for the present
project mainly because of the high resolution flow proportional drain
data collected as part of the Agrovand project in the four drain
seasons between between 1998 and 2002, which included soil particles,
a likely transport path for strongly sorbing pesticides.  The results
of the Agrovand project have been partly documented
in~\citet{petersen200181} (biopores), \citet{petersen2002movement}
(particles and pesticides), \citet{petersen2004movement} (particles
and bromide), and \citet{petersen2008spatio} (anisotropy).
Furthermore, investigations at the site on tillage effects on soil
structure stability and hydraulic properties of the surface layer was
reported in \citet{daraghmeh2008near,daraghmeh2009soil}.

The main focus of the Agrovand project was the influence of tillage on
the soil as a transport medium, so four plots with different tillage
strategies were followed.  In this project we have only studied the
data from plot 4, representing conventional tillage, and only the
first three seasons, where the best data is available.  Apart from
particles, the drain water has been analysed for bromide (one
application), pendimethalin (two applications) and ioxynil (one
application), which we have chosen to include in our simulations.
Soil water status has been followed with piezometers, tensiometers,
and TDR probes.  The most stable results are from the TDR probes, and
they are the only one we have used directly for our calibration.  The
piezometer measurements have been analyzed for use in calibration of
the lower boundary, see appendix~\ref{app:piezometers}.  Furthermore,
transport pathways have been explored using dye tracer, and biopores
have been counted both in the original project, and in more details
forming the basis for the new Daisy biopore model in the present
project~\citep{habekost2,habekost1,habekost3}.

The goal for the simulations presented in this paper is to test two
subcomponents of the newly developed 2D Daisy against real data:
The first is the particle generation and filtration modules; the
second is the slow/fast water movement.  For the later, we will use
the bromide drain data which are available at a high resolution, and
where we have reliable soil measurements to back them up with.  The
pesticide data is presented together with uncalibrated simulation
results as the PLAP sites have more detailed pesticide data
available~\citep{lindhardt2001,vap2009,vap2d}.  The Agrovand data is a
useful supplement though, as we don't have particle data for the PLAP
sites, and the PLAP bromide suffer from the fact that the application
was in spring, meaning an unknown amount have been uptaken by the
crop.  In Agrovand the bromide application was in the autumn,
minimizing the potential plant uptake.  Instead, the Agrovand bromide
may have been affected by the formation of ice in the soil, which is
not simulated.

\chapter{Setup}

The Agrovand data has been used from the beginning of the current
project for developing the new model, giving the final setup a rich
history.
\begin{enumerate}
\item An initial setup was developed for water and bromide using the
  original Daisy model by Tilde Hellsten, as part of her Master
  Thesis~\citep{tilde-agrovand}.
\item A setup for water using the new 2D model was developed by Nanna
  Gudmand-H�yer, as part of her Master Thesis~\citep{nanna-agrovand}.
\item This 2D setup was extended for bromide, particles, and
  pesticides by Mikkel Mollerup, and used as a basis for the PLAP site
  calibration~\citep{vap2d}.
\item Based on the model changes and experience gained made during the
  PLAP site calibration, the setup was recalibrated by Per
  Abrahamsen~\citet{mst-agrovand}.
\end{enumerate}

This history does mean that the setup likely contain parameter choices
that no longer are applicable due to changes in the model, and that a
new setup made from scratch could be simpler or give better results,
had time permitted.

\section{Weather}

All weather data with the exception of precipitation was collected at
a station located at H�jbakkeg�rd.  Three sources were considered for
precipitation.  Hourly measurements 1.2 meter above ground at the
field in the drain seasons, hourly measurements at H�jbakkeg�rd also
1.2 meter above ground, and daily measurements at ground level.

As a starting point, we used the hourly field measurements for the
drain season, supplemented with the hourly measurements from
H�jbakkeg�rd for the rest of the season.  These were compared with the
daily measurements.  Where the daily measurements showed precipitation
but the hourly measurements didn't we examined the TDR measurements
near the surface.  If they indicated precipitation, the daily were
used to supplement the hourly measurements.  Comparison of monthly
sums between the hourly and daily precipitation data indicated no
systematic bias, thus the hourly data were used without correction for
possible effect of wind and snowfall.  

Whether the precipitation falls as snow or rain will obviously affect
the drain flow, especially at short time scale.  Unfortunately, we did
not have direct measurements of the type of precipitation.  A build-in
model of Daisy will let an increasing amount of the precipitation fall
as snow when the air temperature drops below 2$\,^\circ$C.  This works
reasonable well for long time simulations, but not when we as here are
interested in the individual events.  For simplicity, we chose to
fully disable this snow model, so all precipitation in the simulation
will fall as rain.

The final weather data is shown on the top graphs of
figures~\ref{fig:first} to~\ref{fig:last}.

\section{Management}

All seasons had winter wheat with mineral fertilizer, with one plowing
operation between harvest and sowing.  For Daisy, the dates of the
plowing, sowing and harvest is used (table~\ref{tab:crop-man}).
Furthermore, Daisy uses information about the seed bed preparation.  As
we have not enabled nitrogen in the simulation, the fertilization
operations are irrelevant.  We use default parameters for the tillage
operations.  For the harvest, we specify 8 cm stub and that stems and
leaves are left on the field.  However, since we have not enabled a
model for above ground litter, and we are not interested in soil
organic matter, that information is not used in the simulation.

In the 2004 and 2005 seasons, the potential evapotranspiration for a
winter wheat on the experimental field was measured using an eddy
covariance system, and from this a dynamic crop factor was calculated
\citep{kjaersgaard2008crop}.  The default parametrization was
adjusted based on this, and furthermore as part of calibration of soil
water the max penetration depth was increased to 1.5 meter, and the
interception coefficient were lowered to 0.05 mm per LAI.
\begin{table}[htbp]
  \caption{Dates for crop management operations.  The initial crop was
    sowed 1997-9-23.}
  \label{tab:crop-man}
  \centering
  \begin{tabular}{l|lll}\hline
    Operation & 1998 & 1999 & 2000 \\
\hline
    Harvest & 8-20 & 8-20 & 8-20  \\
    Plowing & 9-15 & 9-15 & 9-15 \\
    Sow & 9-23 & 9-27 & 10-18 
  \end{tabular}
\end{table}

Date and amount are specified for pesticide and bromide applications.
The model setup described in \citet{vap2d} was duplicated here, with
field values for DT50 and \koc{} taken from \citet{ppdb20100517}.  No
calibration was done on the pesticides.  See table~\ref{tab:pest-man}.
\begin{table}[htbp]
  \caption{Pesticide and bromide application.}
  \label{tab:pest-man}
  \centering
  \begin{tabular}{l|l|r|r|r}\hline
    Date       & Name          & Amount [g/ha] & DT50 [d] & \koc{} [ml/g] \\
\hline
    1998-11-24 & Bromide       & 34000 & & \\
    1999-11-16 & Pendimethalin &  2000 & \multirow{2}{*}{90}
                                       & \multirow{2}{*}{15744} \\
    \multirow{2}{*}{2000-11-10} 
               & Pendimethalin &  2000 & & \\
               & Ioxynil       &   200 & 5 & 276
  \end{tabular}
\end{table}

All management operations are assumed to be performed at noon.

\section{Soil profile and biopores}
\label{sec:soil-profile}

The soil profile and the description of the drain ditch is based on
the work presented in \citet{habekost3}, where \textsc{isss4} texture
classification was used.  \citet{petersen200181} presents texture and
dry bulk density (\rhob{}) analyses for four depths, which have been
used as basis for the main horizons.  Unfortunately, no measurements
for the C horizon is presented, instead we use the measurement from
the bottom of the B horizon (85--90 cm).  The Ap measurements (10--15
cm) vary between treatments and between spring and autumn, we have
used the spring values for T4 (conventional tillage).  The soil humus
data are from plot A in \citet{petersen2002movement}.  The values used
are summarized in table~\ref{tab:texture}.

\begin{table}[htbp]
  \caption{Soil properties. Depth is specified in cm below soil surface, 
    and the dry bulk density (\rhob{}) specified in g/cm$^3$.
    Humus is given as a percentage of total weight.  For the drain ditch, 
    where the \textsc{isss4} texture classification system was used,
    the mineral soil particle distribution is also given as
    fraction of total weight.  For the other horizons the \textsc{usda3} 
    system was used, and the mineral soil particle distribution is given 
    as percentage of total mineral weight.}
  \label{tab:texture}
  \centering
  \begin{tabular}{rrrrrrrr}\hline
    Horizon & Depth & Clay & Silt & \multicolumn{2}{r}{Sand} & Humus
            & \rhob{} \\
    & & $< 2\;\mu$m & $< 50\;\mu$m & \multicolumn{2}{r}{$< 2\;$mm} & & \\\hline
    Ap & 0--25 & 10.7 & 22.2 & \multicolumn{2}{r}{67.1} & 3.0 & 1.49 \\
    Plow pan & 25--33 &  14.8 & 21.4 & \multicolumn{2}{r}{63.8} & 1.6 & 1.70 \\
    Bt & 33--120 & 22.2 & 19.5 & \multicolumn{2}{r}{58.3} & 1.6 & 1.65 \\
    C & 120--200 & 20.7 & 23.5 & \multicolumn{2}{r}{55.8} & 1.0 &  1.69 \\
    \\
    Area & Depth & Clay & Silt & Fine Sand & Coarse sand & Humus
            & \rhob{} \\
    &  & $< 2\;\mu$m & $< 20\;\mu$m & $< 200\;\mu$m & $< 2\;$mm & & \\\hline
    Drain ditch & 33--120 & 21.3 & 19.0 & 24.4 & 33.9 & 1.4 & 1.65 
  \end{tabular}
\end{table}

Initially, three classes of biopores were used in the simulation based
on \citet{habekost1}, where we focused on the biopores that
potentially had connection to the drain pipes.  We first assumed that
all the deep biopores (the two classes ending in 120 cm) in the drain
ditch would be directly connected to the drain pipes.  Based on
pesticide measurements in drains in the PLAP sites, we decided to
change this so only half the deep biopores in the drain ditch would be
directly connected to the drain pipes \citep{vap2d}.  Compared to the
PLAP simulations, we had additional soil bromide measurements
(section~\ref{sec:brom-cal}), so we decided to add an extra class
ending halfway down.  The measurements of \citet{petersen200181}
indicated a roughly linear decrease of biopore density with depth, so
we chose to use the same density as for the full length biopores.  The
classes are summarized in table~\ref{tab:biopores}.

\begin{table}[htbp]
  \caption{Biopore classes.}
  \label{tab:biopores}
  \centering
  \begin{tabular}{ll|rrrr}\hline
    Depth    & cm      & 0--25 & 0--120 & 30--120 & 0--60\\
    Diameter & mm      & 2    & 4       & 4       & 4 \\
    Density  & m$^{-2}$& 100  & 23      & 7       & 23
  \end{tabular}
\end{table}

The organic matter and nitrogen modules were disabled.

\section{TDR and hydraulic properties}

\Hypres{} was used initially to estimate hydraulic properties for all
horizons. The TDR measurements (see figure~\ref{fig:tdr}) have been
used for calibrating.  The only parameter that has been changed is
$K_{\mbox{sat}}$ (saturated conductivity).  For the surface layer (top
3 cm), this has been decreased to 10\% of the value suggested by
\hypres{}.  In the Bt horizon conductivity has been decreased to 50\%,
and in the C horizon it has been tripled.  The result is shown on
figure~\ref{fig:Rorrende-hor}.

\begin{figure}[htbp] 
  \fig{Rorrende-Ap-Theta}\figright{Rorrende-Ap-K}\\
  \fig{Rorrende-Bt-Theta}\figright{Rorrende-Bt-K}\\
  \fig{Rorrende-C-Theta}\figright{Rorrende-C-K}\\
  \fig{Rorrende-DC-Theta}\figright{Rorrende-DC-K}
  \caption{R{\o}rrende soil hydraulic properties.  \Hypres{} refers to
    parameters estimated according to \citet{hypres}, Daisy to the
    final parametrization (ignoring anisotropy and biopores), and
    surface and plow pan to the conditions at the top of the A and Bt
    horizons.}
  \label{fig:Rorrende-hor}
\end{figure}

Based on \citet{petersen2008spatio} we chose to add an
anisotropy of 12 (meaning horizontal flow is 12 times faster than
vertical) to the plow pan.

\section{Groundwater table and drain water}


An EM38 map of the field indicated that large areas had a sandy
underground \citep{nanna-agrovand}, and the piezometers showed that
these areas had a significantly lower groundwater level (see
appendix~\ref{app:piezometers}).  We estimated that roughly two thirds
of the field did not contribute to the drainage through the
groundwater level.  In Daisy we modelled this by dividing the field
into two columns.  The first column had a free drainage lower boundary
conditions, and represented twice the area of the other column, with
an aquitard bottom.  The aquitard layer was described with a size (2
meters), a conductivity (0.5 mm/h) and the pressure table of the
underlying aquifer.  The pressure table was based on a median
piezometer value (see appendix~\ref{app:piezometers}), and calibrated
to match drain flow (figure~\ref{fig:gwt}).  Note that
piezometer measurements represent pressure 2.3 m below surface, while
the aquifer represents pressure 4 m below surface.  The free drainage
column would still contribute to the drain water through directly
connected biopores.  The TDR measurements were performed in the part
of the field with clay underground, and the comparisons are therefore
done to the column with an aquitard.

\begin{figure}[htbp]
  \begin{center}
    \fig{gwt}
  \end{center}
  \caption{Median measured pressure level at 230 cm below surface
    together with calibrated aquifer pressure level}
  \label{fig:gwt}
\end{figure}

\section{Soil bromide and the secondary domain}
\label{sec:brom-cal}

We have not included cracks in the description of the conductivity
curve, but we still divide water into two domains for the sake of
solute transport.  This division was calibrated based on bromide soil
measurements shown on the top graph on figure~\ref{fig:bromide-acc}.
The simulated dynamics shown on figure~\ref{fig:bromide} were used as
a help.  The two figures are explained in
section~\ref{sec:soil-bromide}.

The division between water into two domains (the primary ``slow''
domain and the secondary ``fast'' domain) is controlled by single
horizon specific parameter, \hlim{}, a pressure head.  If the actual
pressure head ($h_a$) is below \hlim{}, all matrix water will be part
of the primary domain.  Otherwise, the water in the soil corresponding
to \hlim{} is considered part of the primary domain, and any
additional matrix water is considered part of the secondary
domain. The water flux calculated by Richard's equation ($q$) will be
divided so that the primary domain water flux ($q_1$) is
\begin{displaymath}
  q_1 = \frac{K (\hlim)}{K (h_a)}\; q
\end{displaymath}
where $K(h)$ is the hydraulic conductivity at pressure head $h$, and
the secondary domain water flux ($q_2$) is $q_2 = q - q_1$.
Solute transport in the primary domain is calculated with the
convection-dispersion equation, while solute movement in the secondary
domain is handled as pure convection.  A second parameter, $\alpha{}$
determines the speed of solute exchange between the two domains.

The bromide was measured in 25 cm intervals, starting from the soil
surface.  The measurements show the highest bromide concentrations
below 50 cm.  The results were based on 16 random samples of each
plot, and the pattern were similar in the three other
plots~\citep{petersen2004movement}.  Using a plain one domain
convection-dispersion equation, our simulations showed that most
bromide should still be in the top 50 cm.  In other words, this was a
classic case where the convection-dispersion equation, which assumes
full equilibrium between solute content in different pore classes, was
inadequate.  The idea was that by dividing the pore classes in two
domains, and calculating transport separately for each domain, the
bromide could stay in the secondary domain and move down faster.

As an initial guess, we used $\hlim{} = 2\; \mbox{pF}$ and $\alpha =
0.01\; \mbox{h}^{-1}$, the later taken from \citet{jaynes1995field}.
Using these values, our initial results were far worse than with the
pure convection-dispersion equation.  In these simulations, the
bromide would stay in the top 25 cm.  There were two problems: The
soil surface was so dry that much of the solute would enter the
primary domain, and stay relatively protected there.  Lowering \hlim{}
to 3 pF in the soil surface would ensure that all the water (and
solute) would enter the secondary domain.  The second problem was the
long period, over a month, before two large events caused significant
leakage out of the plow layer.  With an $\alpha$ of $0.01\;
\mbox{h}^{-1}$ a month was plenty of time to reach equilibrium, again
causing some of the bromide to be protected in the primary domain.  We
got the best results by lowering $\alpha$ to $0.00003\; \mbox{h}^{-1}$
in the top soil (to the bottom of the plow pan), decreasing it further
had little effect.

As the biopores were the main transport pathways through the plow pan,
we added a new biopore class that ended 60 cm below ground, in order
not to bypass the 50--100 cm area entirely, see
section~\ref{sec:soil-profile}.  This gave a problem for estimation of
$\alpha$ below 33 cm.  A too high value would cause some bromide to
stick just below the plow pan, where it would count as part of the
25-50 cm interval.  A too low value would cause the bromide that were
transported down to 60 cm through the biopores to move too fast below
100 cm.  We never found a good value.  The values used are listed in
table~\ref{tab:secondary}.

\begin{table}[htbp]
  \caption{Two domain solute transport parameters.}
  \label{tab:secondary}
  \centering
  \begin{tabular}{lll}\hline
    Depth [cm] & \hlim{} [pF] & $\alpha$ [h$^{-1}$] \\\hline
    0-33 & 3.0 & 0.00003 \\
    33-  & 2.0 & 0.0001
  \end{tabular}
\end{table}

\section{Particles}

Particles in Daisy are generated on the soil surface as a result of
rainfall, and then transported down through the soil matrix or
biopores.  We use the filter function from~\citet{macro-colloid} for
the matrix domain.  As the matrix domain in Daisy is divided into a
primary and secondary domain, we use different filter coefficients for
the two domains.  We choose values of 80 and 40 m$^{-1}$ for the primary
and secondary domain respectively, in order to stay near the 50 m$^{-1}$
used in~\citet{macro-colloid}. Daisy will (unlike \macro{}) not filter
particles in the biopores, only in the matrix.

For the particle generation we tried multiple models
\citep{Styczen88,EUROSEM,macro-colloid}, but
only~\citet{macro-colloid} gave anything near the desired dynamics.
\citet{macro-colloid} was also the only of the models designed to
match drain measurements, and the only model with a pool of readily
available particles.  We use the values from~\citet{macro-colloid} as
a starting point, except for the maximum particle storage
($M_{\mbox{max}}$) which is estimated based on the clay content as
described in~\citet{mmax}, method 1.  From calibration, we would
initially conclude that the detachment rate coefficient ($k_d$) should
be decreased to 7.5 g/J, the replenishment rate ($k_r$) to 0.1
g/m$^2$/h, and the depth of the soil affected by detachment and
dispersion ($z_i$) to 0.5 mm.  These values were used for PLAP
simulations.  Later we found that reverting to the values
from~\citet{macro-colloid} gave better results, and those values are
used for the present simulations.

The results are discussed in section~\ref{sec:drains}.

\chapter{Results}

The main simulation results are presented together with measured data
on figures~\ref{fig:first} to~\ref{fig:last}, found at the end of the
report (appendix~\ref{app:time-series}).  Additional 2D plots with
simulation results without matching measurements can be found in
appendix~\ref{app:plot-2d}.

\section{Soil water}
\label{sec:tdr}

Figure~\ref{fig:tdr} shows horizontal TDR measurements for different
depths.  The two autumn gaps are after plowing, when the TDR probes
are removed.  At the third season the TDR probes had drifted, and were
left out.  The TDR probes measurements do not include water in the
form of ice, which explain the apparent noise in the measurements
during periods with frost.  Enabling the experimental support for ice
in Daisy showed a good match between upper TDR probes and simulated
water during the two winter periods, supporting the idea that
difference is due to ice (see figure~\ref{fig:tdr-ice}).
Unfortunately, the ice support in Daisy is not yet complete, and
enabling it created too many other problems with the simulation, so it
was disabled for the final runs. The simulation overestimates the
water level near the soil surface, which could possibly be a problem
with the TDR measuring some air.  We may overestimate the dynamics
near the bottom of the plow layer.  The measurements for the bottom
TDR show fast variations during the winter which looks mostly like
noise, something not duplicated in the simulation, with or without
ice.

Figure~\ref{fig:tdr-zoom} shows the same data for the first summer
after installation.  The general water level seems to be slightly
overestimated at the end of the period, except in the 60 cm TDR where
it is underestimated.

\section{Soil bromide}
\label{sec:soil-bromide}

The measured and simulated bromide content in the top four 25 cm
intervals (0-100 cm) is shown on the second graph in
figure~\ref{fig:bromide-acc}.  The period is from right before
application, to right after the soil measurement.

As can be seen, the content of 00-50 cm is slightly overestimated in
the simulation, while the content of 50-100 cm is underestimated.  The
two next graphs below that divide the content in the same intervals
into the primary domain (small pores, slow water movement) and the
secondary domain (large pores, fast water movement).  The remaining
graphs shows bromide transport through the borders between the soil
intervals.  As can be seen, the bromide enter the soil through the
secondary domain, and some move further down through the secondary
domain at 25 cm, but most bromide are moved down through the tertiary
domain (the biopores), There is no significant transport in the
primary domain.  However, at the end of the period the primary domain
dominate storage.

In figure~\ref{fig:bromide-acc-extra} we examine four additional
intervals the same way, namely 25-33 cm (plow pan), 33-50 cm, 100-125
cm (end of long biopores), and 125-150 cm (below biopores).  As can be
seen, the first events bring down bromide with both the medium depth
biopores that end in the 50-75 cm interval, and the deep biopores that
end in the 100-125 cm interval.  But the later events apparently
mostly activate the deep biopores.

Figure~\ref{fig:bromide} shows the usual weather graph at the top.
Next is a graph showing how the water enter the system.  We see that
the first rain after application enter the soil through the secondary
domain.  So does most of the remaining rain, but some events result in
ponding above the threshold for activating surface biopores, as shown
on the third graph.  The bottom five graph correspond to the bottom
five graphs of figure~\ref{fig:bromide-acc}, except the values are not
accumulated. Figure~\ref{fig:bromide-extra} is similar, except that
the four bottom graphs represent the additional intervals from
figure~\ref{fig:bromide-acc-extra}.

\section{Drains}
\label{sec:drains}

The full drain seasons are depicted on
figure~\ref{fig:season9899},~\ref{fig:season9900},
and~\ref{fig:season0001}, while
figure~\ref{fig:season9899zoom},~\ref{fig:season9900zoom},
and~\ref{fig:season0001zoom} focus on a single event within each drain
season.  

\subsection{Water}
\label{sec:drain-water}

The top graph of all figures show precipitation and temperature for
the period.  The next graph shows simulated and observed drain flow,
as well as calibrated aquifer and measured median piezometer pressure.
Below that we get accumulated drain flow, simulated and observed.  By
calibrating the aquifer pressure (see figure~\ref{fig:gwt-sim}), we
were able to match total drain flow, however we consistently
underestimate the dynamics of each event.  Furthermore, in the
beginning of the first season we get too much water, despite using a
very low aquifer pressure compared to the piezometer data, and for the
second season we have the opposite problem.

\subsection{Particle leaching}
\label{sec:particles}

The next two graphs concern particle leaching.  In the first we see
flow proportional measurements of particle leaching, with simulated
values extracted the same time as measurements.  Each data point
represents the accumulated value since last measurement.  We also plot
the simulated reservoir of readily available particles from the
\citet{macro-colloid} model.  The next graph show accumulated values,
as well as simulated water flow directly from surface to drain.  In
general we see that the measurements tend to be taken when Daisy
predict the deep biopores to be active (that is, when there is heavy
rain).  The dynamic simulation rarely match measurements, but the
accumulated numbers show that seen over an event the simulation is
more often a good match.  Not always though, which means Daisy
overestimate total particle leaching the first year, and underestimate
it the last year.  In general, the variation in the measured numbers
is larger than the variation in the simulated numbers.

\subsection{Bromide and pesticides}
\label{sec:drain-solutes}

The next graphs vary by season. Simulated and observed concentration
of bromide and pesticides are shown for the seasons where they were
measured.  For bromide, we also show accumulated values.  For
pesticides, where there are far fewer measurements,
dynamic leaching is shown instead.

The bromide simulation has too high concentration at the beginning of
the season, especially during the first large event that activates the
biopores.  The simulated bromide concentration is too low at the end
of the season.  This trend, with too low concentrations, is continued
the second season.

For both pesticides, the general trend is that we simulate too high
concentrations in the drain water.  However, since we often
underestimate the water flow during the events, the total simulated
drain leaching is closer to what is measured.  We can also
see that the strongly sorbing pendimethalin are almost exclusively
leach together with particles, while ioxynil is found both particle
bound and dissolved.

\chapter{Discussion}

The simulation results can roughly be divided into three categories.
The first category is the results that depend mostly on the upper part
of the system.  These include TDR probes, drain particles, and
pesticide leaching through drains.  The second category is the results
that depend mostly on the lower part of the system.  These include
piezometers and drain water measurements.  The final category is the
bromide measurements in soil and drains.

\section{Upper part of the system}

The fine dynamic match between the TDR probes and simulation results
gives us faith in both our upper boundary, and in our description of
the part of the soil monitored by the probes.  The only caveat here is
the effect of ice and snow, which was not included in our final
simulation.

During calibration, we found that the amount of particles simulated in
drain pipes was robust with regard to changes to the lower boundary,
so we include those results with the upper part of the system.
Getting the right level of particles seen over three seasons using the
parameters from \citet{macro-colloid}, only adjusted for clay content
as specified in~\citet{mmax}, is encouraging.  We would have liked to
see the same variation between seasons as we measured though, and the
timing within events could be better.

The \emph{mass} of simulated pesticide leaching through the drain
pipes is also relatively robust with regard to changes in the lower
boundary as well.  The \emph{concentration} is not, though, as changes
in the lower boundary will greatly affect the amount of water in the
drain pipes.  The explanation is that the simulation has half
(ioxynil) or nearly all (pendimethalin) the amount leached being
particle bound.  This obviously makes the particle model crucial, and
also the \emph{soil enrichment factor} pesticide parameter, which
specifies how more likely the pesticide is to bind to a particle.
That particular parameter were given an initial value of 10000 in
order to see an effect, and has not been calibrated afterwards.  The
sorption kinetic is similarly not based on literature values, nor
calibrated.  A (de)sorption rate of 0.05 h$^{-1}$ was chosen too see
an effect given the Daisy timestep of hour.  With these caveats taken
into account, the results are encouraging.

\section{Lower part of the system}

There are several warning signs for the lower boundary of the systems.
First, less than 5\% of the yearly precipitation finds its way to the
drain pipes, meaning small variations in the total system can lead to
large variations in the drain pipes.  Related to this, the EM38 map
suggest that large parts of the field has a sandy underground, and are
unlikely to contribute to the drain flow.  Finally, the piezometers
show great spatial and temporal variation, and indicate that different
parts of the field may contribute to the drain flow at different
times.  As we sometimes have significant drain flow when the
piezometers show low pressure, this could indicate that local areas of
shallow groundwater may be at play.  The rightmost graph on the second
row on figure~\ref{fig:Rorrende-pF} could be an example of this.

Dividing the field into two parts, one with free drainage and one with
an aquitard bottom, is not enough to catch this spatial variation.
The main problem is our inability to catch the dynamic effect in the
drain that occurs a few hours into a large event.
Figure~\ref{fig:gwt-sim} shows the other side of this, our simulated
groundwater table is much more stable than the median piezometer
measurements, despite the later representing pressure 2.3 m below
ground level.

\section{Bromide and pathways}

The soil measurements show that the largest amounts of bromide should
be located between 50 and 100 cm below soil surface at the end of the
first drain season.  Of the 34 kg/ha applied, 15 were found at that
interval, 7 above, and the remaining 12 were lost.  The same general
pattern were found on the three other plots.  This fits well with the
drain measurements, that shows the largest leaching (with the highest
concentrations) occurring near the end of the drain season.

We were unable to duplicate this in the simulation.  Traditional
convection-dispersion would not move the bromide far enough down.
Distinguishing between transport with slow and fast water tended to
worsen the results, as the bromide stayed long enough in the top soil
after application to move into the primary domain (slow water), where
it would be protected.  An uncertainty of the system was when the
bromide would enter the soil, as the surface was frozen at the time of
application.  Delaying the entrance to the soil in the simulation to
right before the first large event would prevent the bromide from
entering the primary domain, but not leave enough time for it to move
below 50 cm. 

Adding an additional class of biopores that ended at 60 cm did help.
The effect can be seen on the top left graph of
figure~\ref{fig:Rorrende-M-Bromide}.  However, as the top right graph
shows, the main part of the bromide later move down to end of the long
biopores.  The same effect can be seen on figure~\ref{fig:bromide}
and~\ref{fig:bromide-acc}.  The first event activates both biopore
classes, the later events mainly the deep biopores.

In general, this indicates a problem with our model of the pathways in
the system, which will likely have some affect not only bromide but
also on particles and pesticides.

\section{Further work}

There are still more work to be done on calibration of the current
version of the model for the Agrovand dataset.  The problems with the
lower boundary is probably more than we can solve, but the bromide
pathways is a problem that should be solvable with the present model.

The Agrovand dataset can also provide basis for further model
development.  Ice obviously had an influence on the TDR measurements
(see figure~\ref{fig:tdr-ice}, especially the first season.  This
could be used for finishing the ice module of Daisy, and would
increase the trustworthiness of the simulation for that season,
especially when coupled with the drain measurements.  The effect of
frost on particle generation, as examined by
e.g.~\citet{kvarno2006influence} would be relevant.  And of course,
ice may also affect the water pathways, and possibly cast light on the
bromide results.

The largest potential in the dataset resides in the three other plots
with different tillage regimes.  This dataset could be used for
developing a model that included the effect of tillage on soil surface
properties and particle leaching, and consequently on leaching of
strongly sorbing pesticides.  A better model that would include
tillage more directly might help explain the difference we measured
between the three seasons.

Preliminary results from the project \emph{Unders�gelse af
  makroporekontinuitet ved markdr�n og effekter af direkte forbundne
  makroporer p� jords filterfunktion} indicate that the zone around
the drain pipes is wider that assumed in this project, which would
help explain some of the observed drain water dynamics.  Thus,
adjusting the setup to take these results into account would be an
interesting avenue of investigation.

Finally, we need better estimates for (de)sorption rates and the soil
enrichment factor, either from literature or from focused experiment.

\addcontentsline{toc}{chapter}{\numberline{}References}
\bibliography{../../txt/daisy}

\appendix{}

\chapter{Piezometers}
\label{app:piezometers}

A total of 63 piezometers were installed 2.3 m below soil surface for
all four plots, 70 cm, 4 m and 8 m from each side of the drain pipes,
and 10, 40, and 70 m from the sampling wells.

Individual measurements from plot 4 for the two seasons are shown on
figure~\ref{fig:piezo-9899} and~\ref{fig:piezo-9900}.  To impose some
order, we chose to look at the median values, as shown on
figure~\ref{fig:piezo-median-9899} and~\ref{fig:piezo-median-9900}.
The top graphs shows that there generally is higher pressure further
away from the drain pipes, at least in the periods where the pressure
is high and the drains are likely to be active.  When the drains are
inactive, there is no clear trend.  The middle graphs show highest
pressure closest to the well, and that at a distance of 70 m, the
pressure is rarely high enough to indicate drain activity.  The only
clear trend shown in the bottom graph is that the pressure at plot 1
is lower than the pressure at the remaining three plots.

It order to have a single piezometer value for use in calibration of
the lower boundary of the system, we want as many piezometer
measurements as possible, to weed out local variations.  But we only
want piezometer measurements from those part of the field that
contribute to the drain flow.  An EM38 map of the field indicate that
the underground is more dominated by sand around 50 m from the wells,
and that plot 1 likely contain more sand than the other three
\citep{nanna-agrovand}.  As this matches well with our analysis of the
piezometer measurements, we choose to include plot 2, 3 and 4 at 10 m
and 40 m distance from the wells, in our final median piezometer
value, shown on figure~\ref{fig:gwt-sim} together with out calibrated
aquifer pressure, and simulated groundwater table.

\begin{figure}[htbp]
  \begin{center}
    \figtop{piezo-70-9899}
    \figtop{piezo-400-9899}
    \fig{piezo-800-9899}
  \end{center}
  \caption{Pressure at 230 cm below surface, 70 cm (top), 4 m (middle)
    and 8 m (bottom) from drain.  First drain season, plot 4.  The
    labels indicate distance from drain well (in meters) and whether
    the piezometer is located \textbf{N}orth or \textbf{S}outh of the
    drain pipe.}
  \label{fig:piezo-9899}
\end{figure}

\begin{figure}[htbp]
  \begin{center}
    \figtop{piezo-70-9900}
    \figtop{piezo-400-9900}
    \fig{piezo-800-9900}
  \end{center}
  \caption{Pressure at 230 cm below surface, 70 cm (top), 4 m (middle)
    and 8 m (bottom) from drain.  Second drain season, plot 4.  The
    labels indicate distance from drain well (in meters) and whether
    the piezometer is located \textbf{N}orth or \textbf{S}outh of the
    drain pipe.}
  \label{fig:piezo-9900}
\end{figure}

\begin{figure}[htbp]
  \begin{center}
    \figtop{piezo-drain-9899}
    \figtop{piezo-well-9899}
    \fig{piezo-plot-9899}
  \end{center}
  \caption{Median measured pressure level at 230 cm below surface for
    first drain season.  Top graph show distance from drain, middle
    graph distance from well, and bottom graph plot number.}
  \label{fig:piezo-median-9899}
\end{figure}

\begin{figure}[htbp]
  \begin{center}
    \figtop{piezo-drain-9900}
    \figtop{piezo-well-9900}
    \fig{piezo-plot-9900}
  \end{center}
  \caption{Median measured pressure level at 230 cm below surface for
    second drain season.  Top graph show distance from drain, middle
    graph distance from well, and bottom graph plot number.}
  \label{fig:piezo-median-9900}
\end{figure}

\begin{figure}[htbp]
  \begin{center}
    \vspace*{-0.8cm}\fig{gwt-9899}
    \fig{gwt-9900}
    \fig{gwt-0001}
  \end{center}
  \vspace*{-0.8cm}
  \caption{Median measured pressure level at 230 cm below surface
      together with calibrated aquifer pressure level and simulated
      groundwater table. Simulated low value is calculated from
      pressure in lowest unsaturated numeric cell, typically located
      near drain.  Simulated high value is calculated from pressure in
      highest saturated cell, typically farthest away from the drain.
      The sudden jumps of the high value represents situations with
      surface ponding, where the top numeric cell becomes saturated.}
  \label{fig:gwt-sim}
\end{figure}

\newcommand{\figrorrende}[1]{\includegraphics[trim=9mm 0mm 14mm 12mm,clip]{fig/#1}}
\newcommand{\figrorrendel}[1]{\hspace*{-2cm}\figrorrende{#1}}

\chapter{2D plots}
\label{app:plot-2d}

In this appendix we present simulated 2D plots for water, bromide,
pendimethalin, and ioxynil.  There are no measurements to compare
with, a major caveat for both the results and discussion.  We use two
kinds of graphs to capture the 2D structure.

The first kind depict static distribution in the soil.  Each graph has
horizontal distance from drain on the x-axis and height above surface
on the y-axis, using the same scale for both axes.  The graph
represents the the computational soil area used in the simulation.
The right side is the center between two drains (9 meter for
R{\o}rrende), and the bottom is 2 meter, where we use an aquitard
lower boundary with a calibrated aquifer.  The graphs are color coded,
where specific colors represent specific values for the soil at the
end of the month indicated by the graph title.  Each numeric cell in
the computation has a color representing the value within that cell.
Since cells are rectangular, the graphs appear blocky.

The second kind of graph depicts horizontal or vertical movement.  For
the graphs depicting horizontal movement, the y-axis specifies height
above surface (negative number) and the x-axis movement away from
drain (usually also negative).  The horizontal movement at different
distances from the drain pipes are shown as separate plots on each
graph.  For the graphs depicting vertical movement, the axes are
swapped.  The individual plots represent different depths.  We use the
same flow units as we used for the original input, so e.g. pesticide
transport is given in g/ha.

\FloatBarrier
\section{Water}

\subsection{Distribution}

\begin{figure}[htbp]\centering
  \begin{tabular}{ccc}
    \figrorrendel{Rorrende-pF-1998-5} & 
    \figrorrende{Rorrende-pF-1998-6} & 
    \figrorrende{Rorrende-pF-1998-7} \\
    \figrorrendel{Rorrende-pF-1998-8} & 
    \figrorrende{Rorrende-pF-1998-9} & 
    \figrorrende{Rorrende-pF-1998-10} \\
    \figrorrendel{Rorrende-pF-1998-11} & 
    \figrorrende{Rorrende-pF-1998-12} & 
    \figrorrende{Rorrende-pF-1999-1} \\
    \figrorrendel{Rorrende-pF-1999-2} & 
    \figrorrende{Rorrende-pF-1999-3} & 
    \figrorrende{Rorrende-pF-1999-4}\\
    \figrorrendel{Rorrende-pF-1999-5} & 
    \figrorrende{Rorrende-pF-1999-6} & 
    \figrorrende{Rorrende-pF-1999-7} \\
    \figrorrendel{Rorrende-pF-1999-8} & 
    \figrorrende{Rorrende-pF-1999-9} & 
    \figrorrende{Rorrende-pF-1999-10} \\
    \figrorrendel{Rorrende-pF-1999-11} & 
    \figrorrende{Rorrende-pF-1999-12} & 
    \figrorrende{Rorrende-pF-2000-1} \\
    \figrorrendel{Rorrende-pF-2000-2} & 
    \figrorrende{Rorrende-pF-2000-3} & 
    \figrorrende{Rorrende-pF-2000-4}\\
    \figrorrendel{Rorrende-pF-2000-5} & 
    \figrorrende{Rorrende-pF-2000-6} & 
    \figrorrende{Rorrende-pF-2000-7} \\
    \figrorrendel{Rorrende-pF-2000-8} & 
    \figrorrende{Rorrende-pF-2000-9} & 
    \figrorrende{Rorrende-pF-2000-10} \\
    \figrorrendel{Rorrende-pF-2000-11} & 
    \figrorrende{Rorrende-pF-2000-12} & 
    \figrorrende{Rorrende-pF-2001-1} 
  \end{tabular}
  
  \caption{Soil water pressure potential at the end of each month from
    May 1998 (top left) to January 2001 (bottom right).  The y-axis
    denotes depth, the x-axis distance from drain.  There are tick
    marks for every meter.  Blue denotes pF<0, white pF=1, yellow
    pF=2, orange pF=3, red pF=4, and black pF>5.}
\label{fig:Rorrende-pF}
\end{figure}

\subsection{Flow}

\begin{figure}[htbp]
  \centering
  \figtop{Rorrende-water-horizontal-1998}
  \figtop{Rorrende-water-horizontal-1999}
  \fig{Rorrende-water-horizontal-2000}
  
  \caption{Horizontal water flux between 1998-5-1 and 1999-5-1 (top),
    between 1999-5-1 and 2000-5-1 (center), and between 2000-5-1 and
    2001-2-1 (bottom).  The flux is shown on the x-axis (positive away
    from drain) as a function of depth shown on the y-axis.  The graph
    labels are the distance from drain in centimeters.}
  \label{fig:Rorrende-water-horizontal}
\end{figure}

\begin{figure}[htbp]
  \centering
  \figtop{Rorrende-water-1998}
  \fig{Rorrende-water-1999}
  \fig{Rorrende-water-2000}
  
  \caption{Total vertical water flux between 1998-5-1 and
    1999-5-1 (top), between 1999-5-1 and 2000-5-1 (center), and
    between 2000-5-1 and 2001-2-1 (bottom).  The flux is shown on the
    y-axis (positive up) as a function of distance from drain shown on
    the x-axis.  The graph labels are depths in centimeters above
    surface.}
  \label{fig:Rorrende-water-vertical}
\end{figure}

\begin{figure}[htbp]
  \centering
  \figtop{Rorrende-water-biopore-1998}
  \fig{Rorrende-water-biopore-1999}
  \fig{Rorrende-water-biopore-2000}
  
  \caption{Biopore water flux between 1998-5-1 and 1999-5-1 (top),
    between 1999-5-1 and 2000-5-1 (center), and between 2000-5-1 and
    2001-2-1 (bottom).  The flux is shown on the y-axis (positive up)
    as a function of distance from drain shown on the x-axis.  The
    graph labels are depths in centimeters above surface.}
  \label{fig:Rorrende-water-biopore}
\end{figure}

\FloatBarrier
\section{Bromide}

\subsection{Distribution}

\subsection{Transport}

\begin{figure}[htbp]\centering
  \begin{tabular}{ccc}
    \figrorrendel{Rorrende-M-Bromide-1998-11} & 
    \figrorrende{Rorrende-M-Bromide-1998-12} & 
    \figrorrende{Rorrende-M-Bromide-1999-1} \\
    \figrorrendel{Rorrende-M-Bromide-1999-2} & 
    \figrorrende{Rorrende-M-Bromide-1999-3} & 
    \figrorrende{Rorrende-M-Bromide-1999-4} \\
    \figrorrendel{Rorrende-M-Bromide-1999-5} & 
    \figrorrende{Rorrende-M-Bromide-1999-6} & 
    \figrorrende{Rorrende-M-Bromide-1999-7} \\
    \figrorrendel{Rorrende-M-Bromide-1999-8} & 
    \figrorrende{Rorrende-M-Bromide-1999-9} & 
    \figrorrende{Rorrende-M-Bromide-1999-10} \\
    \figrorrendel{Rorrende-M-Bromide-1999-11} & 
    \figrorrende{Rorrende-M-Bromide-1999-12} & 
    \figrorrende{Rorrende-M-Bromide-2000-1} \\
    \figrorrendel{Rorrende-M-Bromide-2000-2} & 
    \figrorrende{Rorrende-M-Bromide-2000-3} & 
    \figrorrende{Rorrende-M-Bromide-2000-4}
  \end{tabular}
  
  \caption{Bromide soil content at the end of each month since first
    application in November 1998 (top left graph) until April 2000
    (bottom right graph).  The y-axis denotes depth, the x-axis
    distance from drain.  There are tick marks for every meter. The
    color scale is white<1 $\mu$g/l, yellow=10 $\mu$g/l, orange=100 $\mu$g/l,
    red=1 mg/l, and black>10 mg/l}
\label{fig:Rorrende-M-Bromide}
\end{figure}

\begin{figure}[htbp]
  \centering
  \fig{Rorrende-Bromide-horizontal-1998}
  \figtop{Rorrende-Bromide-1998}
  \fig{Rorrende-Bromide-biopore-1998}

  
  \caption{Bromide transport between 1998-5-1 and 1999-5-1.  The top
    graph show horizontal transport (top), the center graph show total
    vertical transport, and the bottom graph show biopore transport
    only.  The transport in the top graph is shown on the x-axis
    (positive away from drain) as a function of depth shown on the
    y-axis, with graph labels indicating the distance from drain in
    centimeters.  The transport on the two lowert graphs are shown on
    the y-axis (positive up) as a function of distance from drain
    shown on the x-axis. The graph labels are depths in centimeters above
    surface.}
  \label{fig:Rorrende-Bromide-1998}
\end{figure}

\begin{figure}[htbp]
  \centering
  \fig{Rorrende-Bromide-horizontal-1999}
  \figtop{Rorrende-Bromide-1999}
  \fig{Rorrende-Bromide-biopore-1999}

  
  \caption{Bromide transport between 1999-5-1 and 2000-5-1.  The top
    graph show horizontal transport (top), the center graph show total
    vertical transport, and the bottom graph show biopore transport
    only.  The transport in the top graph is shown on the x-axis
    (positive away from drain) as a function of depth shown on the
    y-axis, with graph labels indicating the distance from drain in
    centimeters.  The transport on the two lowert graphs are shown on
    the y-axis (positive up) as a function of distance from drain
    shown on the x-axis. The graph labels are depths in centimeters above
    surface.}
  \label{fig:Rorrende-Bromide-1999}
\end{figure}

\FloatBarrier
\section{Pendimethalin}

\subsection{Distribution}

\subsection{Transport}

\begin{figure}[htbp]\centering
  \begin{tabular}{ccc}
    \figrorrendel{Rorrende-M-Pendimethalin-1999-11} & 
    \figrorrende{Rorrende-M-Pendimethalin-1999-12} & 
    \figrorrende{Rorrende-M-Pendimethalin-2000-1} \\
    \figrorrendel{Rorrende-M-Pendimethalin-2000-2} & 
    \figrorrende{Rorrende-M-Pendimethalin-2000-3} & 
    \figrorrende{Rorrende-M-Pendimethalin-2000-4} \\
    \figrorrendel{Rorrende-M-Pendimethalin-2000-5} & 
    \figrorrende{Rorrende-M-Pendimethalin-2000-6} & 
    \figrorrende{Rorrende-M-Pendimethalin-2000-7} \\
    \figrorrendel{Rorrende-M-Pendimethalin-2000-8} & 
    \figrorrende{Rorrende-M-Pendimethalin-2000-9} & 
    \figrorrende{Rorrende-M-Pendimethalin-2000-10} \\
    \figrorrendel{Rorrende-M-Pendimethalin-2000-11} & 
    \figrorrende{Rorrende-M-Pendimethalin-2000-12} & 
    \figrorrende{Rorrende-M-Pendimethalin-2001-1}
  \end{tabular}
  
  \caption{Pendimethalin soil content at the end of each month since
    first application in November 1999 (top left graph) until January
    2001 (bottom right graph) .  The y-axis denotes depth, the x-axis
    distance from drain.  There are tick marks for every meter. The
    color scale is white<10 pg/l, yellow=1 ng/l, orange=0.1 $\mu$g/l,
    red=10 $\mu$g/l, and black>1 mg/l}
\label{fig:Rorrende-M-Pendimethalin}
\end{figure}

\begin{figure}[htbp]\centering
  \begin{tabular}{ccc}
    \figrorrendel{Rorrende-C-Pendimethalin-1999-11} & 
    \figrorrende{Rorrende-C-Pendimethalin-1999-12} & 
    \figrorrende{Rorrende-C-Pendimethalin-2000-1} \\
    \figrorrendel{Rorrende-C-Pendimethalin-2000-2} & 
    \figrorrende{Rorrende-C-Pendimethalin-2000-3} & 
    \figrorrende{Rorrende-C-Pendimethalin-2000-4} \\
    \figrorrendel{Rorrende-C-Pendimethalin-2000-5} & 
    \figrorrende{Rorrende-C-Pendimethalin-2000-6} & 
    \figrorrende{Rorrende-C-Pendimethalin-2000-7} \\
    \figrorrendel{Rorrende-C-Pendimethalin-2000-8} & 
    \figrorrende{Rorrende-C-Pendimethalin-2000-9} & 
    \figrorrende{Rorrende-C-Pendimethalin-2000-10} \\
    \figrorrendel{Rorrende-C-Pendimethalin-2000-11} & 
    \figrorrende{Rorrende-C-Pendimethalin-2000-12} & 
    \figrorrende{Rorrende-C-Pendimethalin-2001-1}
  \end{tabular}
  
  \caption{Pendimethalin soil water content at the end of each month
    since first application in November 1999 (top left graph) until
    January 2001 (bottom right graph) .  The y-axis denotes depth, the
    x-axis distance from drain.  There are tick marks for every
    meter. The color scale is white<10 pg/l, yellow=1 ng/l, orange=0.1
    $\mu$g/l, red=10 $\mu$g/l, and black>1 mg/l}
\label{fig:Rorrende-C-Pendimethalin}
\end{figure}

\begin{figure}[htbp]
  \centering
  \fig{Rorrende-Pendimethalin-horizontal-1999}
  \figtop{Rorrende-Pendimethalin-1999}
  \fig{Rorrende-Pendimethalin-biopore-1999}
  
  \caption{Pendimethalin transport between 1999-5-1 and 2000-5-1.  The top
    graph show horizontal transport (top), the center graph show total
    vertical transport, and the bottom graph show biopore transport
    only.  The transport in the top graph is shown on the x-axis
    (positive away from drain) as a function of depth shown on the
    y-axis, with graph labels indicating the distance from drain in
    centimeters.  The transport on the two lowert graphs are shown on
    the y-axis (positive up) as a function of distance from drain
    shown on the x-axis. The graph labels are depths in centimeters above
    surface.}
  \label{fig:Rorrende-Pendimethalin-1999}
\end{figure}

\begin{figure}[htbp]
  \centering
  \fig{Rorrende-Pendimethalin-horizontal-2000}
  \figtop{Rorrende-Pendimethalin-2000}
  \fig{Rorrende-Pendimethalin-biopore-2000}

  \caption{Pendimethalin transport between 2000-5-1 and 2001-2-1.  The top
    graph show horizontal transport (top), the center graph show total
    vertical transport, and the bottom graph show biopore transport
    only.  The transport in the top graph is shown on the x-axis
    (positive away from drain) as a function of depth shown on the
    y-axis, with graph labels indicating the distance from drain in
    centimeters.  The transport on the two lowert graphs are shown on
    the y-axis (positive up) as a function of distance from drain
    shown on the x-axis. The graph labels are depths in centimeters above
    surface.}
  \label{fig:Rorrende-Pendimethalin-2000}
\end{figure}

\FloatBarrier
\section{Ioxynil}

\subsection{Distribution}

\subsection{Transport}

\begin{figure}[htbp]\centering
  \begin{tabular}{ccc}
    \figrorrendel{Rorrende-M-Ioxynil-2000-11} & 
    \figrorrende{Rorrende-M-Ioxynil-2000-12} & 
    \figrorrende{Rorrende-M-Ioxynil-2001-1}
  \end{tabular}
  
  \caption{Ioxynil soil content at the end of each month since first
    application in November 2000 (top left graph) until January 2001
    (bottom right graph) .  The y-axis denotes depth, the x-axis
    distance from drain.  There are tick marks for every meter. The
    color scale is white<10 pg/l, yellow=1 ng/l, orange=0.1 $\mu$g/l,
    red=10 $\mu$g/l, and black>1 mg/l}
\label{fig:Rorrende-M-Ioxynil}
\end{figure}

\begin{figure}[htbp]\centering
  \begin{tabular}{ccc}
    \figrorrendel{Rorrende-C-Ioxynil-2000-11} & 
    \figrorrende{Rorrende-C-Ioxynil-2000-12} & 
    \figrorrende{Rorrende-C-Ioxynil-2001-1}
  \end{tabular}
  
  \caption{Ioxynil soil water content at the end of each month since
    first application in November 2000 (left graph) until January 2001
    (right graph) .  The y-axis denotes depth, the x-axis distance
    from drain.  There are tick marks for every meter. The color scale
    is white<10 pg/l, yellow=1 ng/l, orange=0.1 $\mu$g/l, red=10
    $\mu$g/l, and black>1 mg/l}
\label{fig:Rorrende-C-Ioxynil}
\end{figure}

\begin{figure}[htbp]
  \centering
  \fig{Rorrende-Ioxynil-horizontal-2000}
  \figtop{Rorrende-Ioxynil-2000}
  \fig{Rorrende-Ioxynil-biopore-2000}

  \caption{Ioxynil transport between 2000-11-1 and 2001-2-1.  The top
    graph show horizontal transport (top), the center graph show total
    vertical transport, and the bottom graph show biopore transport
    only.  The transport in the top graph is shown on the x-axis
    (positive away from drain) as a function of depth shown on the
    y-axis, with graph labels indicating the distance from drain in
    centimeters.  The transport on the two lowert graphs are shown on
    the y-axis (positive up) as a function of distance from drain
    shown on the x-axis. The graph labels are depths in centimeters above
    surface.}
  \label{fig:Rorrende-Ioxynil-2000}
\end{figure}

%%% Local Variables: 
%%% TeX-master: "agrovand"
%%% End: 


\chapter{Time series}
\label{app:time-series}

The comparison between measured and simulated numbers are presented in
this appendix.  The figures have a high information density, and have
therefore been allowed to fill most of the page.  Each figure contains
multiple graphs, all of which share the same x-axis.  This structure
is intended to facilitate comparison.  The same figures were used for
calibration.  All figures show precipitation (left axis) and air
temperature (right axis) for the period in the top graph.

All drain measurements were done with flow proportional sampling, with
variable timestep.  For comparison with the hourly simulated results,
samples representing less than one hour of flow very combined.

\section{Soil water content}
\label{sec:soil-water}

The graphs on figure~\ref{fig:tdr} and~\ref{fig:tdr-zoom} shows
simulated and measured volumetric water content at different depths,
and are discussed in section~\ref{sec:tdr}.

\section{Soil bromide content}
\label{sec:soil-bromide-content}

Figures~\ref{fig:bromide},~\ref{fig:bromide-acc},~\ref{fig:bromide-extra},
and~\ref{fig:bromide-acc-extra} concern soil bromide content, and are
discussed in section~\ref{sec:soil-bromide}.

\section{Drain content}
\label{sec:drain-content}

The full drain seasons are depicted on
figure~\ref{fig:season9899},~\ref{fig:season9900},
and~\ref{fig:season0001}, while
figure~\ref{fig:season9899zoom},~\ref{fig:season9900zoom},
and~\ref{fig:season0001zoom} focus on a single event within each drain
season.  See section~\ref{sec:drains}.

Dynamic and accumulated water flow is shown on the two first graphs
under the top graph.  The median piezometer pressure table, and the
calibrated aquifer pressure is shown on the right axis on the dynamic
water flow graph.  See section~\ref{sec:drain-water}.  The next two
graphs show particle leaching.  The dynamic particle leaching
represent particles collected since last measurement.  The particles
may have leached before the time of measurements.  The simulated
reservoir of ready available particles is shown on the same graph.
See section~\ref{sec:particles}

The next graphs depend on season.  For season 1998--1999 and season
1999--2000 (figures~\ref{fig:season9899}~--~\ref{fig:season9900zoom})
the next two graphs depicts bromide concentration, and accumulated
bromide leaching.  For season 2000-2001 they instead depict ioxynil
concentration and mass in drain water.  For the 1999--2000 and
2000--2001 season, we finish off with pendimethalin concentration and
mass.  For the pesticides, we both colloid bound and total amounts are
shown.  See section~\ref{sec:drain-solutes}.

\newgeometry{left=1cm,top=1cm,right=1cm,bottom=1cm,nohead,nofoot}
\pagestyle{empty}
\begin{figure}[htbp]
  \begin{center}
    \figctop{weather} \\
    \figc{theta4cm} \\
    \figc{theta8cm} \\
    \figc{theta12cm} \\
    \figc{theta16cm} \\
    \figc{theta20cm} \\
    \figc{theta24cm} \\
    \figc{theta36cm} \\
    \figc{theta60cm}
  \end{center}
  \caption{\MyID{}TDR measurements.}
  \label{fig:tdr}
  \label{fig:first}
\end{figure}

\begin{figure}[htbp]
  \begin{center}
    \figctop{weather_short} \\
    \figc{theta_short4cm} \\
    \figc{theta_short8cm} \\
    \figc{theta_short12cm} \\
    \figc{theta_short16cm} \\
    \figc{theta_short20cm} \\
    \figc{theta_short24cm} \\
    \figc{theta_short36cm} \\
    \figc{theta_short60cm}
  \end{center}
  \caption{\MyID{}Early TDR measurements.}
  \label{fig:tdr-zoom}
\end{figure}

\begin{figure}[htbp]
  \begin{center}
    \figctop{weather_brominf} \\
    \figc{infiltration}\\
    \figc{pondingdepth}\\
    \figc{brom-input} \\
    \figc{brom-0-25-output} \\
    \figc{brom-25-50-output} \\
    \figc{brom-50-75-output} \\
    \figc{brom-75-100-output}
  \end{center}
  \caption{\MyID{}Bromide dynamics.}
  \label{fig:bromide}
\end{figure}

\begin{figure}[htbp]
  \begin{center}
    \figctop{brom-total} \\
    \figc{brom-primary} \\
    \figc{brom-secondary} \\
    \figc{brom-input-acc} \\
    \figc{brom-0-25-acc} \\
    \figc{brom-25-50-acc} \\
    \figc{brom-50-75-acc} \\
    \figc{brom-75-100-acc}
  \end{center}
  \caption{\MyID{}Accumulated bromide.}
  \label{fig:bromide-acc}
\end{figure}

\begin{figure}[htbp]
  \begin{center}
    \figctop{weather-98-99} \\
    \figc{drainflow-98-99} \\
    \figc{drainflowacc-98-99} \\
    \figc{particles-98-99} \\
    \figc{particlesacc-98-99} \\
    \figc{bromide-98-99} \\
    \figc{brommass-98-99}
  \end{center}
  \caption{\MyID{}Drain season 1998 -- 1999.}
  \label{fig:season9899}
\end{figure}

\begin{figure}[htbp]
  \begin{center}
    \figctop{weather-98-99-zoom} \\
    \figc{drainflow-98-99-zoom} \\
    \figc{drainflowacc-98-99-zoom} \\
    \figc{particles-98-99-zoom} \\
    \figc{particlesacc-98-99-zoom} \\
    \figc{bromide-98-99-zoom} \\
    \figc{brommass-98-99-zoom}
  \end{center}
  \caption{\MyID{}Drain season 1998 --- 1999, single event.}
  \label{fig:season9899zoom}
\end{figure}

\begin{figure}[htbp]
  \begin{center}
    \figctop{weather-99-00} \\
    \figc{drainflow-99-00} \\
    \figc{drainflowacc-99-00} \\
    \figc{particles-99-00} \\
    \figc{particlesacc-99-00} \\
    \figc{bromide-99-00} \\
    \figc{brommass-99-00} \\
    \figc{pendconc-99-00} \\
    \figc{pendmass-99-00}
  \end{center}
  \caption{\MyID{}Drain season 1999 --- 2000.}
  \label{fig:season9900}
\end{figure}

\begin{figure}[htbp]
  \begin{center}
    \figctop{weather-99-00-zoom} \\
    \figc{drainflow-99-00-zoom} \\
    \figc{drainflowacc-99-00-zoom} \\
    \figc{particles-99-00-zoom} \\
    \figc{particlesacc-99-00-zoom} \\
    \figc{bromide-99-00-zoom} \\
    \figc{brommass-99-00-zoom} \\
    \figc{pendconc-99-00-zoom} \\
    \figc{pendmass-99-00-zoom}
  \end{center}
  \caption{\MyID{}Drain season 1999 --- 2000, single event.}
  \label{fig:season9900zoom}
\end{figure}

\begin{figure}[htbp]
  \begin{center}
    \figctop{weather-00-01} \\
    \figc{drainflow-00-01} \\
    \figc{drainflowacc-00-01} \\
    \figc{particles-00-01} \\
    \figc{particlesacc-00-01} \\
    \figc{ioxconc-00-01} \\
    \figc{ioxmass-00-01} \\
    \figc{pendconc-00-01} \\
    \figc{pendmass-00-01}
  \end{center}
  \caption{\MyID{}Drain season 2000 --- 2001.}
  \label{fig:season0001}
\end{figure}

\begin{figure}[htbp]
  \begin{center}
    \figctop{weather-00-01-zoom} \\
    \figc{drainflow-00-01-zoom} \\
    \figc{drainflowacc-00-01-zoom} \\
    \figc{particles-00-01-zoom} \\
    \figc{particlesacc-00-01-zoom} \\
    \figc{ioxconc-00-01-zoom} \\
    \figc{ioxmass-00-01-zoom} \\
    \figc{pendconc-00-01-zoom} \\
    \figc{pendmass-00-01-zoom}
  \end{center}
  \caption{\MyID{}Drain season 2000 --- 2001, single event.}
  \label{fig:season0001zoom}
  \label{fig:last}
\end{figure}

%%% Local Variables: 
%%% mode: latex
%%% TeX-master: nil
%%% End: 


\end{document}

%%% Local Variables: 
%%% mode: latex
%%% TeX-master: t
%%% End: 
