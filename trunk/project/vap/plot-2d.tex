\documentclass[a4paper]{article}

\usepackage[left=1cm,top=1cm,right=1cm,nohead,nofoot]{geometry}
\usepackage[latin1]{inputenc}
\usepackage[T1]{fontenc}
\usepackage[danish]{babel}
\usepackage{natbib}
\bibliographystyle{apalike}
\usepackage{graphicx}
\usepackage{hyperref}
\usepackage{placeins}

\newcommand{\figestrup}[1]{\includegraphics[trim=12mm 0mm 17mm 9mm,clip]{fig/#1}}

\newcommand{\figsilstrup}[1]{\includegraphics[trim=8mm 0mm 12mm 7mm,clip]{fig/#1}}

\begin{document}

\begin{figure}[htbp]\centering
  \begin{tabular}{ccc}
    \figestrup{Estrup-pF-2000-5} & 
    \figestrup{Estrup-pF-2000-6} & 
    \figestrup{Estrup-pF-2000-7} \\
    \figestrup{Estrup-pF-2000-8} & 
    \figestrup{Estrup-pF-2000-9} & 
    \figestrup{Estrup-pF-2000-10} \\
    \figestrup{Estrup-pF-2000-11} & 
    \figestrup{Estrup-pF-2000-12} & 
    \figestrup{Estrup-pF-2001-1} \\
    \figestrup{Estrup-pF-2001-2} & 
    \figestrup{Estrup-pF-2001-3} & 
    \figestrup{Estrup-pF-2001-4}
  \end{tabular}
  
  \caption{Estrup soil potential at the end of each month since first
    application of Bromide.  The z-axis denotes depth, the x-axis
    distance from drain.  There are tick marks for every meter.  Blue
    denote saturation or near saturation (pF<0), white field capacity
    (pF=2), and red wilting point (pF>4.2).}
\label{fig:Estrup-pF-2000}
\end{figure}\FloatBarrier

\begin{figure}[htbp]\centering
  \begin{tabular}{ccc}
    \figestrup{Estrup-pF-2001-5} & 
    \figestrup{Estrup-pF-2001-6} & 
    \figestrup{Estrup-pF-2001-7} \\
    \figestrup{Estrup-pF-2001-8} & 
    \figestrup{Estrup-pF-2001-9} & 
    \figestrup{Estrup-pF-2001-10} \\
    \figestrup{Estrup-pF-2001-11} & 
    \figestrup{Estrup-pF-2001-12} & 
    \figestrup{Estrup-pF-2002-1} \\
    \figestrup{Estrup-pF-2002-2} & 
    \figestrup{Estrup-pF-2002-3} & 
    \figestrup{Estrup-pF-2002-4}
  \end{tabular}
  
  \caption{Estrup soil potential at the end of each month second year
    after application of Bromide.  The z-axis denotes depth, the
    x-axis distance from drain.  There are tick marks for every meter.
    Blue denote saturation or near saturation (pF<0), white field
    capacity (pF=2), and red wilting point (pF>4.2).}
\label{fig:Estrup-pF-2001}
\end{figure}\FloatBarrier

\begin{figure}[htbp]
  \centering
  \includegraphics{fig/Estrup-water-horizontal-2000}
  
  \caption{Estrup total horizontal water flux between 2000-5-1 and
    2001-5-1.  The flux is shown on the x-axis (positive away from
    drain) as a function of depth shown on the y-axis.  The graph
    labels are the distance from drain in centimeters.}
  \label{fig:Estrup-water-horizontal-2000}
\end{figure}\FloatBarrier

\begin{figure}[htbp]
  \centering
  \includegraphics{fig/Estrup-water-2000}
  
  \caption{Estrup total vertical water flux between 2000-5-1 and
    2001-5-1.  The flux is shown on the y-axis (positive up) as a
    function of distance from drain shown on the y-axis.  The graph
    labels are depths in centimeters above surface.}
  \label{fig:Estrup-water-2000}
\end{figure}\FloatBarrier

\begin{figure}[htbp]
  \centering
  \includegraphics{fig/Estrup-water-biopore-2000}
  
  \caption{Estrup total biopore water flux between 2000-5-1 and
    2001-5-1.  The flux is shown on the y-axis (positive up) as a
    function of distance from drain shown on the y-axis.  The graph
    labels are depths in centimeters above surface.}
  \label{fig:Estrup-water-biopore-2000}
\end{figure}\FloatBarrier

\begin{figure}[htbp]
  \centering
  \includegraphics{fig/Estrup-water-horizontal-2001}
  
  \caption{Estrup total horizontal water flux between 2001-5-1 and
    2002-5-1.  The flux is shown on the x-axis (positive away from
    drain) as a function of depth shown on the y-axis.  The graph
    labels are the distance from drain in centimeters.}
  \label{fig:Estrup-water-2001-horizontal}
\end{figure}\FloatBarrier

\begin{figure}[htbp]
  \centering
  \includegraphics{fig/Estrup-water-2001}
  
  \caption{Estrup total vertical water flux between 2001-5-1 and
    2002-5-1.  The flux is shown on the y-axis (positive up) as a
    function of distance from drain shown on the y-axis.  The graph
    labels are depths in centimeters above surface.}
  \label{fig:Estrup-water-2001}
\end{figure}\FloatBarrier

\begin{figure}[htbp]
  \centering
  \includegraphics{fig/Estrup-water-biopore-2001}
  
  \caption{Estrup total biopore water flux between 2001-5-1 and
    2002-5-1.  The flux is shown on the y-axis (positive up) as a
    function of distance from drain shown on the y-axis.  The graph
    labels are depths in centimeters above surface.}
  \label{fig:Estrup-water-biopore-2001}
\end{figure}\FloatBarrier

\begin{figure}[htbp]\centering
  \begin{tabular}{ccc}
    \figestrup{Estrup-M-Bromide-2000-5} & 
    \figestrup{Estrup-M-Bromide-2000-6} & 
    \figestrup{Estrup-M-Bromide-2000-7} \\
    \figestrup{Estrup-M-Bromide-2000-8} & 
    \figestrup{Estrup-M-Bromide-2000-9} & 
    \figestrup{Estrup-M-Bromide-2000-10} \\
    \figestrup{Estrup-M-Bromide-2000-11} & 
    \figestrup{Estrup-M-Bromide-2000-12} & 
    \figestrup{Estrup-M-Bromide-2001-1} \\
    \figestrup{Estrup-M-Bromide-2001-2} & 
    \figestrup{Estrup-M-Bromide-2001-3} & 
    \figestrup{Estrup-M-Bromide-2001-4}
  \end{tabular}
  
  \caption{Estrup Bromide soil content at the end of each month since
    first application of Bromide.  The z-axis denotes depth, the x-axis distance from drain.  There are tick marks for every
    meter. The color scale is white<10 pg/l, yellow=1 ng/l,
    orange=0.1 $\mu$g/l, red=10 $\mu$g/l, and black>1 mg/l}
\label{fig:Estrup-Bromide-2000}
\end{figure}\FloatBarrier

\begin{figure}[htbp]\centering
  \begin{tabular}{ccc}
    \figestrup{Estrup-M-Bromide-2001-5} & 
    \figestrup{Estrup-M-Bromide-2001-6} & 
    \figestrup{Estrup-M-Bromide-2001-7} \\
    \figestrup{Estrup-M-Bromide-2001-8} & 
    \figestrup{Estrup-M-Bromide-2001-9} & 
    \figestrup{Estrup-M-Bromide-2001-10} \\
    \figestrup{Estrup-M-Bromide-2001-11} & 
    \figestrup{Estrup-M-Bromide-2001-12} & 
    \figestrup{Estrup-M-Bromide-2002-1} \\
    \figestrup{Estrup-M-Bromide-2002-2} & 
    \figestrup{Estrup-M-Bromide-2002-3} & 
    \figestrup{Estrup-M-Bromide-2002-4}
  \end{tabular}
  
  \caption{Estrup Bromide soil content at the end of each month second
    year after application of Bromide.  The z-axis denotes depth, the
    x-axis distance from drain.  There are tick marks for every
    meter. The color scale is white<10 pg/l, yellow=1 ng/l, orange=0.1
    $\mu$g/l, red=10 $\mu$g/l, and black>1 mg/l}
\label{fig:Estrup-Bromide-2001}
\end{figure}\FloatBarrier

\begin{figure}[htbp]
  \centering
  \includegraphics{fig/Estrup-Bromide-horizontal-2000}
  
  \caption{Estrup total horizontal Bromide flow between 2000-5-1 and
    2001-5-1.  The flow is shown on the x-axis (positive away from
    drain) as a function of depth shown on the y-axis.  The graph
    labels are the distance from drain in centimeters.}
  \label{fig:Estrup-Bromide-2000-horizontal}
\end{figure}\FloatBarrier

\begin{figure}[htbp]
  \centering
  \includegraphics{fig/Estrup-Bromide-2000}
  
  \caption{Estrup total vertical Bromide flow between 2000-5-1 and
    2001-5-1.  The flow is shown on the y-axis (positive up) as a
    function of distance from drain shown on the y-axis.  The graph
    labels are depths in centimeters above surface.}
  \label{fig:Estrup-Bromide-2000-vertical}
\end{figure}\FloatBarrier

\begin{figure}[htbp]
  \centering
  \includegraphics{fig/Estrup-Bromide-biopore-2000}
  
  \caption{Estrup total biopore Bromide flow between 2000-5-1 and
    2001-5-1.  The flow is shown on the y-axis (positive up) as a
    function of distance from drain shown on the y-axis.  The graph
    labels are depths in centimeters above surface.}
  \label{fig:Estrup-Bromide-biopore-2000}
\end{figure}\FloatBarrier

\begin{figure}[htbp]
  \centering
  \includegraphics{fig/Estrup-Bromide-horizontal-2001}
  
  \caption{Estrup total horizontal Bromide flow between 2001-5-1 and
    2002-5-1.  The flow is shown on the x-axis (positive away from
    drain) as a function of depth shown on the y-axis.  The graph
    labels are the distance from drain in centimeters.}
  \label{fig:Estrup-Bromide-2001-horizontal}
\end{figure}\FloatBarrier

\begin{figure}[htbp]
  \centering
  \includegraphics{fig/Estrup-Bromide-2001}
  
  \caption{Estrup total vertical Bromide flow between 2001-5-1 and
    2002-5-1.  The flow is shown on the y-axis (positive up) as a
    function of distance from drain shown on the y-axis.  The graph
    labels are depths in centimeters above surface.}
  \label{fig:Estrup-Bromide-2001-vertical}
\end{figure}\FloatBarrier

\begin{figure}[htbp]
  \centering
  \includegraphics{fig/Estrup-Bromide-biopore-2001}
  
  \caption{Estrup total biopore Bromide flow between 2001-5-1 and
    2002-5-1.  The flow is shown on the y-axis (positive up) as a
    function of distance from drain shown on the y-axis.  The graph
    labels are depths in centimeters above surface.}
  \label{fig:Estrup-Bromide-biopore-2001}
\end{figure}\FloatBarrier

\begin{figure}[htbp]\centering
  \begin{tabular}{ccc}
    \figestrup{Estrup-M-Glyphosate-2000-5} & 
    \figestrup{Estrup-M-Glyphosate-2000-6} & 
    \figestrup{Estrup-M-Glyphosate-2000-7} \\
    \figestrup{Estrup-M-Glyphosate-2000-8} & 
    \figestrup{Estrup-M-Glyphosate-2000-9} & 
    \figestrup{Estrup-M-Glyphosate-2000-10} \\
    \figestrup{Estrup-M-Glyphosate-2000-11} & 
    \figestrup{Estrup-M-Glyphosate-2000-12} & 
    \figestrup{Estrup-M-Glyphosate-2001-1} \\
    \figestrup{Estrup-M-Glyphosate-2001-2} & 
    \figestrup{Estrup-M-Glyphosate-2001-3} & 
    \figestrup{Estrup-M-Glyphosate-2001-4}
  \end{tabular}
  
  \caption{Estrup Glyphosate soil content at the end of each month
    since first application of Bromide.  The z-axis denotes depth, the
    x-axis distance from drain.  There are tick marks for every
    meter. The color scale is white<10 pg/l, yellow=1 ng/l, orange=0.1
    $\mu$g/l, red=10 $\mu$g/l, and black>1 mg/l}
\label{fig:Estrup-M-Glyphosate-2000}
\end{figure}\FloatBarrier

\begin{figure}[htbp]\centering
  \begin{tabular}{ccc}
    \figestrup{Estrup-C-Glyphosate-2000-5} & 
    \figestrup{Estrup-C-Glyphosate-2000-6} & 
    \figestrup{Estrup-C-Glyphosate-2000-7} \\
    \figestrup{Estrup-C-Glyphosate-2000-8} & 
    \figestrup{Estrup-C-Glyphosate-2000-9} & 
    \figestrup{Estrup-C-Glyphosate-2000-10} \\
    \figestrup{Estrup-C-Glyphosate-2000-11} & 
    \figestrup{Estrup-C-Glyphosate-2000-12} & 
    \figestrup{Estrup-C-Glyphosate-2001-1} \\
    \figestrup{Estrup-C-Glyphosate-2001-2} & 
    \figestrup{Estrup-C-Glyphosate-2001-3} & 
    \figestrup{Estrup-C-Glyphosate-2001-4}
  \end{tabular}
  
  \caption{Estrup Glyphosate soil water concentration at the end of
    each month since first application of Bromide.  The z-axis denotes
    depth, the x-axis distance from drain.  There are tick marks for
    every meter. The color scale is white<10 pg/l, yellow=1 ng/l, orange=0.1
    $\mu$g/l, red=10 $\mu$g/l, and black>1 mg/l}
\label{fig:Estrup-C-Glyphosate-2000}
\end{figure}\FloatBarrier

\begin{figure}[htbp]
  \centering
  \includegraphics{fig/Estrup-Glyphosate-horizontal-2000}
  
  \caption{Estrup total horizontal Glyphosate flow between 2000-5-1 and
    2001-5-1.  The flow is shown on the x-axis (positive away from
    drain) as a function of depth shown on the y-axis.  The graph
    labels are the distance from drain in centimeters.}
  \label{fig:Estrup-Glyphosate-2000-horizontal}
\end{figure}\FloatBarrier

\begin{figure}[htbp]
  \centering
  \includegraphics{fig/Estrup-Glyphosate-2000}
  
  \caption{Estrup total vertical Glyphosate flow between 2000-5-1 and
    2001-5-1.  The flow is shown on the y-axis (positive up) as a
    function of distance from drain shown on the y-axis.  The graph
    labels are depths in centimeters above surface.}
  \label{fig:Estrup-Glyphosate-2000}
\end{figure}\FloatBarrier

\begin{figure}[htbp]
  \centering
  \includegraphics{fig/Estrup-Glyphosate-biopore-2000}
  
  \caption{Estrup total biopore Glyphosate flow between 2000-5-1 and
    2001-5-1.  The flow is shown on the y-axis (positive up) as a
    function of distance from drain shown on the y-axis.  The graph
    labels are depths in centimeters above surface.}
  \label{fig:Estrup-Glyphosate-biopore-2000}
\end{figure}\FloatBarrier

\begin{figure}[htbp]\centering
  \begin{tabular}{ccc}
    \figsilstrup{Silstrup-pF-2000-5} & 
    \figsilstrup{Silstrup-pF-2000-6} & 
    \figsilstrup{Silstrup-pF-2000-7} \\
    \figsilstrup{Silstrup-pF-2000-8} & 
    \figsilstrup{Silstrup-pF-2000-9} & 
    \figsilstrup{Silstrup-pF-2000-10} \\
    \figsilstrup{Silstrup-pF-2000-11} & 
    \figsilstrup{Silstrup-pF-2000-12} & 
    \figsilstrup{Silstrup-pF-2001-1} \\
    \figsilstrup{Silstrup-pF-2001-2} & 
    \figsilstrup{Silstrup-pF-2001-3} & 
    \figsilstrup{Silstrup-pF-2001-4}
  \end{tabular}
  
  \caption{Silstrup soil potential at the end of each month since first
    application of Bromide.  The z-axis denotes depth, the x-axis
    distance from drain.  There are tick marks for every meter.  Blue
    denote saturation or near saturation (pF<0), white field capacity
    (pF=2), and red wilting point (pF>4.2).}
\label{fig:Silstrup-pF-2000}
\end{figure}\FloatBarrier

\begin{figure}[htbp]\centering
  \begin{tabular}{ccc}
    \figsilstrup{Silstrup-pF-2001-5} & 
    \figsilstrup{Silstrup-pF-2001-6} & 
    \figsilstrup{Silstrup-pF-2001-7} \\
    \figsilstrup{Silstrup-pF-2001-8} & 
    \figsilstrup{Silstrup-pF-2001-9} & 
    \figsilstrup{Silstrup-pF-2001-10} \\
    \figsilstrup{Silstrup-pF-2001-11} & 
    \figsilstrup{Silstrup-pF-2001-12} & 
    \figsilstrup{Silstrup-pF-2002-1} \\
    \figsilstrup{Silstrup-pF-2002-2} & &
  \end{tabular}
  
  \caption{Silstrup soil potential at the end of each month second year
    after application of Bromide.  The z-axis denotes depth, the
    x-axis distance from drain.  There are tick marks for every meter.
    Blue denote saturation or near saturation (pF<0), white field
    capacity (pF=2), and red wilting point (pF>4.2).}
\label{fig:Silstrup-pF-2001}
\end{figure}\FloatBarrier

\begin{figure}[htbp]
  \centering
  \includegraphics{fig/Silstrup-water-horizontal-2000}
  
  \caption{Silstrup total horizontal water flux between 2000-5-1 and
    2001-5-1.  The flux is shown on the x-axis (positive away from
    drain) as a function of depth shown on the y-axis.  The graph
    labels are the distance from drain in centimeters.}
  \label{fig:Silstrup-water-2000-horizontal}
\end{figure}\FloatBarrier

\begin{figure}[htbp]
  \centering
  \includegraphics{fig/Silstrup-water-2000}
  
  \caption{Silstrup total vertical water flux between 2000-5-1 and
    2001-5-1.  The flux is shown on the y-axis (positive up) as a
    function of distance from drain shown on the y-axis.  The graph
    labels are depths in centimeters above surface.}
  \label{fig:Silstrup-water-2000}
\end{figure}\FloatBarrier

\begin{figure}[htbp]
  \centering
  \includegraphics{fig/Silstrup-water-biopore-2000}
  
  \caption{Silstrup total biopore water flux between 2000-5-1 and
    2001-5-1.  The flux is shown on the y-axis (positive up) as a
    function of distance from drain shown on the y-axis.  The graph
    labels are depths in centimeters above surface.}
  \label{fig:Silstrup-water-biopore-2000}
\end{figure}\FloatBarrier

\begin{figure}[htbp]
  \centering
  \includegraphics{fig/Silstrup-water-horizontal-2001}
  
  \caption{Silstrup total horizontal water flux between 2001-5-1 and
    2002-3-1.  The flux is shown on the x-axis (positive away from
    drain) as a function of depth shown on the y-axis.  The graph
    labels are the distance from drain in centimeters.}
  \label{fig:Silstrup-water-2001-horizontal}
\end{figure}\FloatBarrier

\begin{figure}[htbp]
  \centering
  \includegraphics{fig/Silstrup-water-2001}
  
  \caption{Silstrup total vertical water flux between 2001-5-1 and
    2002-3-1.  The flux is shown on the y-axis (positive up) as a
    function of distance from drain shown on the y-axis.  The graph
    labels are depths in centimeters above surface.}
  \label{fig:Silstrup-water-2001}
\end{figure}\FloatBarrier

\begin{figure}[htbp]
  \centering
  \includegraphics{fig/Silstrup-water-biopore-2001}
  
  \caption{Silstrup total biopore water flux between 2001-5-1 and
    2002-3-1.  The flux is shown on the y-axis (positive up) as a
    function of distance from drain shown on the y-axis.  The graph
    labels are depths in centimeters above surface.}
  \label{fig:Silstrup-water-biopore-2001}
\end{figure}\FloatBarrier

\begin{figure}[htbp]\centering
  \begin{tabular}{ccc}
    \figsilstrup{Silstrup-M-Bromide-2000-5} & 
    \figsilstrup{Silstrup-M-Bromide-2000-6} & 
    \figsilstrup{Silstrup-M-Bromide-2000-7} \\
    \figsilstrup{Silstrup-M-Bromide-2000-8} & 
    \figsilstrup{Silstrup-M-Bromide-2000-9} & 
    \figsilstrup{Silstrup-M-Bromide-2000-10} \\
    \figsilstrup{Silstrup-M-Bromide-2000-11} & 
    \figsilstrup{Silstrup-M-Bromide-2000-12} & 
    \figsilstrup{Silstrup-M-Bromide-2001-1} \\
    \figsilstrup{Silstrup-M-Bromide-2001-2} & 
    \figsilstrup{Silstrup-M-Bromide-2001-3} & 
    \figsilstrup{Silstrup-M-Bromide-2001-4}
  \end{tabular}
  
  \caption{Silstrup Bromide soil content at the end of each month since
    first application of Bromide.  The z-axis denotes depth, the x-axis distance from drain.  There are tick marks for every
    meter. The color scale is white<10 pg/l, yellow=1 ng/l,
    orange=0.1 $\mu$g/l, red=10 $\mu$g/l, and black>1 mg/l}
\label{fig:Silstrup-Bromide-2000}
\end{figure}\FloatBarrier

\begin{figure}[htbp]\centering
  \begin{tabular}{ccc}
    \figsilstrup{Silstrup-M-Bromide-2001-5} & 
    \figsilstrup{Silstrup-M-Bromide-2001-6} & 
    \figsilstrup{Silstrup-M-Bromide-2001-7} \\
    \figsilstrup{Silstrup-M-Bromide-2001-8} & 
    \figsilstrup{Silstrup-M-Bromide-2001-9} & 
    \figsilstrup{Silstrup-M-Bromide-2001-10} \\
    \figsilstrup{Silstrup-M-Bromide-2001-11} & 
    \figsilstrup{Silstrup-M-Bromide-2001-12} & 
    \figsilstrup{Silstrup-M-Bromide-2002-1} \\
    \figsilstrup{Silstrup-M-Bromide-2002-2} &  & 
  \end{tabular}
  
  \caption{Silstrup Bromide soil content at the end of each month
    second year after application of Bromide.  The z-axis denotes
    depth, the x-axis distance from drain.  There are tick marks for
    every meter. The color scale is white<10 pg/l, yellow=1 ng/l,
    orange=0.1 $\mu$g/l, red=10 $\mu$g/l, and black>1 mg/l}
\label{fig:Silstrup-Bromide-2001}
\end{figure}\FloatBarrier

\begin{figure}[htbp]
  \centering
  \includegraphics{fig/Silstrup-Bromide-horizontal-2000}
  
  \caption{Silstrup total horizontal Bromide flow between 2000-5-1 and
    2001-5-1.  The flow is shown on the x-axis (positive away from
    drain) as a function of depth shown on the y-axis.  The graph
    labels are the distance from drain in centimeters.}
  \label{fig:Silstrup-Bromide-2000-horizontal}
\end{figure}\FloatBarrier

\begin{figure}[htbp]
  \centering
  \includegraphics{fig/Silstrup-Bromide-2000}
  
  \caption{Silstrup total vertical Bromide flow between 2000-5-1 and
    2001-5-1.  The flow is shown on the y-axis (positive up) as a
    function of distance from drain shown on the y-axis.  The graph
    labels are depths in centimeters above surface.}
  \label{fig:Silstrup-Bromide-2000-vertical}
\end{figure}\FloatBarrier

\begin{figure}[htbp]
  \centering
  \includegraphics{fig/Silstrup-Bromide-biopore-2000}
  
  \caption{Silstrup total biopore Bromide flow between 2000-5-1 and
    2001-5-1.  The flow is shown on the y-axis (positive up) as a
    function of distance from drain shown on the y-axis.  The graph
    labels are depths in centimeters above surface.}
  \label{fig:Silstrup-Bromide-biopore-2000}
\end{figure}\FloatBarrier

\begin{figure}[htbp]
  \centering
  \includegraphics{fig/Silstrup-Bromide-horizontal-2001}
  
  \caption{Silstrup total horizontal Bromide flow between 2001-5-1 and
    2002-3-1.  The flow is shown on the x-axis (positive away from
    drain) as a function of depth shown on the y-axis.  The graph
    labels are the distance from drain in centimeters.}
  \label{fig:Silstrup-Bromide-2001-horizontal}
\end{figure}\FloatBarrier

\begin{figure}[htbp]
  \centering
  \includegraphics{fig/Silstrup-Bromide-2001}
  
  \caption{Silstrup total vertical Bromide flow between 2001-5-1 and
    2002-3-1.  The flow is shown on the y-axis (positive up) as a
    function of distance from drain shown on the y-axis.  The graph
    labels are depths in centimeters above surface.}
  \label{fig:Silstrup-Bromide-2001-vertical}
\end{figure}\FloatBarrier

\begin{figure}[htbp]
  \centering
  \includegraphics{fig/Silstrup-Bromide-biopore-2001}
  
  \caption{Silstrup total biopore Bromide flow between 2001-5-1 and
    2002-3-1.  The flow is shown on the y-axis (positive up) as a
    function of distance from drain shown on the y-axis.  The graph
    labels are depths in centimeters above surface.}
  \label{fig:Silstrup-Bromide-biopore-2001}
\end{figure}\FloatBarrier

\begin{figure}[htbp]
  \centering
  \includegraphics{fig/Silstrup-Metamitron-horizontal-2000}
  
  \caption{Silstrup total horizontal Metamitron flow between 2000-5-1 and
    2001-5-1.  The flow is shown on the x-axis (positive away from
    drain) as a function of depth shown on the y-axis.  The graph
    labels are the distance from drain in centimeters.}
  \label{fig:Silstrup-Metamitron-2000-horizontal}
\end{figure}\FloatBarrier

\begin{figure}[htbp]
  \centering
  \includegraphics{fig/Silstrup-Metamitron-2000}
  
  \caption{Silstrup total vertical Metamitron flow between 2000-5-1 and
    2001-5-1.  The flow is shown on the y-axis (positive up) as a
    function of distance from drain shown on the y-axis.  The graph
    labels are depths in centimeters above surface.}
  \label{fig:Silstrup-Metamitron-2000-vertical}
\end{figure}\FloatBarrier

\begin{figure}[htbp]
  \centering
  \includegraphics{fig/Silstrup-Metamitron-biopore-2000}
  
  \caption{Silstrup total biopore Metamitron flow between 2000-5-1 and
    2001-5-1.  The flow is shown on the y-axis (positive up) as a
    function of distance from drain shown on the y-axis.  The graph
    labels are depths in centimeters above surface.}
  \label{fig:Silstrup-Metamitron-biopore-2000}
\end{figure}\FloatBarrier

\begin{figure}[htbp]\centering
  \begin{tabular}{ccc}
    \figsilstrup{Silstrup-M-Glyphosate-2001-5} & 
    \figsilstrup{Silstrup-M-Glyphosate-2001-6} & 
    \figsilstrup{Silstrup-M-Glyphosate-2001-7} \\
    \figsilstrup{Silstrup-M-Glyphosate-2001-8} & 
    \figsilstrup{Silstrup-M-Glyphosate-2001-9} & 
    \figsilstrup{Silstrup-M-Glyphosate-2001-10} \\
    \figsilstrup{Silstrup-M-Glyphosate-2001-11} & 
    \figsilstrup{Silstrup-M-Glyphosate-2001-12} & 
    \figsilstrup{Silstrup-M-Glyphosate-2002-1} \\
    \figsilstrup{Silstrup-M-Glyphosate-2002-2} & & 
  \end{tabular}
  
  \caption{Silstrup Glyphosate soil content at the end of each month
    since first application of Bromide.  The z-axis denotes depth, the
    x-axis distance from drain.  There are tick marks for every
    meter. The color scale is white<10 pg/l, yellow=1 ng/l, orange=0.1
    $\mu$g/l, red=10 $\mu$g/l, and black>1 mg/l}
\label{fig:Silstrup-M-Glyphosate-2001}
\end{figure}\FloatBarrier

\begin{figure}[htbp]\centering
  \begin{tabular}{ccc}
    \figsilstrup{Silstrup-C-Glyphosate-2001-5} & 
    \figsilstrup{Silstrup-C-Glyphosate-2001-6} & 
    \figsilstrup{Silstrup-C-Glyphosate-2001-7} \\
    \figsilstrup{Silstrup-C-Glyphosate-2001-8} & 
    \figsilstrup{Silstrup-C-Glyphosate-2001-9} & 
    \figsilstrup{Silstrup-C-Glyphosate-2001-10} \\
    \figsilstrup{Silstrup-C-Glyphosate-2001-11} & 
    \figsilstrup{Silstrup-C-Glyphosate-2001-12} & 
    \figsilstrup{Silstrup-C-Glyphosate-2002-1} \\
    \figsilstrup{Silstrup-C-Glyphosate-2002-2} &  & 
  \end{tabular}
  
  \caption{Silstrup Glyphosate soil water concentration at the end of
    each month since one year after first application of Bromide.  The
    z-axis denotes depth, the x-axis distance from drain.  There are
    tick marks for every meter. The color scale is white<10 pg/l,
    yellow=1 ng/l, orange=0.1 $\mu$g/l, red=10 $\mu$g/l, and black>1
    mg/l}
\label{fig:Silstrup-C-Glyphosate-2001}
\end{figure}\FloatBarrier

\begin{figure}[htbp]
  \centering
  \includegraphics{fig/Silstrup-Glyphosate-horizontal-2001}
  
  \caption{Silstrup total horizontal Glyphosate flow between 2001-5-1 and
    2002-3-1.  The flow is shown on the x-axis (positive away from
    drain) as a function of depth shown on the y-axis.  The graph
    labels are the distance from drain in centimeters.}
  \label{fig:Silstrup-Glyphosate-2001-horizontal}
\end{figure}\FloatBarrier

\begin{figure}[htbp]
  \centering
  \includegraphics{fig/Silstrup-Glyphosate-2001}
  
  \caption{Silstrup total vertical Glyphosate flow between 2001-5-1 and
    2002-3-1.  The flow is shown on the y-axis (positive up) as a
    function of distance from drain shown on the y-axis.  The graph
    labels are depths in centimeters above surface.}
  \label{fig:Silstrup-Glyphosate-2001-vertical}
\end{figure}\FloatBarrier

\begin{figure}[htbp]
  \centering
  \includegraphics{fig/Silstrup-Glyphosate-biopore-2001}
  
  \caption{Silstrup total biopore Glyphosate flow between 2001-5-1 and
    2002-3-1.  The flow is shown on the y-axis (positive up) as a
    function of distance from drain shown on the y-axis.  The graph
    labels are depths in centimeters above surface.}
  \label{fig:Silstrup-Glyphosate-biopore-2001}
\end{figure}\FloatBarrier

\end{document}
