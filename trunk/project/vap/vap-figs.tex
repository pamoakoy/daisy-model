\documentclass[a4paper]{article}

\usepackage[left=1cm,top=2cm,right=1cm]{geometry}
\usepackage[latin1]{inputenc}
\usepackage[T1]{fontenc}
\usepackage{natbib}
\bibliographystyle{apalike}
\usepackage{graphicx}
\usepackage{hyperref}
\usepackage{fancyhdr}
\usepackage{placeins}
\pagestyle{fancy}
\lhead{\today}

\newcommand{\koc}{$\mbox{K}_{\mbox{\textsc{oc}}}$}
\newcommand{\kclay}{$\mbox{K}_{\mbox{clay}}$}
\newcommand{\kd}{$\mbox{K}_{\mbox{d}}$}

\newcommand{\focus}{\textsc{focus}}
\newcommand{\hypres}{\textsc{hypres}}
\newcommand{\macro}{\textsc{macro}}
\newcommand{\Macro}{\textsc{Macro}}

%% \includeonly{}

\begin{document}

\section*{Daisy simulation of Silstrup and Estrup}

The Danish Pesticide Leaching Assessment Programme (PLAP) has been
monitoring drain and soil water since 1999 in six different locations
in order to evaluate the leaching risk of pesticides.  The drain water
measurements pesticides are found in some drain water measurements,
but not in the horizontal filters 3.5 meter below surface.

In order to better understand the system, and eventually how the
measurements can be better used for assessing potential risk of
contamination of drinking water, the Daisy agricultural model has been
extended with support for those processes we assume are relevant for
transport of pesticides from surface to drain pipes.

To test our understanding as embedded in the model, as well as the
applicability for the PLAP sites, a pilot project with two PLAP sites,
four pesticides, and two years of data have been modelled.  Our
hypothesis is here that we can explain the measured data with the
model.

The most significant measured results are from the Estrup site, so
that was fixed as the first site.  We wanted to use the same years for
both sites to see the results of similar climate on two different
locations.  Hourly weather data from the site had to be present, both
for the modelling period and one growth season before for warmup.  The
initial choice of F{\aa}rdrup for the second site was rejected as it
was not possible to get sufficient quality weather data.  Furthermore,
we wanted the same pesticides on both sites, and both weakly and
strongly sorbing pesticides represented.  Glyphosate had to be one of
them.

Based on these criteria, we chose Silstrup and Estrup, drain seasons
2000-2001 and 2001-2002, with the pesticides glyphosate, fenpropimorph,
dimethoate, and metamitron.

Both sites are described in detail in \citet{lindhardt2001}, with an
overview of the measured data found in \citet{vap2009}.  Estrup is a
pedologically rich site, containing both areas with sand and clay,
peat, thin layers of chalk, and even sand from railroad construction.
Silstrup is also heterogeneous, but less so, with high levels of clay
dominating the area.

\section{Model setup}

\subsection{Weather}

Hourly weather data for Silstrup, Tylstrup and Askov (near Estrup) was
provided by the Faculty of Agricultural Sciences, Aarhus University.
The idea was that Tylstrup data could be used to supplement holes in
the Silstrup data.  Both the Silstrup and Askov data sets contained
several short holes.  We filled those by using values from the
preceding or following hours.  The Silstrup data set ended at
2002-03-12.  The drain season ended 2002-03-20, with 0.6 mm water
collected from the drains the last 8 days.  Baring this in mind, we
chosed to end the simulation 2002-03-11, rather than continue with
data from another station.

The weather data contained air temperature, wind speed, relative
humidity, precipitation, and global radiation are used.  Based on
these data, Daisy can use FAO version of Penman-Monteith equation
\citep{FAO-PM} to calculate reference evapotranspiration (ET0).  From
that, Daisy will calculate potential evapotranspiration (ETc) by using
the crop leaf area index (LAI) with Beer's law to dividing the surface
into canopy and bare soil, and using different factors for each
fraction.  Based partly on \citet{kjaersgaard2008crop}, crop factor of
1.2 and a bare soil factor of 0.6 was chosen.  Actual
evapotranspiration (ETa) is further limited by the capability of the
root system and soil surface to extract water from the soil.

\begin{figure}[htbp]
  \begin{center}
    \includegraphics[trim=0mm 5mm 0mm 0mm,clip]{fig/Silstrup-weather}
    \includegraphics{fig/Estrup-weather}
  \end{center}
  \caption{Accumulated precipitation and hourly values for temperature
    measured at Silstrup (top) and Askov (bottom).  Calculated
    potentiel and simulated actual evapotranspiration are also shown.}
  \label{fig:weather}
\end{figure}
Precipitation, air temperature, ETc and ETa can be seen on
figure~\ref{fig:weather}.

\subsection{Management}

All management data were provided by GEUS.

\subsubsection{Tillage}

Date, type, and depth were specified for all tillage operations.  All
three were entered into Daisy.  In Daisy, the main effects of tillage
are to incroporate some of the surface material into the soil
(depending on the type of tillage operation), and to mix the content
of the soil in the specified depth.  This is not expected to affect
pesticide leaching much.  In the real world, the main effect of
tillage applicable to pesticide leaching is likely to be a change in
the hydraulic properties for the top soil resulting from the tillage
operation.  Since we chose to implement dynamic hydraulic soil
properties for the Silstrup soil surface, the tillage information were
useful there as well.

\subsubsection{Fertilization}

Date, type, and amount were specified for each fertilization event, as
well as N, P and K content of fertilizer.  Of these nutrients, Daisy
can normally only handle N, and N has been disabled for these
simulations to save time.  All fertilization event have been added to
Daisy, but without N there will be no effect of the mineral
fertilizer.  The organic fertilizer will have a minimal effect, the
water content of the fertilizer will be added, and the dry matter
content will be added to the litter where it can catch water and
pesticides, until it become incroporated into the soil by either
tillage operations or earthworm activity.  However, due to the timing
of the applications, both effect is likely to be negigible.

\subsubsection{Crop management}
\label{sec:crop-man}

For sowing information about date, crop and sowing density were
provided.  The Daisy crop parametrizations do not rely on sowing
density, but instead assume that ``standard practise'' is used.  Daisy
has experimental support for a crop model that include sowing density,
but given that the PLAP sites are expected to follow standard
management practise, we found it safer to use the older and better
tested crop parameterizations.

Information about the crow growth was given phenologically (BBCH
stage) and in terms of above ground biomass. No attempt was made
calibrate the crop in order to match this with the simulated
development stage and biomass.  Information about the two important
parameters for the water balance, namely leaf area and root density,
were not provided, but are normally related to the two measured
parameters.

Harvest data included date, stubble height, as well as grain and straw
yields.  Date and stubble height can be directly used by Daisy.  Grain
yield can be used for calibration, however, as crop production is not
the focus of this project, we merely noted that both measured and
simulated yield where within the normal range.  The ratio of between
grain and straw yield was used for a coarse estimation of the fraction
of the crop left on the field as residuals after harvest.  The
residuals play a crusial role in the simulation for the Silstrup
glyphosate leaching.

\begin{table}[htbp]
  \caption{Crop management.}
  \label{tab:crop-man}
  \centering
  \begin{tabular}{c||c|c|c||c|c|c}\hline
    & \multicolumn{3}{c||}{Silstrup} & \multicolumn{3}{c}{Estrup}\\
    Year & Crop          & Sow   & Harvest & Crop          & Sow   & Harvest\\
\hline
    2000 & Fodder Beet   & 04/05 & 15/11   & Spring Barley & 12/4  & 28/8 \\
    2001 & Spring Barley & 9/5   & 5/9     & Peas          & 5/2   & 22/8 \\
    2001 &               &       &         & Winter Wheat  & 19/10 & --- \\
  \end{tabular}
\end{table}

Two calibrations was done on crop management.  The first was to
replace the fodder beet (see table~\ref{tab:crop-man}) with spring
barley.  The Silstrup soil water measurements indicated that we
underestimated ability of the summer 2000 crop to empty the plow layer
soil water (see figure~\ref{fig:Silstrup-theta}).  The spring barley
parametrization did a better job at this than the less tested fodder
beet.  The second served a similar purpose, a hand crafted root
density function that preserved almost the entire root mass in the
plowing layer was added to the two Silstrup crops.  The plow pan was
assumed to be so dense that only a few roots could penetrate through
eart worm channels.  Apart from the soil water measuremenst, both
calibration also served to concentrate the uptake from the zone with
most bromide, in order to explain the low amount of bromide found in
the drain water.  See also section~\ref{sec:cal-bromide}
and~\ref{sec:cal-silstrup-surface}.

\subsubsection{Pesticide and bromide application}

The data for pesticide application consisted of date, amount, and
trade name.  Trade name is is translated to active ingredient using
``Middeldatabasen'' from \textsc{dlbr} Landbrugsinfo
(\url{http://www.landbrugsinfo.dk/}).  Note that Goltix WG application
is given in L/ha, while the active ingredient is specified in g/kg.
To resolve, a density of 1~kg/L is assumed.  For potassium bromide,
bromide content was calculated from molar mass.

The applications are sumarized in table~\ref{tab:man-pest}.
\begin{table}[htbp]
  \caption{Pesticide and bromide application.  Only those active 
    ingredients we track are listed.  }
  \label{tab:man-pest}
  \centering
  \begin{tabular}{c|crl|crl}\hline
    \emph{Silstrup} & Trade name & \multicolumn{2}{c|}{Amount} 
               & Active ingredient & \multicolumn{2}{c}{Active amount} \\
    \hline
    2000-05-22 & Potassium bromide & 30 & kg/ha & Bromide    & 20.14 & kg/ha\\
    2000-05-22 & Goltix WG      & 1   & L/ha & Metamitron    & 700   & g/ha\\
    2000-06-15 & Goltix WG      & 1   & L/ha & Metamitron    & 700   & g/ha\\
    2000-07-12 & Goltix WG      & 1   & L/ha & Metamitron    & 700   & g/ha\\
    2001-06-21 & Tilt Top       & 0.5 & L/ha & Fenpropimorph & 187.5 & g/ha\\
    2001-07-04 & Tilt Top       & 0.5 & L/ha & Fenpropimorph & 187.5 & g/ha\\
    2001-07-16 & Perfektion 500 & 0.6 & L/ha & Dimethoate    & 300   & g/ha\\
    2001-10-25 & Roundup Bio    & 4.0 & L/ha & Glyphosate    & 1440  & g/ha\\
    \hline\hline
    \emph{Estrup} & Trade name & \multicolumn{2}{c|}{Amount} 
               & Active ingredient & \multicolumn{2}{c}{Active amount} \\
    \hline
    2000-05-15 & Potassium bromide & 30 & kg/ha & Bromide & 20.14 & kg/ha\\
    2000-06-15 & Tilt Top       & 0.5 & L/ha & Fenpropimorph & 187.5 & g/ha\\
    2000-06-15 & Perfektion 500 & 0.4 & L/ha & Dimethoate    & 200   & g/ha\\
    2000-07-05 & Tilt Top       & 0.5 & L/ha & Fenpropimorph & 187.5 & g/ha\\
    2000-06-15 & Perfektion 500 & 0.4 & L/ha & Dimethoate    & 200   & g/ha\\
    2000-10-13 & Roundup Bio    & 4.0 & L/ha & Glyphosate    & 1440  & g/ha\\
  \end{tabular}
\end{table}

\subsection{Pesticide and bromide properties}

Of the four pesticides examined, only metamitron and glyphosate were
memasured in concentrations above the detection limit in the examined
dataset.  A single sample at the detection limit were found for
dimethoate, and none were found for fenpropimorph.  No calibration has
therefore been performed on those.  

\subsubsection{Soil sorption and decomposition}

Sorption and decomposition parameters for pesticides are primarily
taken from \citet{ppdb20100517}, with values as shown in
table~\ref{tab:ppdb}.  The database specify a \koc{} value
independently of whether the pesticide is actually sorbed to organic
matter.  We chosed to use a \kd{} values measured in Denmark for the
two main pesticides.  For metamitron, \citet{madsen2000pesticide}
specify \kd{} values together with sil properties for a number of
Danish sites.  Section~\ref{sec:cal-metamitron} describes how we
selected a \kd{} based on those.  For glyphosate, the \kd{} value is
from \citet{gjettermann2009}.  The asorption is not instantaneous, an
adsorption rate of 0.05 h$^-1$ was used as an reasonable initial
guess.  We found no reason to change the value, except for glyphosate,
see section~\ref{sec:cal-glyphosate}.

The effect of depth on decomposition is taken from
\citet{focus2000,focus2002}.  The build in effect of temperature and
humidity for turnover of organic matter in Daisy is also applied for
pesticides.  The default diffusion coefficient used by Daisy for
pesticides of 4.6e-6 cm$^2$/s is used unchanged for all pesticides as
well as for colloids.  For bromide, a value of 2.0e-5 cm$^2$/s is used
instead, reflecting the smaller molecules.  We assume that the
pesticide molecules are all reflected by the roots, so there is no
crop uptake of pesticides.

\begin{table}[htbp]
  \centering
  \caption{Pesticide properties from  \citet{ppdb20100517}. 
    The DT50 value is the decomposition halftime in days.  For both DT50
    and \koc{} we put the value marked  'field' in \citet{ppdb20100517}
    in the center, surrounded by the lower and upper limit found in 
    field studies, as marked in a note in the database. For fenpropimorph
    the \koc{} field value did not fall within the specifed interval.  
    The \kd{} value for glyphosate is from \citet{gjettermann2009}, and
    the \kd{} range for metamitron is from \citet{madsen2000pesticide}.  
    The values used are in \textbf{bold}.}
  \label{tab:ppdb}

  \begin{tabular}{l|ccc}\hline
    Name & DT50 & \koc{} [ml/g] & \kd{} [ml/g] \\\hline
    Dimethoate & 4.6 -- \textbf{7.2} -- 9.8 & 16.25 -- \textbf{30} -- 51.88 & \\
    Fenpropimorph & 8.8 -- \textbf{25.5} -- 50.6 & 2771 -- \textbf{2401} -- 5943 &\\
    Glyphosate & 5 -- \textbf{12} -- 21 & 884 -- 21699 -- 60000  & \textbf{503} \\
    Metamitron & 6.6 -- \textbf{11.1} -- 22.0 & 77.1 -- 80.7 -- 132.5 & 0.14 -- \textbf{4.0}\\
  \end{tabular}
\end{table}

\subsubsection{Surface turnover}

None of our sources had specific information on above ground turnover.
As the glyphosate calibration (see section~\ref{sec:cal-glyphosate})
depend on part of the glyphosate staying in the litter pack for
several days, surface turnover potentially becomes a factor.  The
default value in Daisy of DT50 = 3.5 for pesticides is used.

\subsubsection{Colloids and colloid facilitated transport}

We have no data for colloids, so the parameters for colloid generation
and filtering calibrated for R�rrendeg�rd have been reused for both
sites.  The model itself will adjust to the different clay content.
The pesticides are assumed to be able to sorb to and be transported
with colloids, meaning the colloids will be in competition with the
soil matrix as potential sorbtion sites for the solute form of the
pesticides.  In the lack of any measurements, we have assumed that the
colloids are 10000 more likely to attack to a dissolved soil particle
(a colloid) than to a stationary soil particle (part of the soil
matrix).

\subsubsection{Glyphosate calibration}
\label{sec:cal-glyphosate}

The highest glyphosate contrations in both sites were seen right after
application, or at the first large rain event after application.  This
is unlikely to a function of the pesticide properties, but rather of
the transport routes to the drain.  See~\ref{sec:cal-silstrup-surface}
for how this was calibrated.

The initial simulations showed practically no glyphosate movement once
it entered the soil matrix.  The measurements, however, did show some
late findings of glyphosate in the drain water.  In order to give the
glyphosate a chance to move, we divided the sorption into a weak but
fast and a strong but slow form.  The strong but slow form represent
90\% of the \kd{} value, the weak but fast form the remaining 10\%.
The effect is that the glyphosate is relatively mobile in the
beginning, but becomes less so as more glyphosate becomes sorbed in
the slow form.

\subsubsection{Metamitron calibration}
\label{sec:cal-metamitron}

Adjusting the decompose rate for metamitron has little effect on the
simulation results.  Figure~\ref{fig:Silstrup-C-Metamitron-2000} and
\ref{fig:Silstrup-Metamitron-2000-horizontal} shows why. The
metamitron we find in the drains is the same metamitron that was first
transported vertically to the end of the biopores, and then
horizontally towards the drain.  Since the biopores in the simulation
ends 1.2 meter below surface, this means the metamitron is located
below the 1 meter depth limit for decomposition specified by \focus{}.

In \citet{madsen2000pesticide} sorbtion parameters are measured for a
several Danish sites.  The best correlation for sorption to soil
parameters is for total iron oxide (FeO$_{\mbox{total}}$), the
correlation to organic matter is weak, and no correlation was found to
the easily extracted iron oxide (FeO$_{\mbox{oxalate}}$) which was
measured at Silstrup.  The largest measured \kd{} is 3.1 $\pm$ 0,9
L/kg for Drengsted, and the lowest 0.16 $\pm$ 0.02 L/kg.  We therefore
decided \kd{} should be within the interval 0.14 -- 4.0 L/kg.  A \kd{}
value at the high end of the interval, 4.0 L/kg, gave the best match.

\subsubsection{Bromide calibration}
\label{sec:cal-bromide}

As discussed in section~\ref{sec:crop-man} we wanted the crop to take
up as much bromide at possible, the parameter controling this is
called the crop uptake reclection factor.  Setting it to zero would
give the best results for Silstrup, however on Estrup we had the
oppsite problem, high amounts of bromide was observed in the drain
water, indicating a high value for the crop uptake reclection factor.
It would be possible to justify different values for the two sites, as
there were grown different crops the first year.  However, without any
direct measurements of bromide crop uptake, we found it better to use
a single value.  With a reflection factor of 0.25 we got a good match
for total amount in Silstrup.  In Estrup, it resulted in too little
drain leaching, but still good dynamics.

\subsection{Soil}

The soil setup is based on multiple sources, which will be described
in this section.

\subsubsection{The soil matrix domain}

\paragraph{The primary domain (micropores).}

GEUS had already calibrated the model \macro{}
\citep{jarvis1994simulation,larsbo2003macro} for both sites.  As
\macro{}, like Daisy, solves Richard's Equation, we chose to use the
\macro{} calibration of the hydraulic properties (retention and
conductivity curves) as a basic.  \Macro{} uses a bimodal description
of the hydraulic properties, where the micropore part is identical to
van Genuchten retention curve with Mualem theory for conductivity.
This also happens to be one of the models supported by Daisy.  We made
two changes to the micropore setup.  We increased the hydraulic
conductivity for the plow layer in both sites based on the measurments
depicted on figure~A4.4 and~A4.5 in \citet{vap2005}.  For Silstrup the
boundary hydraulic conductivity ($K_b$) was raised from 0.1 to 1 mm/h
mm/h, and for Estrup from 0.1 to 0.5 mm/h.  However, the low values
used by GEUS are far from unreasonable, as the conductivity of
unprotected soil surface tend to decrease rapidly after heavy rain.
For the Daisy setup, we added a special surface layer with dynamic
hydraulic properties to address this issue (see
section~\ref{sec:surface-plow-pan}).  The other change was the
introduction of 8\% residual water in the B horizon of Silstrup, based
on the relative lack of drying during the summer, as seen on
figure~\ref{fig:Silstrup-theta}.

\paragraph{Soil cracks and anisotropy.}

Unlike \macro{}, Daisy distinguish sharply between macropores small
enough that the cabilary forces are still dominating, and macropores
so large that the cabilary forces are no longer dominating.  On Daisy
terminology, these are called the secondary and tertiary domain
respectively.  The primary domain is the micropores.  Daisy does not
use Richard's Equation for calculating transport in the tertiary
domain.  Richard's Equation is used for both the primary and secondary
domain, and in fact does distinguish between the two.  They are (again
in Daisy terminology) together referred to as the matrix domain.  The
tertiary domain is described in section~\ref{sec:biopores}.

Small soil cracks as those described in \citet{lindhardt2001} are part
of the secondary domain.  A model for how cracks affect the
conductivity based on aperture and density is included in Daisy.  In
\citet{habekost1} an aperture of 50 to 150 $\mu$m is estimated.  In
\citet{jorgensen1998} a value of 78 $\mu$m is used after callibration.
Both sources specify a density of 10 per meter.

In \citet{lindhardt2001} the cracks in the depth interval 75 -- 180 cm
in Siltrup are described as horizontal.  As the model of cracks in
Daisy doesn't include direction, we have decided not to include cracks
in that interval, and instead specify an ansitropy of 100.  This means
the horizontal conductivity is 100 times higher than the vertical,
which is approximately right for for wet soil.  For dry soil this is
wrong, but we don't expact large horizontal gradients in that case
anyway.

For the plowing layer at both sites, we also chose an anistropy of 100
rather than a general modification of the hydraulic conductivity.
They idea behind this is to model the effect increase horizontal
movement due to the surface slope.  The effect is negligible on
Silstrup (figures~\ref{fig:Silstrup-water-2000-horizontal} and
\ref{fig:Silstrup-water-2001-horizontal}) but quite significant on
Estrup (figures~\ref{fig:Estrup-water-2000-horizontal}
and~\ref{fig:Estrup-water-2001-horizontal}).

\citet{lindhardt2001} specify no cracks in Silstrup below 3.5 m,
Biopores and cracks.  For Estrup, we get the best suimulation result
by ignoring all cracks below the plow pan, thus assuming that the
cracks found in \citet{lindhardt2001} are not hydraulically connected.
Note that the high saturated conductivity used in the \macro{}
simulation for the A, B, and C1 horizons are still reflected in Daisy
through the biopores.  It is only the horizontal conductivity (as
Daisy biopores are vertical) that is low, which fit well with Estrup
having a high groundwater table despite the drain pipes being close
together.

\paragraph{Figures.}

Figures~\ref{fig:Silstrup-hor} and~\ref{fig:Estrup-hor} show the
original \macro{} parameterization and the modified parameterization
for Daisy.  Only the vertical conductivity is shown, and as the
conductivity in the tertiary domain in Daisy is infinite, that domain
is not included.  For comparison, we have also elected to show the
effect of the parameters estimated from soil texture by the \hypres{}
pedotransfer function.

\begin{figure}[htbp] 
  \includegraphics{fig/Silstrup-Ap-Theta}\includegraphics{fig/Silstrup-Ap-K}\\
  \includegraphics{fig/Silstrup-B-Theta}\includegraphics{fig/Silstrup-B-K}\\
  \includegraphics{fig/Silstrup-C-Theta}\includegraphics{fig/Silstrup-C-K}
  \caption{Silstrup soil hydraulic properties.  MACRO denotes the
    original parametrization, Daisy the modified parametrization
    (ignoring anisotropy and biopores), and HYPRES refers to
    parameters estimated according to \citet{hypres}.}
  \label{fig:Silstrup-hor}
\end{figure}
\begin{figure}[htbp] 
  \includegraphics{fig/Estrup-Ap-Theta}\includegraphics{fig/Estrup-Ap-K}\\
  \includegraphics{fig/Estrup-B-Theta}\includegraphics{fig/Estrup-B-K}\\
  \includegraphics{fig/Estrup-C1-Theta}\includegraphics{fig/Estrup-C1-K}\\
  \includegraphics{fig/Estrup-C2-Theta}\includegraphics{fig/Estrup-C2-K}
  \caption{Estrup soil hydraulic properties.  \Macro{} denotes the
    original parametrization, Daisy the modified parametrization
    (ignoring anisotropy and biopores), and \hypres{} refers to
    parameters estimated according to \citet{hypres}.}
  \label{fig:Estrup-hor}
\end{figure}

\subsubsection{Soil surface and plow pan}
\label{sec:surface-plow-pan}

Danish agricultural soils usually feature both a plow pan, and a
highly variable conductivity in the soil surface.  These can create
layers of near saurated soil, which is needed for activating the
biopores module in Daisy.  Hence, such layers was added to the soil
description.  The plow pan is defined at the top of the B horizon, but
with different hydraulic properties.  The cracks are removed from the
plow pan, and the hydraulic conductivity in the micropores is reduced
to 10\%.  The surface layer constitute the top of the Ap horizon.
Changing the parameters have not been necessary for Estrup.  For
Silstrup, the hydraulic conductuctivity is temporarily decreased to
0.1\% of the original value, see the description in
section~\ref{sec:cal-silstrup-surface}.

\subsubsection{Fast and slow water}

Water movement in the matrix is calculated by Richard's Equation.
However, for pesticide transport the water is later divided into a
slow moving primary domain consiting of the smallers pores, and a fast
moving secondary domain consiting of the larger pores.  If the horizon
has cracks, the secondary domain water will consist of the water in
the cracks.  If not, the secondary domain water will consist of the
water above pF 2.  We have used pF 1.2 as the limit in other
simulations, but since the retention curves in this simulations are
relatively flat near saturarion, that represented very little water.
Pesticed are tracked independent in the two domains, with an exchange
factor ($\alpha$) at its default value of 0.01 h$^-1$. 

No calibration has been done on these parameters.

\subsubsection{Biopores}
\label{sec:biopores}

The Daisy, once the soil is near saturation, biopores will become
activated, and extract water from the matrix down to -30 cm pressure,
at which point the biupores will deactivate.  The capability of the
biopores to extract water is further limited by the storage capacity
of the biopores themself, and the ability to pass the water back to
the matrix in a deeper layer.  

The biopores a divided into a number of user specified classes, each
defined by density, diameter, where they start and end (including
ending directly in drain).  \citet{lindhardt2001} contain some
information about biopores, but not enough for use by Daisy.  We have
therefore chosen to use a biopore setup based on data measured at
R�rrende specifcally for use by Daisy.

One calibration has been performed, the original setup for R�rrende
had all biopores near the drain end in the drain.  In order to get
more of tail in leaching numbers, half the biopores near the drain now
ends in the soil matrix.

\subsubsection{Groundwater table and drain pipes}

Drain depth and ditance are taken from \citet{lindhardt2001}, and can
be used directly by Daisy.  Automatic meaurements of groundwater
pressure table near the bottom of the part of the soil we have
described in Daisy are being used as a lower boundary condition, just
like net precipitation is the upper boundary condition.  A constant
offset has been added to the measured numbers in order to get the
drain flow right.  The offset has been varying depending on the soil
description (between -40 and 30 cm), for the final setup it ended up
being -4 cm for Silstrup and -5 cm for Estrup.

The simulated groundwater table is hard to define, given that the
model is two dimensional.  We have chosen to show two values, a low
value based on the pressure in the lowest located unsaturated numeric
cell (usually near the drain), and high value based on the pressure in
the highest located saturated numeric cell (usually in the center
between drains).  Measured and simulated groundwater table can be seem
on figure~\ref{fig:gw}.  The frequent zeros for the high value at
Silstrup coresponds to ponding, where the top of the soil will become
saturated.
\begin{figure}[htbp]
  \begin{center}
    \includegraphics[trim=0mm 5mm 0mm 0mm,clip]{fig/Silstrup-gw}
    \includegraphics{fig/Estrup-gw}
  \end{center}
  \caption{Groundwater table at Silstrup (top) and Estrup (bottom).
    Automatic daily measurements at Silstrup are from P3.  Manual
    montly measurement at Estrup until 2000-09-19 are from P3,
    automatic daily measurements from 2000-09-22 are from P1.
    Simulated low value is calculated from pressure in lowest
    unsatured numeric cell, typically located near drain.  Simulated
    high value is calculated from pressure in highest saturated cell,
    typically farthest from the drain.}
  \label{fig:gw}
\end{figure}

\subsubsection{Organic matter and nitrogen}

Inorganic nitrogen has been disabled, in order to save simulation
time.  Initially the organic matter turnover was also disabled.
However, since bioincorporaion of litter into the soil is part of that
module, we had to reenable it as the litter layer appeared to be significant.
No calibration has been done apart from the bioincorporation speed, as
described in section~\ref{sec:cal-silstrup-surface}.

\subsection{Silstrup surface}
\label{sec:cal-silstrup-surface}

Glyphosate

Litter

Bromide

\FloatBarrier
\section{Results}

% \section*{Daisy simulation figures for Silstrup og Estrup}

\subsection*{Silstrup}

\begin{figure}[htbp]
  \begin{center}
    \includegraphics{fig/Silstrup-gw}
  \end{center}
  \caption{Silstrup groundwater table.  Automatic daily measurements
    are from P3.  Simulated low value is calculated from pressure in
    lowest unsatured numeric cell, typically located near drain.
    Simulated high value is calculated from pressure in highest
    saturated cell, typically farthest from the drain.}
  \label{fig:Silstrup-gw}
\end{figure}\FloatBarrier

\begin{figure}[htbp]
  \begin{center}
    \includegraphics[trim=0mm 5mm 0mm 0mm,clip]{fig/Silstrup-theta-SW025cm}\\
    \includegraphics[trim=0mm 5mm 0mm 0mm,clip]{fig/Silstrup-theta-SW060cm}\\
    \includegraphics{fig/Silstrup-theta-SW110cm}
  \end{center}
  \caption{Silstrup soil water content for measurement point S1.}
  \label{fig:Silstrup-theta}
\end{figure}\FloatBarrier

\begin{figure}[htbp]
  \begin{center}
    \includegraphics[trim=0mm 5mm 0mm 0mm,clip]{fig/Silstrup-sc-bromide}\\
    \includegraphics{fig/Silstrup-Bromide-horizontal}
  \end{center}
  \caption{Silstrup soil bromide content in 1.0 m depth (top) and 3.5
    m depth (bottom).  Sim (avg) is the average simulated
    concentration, Sim (fast) is the simulated concentration in the
    large (fast) pores.  S1 and S2 are suction cup measurements.
    H$n$.$m$ refer to measured values in different sections of
    horizontal filters.}
  \label{fig:Silstrup-bromide}
\end{figure}\FloatBarrier

\begin{figure}[htbp]
  \begin{center}
    \includegraphics{fig/Silstrup-horizontal}
  \end{center}
  \caption{Silstrup pesticide concentration in soil water at 3.5 meters depth.}
  \label{fig:Silstrup-horizontal}
\end{figure}\FloatBarrier

\begin{figure}[htbp]
  \begin{center}
    \includegraphics[trim=0mm 5mm 0mm 0mm,clip]{fig/Silstrup-leak150bromide}\\
    \includegraphics[trim=0mm 5mm 0mm 0mm,clip]{fig/Silstrup-leak150}\\
    \includegraphics{fig/Silstrup-leak150acc}
  \end{center}
  \caption{Silstrup simuleret leaching at 1.5 meter, 30 cm under bioporers.}
  \label{fig:Silstrup-leak150}
\end{figure}\FloatBarrier

\begin{figure}[htbp]
  \begin{center}
    \includegraphics[trim=0mm 5mm 0mm 0mm,clip]{fig/Silstrup-drain}\\
    \includegraphics{fig/Silstrup-drain-acc}
  \end{center}
  \caption{Silstrup drain flow, daily values and accumulated.}
  \label{fig:Silstrup-drain}
\end{figure}\FloatBarrier

\begin{figure}[htbp]
  \begin{center}
    \includegraphics[trim=0mm 5mm 0mm 0mm,clip]{fig/Silstrup-Bromide-weekly}\\
    \includegraphics{fig/Silstrup-Metamitron-weekly}
  \end{center}
  \caption{Weekly drain flow of bromide and metamitron.}
  \label{fig:Silstrup-weekly}
\end{figure}\FloatBarrier

\begin{figure}[htbp]
  \begin{center}
    \includegraphics[trim=0mm 5mm 0mm 0mm,clip]{fig/Silstrup-Dimethoate-weekly}\\
    \includegraphics[trim=0mm 5mm 0mm 0mm,clip]{fig/Silstrup-Fenpropimorph-weekly}\\
    \includegraphics{fig/Silstrup-Glyphosate-weekly}
  \end{center}
  \caption{Weekly drain flow of selected pesticides.}
  \label{fig:Silstrup-weekly2}
\end{figure}\FloatBarrier

\begin{figure}[htbp]
  \begin{center}
    \includegraphics[trim=0mm 5mm 0mm 0mm,clip]{fig/Silstrup-Bromide-acc}\\
    \includegraphics{fig/Silstrup-Metamitron-acc}
  \end{center}
  \caption{Accumulated drain flow of bromide and metamitron.}
  \label{fig:Silstrup-bromide-acc}
\end{figure}\FloatBarrier

\begin{figure}[htbp]
  \begin{center}
    \includegraphics[trim=0mm 5mm 0mm 0mm,clip]{fig/Silstrup-Dimethoate-acc}\\
    \includegraphics[trim=0mm 5mm 0mm 0mm,clip]{fig/Silstrup-Fenpropimorph-acc}\\
    \includegraphics{fig/Silstrup-Glyphosate-acc}
  \end{center}
  \caption{Accumulated drain flow of selected pesticides.}
  \label{fig:Silstrup-acc}
\end{figure}\FloatBarrier

\begin{figure}[htbp]
  \begin{center}
    \includegraphics{fig/Silstrup-colloid}
  \end{center}
  \caption{Colloids in drain water.}
  \label{fig:Silstrup-colloids}
\end{figure}\FloatBarrier

\begin{figure}[htbp]
  \begin{center}
    \includegraphics[trim=0mm 5mm 0mm 0mm,clip]{fig/Silstrup-biopore}\\
    \includegraphics{fig/Silstrup-biopore-acc}\\
  \end{center}
  \caption{Biopore activity in different soil layers.  The layers are
    ponded water, soil surface (top 3 cm), the rest of the plowing layer,
    the plow pan, and the the B horizon below plow pan down to 50 cm.}
  \label{fig:Silstrup-biopore}
\end{figure}\FloatBarrier

\begin{figure}[htbp]
  \begin{center}
    \includegraphics[trim=0mm 5mm 0mm 0mm,clip]{fig/Silstrup-biopore-drain}\\
    \includegraphics{fig/Silstrup-biopore-drain-acc}
  \end{center}
  \caption{Drain contribution through biopores from different soil
    layers.  The layers are ponded water, soil surface (top 3 cm), the
    rest of the plowing layer, the plow pan, and the the B horizon
    below plow pan down to 50 cm.}
  \label{fig:Silstrup-biopore-drain}
\end{figure}\FloatBarrier

\begin{figure}[htbp]
  \begin{center}
    \includegraphics[trim=0mm 5mm 0mm 0mm,clip]{fig/Silstrup-weather-glyphosate}\\
    \includegraphics[trim=0mm 5mm 0mm 0mm,clip]{fig/Silstrup-water-glyphosate}\\
    \includegraphics{fig/Silstrup-first-glyphosate}
  \end{center}
  \caption{Silstrup surface water and glyphosate in the first week
    after application.  Top graph shows fluxes affecting surface
    water.  Middle graph shows water storage on surface, as well as
    the water holding capcaity of the litter pack.  Bottom graph track
    the fate of glyphosate on the surface.}
  \label{fig:Silstrup-weather-glyphosate}
\end{figure}\FloatBarrier



\newcommand{\figsilstrup}[1]{\includegraphics[trim=8mm 0mm 12mm 7mm,clip]{fig/#1}}

\subsection*{Silstrup 2D}

\begin{figure}[htbp]\centering
  \begin{tabular}{ccc}
    \figsilstrup{Silstrup-pF-2000-5} & 
    \figsilstrup{Silstrup-pF-2000-6} & 
    \figsilstrup{Silstrup-pF-2000-7} \\
    \figsilstrup{Silstrup-pF-2000-8} & 
    \figsilstrup{Silstrup-pF-2000-9} & 
    \figsilstrup{Silstrup-pF-2000-10} \\
    \figsilstrup{Silstrup-pF-2000-11} & 
    \figsilstrup{Silstrup-pF-2000-12} & 
    \figsilstrup{Silstrup-pF-2001-1} \\
    \figsilstrup{Silstrup-pF-2001-2} & 
    \figsilstrup{Silstrup-pF-2001-3} & 
    \figsilstrup{Silstrup-pF-2001-4}
  \end{tabular}
  
  \caption{Silstrup soil potential at the end of each month since
    first application of bromide.  The z-axis denotes depth, the
    x-axis distance from drain.  There are tick marks for every meter.
    Blue denotes pF<0, white pF=1, yellow pF=2, orange pF=3, red pF=4,
    and black pF>5.}
\label{fig:Silstrup-pF-2000}
\end{figure}\FloatBarrier

\begin{figure}[htbp]\centering
  \begin{tabular}{ccc}
    \figsilstrup{Silstrup-pF-2001-5} & 
    \figsilstrup{Silstrup-pF-2001-6} & 
    \figsilstrup{Silstrup-pF-2001-7} \\
    \figsilstrup{Silstrup-pF-2001-8} & 
    \figsilstrup{Silstrup-pF-2001-9} & 
    \figsilstrup{Silstrup-pF-2001-10} \\
    \figsilstrup{Silstrup-pF-2001-11} & 
    \figsilstrup{Silstrup-pF-2001-12} & 
    \figsilstrup{Silstrup-pF-2002-1} \\
    \figsilstrup{Silstrup-pF-2002-2} & &
  \end{tabular}
  
  \caption{Silstrup soil potential at the end of each month second year
    after application of bromide.  The z-axis denotes depth, the
    x-axis distance from drain.  There are tick marks for every meter.
    Blue denotes pF<0, white pF=1, yellow pF=2, orange pF=3, red pF=4,
    and black pF>5.}
\label{fig:Silstrup-pF-2001}
\end{figure}\FloatBarrier

\begin{figure}[htbp]
  \centering
  \includegraphics{fig/Silstrup-water-horizontal-2000}
  
  \caption{Silstrup total horizontal water flux between 2000-5-1 and
    2001-5-1.  The flux is shown on the x-axis (positive away from
    drain) as a function of depth shown on the y-axis.  The graph
    labels are the distance from drain in centimeters.}
  \label{fig:Silstrup-water-2000-horizontal}
\end{figure}\FloatBarrier

\begin{figure}[htbp]
  \centering
  \includegraphics{fig/Silstrup-water-2000}
  
  \caption{Silstrup total vertical water flux between 2000-5-1 and
    2001-5-1.  The flux is shown on the y-axis (positive up) as a
    function of distance from drain shown on the y-axis.  The graph
    labels are depths in centimeters above surface.}
  \label{fig:Silstrup-water-2000}
\end{figure}\FloatBarrier

\begin{figure}[htbp]
  \centering
  \includegraphics{fig/Silstrup-water-biopore-2000}
  
  \caption{Silstrup total biopore water flux between 2000-5-1 and
    2001-5-1.  The flux is shown on the y-axis (positive up) as a
    function of distance from drain shown on the y-axis.  The graph
    labels are depths in centimeters above surface.}
  \label{fig:Silstrup-water-biopore-2000}
\end{figure}\FloatBarrier

\begin{figure}[htbp]
  \centering
  \includegraphics{fig/Silstrup-water-horizontal-2001}
  
  \caption{Silstrup total horizontal water flux between 2001-5-1 and
    2002-3-1.  The flux is shown on the x-axis (positive away from
    drain) as a function of depth shown on the y-axis.  The graph
    labels are the distance from drain in centimeters.}
  \label{fig:Silstrup-water-2001-horizontal}
\end{figure}\FloatBarrier

\begin{figure}[htbp]
  \centering
  \includegraphics{fig/Silstrup-water-2001}
  
  \caption{Silstrup total vertical water flux between 2001-5-1 and
    2002-3-1.  The flux is shown on the y-axis (positive up) as a
    function of distance from drain shown on the y-axis.  The graph
    labels are depths in centimeters above surface.}
  \label{fig:Silstrup-water-2001}
\end{figure}\FloatBarrier

\begin{figure}[htbp]
  \centering
  \includegraphics{fig/Silstrup-water-biopore-2001}
  
  \caption{Silstrup total biopore water flux between 2001-5-1 and
    2002-3-1.  The flux is shown on the y-axis (positive up) as a
    function of distance from drain shown on the y-axis.  The graph
    labels are depths in centimeters above surface.}
  \label{fig:Silstrup-water-biopore-2001}
\end{figure}\FloatBarrier

\begin{figure}[htbp]\centering
  \begin{tabular}{ccc}
    \figsilstrup{Silstrup-M-Bromide-2000-5} & 
    \figsilstrup{Silstrup-M-Bromide-2000-6} & 
    \figsilstrup{Silstrup-M-Bromide-2000-7} \\
    \figsilstrup{Silstrup-M-Bromide-2000-8} & 
    \figsilstrup{Silstrup-M-Bromide-2000-9} & 
    \figsilstrup{Silstrup-M-Bromide-2000-10} \\
    \figsilstrup{Silstrup-M-Bromide-2000-11} & 
    \figsilstrup{Silstrup-M-Bromide-2000-12} & 
    \figsilstrup{Silstrup-M-Bromide-2001-1} \\
    \figsilstrup{Silstrup-M-Bromide-2001-2} & 
    \figsilstrup{Silstrup-M-Bromide-2001-3} & 
    \figsilstrup{Silstrup-M-Bromide-2001-4}
  \end{tabular}
  
  \caption{Silstrup bromide soil content at the end of each month
    since first application of bromide.  The z-axis denotes depth, the
    x-axis distance from drain.  There are tick marks for every
    meter. The color scale is white<10 pg/l, yellow=1 ng/l, orange=0.1
    $\mu$g/l, red=10 $\mu$g/l, and black>1 mg/l}
\label{fig:Silstrup-Bromide-2000}
\end{figure}\FloatBarrier

\begin{figure}[htbp]\centering
  \begin{tabular}{ccc}
    \figsilstrup{Silstrup-M-Bromide-2001-5} & 
    \figsilstrup{Silstrup-M-Bromide-2001-6} & 
    \figsilstrup{Silstrup-M-Bromide-2001-7} \\
    \figsilstrup{Silstrup-M-Bromide-2001-8} & 
    \figsilstrup{Silstrup-M-Bromide-2001-9} & 
    \figsilstrup{Silstrup-M-Bromide-2001-10} \\
    \figsilstrup{Silstrup-M-Bromide-2001-11} & 
    \figsilstrup{Silstrup-M-Bromide-2001-12} & 
    \figsilstrup{Silstrup-M-Bromide-2002-1} \\
    \figsilstrup{Silstrup-M-Bromide-2002-2} &  & 
  \end{tabular}
  
  \caption{Silstrup bromide soil content at the end of each month
    second year after application of bromide.  The z-axis denotes
    depth, the x-axis distance from drain.  There are tick marks for
    every meter. The color scale is white<10 pg/l, yellow=1 ng/l,
    orange=0.1 $\mu$g/l, red=10 $\mu$g/l, and black>1 mg/l}
\label{fig:Silstrup-Bromide-2001}
\end{figure}\FloatBarrier

\begin{figure}[htbp]
  \centering
  \includegraphics{fig/Silstrup-Bromide-horizontal-2000}
  
  \caption{Silstrup total horizontal bromide flow between 2000-5-1 and
    2001-5-1.  The flow is shown on the x-axis (positive away from
    drain) as a function of depth shown on the y-axis.  The graph
    labels are the distance from drain in centimeters.}
  \label{fig:Silstrup-Bromide-2000-horizontal}
\end{figure}\FloatBarrier

\begin{figure}[htbp]
  \centering
  \includegraphics{fig/Silstrup-Bromide-2000}
  
  \caption{Silstrup total vertical bromide flow between 2000-5-1 and
    2001-5-1.  The flow is shown on the y-axis (positive up) as a
    function of distance from drain shown on the y-axis.  The graph
    labels are depths in centimeters above surface.}
  \label{fig:Silstrup-Bromide-2000-vertical}
\end{figure}\FloatBarrier

\begin{figure}[htbp]
  \centering
  \includegraphics{fig/Silstrup-Bromide-biopore-2000}
  
  \caption{Silstrup total biopore bromide flow between 2000-5-1 and
    2001-5-1.  The flow is shown on the y-axis (positive up) as a
    function of distance from drain shown on the y-axis.  The graph
    labels are depths in centimeters above surface.}
  \label{fig:Silstrup-Bromide-biopore-2000}
\end{figure}\FloatBarrier

\begin{figure}[htbp]
  \centering
  \includegraphics{fig/Silstrup-Bromide-horizontal-2001}
  
  \caption{Silstrup total horizontal bromide flow between 2001-5-1 and
    2002-3-1.  The flow is shown on the x-axis (positive away from
    drain) as a function of depth shown on the y-axis.  The graph
    labels are the distance from drain in centimeters.}
  \label{fig:Silstrup-Bromide-2001-horizontal}
\end{figure}\FloatBarrier

\begin{figure}[htbp]
  \centering
  \includegraphics{fig/Silstrup-Bromide-2001}
  
  \caption{Silstrup total vertical bromide flow between 2001-5-1 and
    2002-3-1.  The flow is shown on the y-axis (positive up) as a
    function of distance from drain shown on the y-axis.  The graph
    labels are depths in centimeters above surface.}
  \label{fig:Silstrup-Bromide-2001-vertical}
\end{figure}\FloatBarrier

\begin{figure}[htbp]
  \centering
  \includegraphics{fig/Silstrup-Bromide-biopore-2001}
  
  \caption{Silstrup total biopore bromide flow between 2001-5-1 and
    2002-3-1.  The flow is shown on the y-axis (positive up) as a
    function of distance from drain shown on the y-axis.  The graph
    labels are depths in centimeters above surface.}
  \label{fig:Silstrup-Bromide-biopore-2001}
\end{figure}\FloatBarrier

\begin{figure}[htbp]
  \centering
  \includegraphics{fig/Silstrup-Metamitron-horizontal-2000}
  
  \caption{Silstrup total horizontal metamitron flow between 2000-5-1 and
    2001-5-1.  The flow is shown on the x-axis (positive away from
    drain) as a function of depth shown on the y-axis.  The graph
    labels are the distance from drain in centimeters.}
  \label{fig:Silstrup-Metamitron-2000-horizontal}
\end{figure}\FloatBarrier

\begin{figure}[htbp]
  \centering
  \includegraphics{fig/Silstrup-Metamitron-2000}
  
  \caption{Silstrup total vertical metamitron flow between 2000-5-1 and
    2001-5-1.  The flow is shown on the y-axis (positive up) as a
    function of distance from drain shown on the y-axis.  The graph
    labels are depths in centimeters above surface.}
  \label{fig:Silstrup-Metamitron-2000-vertical}
\end{figure}\FloatBarrier

\begin{figure}[htbp]
  \centering
  \includegraphics{fig/Silstrup-Metamitron-biopore-2000}
  
  \caption{Silstrup total biopore metamitron flow between 2000-5-1 and
    2001-5-1.  The flow is shown on the y-axis (positive up) as a
    function of distance from drain shown on the y-axis.  The graph
    labels are depths in centimeters above surface.}
  \label{fig:Silstrup-Metamitron-biopore-2000}
\end{figure}\FloatBarrier

\begin{figure}[htbp]\centering
  \begin{tabular}{ccc}
    \figsilstrup{Silstrup-M-Glyphosate-2001-5} & 
    \figsilstrup{Silstrup-M-Glyphosate-2001-6} & 
    \figsilstrup{Silstrup-M-Glyphosate-2001-7} \\
    \figsilstrup{Silstrup-M-Glyphosate-2001-8} & 
    \figsilstrup{Silstrup-M-Glyphosate-2001-9} & 
    \figsilstrup{Silstrup-M-Glyphosate-2001-10} \\
    \figsilstrup{Silstrup-M-Glyphosate-2001-11} & 
    \figsilstrup{Silstrup-M-Glyphosate-2001-12} & 
    \figsilstrup{Silstrup-M-Glyphosate-2002-1} \\
    \figsilstrup{Silstrup-M-Glyphosate-2002-2} & & 
  \end{tabular}
  
  \caption{Silstrup glyphosate soil content at the end of each month
    since first application of bromide.  The z-axis denotes depth, the
    x-axis distance from drain.  There are tick marks for every
    meter. The color scale is white<10 pg/l, yellow=1 ng/l, orange=0.1
    $\mu$g/l, red=10 $\mu$g/l, and black>1 mg/l}
\label{fig:Silstrup-M-Glyphosate-2001}
\end{figure}\FloatBarrier

\begin{figure}[htbp]\centering
  \begin{tabular}{ccc}
    \figsilstrup{Silstrup-C-Glyphosate-2001-5} & 
    \figsilstrup{Silstrup-C-Glyphosate-2001-6} & 
    \figsilstrup{Silstrup-C-Glyphosate-2001-7} \\
    \figsilstrup{Silstrup-C-Glyphosate-2001-8} & 
    \figsilstrup{Silstrup-C-Glyphosate-2001-9} & 
    \figsilstrup{Silstrup-C-Glyphosate-2001-10} \\
    \figsilstrup{Silstrup-C-Glyphosate-2001-11} & 
    \figsilstrup{Silstrup-C-Glyphosate-2001-12} & 
    \figsilstrup{Silstrup-C-Glyphosate-2002-1} \\
    \figsilstrup{Silstrup-C-Glyphosate-2002-2} &  & 
  \end{tabular}
  
  \caption{Silstrup glyphosate soil water concentration at the end of
    each month since one year after first application of bromide.  The
    z-axis denotes depth, the x-axis distance from drain.  There are
    tick marks for every meter. The color scale is white<10 pg/l,
    yellow=1 ng/l, orange=0.1 $\mu$g/l, red=10 $\mu$g/l, and black>1
    mg/l}
\label{fig:Silstrup-C-Glyphosate-2001}
\end{figure}\FloatBarrier

\begin{figure}[htbp]
  \centering
  \includegraphics{fig/Silstrup-Glyphosate-horizontal-2001}
  
  \caption{Silstrup total horizontal glyphosate flow between 2001-5-1 and
    2002-3-1.  The flow is shown on the x-axis (positive away from
    drain) as a function of depth shown on the y-axis.  The graph
    labels are the distance from drain in centimeters.}
  \label{fig:Silstrup-Glyphosate-2001-horizontal}
\end{figure}\FloatBarrier

\begin{figure}[htbp]
  \centering
  \includegraphics{fig/Silstrup-Glyphosate-2001}
  
  \caption{Silstrup total vertical glyphosate flow between 2001-5-1 and
    2002-3-1.  The flow is shown on the y-axis (positive up) as a
    function of distance from drain shown on the y-axis.  The graph
    labels are depths in centimeters above surface.}
  \label{fig:Silstrup-Glyphosate-2001-vertical}
\end{figure}\FloatBarrier

\begin{figure}[htbp]
  \centering
  \includegraphics{fig/Silstrup-Glyphosate-biopore-2001}
  
  \caption{Silstrup total biopore glyphosate flow between 2001-5-1 and
    2002-3-1.  The flow is shown on the y-axis (positive up) as a
    function of distance from drain shown on the y-axis.  The graph
    labels are depths in centimeters above surface.}
  \label{fig:Silstrup-Glyphosate-biopore-2001}
\end{figure}\FloatBarrier


\subsection*{Estrup}

\begin{figure}[htbp]
  \begin{center}
    \includegraphics{fig/Estrup-theta-SW025cm}
  \end{center}
  \caption{Soilstrup soil water content for measurement point S1.}
  \label{fig:Estrup-theta}
\end{figure}\FloatBarrier

\begin{figure}[htbp]
  \begin{center}
    \includegraphics[trim=0mm 5mm 0mm 0mm,clip]{fig/Estrup-sc-bromide}\\
    \includegraphics{fig/Estrup-Bromide-horizontal}
  \end{center}
  \caption{Estrup soil bromide content in 1.0 m depth (top) and 3.5
    m depth (bottom).  Sim (avg) is the average simulated
    concentration, Sim (fast) is the simulated concentration in the
    large (fast) pores.  S1 and S2 are suction cup measurements.
    H$1$.$m$ refer to measured values in different sections of
    horizontal filters.}
  \label{fig:Estrup-bromide}
\end{figure}\FloatBarrier


\begin{figure}[htbp]
  \begin{center}
    \includegraphics{fig/Estrup-horizontal}
  \end{center}
  \caption{Estrup pesticide concentration in soil water at 3.5 meters
    depth.}
  \label{fig:Estrup-horizontal}
\end{figure}\FloatBarrier

\begin{figure}[htbp]
  \begin{center}
    \includegraphics[trim=0mm 5mm 0mm 0mm,clip]{fig/Estrup-leak150bromide}\\
    \includegraphics[trim=0mm 5mm 0mm 0mm,clip]{fig/Estrup-leak150}\\
    \includegraphics{fig/Estrup-leak150acc}
  \end{center}
  \caption{Estrup simuleret leaching at 1.5 meter, 30 cm under bioporers.}
  \label{fig:Estrup-leak150}
\end{figure}\FloatBarrier

\begin{figure}[htbp]
  \begin{center}
    \includegraphics[trim=0mm 5mm 0mm 0mm,clip]{fig/Estrup-drain}\\
    \includegraphics{fig/Estrup-drain-acc}
  \end{center}
  \caption{Estrup drain flow, daily values and accumulated.}
  \label{fig:Estrup-drain}
\end{figure}\FloatBarrier

\begin{figure}[htbp]
  \begin{center}
    \includegraphics[trim=0mm 5mm 0mm 0mm,clip]{fig/Estrup-Bromide-weekly}\\
    \includegraphics{fig/Estrup-Bromide-acc}
  \end{center}
  \caption{Estrup weekly and accumulated drain flow of bromide.}
  \label{fig:Estrup-bromide-weekly}
\end{figure}\FloatBarrier


\begin{figure}[htbp]
  \begin{center}
    \includegraphics[trim=0mm 5mm 0mm 0mm,clip]{fig/Estrup-Dimethoate-weekly}\\
    \includegraphics[trim=0mm 5mm 0mm 0mm,clip]{fig/Estrup-Fenpropimorph-weekly}\\
    \includegraphics{fig/Estrup-Glyphosate-weekly}\\
  \end{center}
  \caption{Estrup weekly drain flow of selected pesticides.}
  \label{fig:Estrup-weekly}
\end{figure}\FloatBarrier

\begin{figure}[htbp]
  \begin{center}
    \includegraphics[trim=0mm 5mm 0mm 0mm,clip]{fig/Estrup-Dimethoate-acc}\\
    \includegraphics[trim=0mm 5mm 0mm 0mm,clip]{fig/Estrup-Fenpropimorph-acc}\\
    \includegraphics{fig/Estrup-Glyphosate-acc}\\
  \end{center}
  \caption{Estrup accumulated drain flow of selected pesticides.}
  \label{fig:Estrup-acc}
\end{figure}\FloatBarrier

\begin{figure}[htbp]
  \begin{center}
    \includegraphics{fig/Estrup-colloid}
  \end{center}
  \caption{Colloids in drain water.}
  \label{fig:Estrup-colloids}
\end{figure}\FloatBarrier

\begin{figure}[htbp]
  \begin{center}
    \includegraphics[trim=0mm 5mm 0mm 0mm,clip]{fig/Estrup-biopore}\\
    \includegraphics{fig/Estrup-biopore-acc}\\
  \end{center}
  \caption{Biopore activity in different soil layers.  The layers are
    ponded water, soil surface (top 3 cm), the rest of the plowing layer,
    the plow pan, and the the B horizon below plow pan down to 50 cm.}
  \label{fig:Estrup-biopore}
\end{figure}\FloatBarrier

\begin{figure}[htbp]
  \begin{center}
    \includegraphics[trim=0mm 5mm 0mm 0mm,clip]{fig/Estrup-biopore-drain}\\
    \includegraphics{fig/Estrup-biopore-drain-acc}
  \end{center}
  \caption{Drain contribution through biopores from different soil
    layers.  The layers are ponded water, soil surface (top 3 cm), the
    rest of the plowing layer, the plow pan, and the the B horizon
    below plow pan down to 50 cm.}
  \label{fig:Estrup-biopore-drain}
\end{figure}\FloatBarrier


\newcommand{\figestrupl}[1]{\hspace*{-1cm}\includegraphics[trim=12mm 0mm 17mm 9mm,clip]{fig/#1}}
\newcommand{\figestrup}[1]{\includegraphics[trim=12mm 0mm 17mm 9mm,clip]{fig/#1}}

\chapter{Estrup 2D dynamics}

In this appendix the simulated 2D dynamics for water, bromide and
pesticides of the Estrup site is presented.  There are no
measurements to compare with.

\begin{figure}[htbp]\centering
  \begin{tabular}{ccc}
    \figestrupl{Estrup-pF-2000-5} & 
    \figestrup{Estrup-pF-2000-6} & 
    \figestrup{Estrup-pF-2000-7} \\
    \figestrupl{Estrup-pF-2000-8} & 
    \figestrup{Estrup-pF-2000-9} & 
    \figestrup{Estrup-pF-2000-10} \\
    \figestrupl{Estrup-pF-2000-11} & 
    \figestrup{Estrup-pF-2000-12} & 
    \figestrup{Estrup-pF-2001-1} \\
    \figestrupl{Estrup-pF-2001-2} & 
    \figestrup{Estrup-pF-2001-3} & 
    \figestrup{Estrup-pF-2001-4}
  \end{tabular}
  
  \caption{Estrup soil potential at the end of each month since first
    application of bromide.  The z-axis denotes depth, the x-axis
    distance from drain.  There are tick marks for every meter.  Blue
    denotes pF<0, white pF=1, yellow pF=2, orange pF=3, red pF=4, and
    black pF>5.}
\label{fig:Estrup-pF-2000}
\end{figure}\FloatBarrier

\begin{figure}[htbp]\centering
  \begin{tabular}{ccc}
    \figestrupl{Estrup-pF-2001-5} & 
    \figestrup{Estrup-pF-2001-6} & 
    \figestrup{Estrup-pF-2001-7} \\
    \figestrupl{Estrup-pF-2001-8} & 
    \figestrup{Estrup-pF-2001-9} & 
    \figestrup{Estrup-pF-2001-10} \\
    \figestrupl{Estrup-pF-2001-11} & 
    \figestrup{Estrup-pF-2001-12} & 
    \figestrup{Estrup-pF-2002-1} \\
    \figestrupl{Estrup-pF-2002-2} & 
    \figestrup{Estrup-pF-2002-3} & 
    \figestrup{Estrup-pF-2002-4}
  \end{tabular}
  
  \caption{Estrup soil potential at the end of each month second year
    after application of bromide.  The z-axis denotes depth, the
    x-axis distance from drain.  There are tick marks for every meter.
    Blue denotes pF<0, white pF=1, yellow pF=2, orange pF=3, red pF=4,
    and black pF>5.}
\label{fig:Estrup-pF-2001}
\end{figure}\FloatBarrier

\begin{figure}[htbp]
  \centering
  \fig{Estrup-water-horizontal-2000}
  
  \caption{Estrup total horizontal water flux between 2000-5-1 and
    2001-5-1.  The flux is shown on the x-axis (positive away from
    drain) as a function of depth shown on the y-axis.  The graph
    labels are the distance from drain in centimeters.}
  \label{fig:Estrup-water-2000-horizontal}
\end{figure}\FloatBarrier

\begin{figure}[htbp]
  \centering
  \fig{Estrup-water-2000}
  
  \caption{Estrup total vertical water flux between 2000-5-1 and
    2001-5-1.  The flux is shown on the y-axis (positive up) as a
    function of distance from drain shown on the y-axis.  The graph
    labels are depths in centimeters above surface.}
  \label{fig:Estrup-water-2000}
\end{figure}\FloatBarrier

\begin{figure}[htbp]
  \centering
  \fig{Estrup-water-biopore-2000}
  
  \caption{Estrup total biopore water flux between 2000-5-1 and
    2001-5-1.  The flux is shown on the y-axis (positive up) as a
    function of distance from drain shown on the y-axis.  The graph
    labels are depths in centimeters above surface.}
  \label{fig:Estrup-water-biopore-2000}
\end{figure}\FloatBarrier

\begin{figure}[htbp]
  \centering
  \fig{Estrup-water-horizontal-2001}
  
  \caption{Estrup total horizontal water flux between 2001-5-1 and
    2002-5-1.  The flux is shown on the x-axis (positive away from
    drain) as a function of depth shown on the y-axis.  The graph
    labels are the distance from drain in centimeters.}
  \label{fig:Estrup-water-2001-horizontal}
\end{figure}\FloatBarrier

\begin{figure}[htbp]
  \centering
  \fig{Estrup-water-2001}
  
  \caption{Estrup total vertical water flux between 2001-5-1 and
    2002-5-1.  The flux is shown on the y-axis (positive up) as a
    function of distance from drain shown on the y-axis.  The graph
    labels are depths in centimeters above surface.}
  \label{fig:Estrup-water-2001}
\end{figure}\FloatBarrier

\begin{figure}[htbp]
  \centering
  \fig{Estrup-water-biopore-2001}
  
  \caption{Estrup total biopore water flux between 2001-5-1 and
    2002-5-1.  The flux is shown on the y-axis (positive up) as a
    function of distance from drain shown on the y-axis.  The graph
    labels are depths in centimeters above surface.}
  \label{fig:Estrup-water-biopore-2001}
\end{figure}\FloatBarrier

\begin{figure}[htbp]\centering
  \begin{tabular}{ccc}
    \figestrupl{Estrup-M-Bromide-2000-5} & 
    \figestrup{Estrup-M-Bromide-2000-6} & 
    \figestrup{Estrup-M-Bromide-2000-7} \\
    \figestrupl{Estrup-M-Bromide-2000-8} & 
    \figestrup{Estrup-M-Bromide-2000-9} & 
    \figestrup{Estrup-M-Bromide-2000-10} \\
    \figestrupl{Estrup-M-Bromide-2000-11} & 
    \figestrup{Estrup-M-Bromide-2000-12} & 
    \figestrup{Estrup-M-Bromide-2001-1} \\
    \figestrupl{Estrup-M-Bromide-2001-2} & 
    \figestrup{Estrup-M-Bromide-2001-3} & 
    \figestrup{Estrup-M-Bromide-2001-4}
  \end{tabular}
  
  \caption{Estrup bromide soil content at the end of each month since
    first application of bromide.  The z-axis denotes depth, the x-axis distance from drain.  There are tick marks for every
    meter. The color scale is white<10 pg/l, yellow=1 ng/l,
    orange=0.1 $\mu$g/l, red=10 $\mu$g/l, and black>1 mg/l}
\label{fig:Estrup-Bromide-2000}
\end{figure}\FloatBarrier

\begin{figure}[htbp]\centering
  \begin{tabular}{ccc}
    \figestrupl{Estrup-M-Bromide-2001-5} & 
    \figestrup{Estrup-M-Bromide-2001-6} & 
    \figestrup{Estrup-M-Bromide-2001-7} \\
    \figestrupl{Estrup-M-Bromide-2001-8} & 
    \figestrup{Estrup-M-Bromide-2001-9} & 
    \figestrup{Estrup-M-Bromide-2001-10} \\
    \figestrupl{Estrup-M-Bromide-2001-11} & 
    \figestrup{Estrup-M-Bromide-2001-12} & 
    \figestrup{Estrup-M-Bromide-2002-1} \\
    \figestrupl{Estrup-M-Bromide-2002-2} & 
    \figestrup{Estrup-M-Bromide-2002-3} & 
    \figestrup{Estrup-M-Bromide-2002-4}
  \end{tabular}
  
  \caption{Estrup bromide soil content at the end of each month second
    year after application of bromide.  The z-axis denotes depth, the
    x-axis distance from drain.  There are tick marks for every
    meter. The color scale is white<10 pg/l, yellow=1 ng/l, orange=0.1
    $\mu$g/l, red=10 $\mu$g/l, and black>1 mg/l}
\label{fig:Estrup-Bromide-2001}
\end{figure}\FloatBarrier

\begin{figure}[htbp]
  \centering
  \fig{Estrup-Bromide-horizontal-2000}
  
  \caption{Estrup total horizontal bromide flow between 2000-5-1 and
    2001-5-1.  The flow is shown on the x-axis (positive away from
    drain) as a function of depth shown on the y-axis.  The graph
    labels are the distance from drain in centimeters.}
  \label{fig:Estrup-Bromide-2000-horizontal}
\end{figure}\FloatBarrier

\begin{figure}[htbp]
  \centering
  \fig{Estrup-Bromide-2000}
  
  \caption{Estrup total vertical bromide flow between 2000-5-1 and
    2001-5-1.  The flow is shown on the y-axis (positive up) as a
    function of distance from drain shown on the y-axis.  The graph
    labels are depths in centimeters above surface.}
  \label{fig:Estrup-Bromide-2000-vertical}
\end{figure}\FloatBarrier

\begin{figure}[htbp]
  \centering
  \fig{Estrup-Bromide-biopore-2000}
  
  \caption{Estrup total biopore bromide flow between 2000-5-1 and
    2001-5-1.  The flow is shown on the y-axis (positive up) as a
    function of distance from drain shown on the y-axis.  The graph
    labels are depths in centimeters above surface.}
  \label{fig:Estrup-Bromide-biopore-2000}
\end{figure}\FloatBarrier

\begin{figure}[htbp]
  \centering
  \fig{Estrup-Bromide-horizontal-2001}
  
  \caption{Estrup total horizontal bromide flow between 2001-5-1 and
    2002-5-1.  The flow is shown on the x-axis (positive away from
    drain) as a function of depth shown on the y-axis.  The graph
    labels are the distance from drain in centimeters.}
  \label{fig:Estrup-Bromide-2001-horizontal}
\end{figure}\FloatBarrier

\begin{figure}[htbp]
  \centering
  \fig{Estrup-Bromide-2001}
  
  \caption{Estrup total vertical bromide flow between 2001-5-1 and
    2002-5-1.  The flow is shown on the y-axis (positive up) as a
    function of distance from drain shown on the y-axis.  The graph
    labels are depths in centimeters above surface.}
  \label{fig:Estrup-Bromide-2001-vertical}
\end{figure}\FloatBarrier

\begin{figure}[htbp]
  \centering
  \fig{Estrup-Bromide-biopore-2001}
  
  \caption{Estrup total biopore bromide flow between 2001-5-1 and
    2002-5-1.  The flow is shown on the y-axis (positive up) as a
    function of distance from drain shown on the y-axis.  The graph
    labels are depths in centimeters above surface.}
  \label{fig:Estrup-Bromide-biopore-2001}
\end{figure}\FloatBarrier

\begin{figure}[htbp]\centering
  \begin{tabular}{ccc}
    \figestrupl{Estrup-M-Glyphosate-2000-5} & 
    \figestrup{Estrup-M-Glyphosate-2000-6} & 
    \figestrup{Estrup-M-Glyphosate-2000-7} \\
    \figestrupl{Estrup-M-Glyphosate-2000-8} & 
    \figestrup{Estrup-M-Glyphosate-2000-9} & 
    \figestrup{Estrup-M-Glyphosate-2000-10} \\
    \figestrupl{Estrup-M-Glyphosate-2000-11} & 
    \figestrup{Estrup-M-Glyphosate-2000-12} & 
    \figestrup{Estrup-M-Glyphosate-2001-1} \\
    \figestrupl{Estrup-M-Glyphosate-2001-2} & 
    \figestrup{Estrup-M-Glyphosate-2001-3} & 
    \figestrup{Estrup-M-Glyphosate-2001-4}
  \end{tabular}
  
  \caption{Estrup glyphosate soil content at the end of each month
    since first application of bromide.  The z-axis denotes depth, the
    x-axis distance from drain.  There are tick marks for every
    meter. The color scale is white<10 pg/l, yellow=1 ng/l, orange=0.1
    $\mu$g/l, red=10 $\mu$g/l, and black>1 mg/l}
\label{fig:Estrup-M-Glyphosate-2000}
\end{figure}\FloatBarrier

\begin{figure}[htbp]\centering
  \begin{tabular}{ccc}
    \figestrupl{Estrup-C-Glyphosate-2000-5} & 
    \figestrup{Estrup-C-Glyphosate-2000-6} & 
    \figestrup{Estrup-C-Glyphosate-2000-7} \\
    \figestrupl{Estrup-C-Glyphosate-2000-8} & 
    \figestrup{Estrup-C-Glyphosate-2000-9} & 
    \figestrup{Estrup-C-Glyphosate-2000-10} \\
    \figestrupl{Estrup-C-Glyphosate-2000-11} & 
    \figestrup{Estrup-C-Glyphosate-2000-12} & 
    \figestrup{Estrup-C-Glyphosate-2001-1} \\
    \figestrupl{Estrup-C-Glyphosate-2001-2} & 
    \figestrup{Estrup-C-Glyphosate-2001-3} & 
    \figestrup{Estrup-C-Glyphosate-2001-4}
  \end{tabular}
  
  \caption{Estrup glyphosate soil water concentration at the end of
    each month since first application of bromide.  The z-axis denotes
    depth, the x-axis distance from drain.  There are tick marks for
    every meter. The color scale is white<10 pg/l, yellow=1 ng/l, orange=0.1
    $\mu$g/l, red=10 $\mu$g/l, and black>1 mg/l}
\label{fig:Estrup-C-Glyphosate-2000}
\end{figure}\FloatBarrier

\begin{figure}[htbp]
  \centering
  \fig{Estrup-Glyphosate-horizontal-2000}
  
  \caption{Estrup total horizontal glyphosate flow between 2000-5-1 and
    2001-5-1.  The flow is shown on the x-axis (positive away from
    drain) as a function of depth shown on the y-axis.  The graph
    labels are the distance from drain in centimeters.}
  \label{fig:Estrup-Glyphosate-2000-horizontal}
\end{figure}\FloatBarrier

\begin{figure}[htbp]
  \centering
  \fig{Estrup-Glyphosate-2000}
  
  \caption{Estrup total vertical glyphosate flow between 2000-5-1 and
    2001-5-1.  The flow is shown on the y-axis (positive up) as a
    function of distance from drain shown on the y-axis.  The graph
    labels are depths in centimeters above surface.}
  \label{fig:Estrup-Glyphosate-2000}
\end{figure}\FloatBarrier

\begin{figure}[htbp]
  \centering
  \fig{Estrup-Glyphosate-biopore-2000}
  
  \caption{Estrup total biopore glyphosate flow between 2000-5-1 and
    2001-5-1.  The flow is shown on the y-axis (positive up) as a
    function of distance from drain shown on the y-axis.  The graph
    labels are depths in centimeters above surface.}
  \label{fig:Estrup-Glyphosate-biopore-2000}
\end{figure}\FloatBarrier



\section{Discussion}

\begin{verbatim}
* Silstrup drain period may be too short due to heterogenity.

* Silstrup drain flow follow first rain, indicate surface flow and anisotropy

* Silstrup Metamitron not p{\aa}virket af halverinsgtid
\end{verbatim}

\section{Conclusion and further work}

It is possible to explain the measured data based on the processes
included in the present model, with some caveats
\begin{itemize}
\item The high degree of heterogeneity found in the Estrup site would
  require a detailed 3D model of the entire area to model
  mechanistically, the current 2D model setup can at best be viewed as
  ``effective parameters''.
\item The drain are activated more abruptly (first year) and sooner
  (second year) in reality, than what we have been able to simulate
  for the Silstrup site.  This is particularly noticeable for the
  Bromide measurements.  
\item Bromide is found in some of the horizontal filters at 3.5 meters
  depth at both sites in the first measurements after application of
  Bromide.  No pesticides are found in those depth though.  It does
  indicate a transport way for non-sorbing solutes that we cannot
  model.  One possibility is large scale fractures, this suggestion is
  supported by other work at GEUS.
\end{itemize}

Furthermore, the work presented in thus report cannot count as a
validation of either the parametrization or the conceptual model,
there are too many unknowns where we have had to guess or calibrate
parameters, and there are likely many different setups that would have
resulted in as good or better fits to the measured results.  It is not
certain that a proper validation is possible, but by applying the
model on more data sets we should be able to gain confidence in it it.

Apart from applying the model on more and larger data sets, two areas
in particular need further work.
\begin{itemize}
\item The surface processes (flow, litter storage, decomposition) are
  very important, especially for the Silstrup site, but only the
  minimal work on those to get an effect have been done in this project.
\item The pesticide processes (colloid transport, sorbtion sites,
  sorption kinetics, and decomposition) are parametrized from
  literature values and ``best guesses'', many of those parameters
  should be adapted to direct local measurements.
\end{itemize}

\addcontentsline{toc}{section}{\numberline{}References}
\bibliography{../../txt/daisy}

\end{document}
