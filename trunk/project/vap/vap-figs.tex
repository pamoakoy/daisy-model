\documentclass[a4paper]{report}

%%\usepackage[left=1cm,top=2cm,right=1cm]{geometry}
\usepackage[top=3cm,bottom=2cm]{geometry}
\usepackage[latin1]{inputenc}
\usepackage[T1]{fontenc}
\usepackage[danish,english]{babel}
\usepackage{natbib}
\bibliographystyle{apalike}
\usepackage{graphicx}
\usepackage{hyperref}
\usepackage{fancyhdr}
\usepackage{placeins}
\pagestyle{fancy}
\lhead{\today}

\newcommand{\koc}{$\mbox{K}_{\mbox{\textsc{oc}}}$}
\newcommand{\kclay}{$\mbox{K}_{\mbox{clay}}$}
\newcommand{\kd}{$\mbox{K}_{\mbox{d}}$}

\newcommand{\focus}{\textsc{focus}}
\newcommand{\hypres}{\textsc{hypres}}
\newcommand{\macro}{\textsc{macro}}
\newcommand{\Macro}{\textsc{Macro}}
\newcommand{\figl}{\hspace*{-2cm}}
\newcommand{\figright}[1]{\includegraphics{fig/#1}}
\newcommand{\fig}[1]{\figl\figright{#1}}
\newcommand{\figtop}[1]{\figl\includegraphics[trim=0mm 5mm 0mm 0mm,clip]{fig/#1}}

% \includeonly{vap-result}

\begin{document}

\chapter*{Daisy 2D simulation of Silstrup and Estrup}

Part of project
\begin{otherlanguage}{danish}
  \begin{it}
    Flerdimensional modelling of vandstr�mning og stoftransport i de
    �verste 1-2 m af jorden i systemer med markdr�n
  \end{it}
\end{otherlanguage}
for the Danish Environmental Protection Agency.
\vspace{1cm}

\begin{bf}
  \begin{large}
    \noindent
    S�ren Hansen \texttt{$<$sha@life.ku.dk$>$}\\
    Per Abrahamsen \texttt{$<$abraham@dina.kvl.dk$>$}\\
    Carsten Petersen \texttt{$<$cpe@life.ku.dk$>$}\\
    %% Marie Habekost Nielsen \texttt{$<$maha@life.ku.dk$>$}\\
    \\
    \today{}\\
  \end{large}
\end{bf}
\vfill\noindent
University of Copenhagen\\
Department of Basic Sciences and Environment\\
Environmental Chemistry and Physics\\
Thorvaldsensvej 40\\
DK-1871 Frederiksberg C\\
Tel: \texttt{$+$45 353 32300}\\
Fax: \texttt{$+$45 353 32398}

\tableofcontents

\chapter{Introduction}

The Danish Pesticide Leaching Assessment Programme (PLAP) has been
monitoring drain and soil water since 1999 at six (five ongoing)
locations in order to evaluate the leaching risk of pesticides.
Pesticides are found in concentrations above 0.1 $\mu$g/L in drain
water, whereas such concentrations are rarely found in the horizontal
filters 3.5 meter below the surface.

In order to better understand the system, and eventually how the
measurements can be better used for assessing potential risk of
contamination of drinking water, the Daisy agricultural model has been
extended by including support for those processes we assume are
relevant for transport of pesticides from surface to drain pipes.

To test our understanding as embedded in the model, as well as the
applicability of the model to the PLAP sites, two PLAP sites, four
pesticides, and two years of data have been modelled as a pilot
project.  Our hypothesis is here that we can explain the measured data
with the model.

The most significant measured results are from the Estrup site, so
that was chosen as one of the sites to be calibrated.  We wanted to
use the same years for both sites to see the results of similar
climate on two different locations.  Hourly weather data from the site
had to be present, both for the modelling period and for one growth
season before for "`warmup"'.  The initial choice of F{\aa}rdrup as
the second site was rejected, as it was not possible to get sufficient
site-sepcific weather data.  Furthermore, we wanted the same
pesticides on both sites, and both weakly and strongly sorbing
pesticides represented.  Glyphosate had to be one of them.

Based on these criteria, we chose Silstrup and Estrup, drain seasons
2000-2001 and 2001-2002, with the pesticides glyphosate, fenpropimorph,
dimethoate, and metamitron.

Both sites are described in details in \citet{lindhardt2001}. An
overview of the measured data can be found in \citet{vap2009}.  Estrup
is a pedologically rich site, containing both areas with sand and
clay, peat, thin layers of chalk, and even sand fill from railroad
construction.  Silstrup is also heterogeneous, but less so, with high
(for Danish soils) levels of clay dominating the area.  Figure 5.13 of
\citet{lindhardt2001} is illustrative.  Based on two profiles,
Silstrup shows the second largest variation in soil texture, but is
consistently the highest or second highest in clay content among the
PLAP sites.  Estrup, on the other hand, shows even larger variation,
and features both the the highest and lowest clay content among the
four loamy soil sites.  

\chapter{Model setup}
\label{cha:setup}

\section{Weather}

Hourly weather data for Silstrup, Tylstrup and Askov (near Estrup) was
provided by Finn Plauborg from the Faculty of Agricultural Sciences,
Aarhus University.  The idea was that Tylstrup data could be used to
fill in gaps in the Silstrup data.  Both the Silstrup and Askov data
sets contained several short gaps.  We filled those by using values
from the preceding or following hours.  The Silstrup data set ended at
2002-03-12.  The drain season ended 2002-03-20, with 0.6 mm water
collected from the drains the last 8 days.  Bearing this in mind, we
chose to end the simulation 2002-03-11, rather than continue with data
from another station.

The weather data used by Daisy consist of air temperature, wind speed,
relative humidity, precipitation, and global radiation.  Based on
these data, Daisy can use the FAO version of the Penman-Monteith
equation \citep{FAO-PM} to calculate reference evapotranspiration
(ET0).  From that, Daisy will calculate potential evapotranspiration
(ETc) by using the crop leaf area index (LAI) with Beer's law to
divide the surface into a canopy covered fraction and bare soil
fraction, and using different factors for each.  Based partly on
\citet{kjaersgaard2008crop}, a canopy factor of 1.2 and a bare soil
factor of 0.6 was chosen, resulting in a combined factor around 1.15
with full canopy for typical crops.  Actual evapotranspiration (ETa)
is further limited by the capability of the root system and the soil
surface to extract water from the soil.

\begin{figure}[htbp]
  \begin{center}
    \figtop{Silstrup-weather}
    \fig{Estrup-weather}
  \end{center}
  \caption{Accumulated precipitation and hourly values for temperature
    measured at Silstrup (top) and the Askov (bottom) station located
    near Estrup.  Calculated accumulated potential and simulated
    actual evapotranspiration are also shown.}
  \label{fig:weather}
\end{figure}
Precipitation, air temperature, ETc and ETa can be seen on
figure~\ref{fig:weather}.

\section{Management}

All management data were provided by Preben Olsen from the Faculty of
Agricultural Sciences, Aarhus University provided via Annette E.\@
Rosenbom from GEUS.

\subsection{Tillage}

Date, type, and depth were specified for all tillage operations.  All
three were entered into Daisy.  In Daisy, the main effects of tillage
are to incorporate some of the surface material into the soil
(depending on the type of tillage operation), and to mix the content
of the soil to the specified depth.  This is not expected to affect
pesticide leaching much.  In the real world, the main effect of
tillage applicable to pesticide leaching is likely to be a change in
the hydraulic properties for the top soil resulting from the tillage
operation.  Since we chose to implement dynamic hydraulic soil
properties for the Silstrup soil surface (see
section~\ref{sec:surface-plow-pan} and~\ref{sec:cal-crust}), the
tillage information were useful there as well.

\subsection{Fertilization}

Date, type, and amount were specified for each fertilization event, as
well as N, P and K content of fertilizer.  Of these nutrients, Daisy
can normally only handle N, and N has been disabled for these
simulations to save time.  All fertilization event have been added to
Daisy management description, but with N disabled the mineral
fertilizer will not affect the simulation.  The organic fertilizer
will have a minimal effect, the water content of the fertilizer will
be added, and the dry matter content will be added to the litter where
it can catch water and pesticides (see section~\ref{sec:cal-litter}),
until it becomes incorporated into the soil by either tillage
operations or earthworm activity.  However, due to the timing of the
applications, both effects are likely to have negligible effect on
pesticide leaching.

\subsection{Crop management}
\label{sec:crop-man}

Information about date, crop and sowing density were provided.  The
default Daisy crop model does not rely on sowing density, but instead
assumes that ``standard practise'' is used.  Daisy has experimental
support for a crop model that includes sowing density, but given that
the PLAP sites are expected to follow standard management practise, we
found it safer to use the better tested parametrizations of the
default model.

Information about the crow growth was given phenologically (BBCH
stage) and in terms of above ground biomass. No attempt was made to
calibrate the crop in order to match this with the simulated
development stage and biomass.  Information about the two important
parameters for the water balance, namely leaf area and root density,
were not provided.  However, those can often be estimated from the
development stage and dry mass. 

Harvest data included date, stubble height, as well as grain and straw
yields.  Date and stubble height can be directly used by Daisy.  Grain
yield can be used for calibration, however as crop production is not
the focus of this project, we merely noted that both measured and
simulated yields where within the normal range.  The ratio between
grain and straw yield was used for a coarse estimation of the fraction
of the crop left on the field as residuals after harvest.  The
residuals play a crucial role in the simulation for the Silstrup
glyphosate leaching, see section~\ref{sec:cal-litter}.

\begin{table}[htbp]
  \caption{Crop management.}
  \label{tab:crop-man}
  \centering
  \begin{tabular}{c||c|c|c||c|c|c}\hline
    & \multicolumn{3}{c||}{Silstrup} & \multicolumn{3}{c}{Estrup}\\
    Year & Crop          & Sow   & Harvest & Crop          & Sow   & Harvest\\
\hline
    2000 & Fodder Beet   & 4/5   & 15/11   & Spring Barley & 12/4  & 28/8 \\
    2001 & Spring Barley & 9/5   & 5/9     & Peas          & 5/2   & 22/8 \\
    2001 &               &       &         & Winter Wheat  & 19/10 & --- \\
  \end{tabular}
\end{table}

Two calibrations were done on crop management.  The first was to
replace the fodder beet (see table~\ref{tab:crop-man}) with spring
barley.  The Silstrup soil water measurements indicated that we
underestimated the ability of the summer 2000 crop to extract water
from the plow layer (see figure~\ref{fig:Silstrup-theta}).  The spring
barley parametrization did a better job than the less tested fodder
beet.  The second calibration served a similar purpose, an ad hoc root
density function that preserved almost the entire root mass in the
plow layer, was added to the two Silstrup crops.  The plow pan was
assumed to be so dense that only a few roots could penetrate through
earthworm channels.  This also matches the lack of seasonal variation
found with the 60 cm TDR probe (figure~\ref{fig:Silstrup-theta}).
Apart from the soil water measurements, both calibrations also served
to concentrate the uptake from the zone with most bromide, in order to
explain the low amount of bromide found in the drain water.  See also
section~\ref{sec:cal-bromide} and~\ref{sec:cal-silstrup-surface}.

\subsection{Pesticide and bromide application}

The data for pesticide application consisted of date, amount, and
trade name.  Trade name was translated to active ingredient using
``Middeldatabasen'' from \textsc{dlbr} Landbrugsinfo
(\url{http://www.landbrugsinfo.dk/}).  For potassium bromide, bromide
content was calculated from molar mass.

The applications are summarized in table~\ref{tab:man-pest}.
\begin{table}[htbp]
  \caption{Pesticide and bromide application.  Only those active 
    ingredients we track are listed.  }
  \label{tab:man-pest}
  \centering
  \begin{tabular}{c|crl|crl}\hline
    \emph{Silstrup} & Trade name & \multicolumn{2}{c|}{Amount} 
               & Active ingredient & \multicolumn{2}{c}{Amount} \\
    \hline
    2000-05-22 & Potassium bromide & 30 & kg/ha & Bromide    & 20.14 & kg/ha\\
    2000-05-22 & Goltix WG      & 1   & kg/ha & Metamitron    & 700   & g/ha\\
    2000-06-15 & Goltix WG      & 1   & kg/ha & Metamitron    & 700   & g/ha\\
    2000-07-12 & Goltix WG      & 1   & kg/ha & Metamitron    & 700   & g/ha\\
    2001-06-21 & Tilt Top       & 0.5 & L/ha & Fenpropimorph & 187.5 & g/ha\\
    2001-07-04 & Tilt Top       & 0.5 & L/ha & Fenpropimorph & 187.5 & g/ha\\
    2001-07-16 & Perfektion 500 & 0.6 & L/ha & Dimethoate    & 300   & g/ha\\
    2001-10-25 & Roundup Bio    & 4.0 & L/ha & Glyphosate    & 1440  & g/ha\\
    \hline\hline
    \emph{Estrup} & Trade name & \multicolumn{2}{c|}{Amount} 
               & Active ingredient & \multicolumn{2}{c}{Amount} \\
    \hline
    2000-05-15 & Potassium bromide & 30 & kg/ha & Bromide & 20.14 & kg/ha\\
    2000-06-15 & Tilt Top       & 0.5 & L/ha & Fenpropimorph & 187.5 & g/ha\\
    2000-06-15 & Perfektion 500 & 0.4 & L/ha & Dimethoate    & 200   & g/ha\\
    2000-07-05 & Tilt Top       & 0.5 & L/ha & Fenpropimorph & 187.5 & g/ha\\
    2000-06-15 & Perfektion 500 & 0.4 & L/ha & Dimethoate    & 200   & g/ha\\
    2000-10-13 & Roundup Bio    & 4.0 & L/ha & Glyphosate    & 1440  & g/ha\\
  \end{tabular}
\end{table}

\section{Pesticide and bromide properties}

Of the four pesticides examined, only metamitron and glyphosate were
measured in concentrations above the detection limit in the examined
data set.  A single sample at the detection limit were found for
dimethoate, and none were found for fenpropimorph.  No calibration has
therefore been performed on those two pesticides.

\subsection{Soil sorption and degradation}

Sorption and degradation parameters for pesticides are primarily taken
from \citet{ppdb20100517}, with values as shown in
table~\ref{tab:ppdb}.  The database specify a \koc{} value
independently of whether the pesticide is actually sorbed to organic
matter.  We chose to use a \kd{} values measured in Denmark for the
two main pesticides.  For metamitron, \citet{madsen2000pesticide}
specify \kd{} values together with soil properties for a number of
Danish sites.  Section~\ref{sec:cal-metamitron} describes how we
selected a \kd{} based on those.  For glyphosate, the \kd{} value is
from \citet{gjettermann2009}.  The adsorption is not instantaneous, an
adsorption rate of 0.05 h$^{-1}$ was used as an reasonable initial
guess for all pesticides.  We found no reason to change the value
during calibration of metamitron, but did for glyphosate as detailed
in section~\ref{sec:cal-glyphosate}.

The effect of depth on degradation is taken from
\citet{focus2000,focus2002}.  The effect of temperature and humidity
for turnover of organic matter in Daisy is also used for pesticides.
The default diffusion coefficient used by Daisy for pesticides of
4.6e-6 cm$^2$/s is used unchanged for all pesticides as well as for
colloids.  A value of 2.0e-5 cm$^2$/s is used instead for the smaller
bromide molecules.  We assume that the pesticide molecules are all
reflected by the roots, so there is no crop uptake of pesticides.

\begin{table}[htbp]
  \centering
  \caption{Pesticide properties from  \citet{ppdb20100517}. 
    The DT50 value is the degradation halftime in days.  For both DT50
    and \koc{} we put the value marked  `field' in \citet{ppdb20100517}
    in the center, surrounded by the lower and upper limit found in 
    field studies, as marked in a note in the database. For fenpropimorph
    the \koc{} field value did not fall within the specified interval.  
    The \kd{} value for glyphosate is from \citet{gjettermann2009}, and
    the \kd{} range for metamitron is from \citet{madsen2000pesticide}.  
    The values used in the simulation are in \textbf{bold}.}
  \label{tab:ppdb}

  \begin{tabular}{l|ccc}\hline
    Name & DT50 [d] & \koc{} [ml/g] & \kd{} [ml/g] \\\hline
    Dimethoate & 4.6 -- \textbf{7.2} -- 9.8 & 16.25 -- \textbf{30} -- 51.88 & \\
    Fenpropimorph & 8.8 -- \textbf{25.5} -- 50.6 & 2771 -- \textbf{2401} -- 5943 &\\
    Glyphosate & 5 -- \textbf{12} -- 21 & 884 -- 21699 -- 60000  & \textbf{503} \\
    Metamitron & 6.6 -- \textbf{11.1} -- 22.0 & 77.1 -- 80.7 -- 132.5 & 0.14 -- \textbf{4.0}\\
  \end{tabular}
\end{table}

\subsection{Surface degradation}

None of our sources had specific information on above ground
degradation.  As the glyphosate calibration (see
section~\ref{sec:cal-glyphosate}) depends on keeping part of the
glyphosate in the litter pack for several days, surface degradation
potentially becomes a factor.  The default value in Daisy of
DT50~=~3.5 for surface degradation of pesticides is used.

\subsection{Colloids and colloid facilitated transport}
\label{sec:coltrans}

We have no data for colloids, so the parameters for colloid generation
and filtering calibrated for R�rrendeg�rd have been reused for both
sites.  The model itself will adjust to the different clay contents.
Simulated colloid leaching is shown in
appendix~\ref{app:col-biopores}.  The pesticides are assumed to be
able to sorb to and be transported with colloids, meaning that the
colloids will be in competition with the soil matrix as potential
sorption sites for the solute form of the pesticides.  This is
difficult to measure directly, and is therefore sometimes used as a
calibration parameter, see e.g.~\citet{sef1000} where a value of 1000
was used.  As a starting point we chose a factor 10 higher than that
(that is, a soil enrichment factor of 10000) to be certain this part
of the model would be tested.  During calibration, we found no reason
to change this initial value.  See also~\citet{mst-agrovand}.

\subsection{Glyphosate calibration}
\label{sec:cal-glyphosate}

The highest glyphosate concentrations in drains were seen at both
sites right after application, or during the first large rain event
after application.  This is unlikely to be a function of the pesticide
properties as such, but rather of the transport pathways to the drain.
See section~\ref{sec:cal-silstrup-surface} for how this was
calibrated.

The initial simulations showed practically no further glyphosate
movement once the glyphosate entered the soil matrix.  The
measurements, however, did show some late findings of glyphosate in
the drain water.  In order to give the glyphosate a chance to move, we
divided the sorption into a weak but fast and a strong but slow form.
The strong but slow form represent 90\% of the \kd{} value, the weak
but fast form the remaining 10\%.  This was a pure calibration
measure, and may not necessarily reflect the chemical properties of
the pesticide.  Two phase kinetics were also observed in
\citet{glyphosate-kinetics}, but at a shorter time-scale.  The effect
is that the glyphosate is relatively mobile in the beginning, but
becomes less so as more glyphosate becomes sorbed in the slow form,
resulting in a better overall match with drain measurements.

\subsection{Metamitron calibration}
\label{sec:cal-metamitron}

Adjusting the degradation rate for metamitron had little effect on the
simulation results.  Figure~\ref{fig:Silstrup-C-Metamitron-2000}
and~\ref{fig:Silstrup-Metamitron-2000-horizontal} shows why. The
metamitron we find in the drains is the same metamitron that was first
transported vertically to the end of the biopores, and then
horizontally towards the drain.  Since the biopores in the simulation
ends 1.2 meter below surface, this means the metamitron is located
below the 1 meter depth limit for degradation specified by \focus{}.

In \citet{madsen2000pesticide} sorption parameters are measured for
several Danish sites.  The best correlation for sorption to soil
parameters is for total iron oxide (FeO$_{\mbox{total}}$), the
correlation to organic matter is weak, and no correlation was found to
the easily extracted iron oxide (FeO$_{\mbox{oxalate}}$) which was
measured at Silstrup.  The largest measured \kd{} is 3.1 $\pm$ 0,9
L/kg at Drengsted, and the lowest 0.16 $\pm$ 0.02 L/kg at Vejen.  We therefore
decided \kd{} should be within the interval 0.14 -- 4.0 L/kg.  A \kd{}
value at the high end of the interval, 4.0 L/kg, gave the best match.

\subsection{Bromide calibration}
\label{sec:cal-bromide}

As discussed in section~\ref{sec:crop-man} we wanted the crop to take
up as much bromide at possible, the parameter controlling this is
called the crop uptake reflection factor.  Setting it to zero would
give the best results for Silstrup, however at Estrup we had the
opposite problem, high amounts of bromide was observed in the drain
water, indicating a high value for the crop uptake reflection factor.
It would be possible to justify different values for the two sites, as
there were grown different crops the first year.  However, without any
direct measurements of bromide crop uptake, we found it better to use
a single value.  With a reflection factor of 0.25 we got a good match
for total amount in Silstrup (figure~\ref{fig:Silstrup-weekly}
and~\ref{fig:Silstrup-bromide-acc}).  In Estrup, this resulted in too
little total drain leaching, but still good leaching dynamics
(figure~\ref{fig:Estrup-bromide-drain}).

\section{Soil}

The soil setup is based on multiple sources, which will be described
in this section.

\subsection{The soil matrix domain}

\subsubsection{The primary domain (micropores)}
\label{sec:cal-primary}

GEUS had already calibrated the model \macro{}
\citep{jarvis1994simulation,larsbo2003macro} for both sites.  The
\macro{} setup was provided by Annette E.\@ Rosenbom from GEUS\@.  As
\macro{}, like Daisy, solves Richard's Equation, we chose to use the
\macro{} calibration of the hydraulic properties (retention and
conductivity curves) as a basis.  \Macro{} uses a bimodal description
of the hydraulic properties, where the micropore part is identical to
van Genuchten retention curve with Mualem theory for conductivity.
This also happens to be one of the models supported by Daisy, so that
part could be used directly.  We made two changes to the micropore
setup: We increased the hydraulic conductivity for the plow layer at
both sites based on the measurements depicted on figure~A4.4 and~A4.5
in \citet{vap2005}.  For Silstrup the boundary hydraulic conductivity
($K_b$) was raised from 0.1 to 1 mm/h, and for Estrup from 0.1 to 0.5
mm/h.  However, the low values used by GEUS are far from unreasonable,
as the conductivity of unprotected soil surface tend to decrease
rapidly after heavy rain.  For the Daisy setup, we added a special
surface layer with dynamic hydraulic properties to address this issue
(see section~\ref{sec:surface-plow-pan}).  The other change was the
introduction of 8\% residual water in the B horizon of Silstrup, based
on the relative lack of drying during the summer, as seen on
figure~\ref{fig:Silstrup-theta}.

\subsubsection{Soil cracks and anisotropy}

Unlike \macro{}, Daisy distinguish sharply between macropores small
enough that the capillary forces are still dominating, and macropores
so large that the capillary forces are no longer a factor.  In Daisy
terminology, these are called the secondary and tertiary domain,
respectively.  The primary domain is the micropores.  The model user
is responsible for specifying both domains, and thus for specifying
for which macropores Daisy should consider the capillary forces
dominating.  Daisy does not use Richard's Equation for calculating
transport in the tertiary domain.  Richard's Equation is used for both
the primary and secondary domain, and in fact does not distinguish
between the two.  They are (again in Daisy terminology) together
referred to as the matrix domain.  The tertiary domain is described in
section~\ref{sec:biopores}.

In the present setup, soil cracks as those described in
\citet{lindhardt2001} have been specified as part of the secondary
domain.  Daisy will use Poiseuille's law for calculating how these
cracks affect the conductivity based on aperture and density.  In
\citet{habekost1} an aperture of 50 to 150 $\mu$m is estimated.  In
\citet{jorgensen1998} a value of 78 $\mu$m is used after calibration.
Both sources specifies a density of 10 per meter.

In \citet{lindhardt2001} the cracks in the depth interval 75 -- 180 cm
in Silstrup are primarily horizontal.  As the secondary domain model
of cracks in Daisy doesn't include direction (they are assumed to be
equally distributed in all directions), we have decided not to use
that model in this interval, and instead specify an anisotropy of 100.
This means the horizontal conductivity is 100 times higher than the
vertical, which fit well with the \macro{} parametrization.  For dry
soil this is wrong, but we don't expect large horizontal hydraulic
gradients in that situation anyway.

For the plow layer at both sites (see table~\ref{tab:profile}), we
also chose an anisotropy of 100 rather than a general modification of
the hydraulic conductivity.  They idea behind this is to model how the
surface slope affect horizontal movement.  The simulation results
shows the effect of this anisotropy is negligible on Silstrup
(figure~\ref{fig:Silstrup-water-horizontal}) but quite significant on
Estrup (figure~\ref{fig:Estrup-water-horizontal}), likely due to
differences in groundwater level.

\citet{lindhardt2001} specifies no cracks at Silstrup below 3.5 m. For
Estrup, the high groundwater level could indicate a poor horizontal
conductivity.  Furthermore, the shape of the bromide drain leaching
curve (figure~\ref{fig:Estrup-bromide-drain}) where the high values
are early also indicate that the drain water are extracted from the
top soil layers.  We therefore assumed that the cracks found in
\citet{lindhardt2001} below 2m are not hydraulically connected, and
thus doesn't influence the conductivity.  Note that the higher
saturated conductivity used in the \macro{} simulation for the B and
C1 horizons are still reflected in Daisy through the biopores.  It is
only the horizontal conductivity (as Daisy biopores are vertical) that
is low.

\subsubsection{Figures and tables}

Figures~\ref{fig:Silstrup-hor} and~\ref{fig:Estrup-hor} show the
original \macro{} parametrization and the modified parametrization for
Daisy.  Only the vertical conductivity is shown, and as the
conductivity in the tertiary domain in Daisy is infinite, that domain
is not included.  For comparison, we have show the effect of the
parameters estimated from soil texture by the \hypres{} pedotransfer
function.  Table~\ref{tab:profile} summarizes the two profiles.

\newcommand{\dd}[2]{#1\hspace{-3mm} &--&\hspace{-3.5mm}#2}
\begin{table}[htbp]
  \caption{Soil profile for the two sites.  Depth is in cm below soil surface.  
    The Note column specifies \emph{Dynamic} conductivity for the soil
    surface layer, \emph{Dense} (low conductivity) for the plow pan, 
    \emph{Anisotropy} for layers with high horizontal hydraulic conductivity,
    and \emph{Cracks} for layers with high near saturated hydraulic 
    conductivity.}
  \label{tab:profile}
  \centering
  \begin{tabular}{rclll||rclll}\hline
    \multicolumn{5}{l||}{\emph{Silstrup}} & \multicolumn{5}{l}{\emph{Estrup}} \\
    \multicolumn{3}{c}{Depth} & Horizon & Note& \multicolumn{3}{c}{Depth} & Horizon & Note \\\hline
    \dd{0}{3}     & Ap (surface) & Dynamic    & \dd{0}{3}     & Ap (surface)    & Dynamic    \\
    \dd{3}{31}    & Ap           & Anisotropy & \dd{3}{27}    & Ap              & Anisotropy \\
    \dd{31}{39}   & B (plow pan) & Dense      & \dd{27}{35}   & B (plow pan)    & Dense \\
    \dd{39}{75}   & B            & Cracks     & \dd{35}{55}   & B               & \\
    \dd{75}{113}  & B            & Anisotropy & \dd{55}{105}  & C1              & \\
    \dd{113}{180} & C            & Anisotropy & \dd{105}{500} & C2              & \\
    \dd{180}{350} & C            & Cracks\\
    \dd{350}{500} & C            & \\
  \end{tabular}
\end{table}
\begin{figure}[htbp] 
  \fig{Silstrup-Ap-Theta}\figright{Silstrup-Ap-K}\\
  \fig{Silstrup-B-Theta}\figright{Silstrup-B-K}\\
  \fig{Silstrup-C-Theta}\figright{Silstrup-C-K}
  \caption{Silstrup soil hydraulic properties.  MACRO denotes the
    original parametrization, Daisy the modified parametrization
    (ignoring anisotropy and biopores), and HYPRES refers to
    parameters estimated according to \citet{hypres}.}
  \label{fig:Silstrup-hor}
\end{figure}
\begin{figure}[htbp] 
  \fig{Estrup-Ap-Theta}\figright{Estrup-Ap-K}\\
  \fig{Estrup-B-Theta}\figright{Estrup-B-K}\\
  \fig{Estrup-C1-Theta}\figright{Estrup-C1-K}\\
  \fig{Estrup-C2-Theta}\figright{Estrup-C2-K}
  \caption{Estrup soil hydraulic properties.  \Macro{} denotes the
    original parametrization, Daisy the modified parametrization
    (ignoring anisotropy and biopores), and \hypres{} refers to
    parameters estimated according to \citet{hypres}.}
  \label{fig:Estrup-hor}
\end{figure}

\subsection{Soil surface and plow pan}
\label{sec:surface-plow-pan}

Danish agricultural soils may feature both a plow pan, and highly
variable conductivity near the soil surface.  These can create layers
of near saturated soil, which is needed for activating the biopores
module in Daisy.  Hence, such layers was added to the soil
description.  The plow pan is defined as the top of the B horizon, but
with different hydraulic properties.  The cracks are removed from the
plow pan, and the hydraulic conductivity in the micropores is reduced
to 10\% \citep{petersen2008spatio}.  The surface layer constitute the
top of the Ap horizon.  Changing the parameters has not been necessary
for Estrup.  For Silstrup, the hydraulic conductivity is temporarily
decreased to 0.1\% of the original value \citep{soilseal}, see the
description in section~\ref{sec:cal-silstrup-surface}.

\subsection{Fast and slow water}
\label{sec:fast-slow}

Water movement in the matrix is calculated by Richard's Equation.
However, for pesticide transport the water is later divided into a
slow moving primary domain consisting of the smaller pores, and a fast
moving secondary domain consisting of the larger pores.  If the
horizon has cracks, the secondary domain water will consist of the
water in the cracks.  If not, the secondary domain water will consist
of the water retained above pF 2.  We have used pF 1.2 as the limit in
other simulations, but since the retention curves in the setup are
relatively flat near saturation, that value represented very little
water.  Pesticides are tracked independent in the two domains, with an
exchange factor ($\alpha$) at its default value of 0.01 h$^-1$.

The initial values were all set as part of the R{\o}rrendeg{\aa}rd
calibration, see \citet{mst-agrovand} for a more detailed discussion.
No further calibration was done on these parameters.

\subsection{Biopores}
\label{sec:biopores}

Biopores are activated once the soil is near saturation, and they
extract water from the matrix down to -30 cm pressure, at which point
the biopores will deactivate
\citep{tofteng2002film,gjettermann2004transport}.  The capability of
the biopores to extract water is further limited by the storage
capacity of the biopores themselves, and the ability to pass the water
back to the soil matrix in a deeper layer.

The biopores a divided into a number of user specified classes, each
defined by density, diameter, where they start and end (including
ending directly in drain).  \citet{lindhardt2001} contain some
information about biopores, but not enough for use by Daisy.  We have
therefore chosen to use a biopore setup based on data measured at
R�rrende specifically for use by
Daisy~\citep{habekost1,habekost2,habekost3}.

One calibration has been applied. The original setup for R�rrende had
all biopores near the drain ended in the drain.  In order to get more
tailing on the simulated leaching curves, half the biopores near the
drain now ends in the soil matrix.  Neither setup is perfect match for
the observations in~\citet{habekost3}, which show that the earth worm
tunnels are generally well connected to the drain pipes, even if they
don't end in the drain pipes.

\subsection{Groundwater table and drain pipes}
\label{sec:gwt}

Depth (1.1 m below ground level) and distance between drain pipes (18
m for Silstrup and 13 m for Estrup)) are taken from
\citet{lindhardt2001}, and can be used directly by Daisy.  Automatic
measurements of groundwater pressure near the bottom of the part of
the soil we have described in Daisy are being used as a lower boundary
condition, just like the net precipitation is used for the upper
boundary condition.  A constant offset has been added to the measured
values in order to get the drain flow right.  The offset has been
varying depending on the soil description during calibration (between
-40 and 30 cm), for the final setup it ended up being -4 cm for
Silstrup and -5 cm for Estrup.  This is less than the spatial
variation shown by the multiple measurement points,
see~\citet{vap2009}.

The simulated groundwater table is not uniquely defined, given that
the model is two dimensional and there can be multiple layers of
saturated soil.  We have chosen to show two values, a low value based
on the pressure in the lowest located unsaturated numeric cell
(usually near the drain), and a high value based on the pressure in
the highest located saturated numeric cell (usually in the center
between drains).  Measured and simulated groundwater table can be seem
on figure~\ref{fig:gw}.  The frequent zeros for the high value at
Silstrup corresponds to ponding.
\begin{figure}[htbp]
  \begin{center}
    \figtop{Silstrup-gw}
    \fig{Estrup-gw}
  \end{center}
  \caption{Groundwater table at Silstrup (top) and Estrup (bottom).
    Automatic daily measurements at Silstrup are from P3.  Manual
    monthly measurement at Estrup until 2000-09-19 are from P3,
    automatic daily measurements from 2000-09-22 are from P1.
    Simulated low value is calculated from pressure in lowest
    unsaturated numeric cell, typically located near drain.  Simulated
    high value is calculated from pressure in highest saturated cell,
    typically farthest away from the drain. See \citet{lindhardt2001}
    for location of P1 and P3.}
  \label{fig:gw}
\end{figure}

\subsection{Organic matter and nitrogen}

Inorganic nitrogen has been disabled in order to save simulation time.
Initially the organic matter turnover was also disabled.  However,
since bioincorporation of litter into the soil is part of that module,
we had to re-enable it as the litter layer appeared to be significant.
No calibration has been done apart from the bioincorporation speed, as
described in section~\ref{sec:cal-silstrup-surface}.

\FloatBarrier
\section{Silstrup surface}
\label{sec:cal-silstrup-surface}

The Silstrup simulations presented two challenges that both were
resolved through calibration of the system surface.  The first
challenge was the measurements of glyphosate in the drain, shown on
figure~\ref{fig:Silstrup-weekly2}, the first week after application of
glyphosate.  The glyphosate is applied 2001-10-25.  The drain
measurements cover the period from the 24'th to 30'th of October.

\subsection{Soil surface crust}
\label{sec:cal-crust}

Figure~\ref{fig:Silstrup-weather-glyphosate} was created to examine
what happened that week.  The two upper graphs concern the water,
which is needed to bring down the glyphosate (shown on the bottom
graph).  All the graphs are from the final simulation.  Precipitation
(top graph) was obviously measured.  Let us start with that.  What we
see is three small precipitation events in the early hours the 27'th,
28'th and 29'th ($<$ 1 mm), followed by a larger event starting at
noon the 29'th.

\begin{figure}[htbp]
  \begin{center}
    \figtop{Silstrup-weather-glyphosate}\\
    \figtop{Silstrup-water-glyphosate}\\
    \fig{Silstrup-first-glyphosate}
  \end{center}
  \caption{Silstrup surface water and glyphosate in the first week
    after application.  Top graph shows fluxes affecting surface
    water.  Middle graph shows water storage on surface, as well as
    the water holding capacity of the litter pack.  Bottom graph track
    the fate of glyphosate on the surface.}
  \label{fig:Silstrup-weather-glyphosate}
\end{figure}

What happened initially was that none of the events would initiate the
biopores, thus no glyphosate in the drains.  As glyphosate \emph{was}
found in the drains, some biopores must have been activated.  The
alternative, that a strongly sorbing pesticide would be able to move
one meter down through the soul matrix in less than a week, was not
considered realistic.

At this point in the simulation, we are more than five month after the
last soil tillage treatment, and nearly two months after harvest.  It
seems likely that the soil surface at this point would have formed a
crust with low hydraulic property.  By calibration, we found that we
generated biopore flow at the large event if we decreased the
hydraulic conductivity of the soil surface to 0.1\%.  As we don't have
a crust formation model implemented, we chose to make this change in
conductivity right after harvest.

\subsection{Litter pack}
\label{sec:cal-litter}

We now got water, but no significant amount of glyphosate, in the
simulated drains.  The explanation for this can also be found in top
graph on figure~\ref{fig:Silstrup-weather-glyphosate}.  The three
first events are too small to activate the biopores, instead the water
would infiltrate through the matrix, bringing with it all the
glyphosate.  Heavy rain and ponding may -- also in Daisy -- release
the glyphosate (possibly colloid bound) from the top soil.  But the
event wasn't that large or violent, so very little glyphosate would be
released that way.  What was needed was a mechanism to protect the
glyphosate on the surface.

The harvest data provided such a mechanism.  The yield was over 7 tons
grains per hectare, and less than 3 ton straws was removed.  This made
it likely that significant amounts of residuals was left on the field.
Furthermore, \citet{gjettermann2009} demonstrated that glyphosate did
not sorb to straws.  Thus, it seemed likely that some of the
glyphosate was kept in the litter pack together with the water from
the small events, and only washed out with the large event.  Using
this mechanism, depending on the size of the litter pack and the
actual precipitation, between 0 and 100\% of the glyphosate might be
stored in the litter pack.  To implement this in Daisy, we needed a
pre-existing and pre-calibrated model, as there were no time for new
model development at this point.

Luckily, this was not hard to find.  Mulching is a known technique to
conserve water, so other people had been interested in the water
dynamics of the litter pack before us.  The model described in
\citet{scopel2004} was a good conceptual fit with Daisy.  In this
model the plant residuals will cover a fraction (calculated by Beer's
law) of the soil based on the amount and type, where they prevent soil
evaporation for the covered area, and catch a corresponding amount of
the precipitation.  The water holding capacity is based on amount and
type of residuals.  In Daisy this was extended to also catch a
fraction of the applied pesticides.  A parametrization based on
millet from \citet{macena2003} was selected.

This left the incorporation of crop residuals from the surface by
earthworms as the remaining calibration parameter.  By decreasing the
maximum speed of incorporation from 0.5 to 0.35 g DM/m$^2$/h we were
able to get a good fit.  As can be seen on the bottom graph of
figure~\ref{fig:Silstrup-weather-glyphosate}, most of the glyphosate
still enters the soil matrix through the three small events, but more
than enough remain to be transported with the biopores at the large
event to match measured data (figure~\ref{fig:Silstrup-weekly2}).

\subsection{Surface water flow}
\label{sec:cal-silstrup-bromide}

The second problem at Silstrup was due to the bromide.  More than one
third of the net precipitation end up in the drains, yet less than 10
percent of the bromide is found there.  Despite our best efforts, we
were not able to make the crop uptake large enough to compensate for
the difference.  The division between fast and slow water we had
inherited from preliminary calibration of Agrovand data
\citep{mst-agrovand} were also inadequate to protect the
bromide.\footnote{This calibration were later changed, unfortunately
  too late to apply on the PLAP data.}  The explanation that gave the
best results was that a significant amount of water fully bypassed the
soil matrix on its way to the drain pipes, thus diluting the drain
water. The bromide was applied 2000-05-22.  The last tillage operation
was 2000-05-03, with no large precipitation events in between (total
precipitation 9.6 mm, highest intensity 1.4 mm/h).  It is therefore
likely that the hydraulic conductivity is still high at that point.
We chose to add crust 2000-06-01 (after 48.4 mm rain, max intensity
4.9 mm/h), setting the hydraulic conductivity down to one percent of
the original.  At this point, the bromide was safely in the soil
matrix

The crust would generate biopore activity, but not necessarily to the
drain (only the biopores at 20 cm to either side of the drain pipes
are assumed to be connected to them).  The biopores not connected to
the drains were not able to take all the water, resulting in ponding
at the rest of the field.  In order to lead some of this water to the
drains, a simple surface water movement was implemented.  When the
ponding is higher than the local detention capacity in any part of the
field, the surplus water is redistributed evenly to the remaining part
of the field.  Using this as a calibration parameter, we found that a
local detention capacity of 2 mm would result in 10\% of the total water
bypassing the drain pipes, and the right amount bromide in the drains
seen over a whole season.

One other observation that points to surface flow possibly being a
real factor is the response time in the drains to precipitation
events.  As the bottom graph on figure~\ref{fig:Silstrup-drain} shows,
the observed drain flow is almost identical the the net precipitation
at the beginning of the drain season.  This suggest a very fast
connection to the drains, which even with the surface flow module we
could not quite match,



\chapter{Results}

In this chapter, dynamic measurements are compared to simulated
results.  We have chosen to present all the measured soil and drain
data we received, even those that for some reason or another have not
been considered in the calibration process.  Data regarding dynamic
crop growth is not presented.  Static data used for the initial setup
(soil physics) and dynamic data used to drive the simulation (weather,
groundwater pressure, and crop management) are presented in
chapter~\ref{cha:setup}.  Daisy will calculate a lot of additional
information, which is useless for validation purposes, but can be
important for interpretation of the results.  We have chosen to put
what we consider the most important of such data (regarding deep
leaching, colloids, biopores and 2D movement) in
appendix~\ref{app:col-biopores} and \ref{app:plot-2d}.  The data
presented in this chapter fall in two broad categories: measurements
of water and solutes within the soil, and measurements of water and
solutes in the drains.  The measurement points referred to throughout
this chapter can be found in \citet{vap2009}.

\section{Soil}

\subsection{TDR measurements}

Plauborg from the Faculty of Agricultural Sciences, Aarhus University,
were responsible for the TDR measurements.  The data was provided by
Annette E.\@ Rosenbom from GEUS.  Soil water content was measured at
both sites using horizontal TDR probes located at the lowest corner of
field.  At Estrup (figure~\ref{fig:Estrup-theta}) we only have data
for 25 cm, at Silstrup (figure~\ref{fig:Estrup-theta}) we have for 25,
60 and 110 cm below soil surface.  The 110 cm probe values show two
distinct curves when plotted as points rather than lines.  The
variation on the 60 cm probe seem to bear little relationship to the
seasons.  The 25 cm probes at both sites are a better match for our
expectations.  The ability of the crop to dry out the soil is larger
than the simulated at both sites.  Also, the simulated high (winter)
level at Silstrup is slightly above the measured high level.

In general, we didn't want to calibrate our soil physics based on
these measurements (e.g.\ by lowering the porosity of the Silstrup Ap
horizon), as the soil physics were based on distributed samples from
the field, and as such more likely to be representative of the field
as a whole, than the TDR measurements.  However, as the bromide
leaching data for Silstrup also lead us to believe that we
underestimated the crop ability to extract water from the top horizon
(containing most of the bromide during the summer), two changes were
made.  The residual water of the B horizon was set to 8\% (up from 0),
and the crop was calibrated so that most of the roots would be
concentrated in the Ap horizon.  See section~\ref{sec:cal-primary} and
section~\ref{sec:crop-man}.

\begin{figure}[htbp]
  \begin{center}
    \figtop{Silstrup-theta-SW025cm}\\
    \figtop{Silstrup-theta-SW060cm}\\
    \fig{Silstrup-theta-SW110cm}
  \end{center}
  \caption{Silstrup soil water content for measurement point S1.}
  \label{fig:Silstrup-theta}
\end{figure}

\begin{figure}[htbp]
  \begin{center}
    \fig{Estrup-theta-SW025cm}
  \end{center}
  \caption{Estrup soil water content for measurement point S1.}
  \label{fig:Estrup-theta}
\end{figure}

\FloatBarrier
\subsection{Suction cups and horizontal filters}

Bromide and pesticide concentration in soil water were measured with
small suction cells one meter below surface, in the same part of the
field as where the TDR's were installed, and 3.5 meter below surface
within large horizontal filters.  The suction cup measurements are
unlikely to be representative for the field as a whole, due to the
large heterogeneity observed.  The horizontal filters, on the other
hand, are placed downstream in the expected general direction of
groundwater flow, and should thus more likely represent the entire
field.

As Daisy keep separate track of solutes in small and large pores (see
section~\ref{sec:fast-slow}), and it is likely that the suction cups
will predominately extract water from the large pores, we have
provided simulation results for concentration in large pores alone, as
well as concentration in total soil water.  Simulated and measured
bromide in both suction cells and filters are shown for Silstrup on
figure~\ref{fig:Silstrup-bromide} and Estrup on
figure~\ref{fig:Estrup-bromide}.  The simulated values for 1 meter are
well within the variation shown by the the suction cups.  The
measurements does hint that the first bromide should arrive earlier
though, especially in Silstrup.  The concentration in the large pores
compared to average does not change this picture.  Variation between
the two is short lived at the time scale of the graphs.  For 3.5
meter, the simulation is still within the general variation, however
the filters clearly show that some bromide find its way to 3.5 meter
very fast (two months after application).

We did not get pesticide measurement data for 3.5 meter depth in time
for this report, but none were above the detection limit anyway.  This
fit well with the simulated results shown on
figure~\ref{fig:pest-horizontal}.

\begin{figure}[htbp]
  \begin{center}
    \figtop{Silstrup-sc-bromide}\\
    \fig{Silstrup-Bromide-horizontal}
  \end{center}
  \caption{Silstrup soil bromide content at 1.0 m depth (top) and 3.5
    m depth (bottom).  Sim (avg) is the average simulated
    concentration, Sim (fast) is the simulated concentration in the
    large (fast) pores.  S1 and S2 are suction cup measurements.
    H$n$.$m$ refer to measured values in different sections of
    horizontal filters.}
  \label{fig:Silstrup-bromide}
\end{figure}

\begin{figure}[htbp]
  \begin{center}
    \figtop{Estrup-sc-bromide}\\
    \fig{Estrup-Bromide-horizontal}
  \end{center}
  \caption{Estrup soil bromide content at 1.0 m depth (top) and 3.5
    m depth (bottom).  Sim (avg) is the average simulated
    concentration, Sim (fast) is the simulated concentration in the
    large (fast) pores.  S1 and S2 are suction cup measurements.
    H$1$.$m$ refer to measured values in different sections of
    horizontal filters.}
  \label{fig:Estrup-bromide}
\end{figure}

\begin{figure}[htbp]
  \begin{center}
    \figtop{Silstrup-horizontal}
    \fig{Estrup-horizontal}
  \end{center}
  \caption{Pesticide concentration in soil water at 3.5 meters depth
    for Silstrup (top) and Estrup (bottom).  The simulated values for
    Estrup are in the order of femtograms per hectare, and not visible
    on a nagogram per hectare scale.}
  \label{fig:pest-horizontal}
\end{figure}

\FloatBarrier
\section{Drain}

Drain water flow was measured continuesly, GEUS provided daily values.
The measurements of bromide and pesticides were done using a mixture
of two sampling methods.  The first is time proportional sampling
where samples are taken at specific time intervals.  The other is flow
proportional sampling, where samples are taken with intervals
proportional to the amount of water flow in the drains.  GEUS has
combined the two into a ``best estimate'' of the total weekly flow,
which is what we have used for calibration.

The water and bromide drain data was provided by Annette E.\@ Rosenbom
from GEUS, with Ruth Grant from DMU, Aarhus University as the
responsible scientist.  The pesticide data was provided by Jeanne
Kj{\ae}r from GEUS.

\subsection{Water}

Calibrating the simulated total drain flow over the two seasons is
``just'' a question of picking the right offset for the measured
ground water pressure (see section~\ref{sec:gwt}).  Getting the length
of the drain seasons right is trickier, and involves calibrating the
soil physics.  Drain flow for Silstrup is shown on
figure~\ref{fig:Silstrup-drain} and for Estrup on
figure~\ref{fig:Estrup-drain}.  For Silstrup the drain season length
is right the first year, but the distribution is more even in the
simulation, compared to the measurements where the flow almost
directly follows the precipitation.  For the second season, the
simulation underestimate water flow at the start of the season, and
compensate by overestimating at the end of the season.  For Estrup we
got an overall good match both seasons, slightly underestimating the
drain flow at the beginning of the first season, while overestimating
the drain flow at the beginning of the second season.

\begin{figure}[htbp]
  \begin{center}
    \figtop{Silstrup-drain}\\
    \fig{Silstrup-drain-acc}
  \end{center}
  \caption{Silstrup drain flow, daily values and accumulated.}
  \label{fig:Silstrup-drain}
\end{figure}

\begin{figure}[htbp]
  \begin{center}
    \figtop{Estrup-drain}\\
    \fig{Estrup-drain-acc}
  \end{center}
  \caption{Estrup drain flow, daily values and accumulated.}
  \label{fig:Estrup-drain}
\end{figure}

\FloatBarrier
\subsection{Bromide and metamitron}

Bromide was a challenge to get right, especially for Silstrup, as
described in section~\ref{sec:cal-silstrup-bromide}.  For Silstrup
(figure~\ref{fig:Silstrup-weekly} and~\ref{fig:Silstrup-bromide-acc})
we get a good match the first year, but the second year the dynamics
are off even if the total amount is right.  The poor second year
dynamics for bromide likely reflects the poor second year dynamics for
water.  For Estrup (figure~\ref{fig:Estrup-bromide-drain}), we
underestimate both the initial leaching the first season, and the
leaching the entire second season.

Metamitron is one of the two pesticides we have interesting data for,
unfortunately only for one site.  By increasing the \kd{} parameter to
the largest value we could defend by literature values (see
section~\ref{sec:cal-metamitron}) we were able to get a good match
with both weekly (figure~\ref{fig:Silstrup-weekly}) and accumulated
(figure~\ref{fig:Silstrup-bromide-acc}) measured values.  The
accumulated values may seem off, but that is only due to two weeks
where the majority of leaching in the simulation occurs, but where the
measured drain water were not analyzed for metamitron.

\begin{figure}[htbp]
  \begin{center}
    \figtop{Silstrup-Bromide-weekly}\\
    \fig{Silstrup-Metamitron-weekly}
  \end{center}
  \caption{Silstrup weekly drain transport of bromide and metamitron.}
  \label{fig:Silstrup-weekly}
\end{figure}

\begin{figure}[htbp]
  \begin{center}
    \figtop{Silstrup-Bromide-acc}\\
    \fig{Silstrup-Metamitron-acc}
  \end{center}
  \caption{Silstrup accumulated drain transport of bromide and metamitron.}
  \label{fig:Silstrup-bromide-acc}
\end{figure}

\begin{figure}[htbp]
  \begin{center}
    \figtop{Estrup-Bromide-weekly}\\
    \fig{Estrup-Bromide-acc}
  \end{center}
  \caption{Estrup weekly and accumulated drain transport of bromide.}
  \label{fig:Estrup-bromide-drain}
\end{figure}

\FloatBarrier
\subsection{Glyphosate,  fenpropimorph, and dimethoate}

The second interesting pesticide is glyphosate, here presented
together with fenpropimorph and dimethoate.  As can be seen on
figure~\ref{fig:Silstrup-acc} and figure~\ref{fig:Estrup-acc} we get
the total amount right for both sites.  The weekly numbers show that
the dynamics is also reasonable for Silstrup
(figure~\ref{fig:Silstrup-weekly2}), but that the simulation
underestimate the later leaching at Estrup
(figure~\ref{fig:Estrup-weekly}).  The early Silstrup simulated
results required a lot of focus on surface processes (see
section~\ref{sec:cal-silstrup-surface}), while the late values are a
result of adjusting the pesticide sorption model (see
section~\ref{sec:cal-glyphosate}).  No (additional) adjustment where
made for Estrup.

There is a single measurement at the detection limit of dimethoate at
Silstrup.  The simulation has three spikes at roughly the same size,
one of them matching the detection.  There are no measurements of
dimethoate above detection limit at Estrup, and none for fenpropimorph
at either site.  The simulation results are in agreement with this, as
the two large spikes simulated at Silstrup both occur before the
measured drain water is analyzed fenpropimorph.

\begin{figure}[htbp]
  \begin{center}
    \figtop{Silstrup-Dimethoate-weekly}\\
    \figtop{Silstrup-Fenpropimorph-weekly}\\
    \fig{Silstrup-Glyphosate-weekly}
  \end{center}
  \caption{Silstrup weekly drain transport of selected pesticides.}
  \label{fig:Silstrup-weekly2}
\end{figure}

\begin{figure}[htbp]
  \begin{center}
    \figtop{Silstrup-Dimethoate-acc}\\
    \figtop{Silstrup-Fenpropimorph-acc}\\
    \fig{Silstrup-Glyphosate-acc}
  \end{center}
  \caption{Silstrup accumulated drain transport of selected pesticides.}
  \label{fig:Silstrup-acc}
\end{figure}

\begin{figure}[htbp]
  \begin{center}
    \figtop{Estrup-Dimethoate-weekly}\\
    \figtop{Estrup-Fenpropimorph-weekly}\\
    \fig{Estrup-Glyphosate-weekly}\\
  \end{center}
  \caption{Estrup weekly drain transport of selected pesticides.}
  \label{fig:Estrup-weekly}
\end{figure}

\begin{figure}[htbp]
  \begin{center}
    \figtop{Estrup-Dimethoate-acc}\\
    \figtop{Estrup-Fenpropimorph-acc}\\
    \fig{Estrup-Glyphosate-acc}\\
  \end{center}
  \caption{Estrup accumulated drain transport of selected pesticides.}
  \label{fig:Estrup-acc}
\end{figure}




\chapter{Discussion}

\section{Comparison between simulated and measured data}

It is possible to explain the measured data based on the processes
included in the present model, with some caveats
\begin{itemize}
\item The high degree of heterogeneity at the Estrup site would
  require a detailed 3D model of the entire area to model
  mechanistically.  The current 2D model setup can at best be viewed
  as ``effective parameters''.
\item The simulated second year drain season for Silstrup is too
  short. This is particularly noticeable for the Bromide measurements.
\item Measurements at both sites show
  (figure~\ref{fig:Silstrup-weekly2} and~\ref{fig:Estrup-weekly}) that
  the initial glyphosate event is followed by a couple of weeks with
  addition drain leaching.  The model show the same, but underestimate
  the size of the later events.  This could be due to easily
  remobilizable glyphosate in proximity of the preferential transport
  system, a process we have not implemented in our model, or for
  Estrup, due to the peat below part of site, which hasn't been
  included in the setup.  As glyphosate doesn't sorb to organic matter
  \citep{gjettermann2009}, any glyphosate that finds it way down to
  the peat through biopores, may potentially slowly move towards the
  drain pipes.  For Estrup, Daisy continues to underestimate the late
  events for the rest of the first drain season.
\item Bromide is found in some of the horizontal filters at 3.5 meters
  depth at both sites in the first measurements after application of
  Bromide.  No pesticides are generally found at this depth though.
  It does indicate a transport way for non-sorbing solutes that we
  cannot currently model.  One possibility is large scale fractures,
  this suggestion is supported by other work at GEUS.
\end{itemize}

Sine we have been developing the model (adding new processes) based on
the measured data, the work presented in this report cannot count as a
model validation.

\section{Deep leaching of pesticides}

Figure~\ref{fig:Silstrup-leak150} and~\ref{fig:Estrup-leak150} show
some deep leaching of pesticides in Silstrup, but apart from a single
event, none for Estrup.  If we look at
figure~\ref{fig:Silstrup-C-Metamitron-2000},~\ref{fig:Silstrup-C-Glyphosate-2001},
and~\ref{fig:Estrup-C-Glyphosate-2000} we see the metamitron moving
downward but being diluted in the process.  The glyphosate is not
visibly moving from beyond the end of the biopores at either site.
The high concentration at the end the biopores is likely mostly a
reflection of a limitation in the model, we have specified all
biopores to end in the same depth, in reality they will end at
different depths.

\section{Process understanding}

Apart from the significance of biopores for pesticide leaching, it is
interesting to note how the two sites are dominated by different
processes.  For Silstrup, surface processes (crust formation, litter,
and overland flow) were dominating the system.  For Estrup, the
majority of the measured leaching can be adequately explained by what
happens in the plow layer.  See also the figures and discussion in
appendix~\ref{app:plot-2d} for further analysis of the simulated
processes.

\section{Localized pesticide parameters}

Due to a communication snafu, we were not aware of the local estimates
of sorption and degradation of some pesticides, documented
in~\citet{vap2003}.  This concerns dimethoate at Estrup, for which we
have no significant measurements, and metamitron at Silstrup, for
which we \emph{do} have significant measurements.  Furthermore,
fenpropimorph has been analyzed at the four remaining PLAP sites.
Both topsoil (0-20 cm) and subsoil (80-100 cm) were analyzed.

A \kd{} value was estimated for both soil depths, but \koc{} only for
the topsoil.  For dimethoate sorption at Estrup, \koc{} was estimated
to 86 mL/g, the value used in Daisy was 30 mL/g.  For metamitron at
Silstrup, \koc{} was estimated to 160 mL/g.  The \kd{} value is 3.5
mL/g in the topsoil, and 0.4 mL/g in the subsoil.  As the organic
content of the subsoil is also 10 \% of the topsoil, using the \koc{}
value seems sensible.  In Daisy we used a \kd{} of 4.0 mL/g.  For
fenpropimorph, the four sites show a span of \koc{} from 1532 mL/g
(Jyndevad) to 7496 mL/g (Sl{\ae}ggerup).  The value used in Daisy is
2401 mL/g.

For dimethoate at Estrup the DT50 value was estimated to be less than 2
days in the top soil, and 74 days in the subsoil.  The value used in
Daisy was 7.2 days in the top soil, which will translate into 24 days
in the subsoil using the \focus{} depth function.  Note that the
\focus{} depth function increase DT50 to infinity (no degradation)
below 1 meter, just under the measured interval of subsoil.
Metamitron decomposition was not analyzed.  For fenpropimorph, DT50
was over 300 for all analyzed subsoils.  For the topsoil, DT50 varied
between 15 and 379.  The value used in Daisy was 25.5 days for the
topsoil, corresponding to 85 days for the subsoil.

We do not believe the difference in parameter values between what was
used by Daisy and what was measured for dimethoate at Estrup are
sufficiently large to signifiantly change the simulated drain
leaching.  However, using the lower measured sorption rate (especially
in the subsoil) for metamitron at Silstrup would likely result in
larger simulated drain leaching, which is not good as we already
overestimate it.  The values used in Daisy for fenpropimorph are
within the span measured at the other sites, except that degradation
apparently descrease faster with depth than asserted by \focus{}.

\section{Further work}

The surface processes (flow, litter storage, degradation) are very
important, especially for the Silstrup site.  This was discovered
late, hence the solutions have been less carefully worked out than we
would desire.  The flow model is nearly non-existing (it just
distribute excess water uniformly on the field), the litter storage
model is based on millet growing in Brazil, and may or may not be the
right choice for spring barley growing in Denmark.  The surface
pesticide degradation parameters were based on an unrelated pesticide
that happened to be in the Daisy pesticide library.

We also need to use localized pesticide parameters like those
available from \citet{vap2003}, as well as get better knowledge of
colloid transport, different sorption sites, and sorption kinetics.
The values for the later are mostly based on a desire to test the
mechanisms in the model, than qualified estimates of the physical and
chemical properties of the system.

\addcontentsline{toc}{chapter}{\numberline{}References}
\bibliography{../../txt/daisy}

\appendix

\chapter{Deep leaching, colloids and biopores}
\label{app:col-biopores}

Figure~\ref{fig:Silstrup-leak150} and~\ref{fig:Estrup-leak150} show
simulated leaching at 150 cm, 30 cm below the end of the biopores at
the two sites.  For bromide, about 20\% of the applied amount is lost
that way.  For Silstrup we see a slow, but steady leaching of
pesticides, in the order of 0.1\% of the applied amount.  For Estrup,
the enly leaching we see is glyphosate, all apparently comming from a
single event.

\begin{figure}[htbp]
  \begin{center}
    \figtop{Silstrup-leak150bromide}\\
    \figtop{Silstrup-leak150}\\
    \fig{Silstrup-leak150acc}
  \end{center}
  \caption{Silstrup simuleret leaching at 1.5 meter, 30 cm under bioporers.}
  \label{fig:Silstrup-leak150}
\end{figure}

\begin{figure}[htbp]
  \begin{center}
    \figtop{Estrup-leak150bromide}\\
    \figtop{Estrup-leak150}\\
    \fig{Estrup-leak150acc}
  \end{center}
  \caption{Estrup simuleret leaching at 1.5 meter, 30 cm under bioporers.}
  \label{fig:Estrup-leak150}
\end{figure}

Colloid simulation is based on R{\o}rrendeg{\aa}rd data, automatically
adjusted for clay content, as discussed in section~\ref{sec:coltrans}.
Figure~\ref{fig:colloids} shows how Sisltrup (with the highest clay
content of the plow layer) has the highest colloid leaching, but the
Estrup number is still higher than what was measured for R�rrendeg�rd
(which has the lowest clay content).

\begin{figure}[htbp]
  \begin{center}
    \figtop{Silstrup-colloid}
    \fig{Estrup-colloid}
  \end{center}
  \caption{Colloids in drain water in Silstrup (top graph) and Estrup
    (bottom).}
  \label{fig:colloids}
\end{figure}

Figure~\ref{fig:Silstrup-biopore} shows all biopore activity at the
top of the Silstrup soil, while
figure~\ref{fig:Silstrup-biopore-drain} shows only the activity in the
biopores directly connected with the drain pipes.  The effect of the
crust added to the simulation 2001-06-01 is clearly visible, instead
of being activated in the plow layer, biopores are now activated on
the surface.  For Estrup, where no crust has been added, events with
biopore activity from the soil surface are rare, and the biopores are
dominated by the plow layer and plow pan.
Figure~\ref{fig:Estrup-biopore} and~\ref{fig:Estrup-biopore-drain}.

\begin{figure}[htbp]
  \begin{center}
    \figtop{Silstrup-biopore}\\
    \fig{Silstrup-biopore-acc}\\
  \end{center}
  \caption{Biopore activity in different soil layers.  The layers are
    ponded water, soil surface (top 3 cm), the rest of the plow layer,
    the plow pan, and the the B horizon below plow pan down to 50 cm.}
  \label{fig:Silstrup-biopore}
\end{figure}

\begin{figure}[htbp]
  \begin{center}
    \figtop{Silstrup-biopore-drain}\\
    \fig{Silstrup-biopore-drain-acc}
  \end{center}
  \caption{Drain contribution through biopores from different soil
    layers.  The layers are ponded water, soil surface (top 3 cm), the
    rest of the plow layer, the plow pan, and the the B horizon
    below plow pan down to 50 cm.}
  \label{fig:Silstrup-biopore-drain}
\end{figure}

\begin{figure}[htbp]
  \begin{center}
    \figtop{Estrup-biopore}\\
    \fig{Estrup-biopore-acc}\\
  \end{center}
  \caption{Biopore activity in different soil layers.  The layers are
    ponded water, soil surface (top 3 cm), the rest of the plow layer,
    the plow pan, and the the B horizon below plow pan down to 50 cm.}
  \label{fig:Estrup-biopore}
\end{figure}

\begin{figure}[htbp]
  \begin{center}
    \figtop{Estrup-biopore-drain}\\
    \fig{Estrup-biopore-drain-acc}
  \end{center}
  \caption{Drain contribution through biopores from different soil
    layers.  The layers are ponded water, soil surface (top 3 cm), the
    rest of the plow layer, the plow pan, and the the B horizon
    below plow pan down to 50 cm.}
  \label{fig:Estrup-biopore-drain}
\end{figure}



\newcommand{\figsilstrupl}[1]{\figl\includegraphics[trim=8mm 0mm 12mm 7mm,clip]{fig/#1}}
\newcommand{\figsilstrup}[1]{\includegraphics[trim=8mm 0mm 12mm 7mm,clip]{fig/#1}}
\newcommand{\fluxtop}[1]{\figl\includegraphics[trim=0mm 10mm 0mm 0mm,clip]{fig/#1}}
\newcommand{\figestrupl}[1]{\hspace*{-1cm}\includegraphics[trim=12mm 0mm 17mm 9mm,clip]{fig/#1}}
\newcommand{\figestrup}[1]{\includegraphics[trim=12mm 0mm 17mm 9mm,clip]{fig/#1}}

\chapter{2D plots}
\label{app:plot-2d}

In this appendix we present simulated 2D plots for water, bromide,
glyphosate, and metamitron.  There are no measurements to compare
with, a major caveat for both the results and discussion.  We use two
kinds of graphs to capture the 2D structure.

The first kind depict static distribution in the soil.  Each graph has
horizontal distance from drain on the x-axis and height above surface
on the y-axis, using the same scale for both axes.  The graph
represents the the computational soil area used in the simulation.
The right side is the center between two drains (9 meter for Silstrup
and 6.5 meter Estrup), and the bottom is 5 meter, where we use the
measured groundwater pressure table as the lower boundary.  The graphs
are color coded, where specific colors represent specific values for
the soil at the end of the month indicated by the graph title.  Each
numeric cell in the computation has a color representing the value
within that cell.  Since cells are rectangular, the graphs appear
blocky.

The second kind of graph depicts horizontal or vertical movement.  For
the graphs depicting horizontal movement, the y-axis specifies height
above surface (negative number) and the x-axis movement away from
drain (usually also negative).  The horizontal movement at different
distances from the drain pipes are shown as separate plots on each
graph.  For the graphs depicting vertical movement, the axes are
swapped.  The individual plots represent different depths.  We use the
same flow units as we used for the original input, so e.g.\ pesticide
transport is given in g/ha.

\FloatBarrier
\section{Water}

\subsection{Distribution}

The Silstrup soil water pressure potential
(figure~\ref{fig:Silstrup-pF-2000} and~\ref{fig:Silstrup-pF-2001})
rarely show any horizontal gradients, in contrast to Estrup
(figure~\ref{fig:Estrup-pF-2000} and~\ref{fig:Estrup-pF-2001}) where
there is a clear horizontal gradient in the drain season.  This
reflects the much higher conductivity of the Silstrup soil, where the
soil down to 3.5 m all have a high saturated horizontal conductivity
due to cracks.  The exception is the plow pan, on top of which we
several times see a build up of water.  The plow pan also acts as a
barrier the other direction, where we at Silstrup (unlike Estrup) see
the plow layer dry out to near wilting point both summers.

\subsection{Flow}

For both sites we see, unsurprisingly, large horizontal flow near the
drain in direction of the drains
(figure~\ref{fig:Silstrup-water-horizontal}
and~\ref{fig:Estrup-water-horizontal}).  For Estrup we also see an
even larger horizontal flow in the plow layer, largest one meter from
the drain.  At Estrup only the plow layer has a good horizontal
conductivity.  For Silstrup, the vertical flow graphs
(figure~\ref{fig:Silstrup-water-2000}
and~\ref{fig:Silstrup-water-2001}) show us that:
\begin{itemize}
\item The deep percolation (the -150 and -200 cm plots, top graph) are
  pretty much unaffected by the position of the drain pipes.
\item The effect of surface flow can be seen on the biopore activity
  (the 0 cm plot, bottom graph).
\item The plow pan contribute relatively little to the total biopore
  activity (-50 cm compared to 0 cm, bottom graph).
\item The area near the drain is far more active than the rest of the
  field for vertical movement, almost exclusively due to the biopores.
\end{itemize}

In contrast, on Estrup (figure~\ref{fig:Estrup-water-2000}
and~\ref{fig:Estrup-water-2001}) the higher groundwater means we get
significant contributions to the drains from below, there is no
significant surface flow or biopore activation on surface, and the
plow pan seems to be an important factor for biopore activation.  The
area above the drain is still much more active than the rest of the
field, and the biopores play a large role in this.

\begin{figure}[htbp]\centering
  \begin{tabular}{ccc}
    \figsilstrupl{Silstrup-pF-2000-5} & 
    \figsilstrup{Silstrup-pF-2000-6} & 
    \figsilstrup{Silstrup-pF-2000-7} \\
    \figsilstrupl{Silstrup-pF-2000-8} & 
    \figsilstrup{Silstrup-pF-2000-9} & 
    \figsilstrup{Silstrup-pF-2000-10} \\
    \figsilstrupl{Silstrup-pF-2000-11} & 
    \figsilstrup{Silstrup-pF-2000-12} & 
    \figsilstrup{Silstrup-pF-2001-1} \\
    \figsilstrupl{Silstrup-pF-2001-2} & 
    \figsilstrup{Silstrup-pF-2001-3} & 
    \figsilstrup{Silstrup-pF-2001-4}
  \end{tabular}
  
  \caption{Silstrup soil water pressure potential at the end of each
    month since first application of bromide.  The y-axis denotes
    depth, the x-axis distance from drain.  There are tick marks for
    every meter.  Blue denotes pF<0, white pF=1, yellow pF=2, orange
    pF=3, red pF=4, and black pF>5.}
\label{fig:Silstrup-pF-2000}
\end{figure}

\begin{figure}[htbp]\centering
  \begin{tabular}{ccc}
    \figsilstrupl{Silstrup-pF-2001-5} & 
    \figsilstrup{Silstrup-pF-2001-6} & 
    \figsilstrup{Silstrup-pF-2001-7} \\
    \figsilstrupl{Silstrup-pF-2001-8} & 
    \figsilstrup{Silstrup-pF-2001-9} & 
    \figsilstrup{Silstrup-pF-2001-10} \\
    \figsilstrupl{Silstrup-pF-2001-11} & 
    \figsilstrup{Silstrup-pF-2001-12} & 
    \figsilstrup{Silstrup-pF-2002-1} \\
    \figsilstrupl{Silstrup-pF-2002-2} & &
  \end{tabular}
  
  \caption{Silstrup soil water pressure potential at the end of each
    month second year after application of bromide.  The y-axis
    denotes depth, the x-axis distance from drain.  There are tick
    marks for every meter.  Blue denotes pF<0, white pF=1, yellow
    pF=2, orange pF=3, red pF=4, and black pF>5.}
\label{fig:Silstrup-pF-2001}
\end{figure}

\begin{figure}[htbp]\centering
  \begin{tabular}{ccc}
    \figestrupl{Estrup-pF-2000-5} & 
    \figestrup{Estrup-pF-2000-6} & 
    \figestrup{Estrup-pF-2000-7} \\
    \figestrupl{Estrup-pF-2000-8} & 
    \figestrup{Estrup-pF-2000-9} & 
    \figestrup{Estrup-pF-2000-10} \\
    \figestrupl{Estrup-pF-2000-11} & 
    \figestrup{Estrup-pF-2000-12} & 
    \figestrup{Estrup-pF-2001-1} \\
    \figestrupl{Estrup-pF-2001-2} & 
    \figestrup{Estrup-pF-2001-3} & 
    \figestrup{Estrup-pF-2001-4}
  \end{tabular}
  
  \caption{Estrup soil water pressure potential at the end of each
    month since first application of bromide.  The y-axis denotes
    depth, the x-axis distance from drain.  There are tick marks for
    every meter.  Blue denotes pF<0, white pF=1, yellow pF=2, orange
    pF=3, red pF=4, and black pF>5.}
\label{fig:Estrup-pF-2000}
\end{figure}

\begin{figure}[htbp]\centering
  \begin{tabular}{ccc}
    \figestrupl{Estrup-pF-2001-5} & 
    \figestrup{Estrup-pF-2001-6} & 
    \figestrup{Estrup-pF-2001-7} \\
    \figestrupl{Estrup-pF-2001-8} & 
    \figestrup{Estrup-pF-2001-9} & 
    \figestrup{Estrup-pF-2001-10} \\
    \figestrupl{Estrup-pF-2001-11} & 
    \figestrup{Estrup-pF-2001-12} & 
    \figestrup{Estrup-pF-2002-1} \\
    \figestrupl{Estrup-pF-2002-2} & 
    \figestrup{Estrup-pF-2002-3} & 
    \figestrup{Estrup-pF-2002-4}
  \end{tabular}
  
  \caption{Estrup soil water pressure potential at the end of each
    month second year after application of bromide.  The y-axis
    denotes depth, the x-axis distance from drain.  There are tick
    marks for every meter.  Blue denotes pF<0, white pF=1, yellow
    pF=2, orange pF=3, red pF=4, and black pF>5.}
\label{fig:Estrup-pF-2001}
\end{figure}

\begin{figure}[htbp]
  \centering
  \figtop{Silstrup-water-horizontal-2000}
  \fig{Silstrup-water-horizontal-2001}
  
  \caption{Silstrup total horizontal water flux between 2000-5-1 and
    2001-5-1 (top) and between 2001-5-1 and 2002-3-1 (bottom).  The
    flux is shown on the x-axis (positive away from drain) as a
    function of depth shown on the y-axis.  The graph labels are the
    distance from drain in centimeters.}
  \label{fig:Silstrup-water-horizontal}
\end{figure}

\begin{figure}[htbp]
  \centering
  \figtop{Estrup-water-horizontal-2000}
  \fig{Estrup-water-horizontal-2001}
  
  \caption{Estrup total horizontal water flux between 2000-5-1 and
    2001-5-1 (top) and between 2001-5-1 and 2002-5-1 (bottom).  The
    flux is shown on the x-axis (positive away from drain) as a
    function of depth shown on the y-axis.  The graph labels are the
    distance from drain in centimeters.}
  \label{fig:Estrup-water-horizontal}
\end{figure}

\begin{figure}[htbp]
  \centering
  \figtop{Silstrup-water-2000}
  \fig{Silstrup-water-biopore-2000}
  
  \caption{Silstrup vertical water flux between 2000-5-1 and
    2001-5-1.  Top graph show total flux, bottom graph only biopores.  The flux is shown on the y-axis (positive up) as a
    function of distance from drain shown on the x-axis.  The graph
    labels are depths in centimeters above surface.}
  \label{fig:Silstrup-water-2000}
\end{figure}

\begin{figure}[htbp]
  \centering
  \figtop{Silstrup-water-2001}
  \fig{Silstrup-water-biopore-2001}
  
  \caption{Silstrup vertical water flux between 2001-5-1 and 2002-3-1.
    Top graph show total flux, bottom graph only biopores.  The flux
    is shown on the y-axis (positive up) as a function of distance
    from drain shown on the x-axis.  The graph labels are depths in
    centimeters above surface.}
  \label{fig:Silstrup-water-2001}
\end{figure}

\begin{figure}[htbp]
  \centering
  \figtop{Estrup-water-2000}
  \fig{Estrup-water-biopore-2000}
  
  \caption{Estrup vertical water flux between 2000-5-1 and 2001-5-1.
    Top graph show total flux, bottom graph only biopores. The flux is
    shown on the y-axis (positive up) as a function of distance from
    drain shown on the x-axis.  The graph labels are depths in
    centimeters above surface.}
  \label{fig:Estrup-water-2000}
\end{figure}

\begin{figure}[htbp]
  \centering
  \figtop{Estrup-water-2001}
  \fig{Estrup-water-biopore-2001}
  
  \caption{Estrup vertical water flux between 2001-5-1 and 2002-5-1.
    Top graph show total flux, bottom graph only biopores. The flux is
    shown on the y-axis (positive up) as a function of distance from
    drain shown on the x-axis.  The graph labels are depths in
    centimeters above surface.}
  \label{fig:Estrup-water-2001}
\end{figure}

\FloatBarrier
\section{Bromide}

\subsection{Distribution}

Like for water, there is hardly any horizontal gradients worth
speaking of for bromide at Silstrup
(figure~\ref{fig:Silstrup-Bromide-2000}
and~\ref{fig:Silstrup-Bromide-2001}).  The bromide is mostly contained
within the plow layer the first summer, but at the end of the drain
season, the bromide is everywhere.  Estrup shows a different pattern
(figure~\ref{fig:Estrup-Bromide-2000}
and~\ref{fig:Estrup-Bromide-2001}).  At the end of summer, most of the
bromide has left the plow layer, and the upward direction of the water
flow below the drain pipes keep that part of the soil relatively clear
of bromide.  In the second year, the horizontal flow of of water in
the plow layer is resulting in the soil above drain pipes also being
cleared of bromide.

\subsection{Transport}

The most interesting thing to note about the horizontal bromide
transport is that the rather small horizontal flow of water depicted
on the top graph of figure~\ref{fig:Silstrup-water-horizontal}
translate into a much more significant transport of bromide shown on
figure~\ref{fig:Silstrup-Bromide-horizontal}.  This indicates that the
horizontal water flow happens early, when the bromide concentration of
the plow layer is still high.  The bottom graph of
figures~\ref{fig:Silstrup-Bromide-horizontal}
and~\ref{fig:Estrup-Bromide-horizontal} both show less horizontal transport
the second year, especially in the plow layer.

Figure~\ref{fig:Silstrup-Bromide-2000-vertical} shows us that all the
bromide enter through the matrix, and only half the bromide leaver the
top 25 cm.  We also see the biopores being activated between -25 and
-50 cm, indicating the plow pan being significant.  The drain pipes
only visibly affect the transport right on top of them (-100 cm),
where most of the transport is through biopores.  The second year
(figure~\ref{fig:Silstrup-Bromide-2001-vertical}) does not show much
transport at all, except right above the pipes like the year before.  For
Estrup, we see a strong matrix transport with right above the pipes, with
some contributions from biopores
(figure~\ref{fig:Estrup-Bromide-2000-vertical}).  The bromide leaching
from the top 25 cm is slightly higher than for Silstrup, and dominated
by matrix transport.  The second year
(figure~\ref{fig:Estrup-Bromide-2001-vertical}) we get contribution to
the drains from both above and below, almost exclusively through
matrix transport.

\begin{figure}[htbp]\centering
  \begin{tabular}{ccc}
    \figsilstrupl{Silstrup-M-Bromide-2000-5} & 
    \figsilstrup{Silstrup-M-Bromide-2000-6} & 
    \figsilstrup{Silstrup-M-Bromide-2000-7} \\
    \figsilstrupl{Silstrup-M-Bromide-2000-8} & 
    \figsilstrup{Silstrup-M-Bromide-2000-9} & 
    \figsilstrup{Silstrup-M-Bromide-2000-10} \\
    \figsilstrupl{Silstrup-M-Bromide-2000-11} & 
    \figsilstrup{Silstrup-M-Bromide-2000-12} & 
    \figsilstrup{Silstrup-M-Bromide-2001-1} \\
    \figsilstrupl{Silstrup-M-Bromide-2001-2} & 
    \figsilstrup{Silstrup-M-Bromide-2001-3} & 
    \figsilstrup{Silstrup-M-Bromide-2001-4}
  \end{tabular}
  
  \caption{Silstrup bromide soil content at the end of each month
    since first application of bromide.  The y-axis denotes depth, the
    x-axis distance from drain.  There are tick marks for every
    meter. The color scale is white<10 pg/l, yellow=1 ng/l, orange=0.1
    $\mu$g/l, red=10 $\mu$g/l, and black>1 mg/l}
\label{fig:Silstrup-Bromide-2000}
\end{figure}

\begin{figure}[htbp]\centering
  \begin{tabular}{ccc}
    \figsilstrupl{Silstrup-M-Bromide-2001-5} & 
    \figsilstrup{Silstrup-M-Bromide-2001-6} & 
    \figsilstrup{Silstrup-M-Bromide-2001-7} \\
    \figsilstrupl{Silstrup-M-Bromide-2001-8} & 
    \figsilstrup{Silstrup-M-Bromide-2001-9} & 
    \figsilstrup{Silstrup-M-Bromide-2001-10} \\
    \figsilstrupl{Silstrup-M-Bromide-2001-11} & 
    \figsilstrup{Silstrup-M-Bromide-2001-12} & 
    \figsilstrup{Silstrup-M-Bromide-2002-1} \\
    \figsilstrupl{Silstrup-M-Bromide-2002-2} &  & 
  \end{tabular}
  
  \caption{Silstrup bromide soil content at the end of each month
    second year after application of bromide.  The y-axis denotes
    depth, the x-axis distance from drain.  There are tick marks for
    every meter. The color scale is white<10 pg/l, yellow=1 ng/l,
    orange=0.1 $\mu$g/l, red=10 $\mu$g/l, and black>1 mg/l}
\label{fig:Silstrup-Bromide-2001}
\end{figure}

\begin{figure}[htbp]\centering
  \begin{tabular}{ccc}
    \figestrupl{Estrup-M-Bromide-2000-5} & 
    \figestrup{Estrup-M-Bromide-2000-6} & 
    \figestrup{Estrup-M-Bromide-2000-7} \\
    \figestrupl{Estrup-M-Bromide-2000-8} & 
    \figestrup{Estrup-M-Bromide-2000-9} & 
    \figestrup{Estrup-M-Bromide-2000-10} \\
    \figestrupl{Estrup-M-Bromide-2000-11} & 
    \figestrup{Estrup-M-Bromide-2000-12} & 
    \figestrup{Estrup-M-Bromide-2001-1} \\
    \figestrupl{Estrup-M-Bromide-2001-2} & 
    \figestrup{Estrup-M-Bromide-2001-3} & 
    \figestrup{Estrup-M-Bromide-2001-4}
  \end{tabular}
  
  \caption{Estrup bromide soil content at the end of each month since
    first application of bromide.  The y-axis denotes depth, the x-axis distance from drain.  There are tick marks for every
    meter. The color scale is white<10 pg/l, yellow=1 ng/l,
    orange=0.1 $\mu$g/l, red=10 $\mu$g/l, and black>1 mg/l}
\label{fig:Estrup-Bromide-2000}
\end{figure}

\begin{figure}[htbp]\centering
  \begin{tabular}{ccc}
    \figestrupl{Estrup-M-Bromide-2001-5} & 
    \figestrup{Estrup-M-Bromide-2001-6} & 
    \figestrup{Estrup-M-Bromide-2001-7} \\
    \figestrupl{Estrup-M-Bromide-2001-8} & 
    \figestrup{Estrup-M-Bromide-2001-9} & 
    \figestrup{Estrup-M-Bromide-2001-10} \\
    \figestrupl{Estrup-M-Bromide-2001-11} & 
    \figestrup{Estrup-M-Bromide-2001-12} & 
    \figestrup{Estrup-M-Bromide-2002-1} \\
    \figestrupl{Estrup-M-Bromide-2002-2} & 
    \figestrup{Estrup-M-Bromide-2002-3} & 
    \figestrup{Estrup-M-Bromide-2002-4}
  \end{tabular}
  
  \caption{Estrup bromide soil content at the end of each month second
    year after application of bromide.  The y-axis denotes depth, the
    x-axis distance from drain.  There are tick marks for every
    meter. The color scale is white<10 pg/l, yellow=1 ng/l, orange=0.1
    $\mu$g/l, red=10 $\mu$g/l, and black>1 mg/l}
\label{fig:Estrup-Bromide-2001}
\end{figure}

\begin{figure}[htbp]
  \centering
  \figtop{Silstrup-Bromide-horizontal-2000}
  \fig{Silstrup-Bromide-horizontal-2001}
  
  \caption{Silstrup total horizontal bromide transport between 2000-5-1 and
    2001-5-1 (top) and between 2001-5-1 and 2002-3-1 (bottom).  The
    transport is shown on the x-axis (positive away from drain) as a
    function of depth shown on the y-axis.  The graph labels are the
    distance from drain in centimeters.}
  \label{fig:Silstrup-Bromide-horizontal}
\end{figure}

\begin{figure}[htbp]
  \centering
  \figtop{Estrup-Bromide-horizontal-2000}
  \fig{Estrup-Bromide-horizontal-2001}
  
  \caption{Estrup total horizontal bromide transport between 2000-5-1 and
    2001-5-1 (top) and between 2001-5-1 and 2002-5-1 (bottom).  The
    transport is shown on the x-axis (positive away from drain) as a
    function of depth shown on the y-axis.  The graph labels are the
    distance from drain in centimeters.}
  \label{fig:Estrup-Bromide-horizontal}
\end{figure}

\begin{figure}[htbp]
  \centering
  \figtop{Silstrup-Bromide-2000}
  \fig{Silstrup-Bromide-biopore-2000}
  
  \caption{Silstrup total (top) and biopores (bottom) vertical bromide
    transport between 2000-5-1 and 2001-5-1.  The transport is shown on the
    y-axis (positive up) as a function of distance from drain shown on
    the x-axis.  The graph labels are depths in centimeters above
    surface.}
  \label{fig:Silstrup-Bromide-2000-vertical}
\end{figure}

\begin{figure}[htbp]
  \centering
  \figtop{Silstrup-Bromide-2001}
  \fig{Silstrup-Bromide-biopore-2001}
  
  \caption{Silstrup total (too) and biopore (bottom) vertical bromide
    transport between 2001-5-1 and 2002-3-1.  The transport is shown on the
    y-axis (positive up) as a function of distance from drain shown on
    the x-axis.  The graph labels are depths in centimeters above
    surface.}
  \label{fig:Silstrup-Bromide-2001-vertical}
\end{figure}

\begin{figure}[htbp]
  \centering
  \figtop{Estrup-Bromide-2000}
  \fig{Estrup-Bromide-biopore-2000}
  
  \caption{Estrup total (top) and biopore (bottom) vertical bromide
    transport between 2000-5-1 and 2001-5-1.  The transport is shown on the
    y-axis (positive up) as a function of distance from drain shown on
    the x-axis.  The graph labels are depths in centimeters above
    surface.}
  \label{fig:Estrup-Bromide-2000-vertical}
\end{figure}

\begin{figure}[htbp]
  \centering
  \figtop{Estrup-Bromide-2001}
  \fig{Estrup-Bromide-biopore-2001}
  
  \caption{Estrup total (top) and biopore (bottom) vertical bromide
    transport between 2001-5-1 and 2002-5-1.  The transport is shown on the
    y-axis (positive up) as a function of distance from drain shown on
    the x-axis.  The graph labels are depths in centimeters above
    surface.}
  \label{fig:Estrup-Bromide-2001-vertical}
\end{figure}

\FloatBarrier
\section{Metamitron}

\subsection{Distribution}

Figure~\ref{fig:Silstrup-M-Metamitron-2000} shows the metamitron
entering first the plow layer, and later being transported to the end
of the biopores, indicating that the plow pan could be important for
metamitron dynamics.  The metamitron eventually disappear from the
plow layer, but linger at the end of the biopores (where there is no
degradation).  It is more likely diluted than removed.
Figure~\ref{fig:Silstrup-C-Metamitron-2000} shows concentration in soil
water, where four months after application only the soil near the end
of the biopores show concentrations near the limit for drinking water
(0.1 $\mu$g/l).

\subsection{Transport}


Figure~\ref{fig:Silstrup-Metamitron-2000-vertical} shows that most of
the metamitron enter the soil through the matrix, and only above the
drains are there a significant contribution from the biopores. We can
also see that the vertical movement within the soil is almost
exclusively through biopores.  Since Daisy does not have a model for
transport of solutes on the surface, the reason for the decline in
metamitron entering the soil away from the drain pipes must be surface
degradation.

Figure~\ref{fig:Silstrup-Metamitron-2000-horizontal} shows the largest
horizontal transport near the top of the soil.  Likely because the majority
of the metamitron enters the soil through the matrix, and does not
move much further down.

\begin{figure}[htbp]
  \centering
  \fig{Silstrup-Metamitron-horizontal-2000}
  
  \caption{Silstrup total horizontal metamitron transport between 2000-5-1 and
    2001-5-1.  The transport is shown on the x-axis (positive away from
    drain) as a function of depth shown on the y-axis.  The graph
    labels are the distance from drain in centimeters.}
  \label{fig:Silstrup-Metamitron-2000-horizontal}
\end{figure}

\begin{figure}[htbp]\centering
  \begin{tabular}{ccc}
    \figsilstrupl{Silstrup-M-Metamitron-2000-5} & 
    \figsilstrup{Silstrup-M-Metamitron-2000-6} & 
    \figsilstrup{Silstrup-M-Metamitron-2000-7} \\
    \figsilstrupl{Silstrup-M-Metamitron-2000-8} & 
    \figsilstrup{Silstrup-M-Metamitron-2000-9} & 
    \figsilstrup{Silstrup-M-Metamitron-2000-10} \\
    \figsilstrupl{Silstrup-M-Metamitron-2000-11} & 
    \figsilstrup{Silstrup-M-Metamitron-2000-12} & 
    \figsilstrup{Silstrup-M-Metamitron-2001-1} \\
    \figsilstrupl{Silstrup-M-Metamitron-2001-2} & 
    \figsilstrup{Silstrup-M-Metamitron-2001-3} & 
    \figsilstrup{Silstrup-M-Metamitron-2001-4}
  \end{tabular}
  
  \caption{Silstrup metamitron soil content at the end of each month
    since first application of bromide.  The y-axis denotes depth, the
    x-axis distance from drain.  There are tick marks for every
    meter. The color scale is white<10 pg/l, yellow=1 ng/l, orange=0.1
    $\mu$g/l, red=10 $\mu$g/l, and black>1 mg/l}
\label{fig:Silstrup-M-Metamitron-2000}
\end{figure}

\begin{figure}[htbp]\centering
  \begin{tabular}{ccc}
    \figsilstrupl{Silstrup-C-Metamitron-2000-5} & 
    \figsilstrup{Silstrup-C-Metamitron-2000-6} & 
    \figsilstrup{Silstrup-C-Metamitron-2000-7} \\
    \figsilstrupl{Silstrup-C-Metamitron-2000-8} & 
    \figsilstrup{Silstrup-C-Metamitron-2000-9} & 
    \figsilstrup{Silstrup-C-Metamitron-2000-10} \\
    \figsilstrupl{Silstrup-C-Metamitron-2000-11} & 
    \figsilstrup{Silstrup-C-Metamitron-2000-12} & 
    \figsilstrup{Silstrup-C-Metamitron-2001-1} \\
    \figsilstrupl{Silstrup-C-Metamitron-2001-2} & 
    \figsilstrup{Silstrup-C-Metamitron-2001-3} & 
    \figsilstrup{Silstrup-C-Metamitron-2001-4}
  \end{tabular}
  
  \caption{Silstrup metamitron soil water concentration at the end of
    each month since first application of bromide.  The y-axis denotes
    depth, the x-axis distance from drain.  There are tick marks for
    every meter. The color scale is white<10 pg/l, yellow=1 ng/l, orange=0.1
    $\mu$g/l, red=10 $\mu$g/l, and black>1 mg/l}
\label{fig:Silstrup-C-Metamitron-2000}
\end{figure}

\begin{figure}[htbp]
  \centering
  \figtop{Silstrup-Metamitron-2000} 
  \fig{Silstrup-Metamitron-biopore-2000}
 
  \caption{Silstrup total (top) and biopore (bottom) vertical
    metamitron transport between 2000-5-1 and 2001-5-1.  The transport is shown
    on the y-axis (positive up) as a function of distance from drain
    shown on the x-axis.  The graph labels are depths in centimeters
    above surface.}
  \label{fig:Silstrup-Metamitron-2000-vertical}
\end{figure}

\FloatBarrier
\section{Glyphosate}

Unfortunately, the glyphosate was applied on different years for the
sites, making them less comparable.  Nonetheless, comparing with the
rest of the data, the differences seem to be more a result of the
respective soils than difference in weather.

\subsection{Distribution}

On figure~\ref{fig:Silstrup-M-Glyphosate-2001} (Silstrup) we can see
the glyphosate entering the soil in three different places.  The soil
surface, the bottom of the short biopores that end right above the
plow pan, and the end of the deep biopores than end 1.2 meter below
the surface.  The glyphosate within the plow layer is then mixed by a
soil tillage operation.  The leaching below 2 meter is hardly visible,
but there is clearly some redistribution within the biopore active
soil.  If we look at the concentration in soil
water~\ref{fig:Silstrup-C-Glyphosate-2001} we see a clear decrease in
the plow layer, which can be explained by a combination of degradation
and dilution as the water content is increasing (see
figure~\ref{fig:Silstrup-pF-2001}).

At Estrup, the glyphosate hardly even move out of the plow layer
(figure~\ref{fig:Estrup-M-Glyphosate-2000}).  If we look at the soil
water concentration (figure~\ref{fig:Estrup-M-Glyphosate-2000}), it is
only above the limit for drinking water within the plow layer, except
for the first month where it is near the limit in a area above the
drain pipes.  The reason for this is that the water table at the time
is lower above the drain pipes (see figure~\ref{fig:Estrup-pF-2000}),
and the biopores will mainly empty in unsaturated soil.  Looking one
year further ahead (figure~\ref{fig:Estrup-C-Glyphosate-2001}) we see
the glyphosate above 1 meter being degraded, and the glyphosate below
1 meter going nowhere.

\subsection{Transport}

The horizontal transport (figure~\ref{fig:Glyphosate-horizontal}) reflect
the location in the soil, at Silstrup we see some horizontal transport at
the top of the soil, at the bottom of the short biopores, and at the
bottom of the deep biopores.  At Estrup, we plow shortly after
application.  The plow operation as defined in Daisy distributes the
glyphosate from the surface to the bottom half of the plow layer.
Which is where we see the horizontal transport.

At Silstrup (figure~\ref{fig:Silstrup-Glyphosate-2001-vertical}) most
of the glyphosate enters the soil through the matrix, but only the
part entering the soil through biopores is transported further down.
Unlike for metamitron
(figure~\ref{fig:Silstrup-Metamitron-2000-vertical}), less glyphosate
enter the soil above the drain pipes, indicating that the glyphosate
spend more time on the surface.  For Estrup
(figure~\ref{fig:Estrup-Glyphosate-2000}) there is no horizontal
variation in how much glyphosate enter the soil, none of it does so
through the biopores.  There is some matrix transport 25 cm below surface
(the plowing operation put most glyphosate 22 cm below surface),
further down there is some biopore facilitated transport above the
drains.

\begin{figure}[htbp]\centering
  \begin{tabular}{ccc}
    \figsilstrupl{Silstrup-M-Glyphosate-2001-5} & 
    \figsilstrup{Silstrup-M-Glyphosate-2001-6} & 
    \figsilstrup{Silstrup-M-Glyphosate-2001-7} \\
    \figsilstrupl{Silstrup-M-Glyphosate-2001-8} & 
    \figsilstrup{Silstrup-M-Glyphosate-2001-9} & 
    \figsilstrup{Silstrup-M-Glyphosate-2001-10} \\
    \figsilstrupl{Silstrup-M-Glyphosate-2001-11} & 
    \figsilstrup{Silstrup-M-Glyphosate-2001-12} & 
    \figsilstrup{Silstrup-M-Glyphosate-2002-1} \\
    \figsilstrupl{Silstrup-M-Glyphosate-2002-2} & & 
  \end{tabular}
  
  \caption{Silstrup glyphosate soil content at the end of each month
    since one year after the first application of bromide.  The y-axis
    denotes depth, the x-axis distance from drain.  There are tick
    marks for every meter. The color scale is white<10 pg/l, yellow=1
    ng/l, orange=0.1 $\mu$g/l, red=10 $\mu$g/l, and black>1 mg/l}
\label{fig:Silstrup-M-Glyphosate-2001}
\end{figure}

\begin{figure}[htbp]\centering
  \begin{tabular}{ccc}
    \figsilstrupl{Silstrup-C-Glyphosate-2001-5} & 
    \figsilstrup{Silstrup-C-Glyphosate-2001-6} & 
    \figsilstrup{Silstrup-C-Glyphosate-2001-7} \\
    \figsilstrupl{Silstrup-C-Glyphosate-2001-8} & 
    \figsilstrup{Silstrup-C-Glyphosate-2001-9} & 
    \figsilstrup{Silstrup-C-Glyphosate-2001-10} \\
    \figsilstrupl{Silstrup-C-Glyphosate-2001-11} & 
    \figsilstrup{Silstrup-C-Glyphosate-2001-12} & 
    \figsilstrup{Silstrup-C-Glyphosate-2002-1} \\
    \figsilstrupl{Silstrup-C-Glyphosate-2002-2} &  & 
  \end{tabular}
  
  \caption{Silstrup glyphosate soil water concentration at the end of
    each month since one year after first application of bromide.  The
    y-axis denotes depth, the x-axis distance from drain.  There are
    tick marks for every meter. The color scale is white<10 pg/l,
    yellow=1 ng/l, orange=0.1 $\mu$g/l, red=10 $\mu$g/l, and black>1
    mg/l}
\label{fig:Silstrup-C-Glyphosate-2001}
\end{figure}

\begin{figure}[htbp]\centering
  \begin{tabular}{ccc}
    \figestrupl{Estrup-M-Glyphosate-2000-5} & 
    \figestrup{Estrup-M-Glyphosate-2000-6} & 
    \figestrup{Estrup-M-Glyphosate-2000-7} \\
    \figestrupl{Estrup-M-Glyphosate-2000-8} & 
    \figestrup{Estrup-M-Glyphosate-2000-9} & 
    \figestrup{Estrup-M-Glyphosate-2000-10} \\
    \figestrupl{Estrup-M-Glyphosate-2000-11} & 
    \figestrup{Estrup-M-Glyphosate-2000-12} & 
    \figestrup{Estrup-M-Glyphosate-2001-1} \\
    \figestrupl{Estrup-M-Glyphosate-2001-2} & 
    \figestrup{Estrup-M-Glyphosate-2001-3} & 
    \figestrup{Estrup-M-Glyphosate-2001-4}
  \end{tabular}
  
  \caption{Estrup glyphosate soil content at the end of each month
    since first application of bromide.  The y-axis denotes depth, the
    x-axis distance from drain.  There are tick marks for every
    meter. The color scale is white<10 pg/l, yellow=1 ng/l, orange=0.1
    $\mu$g/l, red=10 $\mu$g/l, and black>1 mg/l}
\label{fig:Estrup-M-Glyphosate-2000}
\end{figure}

\begin{figure}[htbp]\centering
  \begin{tabular}{ccc}
    \figestrupl{Estrup-C-Glyphosate-2000-5} & 
    \figestrup{Estrup-C-Glyphosate-2000-6} & 
    \figestrup{Estrup-C-Glyphosate-2000-7} \\
    \figestrupl{Estrup-C-Glyphosate-2000-8} & 
    \figestrup{Estrup-C-Glyphosate-2000-9} & 
    \figestrup{Estrup-C-Glyphosate-2000-10} \\
    \figestrupl{Estrup-C-Glyphosate-2000-11} & 
    \figestrup{Estrup-C-Glyphosate-2000-12} & 
    \figestrup{Estrup-C-Glyphosate-2001-1} \\
    \figestrupl{Estrup-C-Glyphosate-2001-2} & 
    \figestrup{Estrup-C-Glyphosate-2001-3} & 
    \figestrup{Estrup-C-Glyphosate-2001-4}
  \end{tabular}
  
  \caption{Estrup glyphosate soil water concentration at the end of
    each month since first application of bromide.  The y-axis denotes
    depth, the x-axis distance from drain.  There are tick marks for
    every meter. The color scale is white<10 pg/l, yellow=1 ng/l, orange=0.1
    $\mu$g/l, red=10 $\mu$g/l, and black>1 mg/l}
\label{fig:Estrup-C-Glyphosate-2000}
\end{figure}

\begin{figure}[htbp]\centering
  \begin{tabular}{ccc}
    \figestrupl{Estrup-C-Glyphosate-2001-5} & 
    \figestrup{Estrup-C-Glyphosate-2001-6} & 
    \figestrup{Estrup-C-Glyphosate-2001-7} \\
    \figestrupl{Estrup-C-Glyphosate-2001-8} & 
    \figestrup{Estrup-C-Glyphosate-2001-9} & 
    \figestrup{Estrup-C-Glyphosate-2001-10} \\
    \figestrupl{Estrup-C-Glyphosate-2001-11} & 
    \figestrup{Estrup-C-Glyphosate-2001-12} & 
    \figestrup{Estrup-C-Glyphosate-2002-1} \\
    \figestrupl{Estrup-C-Glyphosate-2002-2} & 
    \figestrup{Estrup-C-Glyphosate-2002-3} & 
    \figestrup{Estrup-C-Glyphosate-2002-4}
  \end{tabular}
  
  \caption{Estrup glyphosate soil water concentration at the end of
    each month since first application of bromide.  The y-axis denotes
    depth, the x-axis distance from drain.  There are tick marks for
    every meter. The color scale is white<10 pg/l, yellow=1 ng/l, orange=0.1
    $\mu$g/l, red=10 $\mu$g/l, and black>1 mg/l}
\label{fig:Estrup-C-Glyphosate-2001}
\end{figure}

\begin{figure}[htbp]
  \centering
  \fig{Silstrup-Glyphosate-horizontal-2001}
  \fig{Estrup-Glyphosate-horizontal-2000}
  
  \caption{Silstrup total horizontal glyphosate transport between 2001-5-1
    and 2002-3-1 and Estrup total horizontal glyphosate transport between
    2000-5-1 and 2001-5-1. The transport is shown on the x-axis (positive
    away from drain) as a function of depth shown on the y-axis.  The
    graph labels are the distance from drain in centimeters.}
  \label{fig:Glyphosate-horizontal}
\end{figure}

\begin{figure}[htbp]
  \centering
  \figtop{Silstrup-Glyphosate-2001}
  \fig{Silstrup-Glyphosate-biopore-2001}
  
  \caption{Silstrup total (top) and biopore (bottom) vertical
    glyphosate transport between 2001-5-1 and 2002-3-1.  The transport is shown
    on the y-axis (positive up) as a function of distance from drain
    shown on the x-axis.  The graph labels are depths in centimeters
    above surface.}
  \label{fig:Silstrup-Glyphosate-2001-vertical}
\end{figure}

\begin{figure}[htbp]
  \centering
  \figtop{Estrup-Glyphosate-2000}
  \fig{Estrup-Glyphosate-biopore-2000}
  
  \caption{Estrup total (top) and biopore (bottom) vertical glyphosate
    transport between 2000-5-1 and 2001-5-1.  The transport is shown on the
    y-axis (positive up) as a function of distance from drain shown on
    the x-axis.  The graph labels are depths in centimeters above
    surface.}
  \label{fig:Estrup-Glyphosate-2000}
\end{figure}


\end{document}

%%% Local Variables: 
%%% mode: latex
%%% TeX-master: t
%%% End: 
