\documentclass{article}

\begin{document}

\section{Graphical User Interfaces and Scientific Models}

Modelling is an integral part of the scientific method, and the advent
of computers have made scientific predictions made with complex
numerical models feasible.  Computer programming has thus become an
important tool for scientific research.

Computer programs originating from scientific research are rarely
characterized by easy of use or accessibility from people outside the
research groups who created the programs.  Which is a pity, as this
seriously impair the dissimination of the research results if anyone
who wants to build upon them have to recreate the model from published
papers.  Especially for large system model that aggregate a number of
simpler models this can be an immense barrier.  

There are two root causes for this problem.  First, the primary users
of the model are the scientists themselves, who are of caurse expert
users with initimate knowledge of both the model and the software
implementing it.  The need of these users would be flexibility, rather
than simplicity.

The second is due to the way the research grant system work.  Grants
aren't given to programming work, and the only relevant dissimination
is considered scientific publication.  This means that most grants
will only cover the basic modifications to the software needed to
prove or disprove specific hypothesises, and not adjustments in the
user interface to needed make the new knowlede accessible that way.
Which again means that even if a non-specialist user-interface is
developed, it will not be maintained to cover new functionality. 

\subsection{Graphical User Interfaces and Daisy}

Daisy is a scientific model which has generated interest outside the
research community, leading to three attempts to create a graphical
user interfaces.  The first was a standalone interface written in
Delphi by a pair of student programmers, the second was a integrated
GIS version written by DHI, and the third a web-interface developed at
DJF.  The Delphi version never quite got to the point where we were
ready to distribute it, before the the funding was used up.  A
graphical log file viewer did for a fast presentation of the output
did came out of the project.  The GIS version is a current commercial
product, but some frictions do occur when people meet the limitations
of the GIS interface, and want to use features from the research
version.  Since the GIS interface tend to be build upon older versions
of Daisy, mixing features from the latest research version can meet
incompatibilities.  The web interface is a specific design for
predicting the need for an extra fertilizer application after heavy
rainfall.

Our proposal is to develop a GUI for Daisy that is closely so
integrated with the Daisy kernal that changes and new features in
Daisy will automatically be reflected in the GUI.  This requires a
greater initial investment than a stand-alone GUI, but has the benefit
of being maintainable, even without continues funding.  

\section{The Daisy Model}

Daisy is a model of an agro-ecosystem, that is, a model of the
processes that happens at field scale.  It can be used to predict both
environmental impact and production of different agronomic practices,
which leads to interest in both environmental protection offices and
farmers. 

Daisy is a mechanistic, deterministic, simulation model.  It is
mechanistic because it attempts to model the actual physical, chemical
and biological processes that occurs in the field.  It is
deterministic because it doesn't attempt to model out lack of
knowledge of the state of the system, but assumes the state is fully
specified.  It is a simulation model because we model what happens
following the time axis, from an initial state, through one time step
to a new state, and so on until we end the simulation.

\subsection{Structure}

The design of the Daisy program is build upon three layers.  The top
layer is called a component.  A component is meant to represent a
process in the domain being modelled, such as for example
nitrification, the process that turns ammonium into nitrate, but it
also used for software artifacts such as input and output code. The
second layer is a model.  A model is mathematic description of the
process.  In the code, a process is implemented as a base class which
exports its interface.  The models are implemented as derived,
concrete classes, with no exported interface.  All access to the model
is through the base classe interface.  The models register themself in
a ``librarian'' template class (these days known as a factory
pattern), associated with the base class. The librarian is then
responsible for instantiating the derived classes.

The third layer is the model parameterizations.  When a model register
itself in the appropriate library, it also register the parameters and
state variables associated with the model, as well as default values,
type information, and textual description of parameters and variables.
One particular parameter type is ``model'', which allow the parameter
to specify a registered parameterization of a model for a specific
process.  This allows for a hierarchical system description, where all
the agro-ecological processes are ultimately part of the Daisy model.

\subsection{Input, Output and Documentation}

This design is used to automate the input and output procedures, and
to automatically generate a reference manual.  The input system works
by reading a file containing named parameterizations of a given
process.  Each new parameterization is derived from an existing
parameterization or model.  The input system will then read parameters
of the given model, and check that their type matches what is
specified in the library.  The new parameterization is then registered
in the library, using the same model as the base parameterization.

The output system is implemented as a component, with models that
create tables or checkpoints.  The table model allow the user to
specify which state variables to print in each column, and how
often. 

The document system simply creates a (\LaTeX{}) file with processes as
chapters, models and paramterizations as sections, and an autmatically
generated table of parameters, state variables, their types and a
description string, all found in the komponent libraries.

This mean that any change to an existing model, or any new model
introduced to the system, automatically will be reflected in both the
input and output subsystems, as well as documented in the reference
manual. 

The goal is to make the same true for the graphical user interface.

\section{Graphical User Interfaces and Daisy}

Not perfect, but maintainable.


files

libraries

dependencies

...

text 

model form

subset views (is parameterizations)


automatisk GUI (advanceret)

Specialiseret GUI for bestemte brugere.  V�lg parametre fra filer, i
stil med log parameteriseringer.

\end{document}
