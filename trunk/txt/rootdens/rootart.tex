\documentclass[a4paper,11pt,twoside]{article}
\usepackage{a4}
\usepackage[T1]{fontenc}
\usepackage[latin1]{inputenc}
\usepackage{amsmath}
\usepackage{hyperref}
\usepackage{natbib}
\usepackage{graphicx}
\newcommand{\daisy}{Daisy}
\newcommand{\Daisy}{Daisy}
\newcommand{\cplusplus}%
{{\leavevmode{\rm{\hbox{C\hskip -0.1ex\raise 0.5ex\hbox{\tiny ++}}}}}}
\newcommand{\Cplusplus}{\cplusplus}
\newcommand{\mshe}{Mike/\textsc{she}}
\newcommand{\wintel}{\texttt{win32}}
\newcommand{\dll}{\textsc{dll}}
\newcommand{\Dll}{\textsc{Dll}}
\newcommand{\gui}{\textsc{gui}}
\newcommand{\Gui}{\textsc{Gui}}
\newcommand{\unix}{Unix}
\newcommand{\dhi}{\textsc{dhi}}
\newcommand{\Dhi}{\textsc{Dhi}}
\newcommand{\api}{\textsc{api}}
\newcommand{\Api}{\textsc{Api}}
\newcommand{\lai}{\textsc{lai}}
\newcommand{\Lai}{\textsc{Lai}}
%\newcommand{\url}[1]{\linebreak[4]\texttt{<URL:#1>}}

%%% Local Variables: 
%%% mode: latex
%%% TeX-master: t
%%% End: 


% MMO
% L_z -> L(z)
% L_{z,x} -> L(x,z)

\newcommand{\attmark}[1]{\mbox{$^{\mbox{\footnotesize #1}}$}}
\newcommand{\twod}{\textsc{2d}}
\newcommand{\oned}{\textsc{1d}}
\newcommand{\threed}{\textsc{3d}}
\begin{document}

\title{A simple parametric \twod{} root density distribution model
  for row crops.}  

\author{Per Abrahamsen\attmark{a,}\footnote{Corresponding author.
    Present address: IGM LIFE KU, Thorvaldsensvej 40, DK-1871
    Frederiksberg.  Email: pa@life.ku.dk}, 
  Mathias Neumann Andersen\attmark{b}, Xuebin Qi\attmark{c},\\
  Guitong Li\attmark{d}, Mikkel Mollerup\attmark{a},
  and S�ren Hansen\attmark{a}\\
  \\
  \small\em\attmark{a}University of Copenhagen, Denmark\\
  \small\em\attmark{b}Aarhus University,
  Denmark\\
  \small\em\attmark{c}Chinese Academy of Agricultural Sciences, China\\
  \small\em\attmark{d}China Agricultural University, China}

\maketitle

\begin{abstract}
  Partial root zone drying (PRD) is an irrigation technique that in some
  cases can increase the water usage efficiency.  A \twod{} model of
  water movement in the soil is needed to predict how to most
  efficiently apply irrigation in order to achieve the PRD effect.  This
  model must include a \twod{} root model.  In this paper we extend an
  empirical root density distribution based on densely populated
  homogeneous fields (\oned{}) to row crops (\twod{}).  The row crops
  are modeled as having a uniform density in the direction parallel to
  the rows, but variable in the direction perpendicular to the row.  In
  each case we demonstrate how to find the distribution parameters from
  the root dry matter and the size of the root zone.

  The SAFIR EU project included \twod{} root density measurements
  performed at two sites, near Zhengzhou and Beijing respectively, for
  two experimental years and on multiple plots.  Both the \oned{} and
  \twod{} model are fitted to these datasets.  Based on a statistical
  F-test we conclude that the \twod{} model is superior when the root
  zone width is less than the distance between rows.  The median R$^2$
  for all the examined datasets is 0.86, and a visual inspection
  confirms that the model catches both the horizontal and vertical
  root density variation.
\end{abstract}

\section{Introduction}

Partial root zone drying (PRD) is an irrigation technique where
irrigation is applied on only part of the root zone, dividing it into
a wet and a dry zone.  The idea is then that root signals from the dry
zone will lower the hydraulic conductivity through of the stomata, and
thus reduce the transpiration from the wet zone.  Used properly, the
technique has been shown to provide significant water savings
(MNA-TODO: REF).  For PRD to work, the irrigation scheme must
alternate between what part of the root zone is wet, and what part is
kept dry. (MNA-TODO: REF-SAFIR/Fulai) recommends that (MNA-TODO:
SAFIR/Fulai strategy).  When these conditions are fulfilled is
determined by the movement of water through the soil matrix from the
wet zone to the dry zone, and the water uptake through the plant
roots.  The established solution to the first is based on Richards
equation \citep{richardseq}.  The formulation must be at least \twod{}
to include both the vertical and a horizontal component.

The issue of water uptake from the soil matrix through the roots can
be broken into the total uptake (approximately equal to the
transpiration), and the spatial distribution within the root zone.
With PRD irrigation, the transpiration will depend on the
distribution, due to the reliance of root signals, see
e.g.\@~\cite{ssoc}.  This makes the spatial distribution even more
important.  Several models incorporating this exists,
e.g.\@~\cite{vrugt2001calibration} proposes a flexible, empirical
\twod{} model with a radial geometry, which \cite{apri} successfully
use to model PRD irrigation of a vineyard.  A static description of
the root density based on measurements was used in the later
application.  For perennial plants using a static description during
the irrigation season can make sense, but for annual plants, such as
most vegetables, the root zone is too dynamic.  A root growth model
for row crops based on diffusion theory was proposed
in~\cite{acock1996convective}. In general the model required a
numerical solution, but analytical solutions for special cases were
found by~\cite{willigen2002two}.

The \twod{} radial and the \threed{} root density models typically
include only a single plant.  In contrast, \oned{} models by their
nature will be concerned with a whole population, like a forest or an
agricultural crop.  An example of the later would be the model found
by~\cite{gp74} in a meta-analysis, where they found an exponential
decrease with depth to match the observations.  The root models
mentioned above are parametric, the root are described as a density
function, where the parameters are position in space relative to the
plant and soil surface.  In contrast architectural root models track
the structure and branches of the roots.  See \cite{wang2004modelling}
for an overview of existing \oned{}, \twod{}, and \threed{} root
models, both parametric and architectural.  

For looking at the root zone of a single plant, radial coordinates are
a good match.  But when the root zones of the plants overlap, the
problem becomes \threed{}.  Solving Richards equation in \threed{} is
complex and results in long simulation times.  Finding a \twod{}
approximation is therefore desirable.  In this paper we therefore
propose a \twod{} model with Euclidean coordinates for row crops.  The
assumption is that the row is sufficiently dense in the direction of
the rows that we can ignore that dimension.  The two dimensions then
become depth, and horizontal distance from the row.  The assumption is
obviously from for the very young crop, but at that stage PRD
irrigation is not feasible.

The new root zone model was used in the SAFIR (SAFe IRrigation) EU
project \citep{safir2010,ssoc} as part of the Daisy agro-ecological
model \citep{daisy-fertilizer,daisy-ems}.  The SAFIR project had two
aspects, the first being assessment of the risk for human health with
regard to use of waster water for irrigation
\citep{Forslund2010440,Surdyk2010451}, and water saving irrigation
techniques, in particular PRD \citep{Jensen2010403}.  There were five
experimental sites, three in Europe, and two in China, and three crops,
potatoes, fresh tomatoes, and processing tomatoes.  For the two
Chinese sites root density was measured. The original \oned{} Daisy
model was extended to \twod{} \citep{1dvs2d}, and used for modelling
the experiments.  The original Daisy crop model calculates root dry
mass and root zone depth, and assumes a root density distribution
following \cite{gp74}.  With the \twod{} extension root zone width was
added to the crop model, as well as the root zone density distribution 
model discussed in this paper.

\section{Model theory}

Our goal is to describe the two dimensional root density distribution
for row crops in terms of root dry mass, root zone depth, and root
zone width.  These values can be provided by a general crop growth
model (like the one found in Daisy), and furthermore have the
advantage that they have a physical meaning.  However, first we will
describe for a traditional one dimensional root density distribution
in terms of root dry mass and root zone width.  This will provide a
reference to compare with later.

\subsection{Densely populated fields}
\label{sec:model-1d}

In accordance with \cite{gp74}, the root density distribution $L_z$
for a crop can be described by
\begin{equation}
  L_z = L_0\, e^{-a z}
  \label{eq:g+p}
\end{equation}
where $L_0$ is the root density at the soil surface, $a$ is a
distribution parameter, and $z$ is the depth below soil surface.

We here assume that the density is uniformly distributed on the
horizontal plane, an assumption that fails with e.g.\ row crops.

The parameters $a$ and $L_0$ will both vary with time.  For a
production oriented simulation model like Daisy, it can be more
convenient to specify the density in terms of accumulated root dry
matter $M_r$ and total root depth $d_c$\citep{daisyN}.

We define the root depth at the lowest depth where the root density is
at above specified threshold $L_m$.  By inserting this in
\eqref{eq:g+p}, we get
\begin{equation}
  L_m = L_{d_c} = L_0\, e^{-a d_c}
  \label{eq:root-depth}
\end{equation}

We convert the root mass to root length $l_r$ by assuming the specific
root length $S_r$ is a known constant
\begin{equation}
  l_r = S_r \, M_r
  \label{eq:root-length}
\end{equation}

The total root length is also the integral of the root density over
the profile
\begin{equation}
  l_r = \int_0^{\infty} L_z \: dz 
      = \int_0^{\infty} L_0\, e^{-a z} \, dz 
      = \frac{L_0}{a}
  \label{eq:root-integral}
\end{equation}

By inserting the expression we get for $L_0$ from
\eqref{eq:root-integral} in \eqref{eq:root-depth} we get
\begin{equation}
  L_m = l_r \, a \, e^{-a d_c}
  \label{eq:a-only}
\end{equation}

If we substitute $W = -a d_c$ and isolate the known values on the right
side this gives us
\begin{equation}
   W e^W = - L_m \frac{d_c}{l_r}
  \label{eq:Lambert}
\end{equation}
The solution to this equation with regard to $W$ happens to be the
definition of the Lambert-W function \citep{euler83,lambert58}.  The
function on the left hand side of the equation is depicted on
figure~\ref{fig:W}. 
\begin{figure}[htbp]
  \input{rootdens_W}
  \caption{$W e^W$}
  \label{fig:W}
\end{figure}

Since we now know the value for $W$, we can find the desired density
parameters $L_0$ and $a$ by substituting back
\begin{eqnarray}
   a   &=& -W / d_c\\\label{eq:a-solved}
   L_0 &=& \frac{L_m}{e^{-a d_c}} = L_m e^{a d_c}\label{eq:L0-found}
\end{eqnarray}

\subsubsection{Numeric solution to $W$}
\label{sec:W}

We start by dividing the functions into monotonic intervals by finding
the derivative
\begin{equation}
  \frac{d W\, e^W}{dW} = e^W + W e^W
  \label{eq:derived-W}
\end{equation}
The equation
\begin{equation}
  e^W + W e^W = 0
  \label{eq:derived-W-solutions}
\end{equation}
has one solution, $W=-1$. The expression $W\, e^W$ is decreasing below
$-1$ and increasing above $-1$.  Thus, $W=0$ is a global minimum.

Since $\lim_{W\to-\infty} W\,e^W=0$ we get a single solution when
$-L_m \frac{d_c}{l_r}$ is exactly at the bottom point ($-1 e^{-1}$), two
when it is above (it is never positive), and none when it is below.
The later situation corresponds to the case where there is
insufficient root $l_r$ to satisfy the minimal root density $L_m$
within the given root zone $d_c$.

Both solutions are valid, but represent different distributions.
\begin{itemize} 
\item The solution for $W < -1$ represents a large $a$ parameter. From
  \eqref{eq:L0-found} we see this also means $L_0$ is large.  Thus,
  the solution corresponds to a root zone with a high density near the
  top that decreases rapidly to $L_m$ at the bottom of the root zone,
  and continues to decrease so only a small contribution to the total
  root length come from below he root zone.
\item The solution for $W > -1$ (and thus small values of $a$ and
  $L_0$) corresponds to a low root density near the top that
  decreases slowly, and thus gives a larger contribution to the total
  root length from below the root zone.
\end{itemize}
As the total root length increases, pressing $W$ towards $0$ or
$-\infty$, the difference between the solutions grow.  When there is
just enough roots to satisfy the constraints at $W = -1$, the two
solutions converges to one.  As we like our roots to stay mostly
within the root zone, we choose the solution for $W < -1$.  We can
thus find $W$ numerically using Newton's method and an initial guess
of $-2$.

\subsubsection{Limited growth}
\label{sec:soillim}

The distribution in \eqref{eq:g+p} implies a gradual decrease of roots
going towards but never reaching zero.  For some soils this doesn't
match what we observe, rather than a gradual decrease, there is sharp
decrease at a specific depth, as the roots are unable to penetrate
further down.  To handle this, we divide the root depth into a crop
specific and soil independent potential root depth $d_c$, and soil
specific and crop independent maximum root depth $d_s$.  The actual
root depth $d_a$ is then the shallowest of these two.
\begin{equation}
  d_a = \min (d_c, d_s)
  \label{eq:actual-depth}
\end{equation}

We now create a modified root density function $L_z^\prime$ by
defining it to zero below $d_a$, and scaled to preserve mass balance
above.
\begin{equation}
  L_z^\prime =
  \begin{cases}
    k^\prime L_z & \text{if $z \leq d_a$}\\
    0 & \text{if $z > d_a$}
  \end{cases}
  \label{eq:limited-depth}
\end{equation}
\begin{equation}
  k^\prime = \frac{l_r}{\int_0^{d_a} L_z \, dz}
          = \frac{l_r\; a}{L_0 - L_0  \; e^{-a\; d_a}}
  \label{eq:scale-factor}
\end{equation}

\subsection{Row crops}

We can describe a row crop with a two dimensional model by assuming
that the plants are densely packed in the row.  Our second dimension
$x$ is horizontal, orthogonal to the row.  The root density at a
specific point can be denoted $L_{z,x}$, and we choose a coordinate
system where $L_{0,0}$ is the the root density in the top center of
the row.  We then define the following root distribution
\begin{equation}
  L_{z,x} = L_{0,0}\, e^{-a_z z} e^{-a_x |x|}
  \label{eq:Lzx}
\end{equation}
where $a_z$ and $a_z$ control the density decrease in the two
dimensions.

\subsubsection{Finding the parameters}

To find the parameters $a_z$, $a_x$ and $L_{0,0}$, we assume as before
that the root depth and root mass are known, and now additionally that
the root zone radius at soil surface $w_c$ is known.  We define the
root zone depth $d_c$ to be the depth right below the row ($x = 0$)
where the root density is $L_m$.  As $x=0$ is the place where
\eqref{eq:Lzx} predicts the highest density, the average root density
at that depth will be below $L_m$.  Similarly, we define the radius
$w_c$ as the horizontal distance from the row where the root density
at the surface ($z = 0$) is equal to $L_m$.
\begin{equation}
  L_m = L_{d,0} = L_{0,r}
  \label{eq:minroot}
\end{equation}
See figure~\ref{fig:row} for an illustration.
\begin{figure}[htbp]
  \input{row}
  \caption{The \twod{} root zone of a single row of crops.  The x-axis
    denotes horizontal distance to the row, and the z-axis height
    above ground level.  The highest root density ($L_{0,0}$) can be
    found in the row at ground level.  The root density decreases
    exponentially, with both horizontal and vertical distance.  Four
    root density contours are shown.  The three innermost contours
    each represents a fixed decrease in root density.  The last
    contour represents the threshold value, $L_m$, and defines both
    the root zone depth ($d_c$) and the root zone radius ($w_r$).}
  \label{fig:row}
\end{figure}

The total root length on one side of the row ($l_R$), which we assume
is known from our crop model, is the integral of the root density over
the half plane
\begin{equation}
    l_R = \int_0^{\infty} \int_0^{\infty} L_{z,x} \: dz \, dx = \frac{L_{0,0}}{a_z a_x}
%%        = \int_0^{\infty} \int_0^{\infty} L_{0,0}\, e^{-a_z z} e^{-a_x |x|} \: dz \, dx
  \label{eq:root-integral2}
\end{equation}
Thus \eqref{eq:Lzx} can be rewritten 
\begin{equation}
  L_{z,x} = l_R\, a_z\, a_x \, e^{-a_z z} e^{-a_x |x|}  
  \label{eq:azx}
\end{equation}
By using \eqref{eq:azx} in \eqref{eq:minroot} we get
\begin{eqnarray}
  L_m &=& l_R\, a_z\, a_x \, e^{-a_z d} \\\label{eq:Ld0}
  L_m &=& l_R\, a_z\, a_x \, e^{-a_x r}\label{eq:L0r}
\end{eqnarray}
Thus $e^{-a_z d} = e^{-a_x r}$ or
\begin{equation}
  a_x = \frac{d_c}{w_c} a_z
  \label{eq:aztoax}
\end{equation}
By inserting \eqref{eq:aztoax} in \eqref{eq:Ld0} we get
\begin{equation}
  L_m = l_R\, a_z\, \frac{d}{r} a_z \, e^{-a_z d}
  \label{eq:azeq1}
\end{equation}
If we substitute 
\begin{equation}
  Q = -a_z d
  \label{eq:Qaz}
\end{equation}
and isolate the known values on the right side, this gives us:
\begin{equation}
  Q^2\, e^Q = L_m \frac{d \, r}{l_R}
  \label{eq:logsquare}
\end{equation}
The left hand side expression is illustrated in figure~\ref{fig:Q}.
Unlike \eqref{eq:Lambert}, nobody bothered to give the solution to
\eqref{eq:logsquare} a name.  Knowing $Q$ we can find
$a_z$~\eqref{eq:Qaz}, $a_x$~\eqref{eq:aztoax}, and
$L_{0,0}$~\eqref{eq:root-integral2}.
\begin{figure}[htbp]
  \input{rootdens_Q}
  \caption{$Q^2 e^Q$}
  \label{fig:Q}
\end{figure}

\subsubsection{Numeric solution to $Q$}
\label{sec:Q}

We start by dividing the expression into monotonic intervals by
finding the derivative
\begin{equation}
  \frac{d (Q^2\, e^Q)}{dQ} = 2 Q e^Q + Q^2 e^Q
  \label{eq:derived}
\end{equation}
The equation
\begin{equation}
  2 Q e^Q + Q^2 e^Q = 0
  \label{eq:derived-solutions}
\end{equation}
has two solutions, $Q=0$ and $Q=-2$, and the expression $Q^2\, e^Q$ is
increasing below $-2$, decreasing between $-2$ and $0$, and increasing
above $0$. Thus, $Q=0$ is a local (and in this case also global)
minimum, and $Q=-2$ is a local maximum.

We are not interested in positive values for $Q$, they correspond to
negative values for $a_z$, the simplification in
\eqref{eq:root-integral2} are only valid if $a_z > 0$. Since
$\lim_{Q\to-\infty} Q^2\,e^Q=0$ we get a single negative solution when
$L_m \frac{d \, r}{l_R}$ is exactly at the top point ($2^2 e^{-2}$),
two when it is smaller (it is never negative), and none when it is
larger.  The later situation corresponds to the case where there is
insufficient root $l_R$ to satisfy the minimal root density $L_m$
within the given root zone.

Both negative solutions are valid, but represent different
distributions.  The solution for $Q > -2$ corresponds to a root
distribution with a large fraction of the root length being located
from outside the root zone, hence we choose the solution for $Q < -2$.
We can find $Q$ numerically using Newton's method and an initial guess
of $-3$.

\subsubsection{Multiple rows}
\label{sec:model-rows}

If the rows are close, the root systems will overlap as shown on
figure~\ref{fig:zigzag}.
\begin{figure}[htbp]
  \input{rootdens_L}
  \caption{The x-axis represents the distance from a row to the
    midpoint between it and the row to its right.  The y-axis is the
    root density for roots originating in a specific row.  The top
    line represents the roots from the row itself.  The next line the
    roots from the row to the right.  And the last line the roots from
    the row to the left.  In theory, all the rows on the field will
    contribute some roots to the interval.  The root density in the
    interval will be the sum of all the individual contributions.}
  \label{fig:zigzag}
\end{figure}

If $R$ is the distance between rows, and we assume an infinite number
of identical rows, this can be expressed by the equation
\begin{equation}
  L^*_{z,x} =
    \begin{cases}
       \sum_{i=0}^{\infty} (L_{z,x + i R} + L_{z,R + i R - x}) & \text{if $x < R/2$}\\
                                                    0  & \text{if $x \geq R/2$}
    \end{cases}
  \label{eq:Lzxstar}
\end{equation}

Using \eqref{eq:Lzx} and the rules for geometric series we can rewrite
the first case to get rid of the sum
\begin{equation}
  \begin{array}{rl}
     & \sum_{i=0}^{\infty} (L_{z,x + i R} + L_{z,R + i R - x})\\
%%    =& L_{0,0}\, e^{-a_z z} 
%%       \sum_{i=0}^{\infty} (e^{-a_x (x + i R)} + e^{-a_x (R + i R - x)})\\
    =& L_{0,0}\, e^{-a_z z} (       e^{-a_x x} \sum_{i=0}^{\infty} e^{-a_x i R} 
                          + e^{-a_x (R - x)} \sum_{i=0}^{\infty} e^{-a_x i R})\\
%%    =& L_{0,0}\, e^{-a_z z} (e^{-a_x x} + e^{-a_x (R - x)})
%%       \sum_{i=0}^{\infty} e^{-a_x i R}) \\
    =& L_{0,0}\, e^{-a_z z} (e^{-a_x x} + e^{-a_x (R - x)})
       \sum_{i=0}^{\infty} ((\frac{1}{e})^{a_x R})^i\\
    =& \frac{L_{0,0}\, e^{-a_z z} (e^{-a_x x} + e^{-a_x (R - x)})}
            {1 - \frac{1}{e}^{a_x R}}\\
  \end{array}
  \label{eq:Lzxstar-solved}
\end{equation}
Two examples of the resulting root density distribution can be found
on figure~\ref{fig:tworows}.

\subsubsection{Mapping between the models}
\label{sec:mapping}

We would like to retain our original distribution \eqref{eq:g+p} when
ignoring the x dimension.  We couldn't do that when looking only at
the root system for a single row, as it is infinitely wide and thus
has an average density of zero.  However, if we look at the roots of
single row, we get
\begin{equation}
  L_z = \frac{2 \int_0^{\infty} L_{z,x}\, dx}{R}
  \label{eq:x-integrated}
\end{equation}
We multiply by two as we assume the two sides of the rows are
identical.  By integrating to $\infty$ rather than just $R/2$ we do
include roots from outside the row.  However, because the system has
an infinite number of identical rows, the amount of roots from the
crop outside its own row is exactly the same as the amount of roots
from other rows inside the row we are examining.

Inserting \eqref{eq:Lzx} and \eqref{eq:g+p} in \eqref{eq:x-integrated} we get
\begin{equation}
  \begin{array}{rcl}
    L_0\, e^{-a z} &=& \frac{2}{R} \int_{0}^{\infty} L_{0,0}\, e^{-a_z z} e^{-a_x x} dx\\
                 &=& \frac{2 L_{0,0}\, e^{-a_z z}}{R} \int_{0}^{\infty} e^{-a_x x} dx\\
                 &=& L_{0,0}\, e^{-a_z z} \frac{0 - 1}{-a_x}\\
                 &=& \frac{2 L_{0,0}}{R a_x} e^{-a_z z}\\
  \end{array}
  \label{eq:1d2d}
\end{equation}
So we get
\begin{eqnarray}
  a_z &=& a\\\label{eq:azisa}
  L_{0,0} &=& � a_x R L_0\\\label{eq:L00L0}
  L_0 &=& \frac{2 L_{0,0}}{a_x R}\label{eq:L0L00}
\end{eqnarray}
as the equations to use when switching between the one and two
dimensional descriptions.

\section{Data description and methodology}

Root sampling were performed at two experimental fields, the first
being operated by the Chinese Academy for Agricultural Sciences (CAAS)
and located near Zhengzhou in the Henan province, and the second
operated by the Chinese Agricultural University and located near
Beijing.  Potatoes were grown at the CAAS site, while the CAU site
were growing tomatoes.

\subsection{Treatments}

The SAFIR project had two main aspects, the first was the reuse of
waste water for irrigation, with emphasis on human safety.  All
treatments were fertilized, so the constituents of the waste water
would be unlikely to affect crop growth.  Hence, for crop growth the
different irrigation sources could be seen as replicates.  The second
aspect was how the design of the irrigation system could save water.
Two irrigation methods and three irrigation strategies were
considered.  The irrigation methods were subsurface drip irrigation
and furrow irrigation.  The irrigation strategies were full
irrigation, deficit irrigation (70\% of full irrigation), and partial
rootzone drying (like deficit, but altering sides of the crop).

At the CAAS site, root density sampling was performed in 2006 and
2008.  Unfortunately, the 2006 harvest failed, so we have discarded
that data set.  All treatments were done with three replicates, but
due to the labour intensive nature of root density sampling, it was
only performed on selected plots as outlined in
table~\ref{tab:caas2008treatments} and~\ref{tab:cautreatments}.

\begin{table}[htbp]
  \caption{CAAS 2008 treatments with root density measurements.  
    All treatments used secondary treated wastewater.  Some treatments
    also used a sand filter (SF), addition plus removal of heavy
    metals (HM), or ultraviolet light (UV).}
  \label{tab:caas2008treatments}
  \begin{tabular}{llll}\\\hline
Plot	& Source	& Method	& Strategy\\\hline
3	& SF + HM + UV	& Subsurface drip	& Partial rootzone drying\\
6	& SF + HM + UV	& Subsurface drip	& Full irrigated\\
9	& SF + HM + UV	& Subsurface drip	& Deficit Irrigation\\
12	& SF + HM	& Subsurface drip	& Deficit Irrigation\\
15	& SF + HM	& Subsurface drip	& Partial rootzone drying\\
18	& SF + HM	& Subsurface drip	& Full irrigated\\
21	& 	& Furrow	& Partial rootzone drying\\
24	& 	& Furrow	& Full irrigated
  \end{tabular}
\end{table}

\begin{table}[htbp]
  \caption{CAU treatments with root density measurements.  
    All treatments used tap water.  The strategy used was either 
    partial rootzone drying (PRD) or full irrigation (FI).}
  \label{tab:cautreatments}
  \begin{tabular}{lll|lll}\\\hline
\multicolumn{3}{c|}{2007} & \multicolumn{3}{c}{2008}\\
Plot	& Method	& Strategy	& Plot	& Method	& Strategy\\\hline
1	& Subsurface drip	& PRD	& 3	& Subsurface drip	& PRD\\
2	& Subsurface drip	& PRD	& 6	& Subsurface drip	& FI\\
3	& Subsurface drip	& PRD	& 9	& Furrow	& PRD\\
4	& Subsurface drip	& FI	& 12	& Furrow	& FI\\
5	& Subsurface drip	& FI	& 15	& Subsurface drip	& FI\\
\end{tabular}
\end{table}

\subsection{Root sampling}

Both sites used ridge systems.  To describe it, we use a coordinate
system with the x-axis orthogonal to the ridges with x=0 cm representing
the top of the ridge, the y-axis along the ridges with y=0 cm
representing the plant position, and the z-axis representing height
above the undisturbed ground level, positive upwards.  At the CAAS
site, roots were sampled in a $3\times2\times7$ grid.  The sampling
along the x-axis corresponded to the top of the ridge, the middle of
the ridge wall, and the bottom of the valley (x=0 cm, x=18.75 cm,
x=37.5 cm).  The sampling along the y-axis corresponded to the plant
location, and the middle between two plants (y=0 cm, y=15 cm).  At the
z-axis, there were sampling in ten  cm intervals starting just below
the undisturbed ground level (z=-5 cm) at the top of the ridge, and
starting 10 and 20 cm lower for sampling at the middle and bottom of
the ridge system.  The sampling points are illustrated on
figure~\ref{fig:sample-caas}.  A the CAU site, the sampling were done
in a cross shape, with five sample locations at the top of the ridge
(x=0 cm; and y=-20 cm, y=-10 cm, y=0 cm, y=10 cm, y=20 cm), and five sample
locations across the ridge system (y=0 cm; and x=-30 cm, x=-15 cm,
x=15 cm, x=0 cm, x=30 cm).  The sampling along the top of the ridge
started at z=15 cm, that is above undisturbed ground level.  The
sampling across the ridge started at z=-5 cm.  The samplings were
performed in 10  cm intervals, the lowest varied between plots and with
the horizontal distance to the plant, the lowest were at z=-55 cm. See
figure~\ref{fig:sample-cau}.

\begin{figure}[htbp]
  \input{sample-caas}
  \caption{Sampling points for CAAS site.  All distances are in cm.
    Root density has been sampled at the center of the ridge (x=0), at
    the center of the furrow (x=37.5), and halfway between the furrow
    and ridge (x=18.75).  This sampling was done both at the plant
    location (y=0), and between the plants in the row (y=15).  The
    sampling depth starts at soil surface and continues in 10 cm
    intervals to -70 cm.  There are no samples in the top of the
    ridge, above the original surface level.}
  \label{fig:sample-caas}
\end{figure}

\begin{figure}[htbp]
  \input{sample-cau}
  \caption{Sampling points for CAU site.  All distances are in cm.
    Root density has been sampled at a cross shape with center at
    plant location, with samples 10 and 20 cm at each side of the
    plant location within the ridge, and 15 and 30 cm at each side
    across the ridge system.  The sampling depths start 15 centimeters
    above the original ground level within the ridge, and continues
    down to 55 cm at the plant location and 10 cm away, down to 45 cm
    15 cm away, and down to 35 cm 20 and 30 cm away. We only use the
    data from the samplings across the ridge system and below the
    original ground level, marked $+$ on the figure.}
  \label{fig:sample-cau}
\end{figure}

\subsection{Fitting the data to the model}

The root density model presented in this paper assumes flat soil.  To
solve this, a virtual soil surface corresponding to the undisturbed
ground level (z=0 cm) was used in the model.  Furthermore, the root
density model is \twod{}, while the root data for both sites is
\threed{}.  The root model describes the x-axis (position between row)
and the z-axis (height above ground), but not the y-axis (position
within row).  For the CAAS dataset, we have similar datasets for two
y-values, and we have included both datasets, thus fitting a \twod{}
model to a \threed{} dataset.  From the point of view of the model, we
have two observations for each (x, z) pair.  For the CAU dataset, all
the observations where y$\neq$0 cm have x=0 cm.  Including these would
give an unbalanced dataset.  So for the CAU dataset we have only used
the measurements where y=0 cm.  Furthermore, we have ignored the
measurements with z$>$0 cm, as illustrated on
figure~\ref{fig:sample-cau}.  For each year, we have dataset for each
plot, plus on additional dataset consisting of all measurements from
the same site, in effect a dataset where we considered the different
treatments as replicates.

We used fixed values for the minimal root density ($L_m = 0.1$
cm/cm$^3$) and specific root length ($S_r = 100$ m/g).  The distance
between rows ($R$) were taken from the experimental setup, 75 cm for
CAAS and 80 cm for CAU.  Given these, we could create a function that
calculated the coefficient of determination (R$^2$) for the \oned{}
model based on root dry mass ($M_r$) and root zone depth ($d_c$) as
described in section~\ref{sec:model-1d}.  Imposing a soil limit (see
section~\ref{sec:soillim}) did not improve the fit, so we used $d_a =
d_c$.  For the \twod{} model, we created a similar function based on
section~\ref{sec:model-rows} that in addition took the horizontal root
zone radius ($w_c$) as a parameter.  For each dataset, we found the
parameter values that gave the highest R$^2$ using
the~\cite{nelder1965simplex} simplex algorithm.

The \oned{} model has two free variables ($M_r$ and $d_c$) as used
above, while the \twod{} model has three free variables ($M_r$, $d_c$,
and $w_c$).  Furthermore, as demonstrated in
section~\ref{sec:mapping}, the \oned{} model is special case of the
\twod{} model.  This means we can use an F-test on the hypothesis that
the \twod{} provide no significant advantage over the \oned{} model.
We used $p=0.05$ as the test criteria.

\section{Results}

The phenological data is shown together with the root sampling dates on
table~\ref{tab:caas-phenology} and~\ref{tab:cau-phenology}.  Note that
the 2008 sampling were later than the 2007 sampling at the CAU site.
Final yield are shown on figure~\ref{tab:yield}, and are above Chinese
average in all cases except plot 21 for the CAAS site.

\begin{table}[htbp]
  \caption{CAAS phenology and root sampling.}
  \label{tab:caas-phenology}
  \begin{tabular}{ll}\\\hline
2008-03-15	& Sprouting/Bud development\\
2008-03-25	& Emergence\\
2008-04-10	& Tuber initiation\\
2008-04-15	& First individual buds\\
2008-04-20	& Beginning of flowering\\
2008-05-20	& First berries visible\\
2008-06-03	& Beginning of leaf yellowing\\
2008-06-10   & \textbf{Root density sample}\\
2008-06-26	& Final harvest\\
  \end{tabular}
\end{table}

\begin{table}[htbp]
  \caption{CAU phenology and root sampling.}
  \label{tab:cau-phenology}
  \begin{tabular}{lll}\\\hline
2007-04-20	& 2008-03-10	& Leaf development\\
2007-05-10	& 2008-04-27	& Leaf development\\
2007-05-25	& 2008-05-12	& Leaf development\\
2007-05-30	& 	& \textbf{Root density sample}\\
2007-06-08	& 2008-05-21	& Inflorescence emergence\\
2007-06-11	& 2008-05-25	& Flowering\\
2007-06-14	& 2008-05-29	& Development of fruit\\
2007-06-16	& 2008-06-07	& Inflorescence emergence\\
2007-06-20	& 2008-06-12	& Flowering\\
	& 2008-06-15 & \textit{\textbf{Root density sample}}\\
2007-06-23	& 2008-06-20	& Development of fruit\\
2007-06-27	& 2008-06-30	& Inflorescence emergence\\
2007-07-01	& 2008-07-03	& Flowering\\
2007-07-05	& 	& \textbf{Root density sample}\\
2007-07-10	& 2008-07-16	& Ripening of fruit and seed\\
2007-07-20	& 2008-07-24	& Ripening of fruit and seed\\
	& 2008-07-24 & \textit{\textbf{Root density sample}}\\
2007-07-25	& 2008-08-01	& Ripening of fruit and seed\\
  \end{tabular}
\end{table}

\begin{table}[htbp]
  \caption{Final yield.  All numbers are wet weight.  The ``China''
    number is the average Chinese yield that year, as reported by FAO 
    (\url{http://faostat.fao.org/}).  The ``Average'' number is the 
    average yield for the listed plots.  Yield information is missing 
    for CAU 2008 plot 15.}
  \label{tab:yield}

  \begin{tabular}{ll|ll|ll}\\\hline
\multicolumn{2}{c|}{CAAS 2008} & \multicolumn{2}{c|}{CAU 2007} & \multicolumn{2}{c}{CAU 2008} \\
Plot	& Mg/ha	& Plot	& Mg/ha	& Plot	& Mg/ha\\\hline
China	& 14.8	& China	& 23.1	& China	& 23.3\\\hline
Average	& 19.9	& Average	& 44.3	& Average	& 46.9\\
3	& 20.6	& 1	& 41.8	& 3	& 45.1\\
6	& 23.2	& 2	& 43.5	& 6	& 41.8\\
9	& 18.3	& 3	& 51.5	& 9	& 48.8\\
12	& 20.0	& 4	& 41.7	& 12	& 51.9\\
15	& 18.0	& 5	& 43.0	& 15	& n/a\\
18	& 22.4	& 	& 	& 	& \\
21	& 14.4	& 	& 	& 	& \\
24	& 22.5	& 	& 	& 	& 
  \end{tabular}
\end{table}

Table~\ref{tab:caas-ftest} and~\ref{tab:cau-ftest} shows the best fit
root parameters for the \twod{} root density model, as well as the
coefficient of determination (R$^2$) for both the \twod{} and \oned{}
models, and the F-test value.  The horizontal root zone radius is the
parameter that shows the largest variation between treatments, and
root zone depth is the parameter that shows least variation.  With the
exception of the early 2008 sampling of CAU plot 12, the \twod{} R$^2$
is always above 0.55, with a median of 0.86.  The potato (CAAS) tend
to have a wider root zone and more root mass than the tomato (CAU),
but no clear difference in root zone depth.  The \oned{} R$^2$ were
much lower (less than 0.20) for the plots where the estimated \twod{}
root zone radius was less than half the distance between rows.  The
F-test showed that the \twod{} model gave a significant better fit
than the \oned{} model for all the aggregate datasets, half the CAAS
datasets, and all but but three of the CAU datasets.

\begin{table}[htbp]
  \caption{CAAS model fit.  The best fit for root zone depth ($d_c$),
    root zone diameter ($2\; w_c$), and total root dry matter ($M_r$) is show
    for each plot.  The `All' plot indicate a dataset containing all
    the individual plots.  Furthermore, the R$^2$ for both the \oned{}
    and \twod{} fits are listed, as well as an F test indicating whether
    the \twod{} model provides a significantly better fit.} 
\label{tab:caas-ftest}

\begin{tabular}{rrrrrrrrr}\\\hline
Plot	& $d_c$ [cm]	& $2\;w_c$ [cm]	& $M_r$ [Mg/ha]	& \twod{} R$^2$	& \oned{} R$^2$	& F	& F (0.05)\\\hline
All	& 93	& 216	& 0.65	& 0.61	& 0.60	& \textbf{5.42}	& 3.88\\
3	& 91	& 265	& 1.22	& 0.93	& 0.92	& 2.80	& 4.14\\
6	& 81	& 743	& 1.11	& 0.88	& 0.88	& 0.03	& 4.14\\
9	& 106	& 151	& 0.52	& 0.94	& 0.89	& \textbf{25.23}	& 4.14\\
12	& 88	& 151	& 0.39	& 0.94	& 0.91	& \textbf{17.75}	& 4.14\\
15	& 87	& $\infty$	& 0.91	& 0.75	& 0.75	& 0.00	& 4.14\\
18	& 87	& 82	& 0.22	& 0.88	& 0.79	& \textbf{23.80}	& 4.14\\
21	& 88	& 99	& 0.28	& 0.86	& 0.79	& \textbf{16.53}	& 4.14\\
24	& 132	& 164	& 0.74	& 0.73	& 0.69	& 4.10	& 4.14\\
\end{tabular}
\end{table}

\begin{table}[htbp]
  \caption{CAU model fit.  The best fit for root zone depth ($d_c$),
    root zone diameter ($2 w_c$), and total root dry matter ($M_r$) is show
    for each plot.  The `All' plot indicate a dataset containing all
    the individual plots.  Furthermore, the R$^2$ for both the \oned{}
    and \twod{} fits are listed, as well as an F test indicating whether
    the \twod{} model provides significantly better fit.} 
  \label{tab:cau-ftest}

  \begin{tabular}{rrrrrrrrr}\\\hline
Plot	& $d_c$ [cm]	& $2 w_c$ [cm]	& $M_r$ [Mg/ha]	& \twod{} R$^2$	& \oned{} R$^2$	& F	& F (0.05)\\\hline
\multicolumn{8}{c}{2007-05-30} \\
All	& 78	& 96	& 0.18	& 0.63	& 0.49	& \textbf{43.69}	& 3.93\\
1	& 56	& 48	& 0.11	& 0.84	& 0.40	& \textbf{53.28}	& 4.38\\
2	& 83	& 75	& 0.17	& 0.76	& 0.46	& \textbf{23.72}	& 4.38\\
3	& 83	& 271	& 0.23	& 0.84	& 0.83	& 1.00	& 4.32\\
4	& 90	& 259	& 0.24	& 0.83	& 0.82	& 1.09	& 4.32\\
5	& 72	& 47	& 0.15	& 0.96	& 0.31	& \textbf{309.76}	& 4.38\\\hline
\multicolumn{8}{c}{2007-07-05}\\
All	& 112	& 64	& 0.33	& 0.67	& 0.19	& \textbf{172.94}	& 3.92\\
1	& 85	& 39	& 0.20	& 0.88	& 0.16	& \textbf{121.21}	& 4.32\\
2	& 94	& 40	& 0.22	& 0.94	& 0.15	& \textbf{276.04}	& 4.32\\
3	& 137	& 97	& 0.51	& 0.76	& 0.34	& \textbf{36.52}	& 4.32\\
4	& 126	& 89	& 0.50	& 0.84	& 0.32	& \textbf{67.25}	& 4.32\\
5	& 93	& 39	& 0.21	& 0.94	& 0.15	& \textbf{265.82}	& 4.35\\\hline
\multicolumn{8}{c}{2008-06-15}\\
All	& 124	& 63	& 0.27	& 0.56	& 0.14	& \textbf{85.36}	& 3.95\\
3	& 72	& 31	& 0.14	& 0.89	& 0.14	& \textbf{96.81}	& 4.54\\
6	& 99	& 69	& 0.23	& 0.86	& 0.37	& \textbf{54.02}	& 4.54\\
9	& 144	& 63	& 0.30	& 0.83	& 0.15	& \textbf{58.11}	& 4.54\\
12	& 245	& 122	& 0.63	& 0.30	& 0.11	& 4.09	& 4.54\\
15	& 125	& 44	& 0.24	& 0.85	& 0.12	& \textbf{70.26}	& 4.54\\\hline
\multicolumn{8}{c}{2008-07-24}\\
All	& 108	& 45	& 0.25	& 0.92	& 0.13	& \textbf{894.32}	& 3.95\\
3	& 120	& 42	& 0.28	& 0.95	& 0.06	& \textbf{260.20}	& 4.60\\
6	& 95	& 44	& 0.22	& 0.96	& 0.17	& \textbf{281.54}	& 4.54\\
9	& 115	& 53	& 0.28	& 0.94	& 0.15	& \textbf{184.84}	& 4.54\\
12	& 101	& 40	& 0.23	& 0.96	& 0.17	& \textbf{282.93}	& 4.60\\
15	& 113	& 47	& 0.26	& 0.89	& 0.11	& \textbf{101.73}	& 4.60\\
  \end{tabular}
\end{table}

The aggregate datasets have an R$^2$ for the \twod{} model are lower
than for the individual datasets, except for the two 2008 CAU
samplings.  We see on figure~\ref{fig:caas2008}, \ref{fig:cau2007},
and~\ref{fig:cau2008} that the CAU 2008 aggregate datasets also have
the lowest standard deviation between plots.  The CAAS 2008 aggregate
dataset has the highest standard deviation, reflecting the fact that
half the model parameter fits shows a wide root zone, and the other
half a narrow root zone.  Both plots using subsoil drip irrigation and
partial rootzone drying for the CAAS site show a wide root zone, and
both plots with subsoil drip and ``normal'' deficit irrigation show a
relatively narrow root zone.  However, one of the fully irrigated
subsoil plots has a wide root zone, and the other a narrow root zone.
The three samplings at the CAU site that shows a wide root zone
represent both irrigation methods and both irrigation strategies.  The
two 2008 sampling at CAU shows a tendency of the roots to concentrate
near the center of the row, as illustrated on
figure~\ref{fig:tworows}.  The tendency is partly supported by the
2007 data, where the root zone sampling was performed earlier
(table~\ref{tab:cau-phenology}).

\begin{figure}[htbp]
  \input{CAAS-2008-0}\\
  \input{CAAS-2008-18}\\
  \input{CAAS-2008-37}
  \caption{Estimated and observed root density for the 2008 CAAS root
    density sampling.  The top graph show data from the plant row,
    the center graph 18.75 cm from the row, and the bottom graph 30
    cm from the plant rows.  The y-axes represent depth below the
    original ground level in cm, the x-axes represent root 
    density in cm/cm$^3$.  The curvy lines are the modelled root
    density.  The error bars represent mean and standard deviation for
    all treatments}
  \label{fig:caas2008}
\end{figure}

\begin{figure}[htbp]
  \input{CAU-2007a-0}\input{CAU-2007b-0}\\
  \input{CAU-2007a-15}\input{CAU-2007b-15}\\
  \input{CAU-2007a-30}\input{CAU-2007b-30}
  \caption{Estimated and observed root density for the 2007 CAU root
    density sampling.  The left side is the May 30 sampling, the right
    side the July 5 sampling.  The top graphs show data from the plant
    row, the center graphs 15 cm from the row, and the bottom graphs
    30 cm from the plant rows.  The y-axes represent depth below the
    original ground level in cm, the x-axes represent root density in
    cm/cm$^3$.  The curvy lines are the modelled root density.  The
    error bars represent mean and standard deviation for all
    treatments, including measurements at both sides of the row.}
  \label{fig:cau2007}
\end{figure}

\begin{figure}[htbp]
  \input{CAU-2008a-0}\input{CAU-2008b-0}\\
  \input{CAU-2008a-15}\input{CAU-2008b-15}\\
  \input{CAU-2008a-30}\input{CAU-2008b-30}
  \caption{Estimated and observed root density for the 2008 CAU root
    density sampling.  The left side is the June 15 sampling, the
    right side the July 24 sampling.  The top graphs show data from
    the plant row, the center graphs 15 cm from the row, and the
    bottom graphs 30 cm from the plant rows.  The y-axes represent
    depth below the original ground level in cm, the x-axes represent
    root density in cm/cm$^3$.  The curvy lines are the modelled root
    density.  The error bars represent mean and standard deviation for
    all treatments, including measurements at both sides of the row.}
  \label{fig:cau2008}
\end{figure}

\begin{figure}[htbp]
  \input{CAU2008a}\\
  \input{CAU2008b}
  \caption{Estimated root zone for the CAU site.  The top graph shows
    the 15-06-2008 root density distribution and the bottom graph the
    24-07-2008 root density distribution. Two interacting rows at x =
    0 and 80 cm are shown.  The lines represent root density contours
    in cm/cm$^3$.  The axes show height above ground and horizontal
    position relative to the first row in cm.}
  \label{fig:tworows}
\end{figure}

\section{Discussion and concluding remarks}

A major source of uncertainty is the mapping of the 3D ridge geometry
to a 2D flat surface.  For both sites, we ignore the root mass that
are placed in the top of the ridge.  We also ignore that the ridge
valley does not contain any soil.  For the CAU site we furthermore
ignore the third dimension, using only measurements at y = 0 cm where
the root density is highest.  These systematic errors affect the
estimate of all three parameters, but especially the root dry mass.
However, a comparison between plots at the same site should still be
valid.

The statistical analysis (table~\ref{tab:caas-ftest}
and~\ref{tab:cau-ftest}) shows us that the presented \twod{} model
constitute a significant improvement over the \oned{} model,
especially in the cases where the root zone from the individual row is
narrow.  Furthermore, visual inspection of observed data vs.\@
predicted data (figure~\ref{fig:caas2008},~\ref{fig:cau2007},
and~\ref{fig:cau2008}) shows that the model provides a good match for
both the horizontal and vertical changes in root density.  Based on
this we believe that the model will be useful for the intended purpose
of simulating soil water dynamics in row crops.

Additionally, we believe that conversion of root density measurements
to general crop parameters (root dry mass, root zone depth, and root
zone radius) will be useful when analysing the effect of treatments,
climate, or soil type on root development.  Such an analysis is
outside the scope of this paper, but the numbers presented on
table~\ref{tab:caas-ftest} and~\ref{tab:cau-ftest} do not give a clear
indication of the difference in treatments being the main factor in
the difference between plots for the present experiment.

\section{Software and data availability}

Software to estimate root mass and root zone depth from \oned{} root
density data, and in addition root zone width from \twod{} root
density data, can be found at \url{http://www.daisy-model.org/} (look
under \texttt{Wiki}, \texttt{\textsc{gp2d}}).  Also supported is
estimating root density distribution from root mass, root zone depth,
and optionally, root zone width.  The program code is written in
\cplusplus{} and is covered by on open source license (\textsc{gnu
  lgpl}), so it can be freely incorporated in other models.

The root density data from the two field sites will be made available
at \url{http://www.safir4eu.org/} from MNA-TODO-XXXX-DATE
(look under \texttt{Trials}).

\section*{Acknowledgements}

The research was partly funded by the EU-DG XII (FP6 Contract No. CT
Food-2005-023168 SAFIR).

\bibliographystyle{elsart-harv}
\bibliography{../daisy}

\vfill{}\section*{List of symbols}

\begin{tabular}{lll}
Symbol  & Unit    & Description\\\hline
$a$     & L$^{-1}$ & Root density distribution parameter\\
$a_z$   & L$^{-1}$ & Vertical root density distribution parameter\\
$a_x$   & L$^{-1}$ & Horizontal root density distribution parameter\\
$d_a$   & L       & Soil limited root depth\\
$d_c$   & L       & Crop potential root depth\\
$d_s$   & L       & Soil maximum root depth\\ 
$k^\prime$   &         & Soil root limit factor\\ 
$l_r$   & L/L$^2$ & Total root length per area\\ 
$l_R$   & L/L     & Total root length per length of row on one side\\ 
$L_0$   & L/L$^3$ & Average root density at soil surface\\
$L_{0,0}$& L/L$^3$ & Root density in row at soil surface\\
$L_m$   & L/L$^3$ & Minimal root density\\ 
$L_z$   & L/L$^3$ & Root density at soil depth $z$\\
$L_z^\prime$ & L/L$^3$ & Soil limited root density at soil depth $z$\\ 
$L_{z,x}$& L/L$^3$ & Root density at soil depth $z$ and distance $x$ from row\\
$L^*_{z,x}$& L/L$^3$ & Root density from multiple rows\\
$M_r$   & M/L$^2$& Total root dry matter\\ 
$Q$     &         & Substitution variable\\
$R$     & L       & Distance between rows\\
$S_r$   & L/M    & Specific root length\\ 
$W$     &         & Lambert-W function\\ 
$w_c$     & L     & Horizontal root zone radius\\
$x$     & L       & Horizontal distance from row \\
$z$     & L       & Soil depth \\
\end{tabular}
\vfill{}

\pagebreak[4]
\section*{Questions}

\begin{itemize}
\item Where to publish?
  \begin{itemize}
  \item Environmental Software \& Modelling\\
    The motivation is agricultural.
  \item Agricultural Water Management\\
    They hate math and models.
  \item Plant \& Soil
    Seem to like math and models
  \end{itemize}
\item MNA: Who should be co-authors from China?
\item I think most of the text is not stringent or ``to-the-point''
  enough.  Not sure what to do about it.
\item Should the article be shorter?  If so, what to cut?
\item One could drop figure~\ref{fig:W} and~\ref{fig:Q} and
  section~\ref{sec:W} and~\ref{sec:Q} and just say
  ``Equation~\ref{eq:Lambert}/\ref{eq:logsquare} can be solved
  numerically to find W/Q'' to save space.  I wouldn't like it though.
\item Drop table~\ref{tab:yield}? (Yield)
\item Drop table~\ref{tab:caas-phenology} and~\ref{tab:cau-phenology}?
  (Phenology)
\item MNA: Should we have more details on treatments and experimental
  protocol?
\item Should we have more references in the introduction?  We can add
  lots of references on PRD and on various root models
\item CAU: What is the ridge geometry?  And zero level?
\item I'm still uncertain on the F-test is done right.  I followed the
  explanation in Wikipedia, which was the only one I could understand.
  Should I have more details?  Like dimensions for the F-distribution?
\item Should RMSE numbers be included?
\item MNA: PRD references?
\item SHA (and MMO): Notation: $L\,(z)$ or $L_z$?
\item MNA: When will SAFIR root data become public?
\item Are the \cite{richardseq}, \cite{nelder1965simplex},
  \citep{euler83}, and \cite{lambert58} citations too silly?
\end{itemize}

\end{document}
