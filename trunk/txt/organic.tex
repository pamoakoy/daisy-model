\documentclass[a4paper]{article}
\usepackage{verbatim}
\newcommand{\daisy}{Daisy}
\newcommand{\Daisy}{Daisy}
\newcommand{\cplusplus}%
{{\leavevmode{\rm{\hbox{C\hskip -0.1ex\raise 0.5ex\hbox{\tiny ++}}}}}}
\newcommand{\Cplusplus}{\cplusplus}
\newcommand{\mshe}{Mike/\textsc{she}}
\newcommand{\wintel}{\texttt{win32}}
\newcommand{\dll}{\textsc{dll}}
\newcommand{\Dll}{\textsc{Dll}}
\newcommand{\gui}{\textsc{gui}}
\newcommand{\Gui}{\textsc{Gui}}
\newcommand{\unix}{Unix}
\newcommand{\dhi}{\textsc{dhi}}
\newcommand{\Dhi}{\textsc{Dhi}}
\newcommand{\api}{\textsc{api}}
\newcommand{\Api}{\textsc{Api}}
\newcommand{\lai}{\textsc{lai}}
\newcommand{\Lai}{\textsc{Lai}}
%\newcommand{\url}[1]{\linebreak[4]\texttt{<URL:#1>}}

%%% Local Variables: 
%%% mode: latex
%%% TeX-master: t
%%% End: 

\usepackage[latin1]{inputenc}

\begin{document}

\title{\textbf{Initializing organic matter pools}}
\author{
    Per Abrahamsen, S�ren Hansen and Henrik Svendsen\\
    The Royal Veterinary and Agricultural University\\
    Department of Agricultural Sciences\\
    Laboratory for Agrohydrology and Bioclimatology
}
\date{\today}
\maketitle

\begin{abstract}
Organic matter models using multiple pools for soil biomass and soil
organic matter have proved able to simulate both short term and long
term change in humus content of agricultural soil.  However, these
pools do not correspond to measurable physical quantities, and are
therefore difficult for a non-expert to understand and use, and
fragile with respect to changes in the model.  An alternative
sometimes used is to assume the organic matter is in a state of
equilibrium.  Unfortunately, the time to reach equilibrium is best
measured in centuries, so this assumptions is unlikely to hold true.
For the same reason, the use of a warmup period cannot replace the
need for a good initial partitioning.  In this paper we propose a
milder assumption, namely a quasi-equilibrium where all but the
slowest pool is in equilibrium with the amount of carbon input.  This
assumption allows the non-expert user to initialize the model, and is
robust with regard to model changes.
\end{abstract}

\tableofcontents
\pagebreak

\section{Soil organic matter modelling}

SH: referencer til vigtigste org.matter modeller.

It is conventional to divide the organic matter in the soil into three
fractions.  First we have the freshly entered organic matter, which
can still be traced back to its origin.  For a cultivated soil, this
might include organic fertilizer, crop residuals, including
rhizodeposition and dead leaves incorporated to the soil by
earthworms.  This fraction is conventionally called \emph{added
  organic matter}, or \textsc{aom}.  Then we have the \emph{soil
  microbial biomass}, or \textsc(smb), the living part of the organic
matter, excluding roots.  Finally we have the humus, the \emph{soil
  organic matter} (\textsc{som}), which can no longer be traced back
to its origin.  The dynamics of the system consist of input in the
form of new added organic matter, and turnover in the form the soil
biomass eating the other matter (and itself).

In numeric models, these fractions may be further divided into smaller
pools, the content of each pool assumed by the model to have uniform
properties, e.g.\ similar turnover rate and the same C/N ratio.
Having two \textsc{som} and two \textsc{smb} pools allows a well
calibrated model to capture both the short term (see XXX) and long
term (XXX) dynamics of the system.  If you are only interested in
accuracy on a single time scale, less pools may be needed.  The number
of \textsc{aom} pools needed also depends on what you are simulating.
For batch experiments and some long time scenarios, a single pool may
be enough.  To simulate input sources from real farming, two pools per
source are in general needed per source.

In several comparisons (see XXX, YYY, ZZZ), \daisy{} (see XYZZY) has
been among the best models for both short and long term predictions of
soil matter, and \daisy{} will be the reference for the rest of this
article.  Apart from soil organic matter, \daisy{} also simulate a
number of other processes, such as water, heat and nitrogen dynamics
in the soil, as well as biocliamte, crop development and management.

\section{Evolution of the Daisy model}

The purpose of the \daisy{} model is to simulate real farming
practice, both on the short and the long term.  Thus, we get a system
with two pools for each of \textsc{som} and \textsc{smb}, and two pools for
each type fertilizer applied or crop residual left on the field.  The
original model is depicted in figure~\ref{fig:om1}.

FIG{om1}: The original Daisy OM parameterization.

PA: turnover rates, fractions, efficiency, maintenance, (abiotic factors).

The \daisy{} software allows the user to adjust all of these
parameters.  It also allows the user to specify the number of \textsc{som}
and \textsc{smb} pools, as well as the number of \textsc{aom} pools for each
fertilizer application and each crop residual type.  It thus provides
a good basis for experimentation.

The first such experimental change that made it back into \daisy{} was
made by Torsten M�ller (see XXX 19??), who adjusted the turnover rates
of the \textsc{smb} pools so the biomass content of the soil better
matched the levels measured at the fields.  The change did not affect
the long time dynamics of the systen.  The second such change was made
by Anders Sanders (see XXX 19??).  The was a complete recalibration
that took into account the carbon input from rhizodepositions.  This
change was more radical, involving both turnover rates and directions
of flow, and made the system much more adaptable to new levels of
input, another effect which has also been observed emperically (see
DJF repport on humus levels).

The current calibration is depicted in~\ref{fig:om2}.  The purpose of
the SOM3 pool in that figure will be explained later.

FIG{om2}: Ny model.

Some of the experimental changes affecting soil organic matter that we
are currently working on, and which may be included in future versions
of \daisy{}, are dissolved organic matter (see BGJ), another
recalibration based on a larger experimental database and with special
focus on the effect of clay (see Biomod/Bj�rn/AP), and some wishlist
items includes phosphor dynamics and a calibration for forrest soil.
The conclusion is that model has changed in the past, and will
continue to change in the future.

\section{Expanding user base}

The original development of \daisy{} was founded by the national
environmental agency, and the first non-research use was made by
expert users who evaluated the national environmental plans for
limiting nitrogen contamination of groundwater and surface water (see
VMPuse).  Since then, \daisy{} has been used in other parts of the
environmental and agricultural agencies (see NORVANA XXX), as well as
increasingly by local authorities for evaluating regional plans (see
�rhus Amt og XXX), and recently also for evaluation of applications
for changed farm practice at an individual farm level (see VVM,
PLANTINFO).  For the last case, \daisy{} has also been used by
consulting agencies representing the other side, namely farmers
seeking permission to change farming practice.

The effect of this increased use has been that we no longer can rely
on the users being experts.  Therefore, a project was started in order
to provide standardized procedures for setting \daisy{} simulations
(see XXX).  As part of this project, most components of \daisy{} was
evaluated in order to find parameters and submodels who could be
initialized automatically from default values, or from other
parameters.  These parameters and submodels are still available to the
expert user, but will be hidden for the user who do not have the
expertice or data available.  The result is that the software now
appears much simpler to the user, even if the substantial changes to
the model have been made.

The partition of the organic matter into the various soil pools have
been a particular stumbling block for most users, partly because it
does not reflect an easily identificable property of the system being
modellen.  It is, to a large degree, a modelling artifact.

It is common to use an equilibrium assumption for initialising models
where measurements are not available.  For example, in \daisy{} we
assume that the soil water in the bottom of the unsaturated zone is in
equilibrium with the groundwater, as specified by Darcy's equation.
There are two reasons equilibrium is often a good assumption.  For
some systems, it will be correct most of the time, because they quicly
reach equilibrium compared to the externaly imposed changes.  In other
cases, the dynamic of the system quickly dominate over the initial
condition, so an equilibrium is good for giving a reasonable initial
state.

As part of Sander's recalibration work, he also found that the ratio
between the \textsc{som1} and \textsc{som2} for a system in equilibrium will
49\% vs 51\%.  Since \daisy{} after his recalibration required users
to readjust the partitioning between \textsc{som1} and \textsc{som2}, there
was a strong tendency that users just picked the equilibrium values.

\section{Long term simulations}

As part of the standardization project mentioned earlier (see XXX),
Henrik Svendsen used \daisy{} to simulate the effect on soil organic
matter of changing farming practice.  Figure~\ref{fig:clayom} shows a
system that goes from equilibrium at a high level of carbon input, to
equilibirum at a low level of carbon input.  Danish farming practice
has been used, as well as Danish climate and soil.  The soil has a for
Danish conditions high clay content (10\%).
Figure~\ref{fig:clayration} shows the \textsc{som1}/\textsc{som2}
ratio, which is can be used for initializing \daisy{}.  Finally,
figure~\ref{fig:sandratio} shows the ration for simulation on a coarse
sandy soil going from a low to high input equilibrium.

FIG{fig:clayom}: Long term simulation of organic matter content from
changing from high to low input on a clayish loam.

FIG{fig:clayratio}: Long term simulation of \textsc{som1}/\textsc{som2} ratio
from changing from high to low input on a clayish loam.

FIG{fig:sandratio}: Long term simulation of \textsc{som1}/\textsc{som2} ratio
from changing from low to high input on a coarse sandy soil

As can be seen, the time to reach a new equilibirum after a change in
farming practice is measured in centuries, which makes it safe to
assume that no Danish farming land is in equilibrium.  At such a time
scale, not only farming practice but also climate is going to change.

As a conclusion, we need a way to initialize the soil organic matter
that is robust with regard to changes in the model or model
parameterization, and at the same time expressible in terms of
physical properties the user can understand.  And we cannot use the
most obvious initialization condition, namely an equilibrium
assumption.

\section{Equations}

The the sytem of equations.

4 equations, nine unknown.

total humus known, 5 equations (still nine unknowns)

assume equilibrium, 5 eq, 5 unknown 


Assume quasi-equilibrium with input.

Attempt to provoke background mineralization.

Assumptions and limitations.

-- clay, T og h

T=10 dg

-- top horizons.

\section{Results}

Validation.  Lange tidsserier af Per og S�ren.

FIG: SOM1/SOM2 ratio (sim + eq)  after change in management high->low, clay
FIG: SOM1/SOM2 ratio (sim + eq)  after change in management low->high,
sand

lim 70\%

Brug i standardisering.

\end{document}

%%% Local Variables: 
%%% mode: latex
%%% TeX-master: t
%%% End: 
