\documentclass[a4paper]{article}
\usepackage{verbatim}
\newcommand{\daisy}{Daisy}
\newcommand{\Daisy}{Daisy}
\newcommand{\cplusplus}%
{{\leavevmode{\rm{\hbox{C\hskip -0.1ex\raise 0.5ex\hbox{\tiny ++}}}}}}
\newcommand{\Cplusplus}{\cplusplus}
\newcommand{\mshe}{Mike/\textsc{she}}
\newcommand{\wintel}{\texttt{win32}}
\newcommand{\dll}{\textsc{dll}}
\newcommand{\Dll}{\textsc{Dll}}
\newcommand{\gui}{\textsc{gui}}
\newcommand{\Gui}{\textsc{Gui}}
\newcommand{\unix}{Unix}
\newcommand{\dhi}{\textsc{dhi}}
\newcommand{\Dhi}{\textsc{Dhi}}
\newcommand{\api}{\textsc{api}}
\newcommand{\Api}{\textsc{Api}}
\newcommand{\lai}{\textsc{lai}}
\newcommand{\Lai}{\textsc{Lai}}
%\newcommand{\url}[1]{\linebreak[4]\texttt{<URL:#1>}}

%%% Local Variables: 
%%% mode: latex
%%% TeX-master: t
%%% End: 

\usepackage[latin1]{inputenc}

\begin{document}

\title{\textbf{Initializing organic matter pools}}
\author{
    Per Abrahamsen, S�ren Hansen and Henrik Svendsen\\
    The Royal Veterinary and Agricultural University\\
    Department of Agricultural Sciences\\
    Laboratory for Agrohydrology and Bioclimatology
}
\date{\today}
\maketitle

\begin{abstract}
Organic matter models using multiple pools for soil biomass and soil
organic matter have proved able to simulate both short term and long
term change in humus content of agricultural soil.  However, these
pools do not correspond to measurable physical quantities, and are
therefore difficult for a non-expert to understand and use, and
fragile with respect to changes in the model.  An alternative
sometimes used is to assume the organic matter is in a state of
equilibrium.  Unfortunately, the time to reach equilibrium is best
measured in centuries, so this assumptions is unlikely to hold true.
For the same reason, the use of a warmup period cannot replace the
need for a good initial partitioning.  In this paper we propose a
milder assumption, namely a quasi-equilibrium where all but the
slowest pool is in equilibrium with the amount of carbon input.  This
assumption allows the non-expert user to initialize the model.
\end{abstract}

\tableofcontents
\pagebreak

\section{Soil organic matter modelling}

SH: referencer til vigtigste org.matter modeller.

It is conventional to divide the organic matter in the soil into three
fractions.  First we have the freshly entered organic matter, which
can still be traced back to its origin.  For a cultivated soil, this
might include organic fertilizer, crop residuals, including
rhizodeposition and dead leaves incorporated to the soil by
earthworms.  This fraction is conventionally called \emph(added
organic matter), or \sc(aom).  Then we have the soil microbial
biomass, or \sc(smb), the living part of the organic matter, excluding
roots.  Finally we have the humus, the soil organic matter (\sc(som)),
which can no longer be traced back to its origin.  The dynamics of the
system consist of input in the form of new added organic matter, and
turnover in the form the soil biomass eating the other matter (and
itself) and dying.

In numeric models, these fractions may be further divided into smaller
pools, the content of each pool assumed by the model to have uniform
properties, e.g.\ similar turnover rate and the same C/N ratio for
models that are concern with nitrogen.  Having two \sc{som} and two
\sc{smb} pools allows a well calibrated model to capture both the
short term (see XXX) and long term (XXX) dynamics of the system.  If
you are only interested in accuracy on a single time scale, less pools
may be needed.  The number \sc{aom} pools needed also depends on what
you are simulating.  For batch experiments and some long time
scenarios, a single pool may be enough.  To simulate input sources
from real farming, two pools per source are in general needed for each
source.  

PA: sammenligninger, Daisy er bedst!

\section{Evolution of the Daisy model}

PA: Daisy kan en masse.

The purpose of the \daisy{} model is to simulate real farming
practice, both on the short and the long term.  Thus, we get a system
with two pools for each of \sc{som} and \sc{smb}, and two pools for
each type fertilizer applied or crop residual left on the field.  The
original model is depicted in figure~\ref{fig:om1}.

FIG{om1}: The original Daisy OM parameterization.

PA: turnover rates, fractions, efficiency, maintenance, (abiotic factors).

The \daisy{} software allows the user to adjust all of these
parameters.  It also allows the user to specify the number of \sc{som}
and \sc{smb} pools, as well as the number of \sc{aom} pools for each
fertilizer application and each crop residual type.  It thus provides
a good basis for experimentation.

The first such experimental change that made it back into \daisy{} was
made by Torsten M�ller (see XXX 19??), who adjusted the turnover rates
of the \sc{smb} pools so the biomass content of the soil better
matched the levels measured at the fields.  The change did not affect
the long time dynamics of the systen.  The second such change was made
by Anders Sanders (see XXX 19??).  The was a complete recalibration
that took into account the carbon input from rhizodepositions.  This
change was more radical, involving both turnover rates and directions
of flow, and made the system much more adaptable to new levels of
input, another effect which has also been observed emperically (see
DJF repport on humus levels).

Some of the experimental changes affecting soil organic matter that we
are currently working on, and which may be included in future versions
of \daisy{}, are dissolved organic matter (see BGJ), a 

PA: Fremtidige �ndringer DOM, Biomod, P, forrest

PA: konklusion: model �ndrer sig

\section{Expanding user base}

PA: hvem bruger det?

It is common to use an equilibrium assumption for initialising models
where measurements are not available.  For example, in \daisy{}we
assume that the soil water in the bottom of the unsaturated zone is in
equilibrium with the groundwater, as specified by Darcy's equation.
There are two reasons this is often a good assumption.  For some
systems, it will be correct most of the time, because they quicly
reach equilibrium compared to the externaly imposed changes.  In other
cases, the dynamic of the system quickly dominate over the initial
condition, so an equilibrium is good for giving a reasonable initial
state.

As part of Sander's recalibration work, he also found that the ratio
between the \sc{som1} and \sc{som2} for a system in equilibrium will
49\% vs 51\%.   

PA: sanders equilibrium

PA: tove og kresten

PA: the standardization project

PA: konklusion: behov for forst�elige og model-uafh�ngige begreber,
equilibrium assumption uholbar.

\section{Equations}

The the sytem of equations.

Assume quasi-equilibrium with input.

Attempt to provoke background mineralization.

Assumptions and limitations.

-- clay, T og h

\section{Results}

Validation.  Lange tidsserier af Per og S�ren.

Brug i standardisering.

\end{document}

%%% Local Variables: 
%%% mode: latex
%%% TeX-master: t
%%% End: 
