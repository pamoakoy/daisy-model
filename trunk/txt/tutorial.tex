%%% tutorial.tex -- A tutorial for the Daisy crop/soil system.

\documentclass[a4paper,11pt]{article}
\usepackage{a4}
\usepackage{hyperref}

\bibliographystyle{apalike}

\newcommand{\daisy}{Daisy}
\newcommand{\Daisy}{Daisy}
\newcommand{\cplusplus}%
{{\leavevmode{\rm{\hbox{C\hskip -0.1ex\raise 0.5ex\hbox{\tiny ++}}}}}}
\newcommand{\Cplusplus}{\cplusplus}
\newcommand{\mshe}{Mike/\textsc{she}}
\newcommand{\wintel}{\texttt{win32}}
\newcommand{\dll}{\textsc{dll}}
\newcommand{\Dll}{\textsc{Dll}}
\newcommand{\gui}{\textsc{gui}}
\newcommand{\Gui}{\textsc{Gui}}
\newcommand{\unix}{Unix}
\newcommand{\dhi}{\textsc{dhi}}
\newcommand{\Dhi}{\textsc{Dhi}}
\newcommand{\api}{\textsc{api}}
\newcommand{\Api}{\textsc{Api}}
\newcommand{\lai}{\textsc{lai}}
\newcommand{\Lai}{\textsc{Lai}}
%\newcommand{\url}[1]{\linebreak[4]\texttt{<URL:#1>}}

%%% Local Variables: 
%%% mode: latex
%%% TeX-master: t
%%% End: 


\begin{document}

\section*{\Daisy{} Tutorial}

\begin{abstract}
  Welcome to the \daisy{} tutorial.  This document is intended to
  teach the use of \daisy{} by example.  It is not by any stretch of
  the imagination a reference to all the features of \daisy{}, for
  that see~\cite{dina81}.  It also does not attempt to explain the
  mathematical and physical models implemented by the \daisy{}
  software, for that see~\cite{daisy-def} and~\cite{daisy-new}.
  
  I hope that you will able to get started with \daisy{} from these
  examples, and when you need to go beyond will be familiar enough
  with the system to make good use of the reference manual.  \Daisy{}
  has an amazing number of adjustable dials and wheels, only a tiny
  fraction of which are explained here.  However, they are all at
  least mentioned in the reference manual, which is mostly
  automatically generated from the \texttt{daisy} executable itself.
  
  If you have questions or suggestions to either the manual or the
  \daisy{} system itself, feel free to contact the authors at
  \href{mailto:daisy@dina.kvl.dk}{\texttt{$<$daisy@dina.kvl.dk$>$}}.
\end{abstract}

Per Abrahamsen, \today{}.

\tableofcontents

\section{Retrieving and installing the files}
\label{sec:install}

We start by retrieving the files.  With a web browser, go to the
address~\url{http://www.dina.kvl.dk/~daisy/}, the home page of the
\daisy{} project.  From there, you should download the following
items:
\begin{description}
\item[Daisy Description] For information about the mathematical and
  physical models in \daisy{}.
\item[Daisy Tutorial] You already got this, but if you are reading
  this from an old printout, you might want to get a newer version.
\item[Daisy Program Reference Manual] Sooner or later, you will need
  the reference manual.  I suggest that you get the \textsc{pdf}
  version (see under the ``Distribution'' header), which is both
  properly formatted (unlike the \textsc{html} version), easy to
  browse, and easy to print.
\item[command line executable] Download the version for the platform
  you use.  If you don't know what I mean by this, you are most likely
  using ``win32''.
\item[parameter library] This is a zip file containing containing
  standard parameterizations of e.g.\ crops.  You will need this.  You
  will also need a program to unzip the file, download it from a place
  like \url{http://www.tucows.com/} if you haven't already got it.
  Search for unzip.
\item[weather file] The weather file contain observations from the KVL
  research station in Taastrup, Denmark.  Most likely, you will
  eventually want weather data from closer to home, but this file will
  get you started.
\item[log file viewer] \Daisy{} stores the results of its simulations
  in log files (with the extension \texttt{.dlf}).
  \texttt{ShowDaisyOutput.exe} is a win32 program that will show the
  content of these log files graphically, it can even animate some of
  the data.
\end{description}

Store all the files in the same directory, unzip the parameter
library, open a shell or command prompt, and run the command
\texttt{daisy test.dai} to check that everything works.  It should
start printing dates to show the progress of the simulation.  After
simulation has finished, you can browse the `\texttt{daisy.log}' file
to see the output again.

\section{Example}

The \texttt{test.dai} file contain all the information about the
simulation.  We call such a file a \daisy{} setup file, or a
\texttt{.dai} file for short.  By convention, we give such files the
\texttt{.dai} extension, even though \daisy{} doesn't care.  Teaching
you to write \daisy{} setup files is the main purpose of this
tutorial.

First, let's examine line by line the content of the \texttt{test.dai}
file (see appendix~\ref{app:test} for the full version).  The first
lines are

\begin{verbatim}
  ;;; test.dai -- Sample file using the Daisy libraries.

  (description "Simulation for use in tutorial.")
\end{verbatim}

Semicolons are used for comments, everything from the semicolon to the
end of the line is ignored by \daisy{}.  This can (and should!) be
used to put information about the simulation there for the person
reading the file.  That person may very well be you, who some month
(or years) later have forgotten about the purpose of this particular
simulation.

For the same reason, providing a description as above is also a good
idea.  The description will be written to the log files produced by
\daisy{}, allowing you to later see what log files come from what
simulation.

\subsection{Input files}
\label{sec:ex-input}

\begin{verbatim}
  ;; Use standard parameterizations.
  (input file "tillage.dai")
  (input file "crop.dai")
  (input file "log.dai")
\end{verbatim}

Here, we tell \daisy{} to read three additional files:
\texttt{tillage.dai} containing descriptions of various standard
tillage operations, such as plowing and seed bed preparation;
\texttt{crop.dai} which contains descriptions of crops like spring
barley or rye; and \texttt{log.dai} with directives for \daisy{} to
produce files with information about various aspects of the
simulation, such as the nitrogen or water balance.  You can read more
about input files in section~\ref{sec:files}.

\subsection{Weather}

\Daisy{} need to know what the weather is during the simulation.  This
is usually read from a file.

\begin{verbatim}
  ;; Weather data.
  (weather default "weather.dwf")
\end{verbatim}

Here we specify where \daisy{} should find the weather data.  \Daisy{}
can read weather data in several different formats, the
\texttt{default} format is \texttt{.dwf}, short for \emph{D}aisy
\emph{W}eather \emph{F}ormat.  The format is specified in an appendix
 of \cite{dina81}.

\subsection{Horizons}
\label{sec:ex-hor}
\label{sec:background}

Now we get to main part of the simulation, namely the soil profile, or
\emph{column}.  In \daisy{}, the column is a one dimensional
description of an agricultural system, with the weather at the top and
the groundwater at the bottom.  Before coming to that, we need to
define two horizons.  A horizon is a vertical layer of soil with
similar chemical and physical properties.  If you make a vertical cut
in the soil, their is usually a clear visual distinction between the
horizons.

\begin{verbatim}
  ;; We have some very sandy soil.
  (defhorizon Ap default
    (clay 8.0 [])
    (silt 10.5 [])
    (coarse_sand 63.4 [])
    (fine_sand 16.5 [])
    (humus 1.12 [])
    (dry_bulk_density 1.5 [g/cm^3])
    (SOM_fractions 0.66 0.34 []))
\end{verbatim}

Here, we define a horizon named \texttt{Ap} (\texttt{defhorizon} is
short for ``define horizon'').  Like the place where we specified the
weather, the \texttt{default} keyword means that we use the default
model for specifying horizons.  At this point it is our only choice,
but that will soon change.

The first four parameters, \texttt{clay}, \texttt{silt},
\texttt{coarse\_sand}, and \texttt{fine\_sand} together define the
soil texture.  The texture affects many aspects of the simulation, for
example will the ammonium adsorption depend on the clay content, and
if not otherwise specified, the hydraulic parameters such as soil
water retention characteristics are also estimated from the texture, a
sandy soil is less capable of storing water than a clay soil.

The fifth soil constituent is the \texttt{humus}, which among other thing
creates the background mineralization, thus providing the soil with
nitrogen.  The numbers given for the five soil constituents are
relative, that is they are weighted to the total.  So you can specify
them as fractions or percentage or whatever, as you wish.

The \texttt{dry\_bulk\_density} is, as the name implies, the densisty
of the dry soil.  It affects the hydraulic properties of the soil and
the total humus content.  If you haven't measured this value, you can
use a different function for estimating the hydraulic properties, by
replacing it with
\begin{verbatim}
  (hydraulic Cosby_at_al)
\end{verbatim}
This pedotransfer function has been developed in America, and does not
require knowledge of the dry bulk density.  Even if you know the dry
bulk density, using \texttt{Cosby\_et\_al} might be a better choise
for simulation in that continent.  The default pedotransfer function,
\texttt{hypres}, has been developed based on European (especially
German) soil measurements.

The last parameter, \texttt{SOM\_fractions}, divides the humus into a
slow and a fast pool.  This affects background mineralization.  For a
system in perfect equilibrium (input of organic matter equal
tuinover), the value will be

\begin{verbatim}
  (SOM_fractions 0.49 0.51)
\end{verbatim}

For farms with a long term humus build-up, more of the humus will be
in the second (fast) pool.  For farms where the amount of humus is
decreasing, slow first (slow) fraction will be larger.  New farmland
and farmland using mineral fertilizer tend to have decreasing humus,
and thus a larger fraction of humus in the slow pool.  Note that we
are talking changes over decades and centuries or even millennia, if
you switched to organic fertilizer five years ago, that is unlikely to
have a measurable effect on the SOM partitioning.

For horizons below the root zone, if there is any significant amount
of humus may be leftover from old tree roots, and slowly decreasing.
In that case, the vast majority (maybe 95\%) is in the first, slow
pool.

Note that all the parameters end with a \verb|[]|.  This is the
dimension surrounded by square brackets.  Since fractions are
dimnesionsless, there is nothing in the brackets.  More interesting
dimensions will occur later.

The next horizon is simpler.

\begin{verbatim}
  (defhorizon C Ap
    (humus 0.12 [])                     ;Less humus.
    (SOM_fractions 0.80 0.20 []))       ;Slower background mineralization.
\end{verbatim}

Here, instead of \texttt{default}, we write \texttt{Ap}.  This is the
name of the horizon we just defined, and will cause \daisy{} to reuse
the parameter values in the new horizon named \texttt{C} we
define here.  In particular, we give it the same texture as the
\texttt{Ap} horizon.  However, since the horizon is intended to
describe lower soil, we specify less humus, and put more of the humus
in the slow pool.  This will cause a much lower background
mineralization.

\subsection{Column}
\label{ex:column}

Now we have described two horizons, we can describe a column.

\begin{verbatim}
  ;; We build the column from the horizons.
  (defcolumn Andeby default
    "Data collected by F.Guf at the B.And farm, Andeby, 2002."
    (Soil (horizons (-20 [cm] Ap) (-2.5 [m] C))
          (MaxRootingDepth 60.0 [cm]))
    (Groundwater deep))
\end{verbatim}

Here, we name the column \texttt{Andeby}, and use the \texttt{default}
model for \daisy{} columns\footnote{An alternative at this point would
  be to use the \texttt{inorganic} column model, which does not
  perform any calculations regarding organic matter, thus saving
  time.}.  The string following the \texttt{default} serves as a
description, it is a good place to note who collected the date, from
where, and when.

In \daisy{}, a column consist of many separate subsystems.  Most of
these have default values, and does thus not need to be specified.
However, we need a description of the soil, and we need to tell
\daisy{} where the groundwater is.  See section~\ref{sec:column} for
other examples of other subsystems.

\begin{verbatim}
    (Soil (horizons (-20 [cm] Ap) (-2.5 [m] C))
\end{verbatim}
The first soil parameter, \texttt{horizons}, specify that the soil is
divided into two horizons, the first span the top 20 cm of the soil,
and the second span the soil from there down to 2.5 meters depth.
There can be as many horizons as you wish, if you had a third it might
be described as below.

\begin{verbatim}
  (horizons (-20 [cm] Ap) (-1 [m] B) (-2.5 [m] C))
\end{verbatim}

The B horizon would then span the depth from 20 cm to 1
meter.

You may note that we specify the depth in \verb|[cm]| for the first
depth, and \verb|[m]| for the next.  If you don't specify a dimension,
\daisy{} will expect numbers to be in centimeters, however if you
specify the dimension as \verb|[m]| knows how to convert.  It is good
idea always to specify the dimension, as this will allow \daisy{} to
check that dimension really is what you think it is.
Misunderstandings about dimensions are a very common source of errors.

The other mandatory soil parameter is \texttt{MaxRootingDepth}.  This
is a soil imposed barrier for the roots, they will not penetrate below
that depth.  To make writing \daisy{} setup files fun, as well as to
keep you alert, this number is specified as positive depth, unlike the
layer endpoints which were specified as negative heights.  \Daisy{}
will tell you if you get it wrong.

After specifying the Soil, we must tell \daisy{} the location of the
groundwater.  Here, we specify that the groundwater is located so deep
that it does not directly affect the simulation.  Other options
include a constant groundwater table, field drainage, and a variable
groundwater table specified in a file.  See
section~\ref{sec:groundwater} for examples.  Apart from soil and
groundwater, there are many more submodels you can specify here,
see section~\ref{sec:column} for more examples.

We have now specified a column, and given it the name \texttt{Andeby}.
Now we just need to tell \daisy{} to use this column for the
simulation.

\begin{verbatim}
  ;; Use it.
  (column Andeby)
\end{verbatim}

\subsection{Time}

After specifying the most complex \daisy{} parameter, the column, we
relax with the simplest parameter.

\begin{verbatim}
  ;; Simulation start date.
  (time 1986 12 1 1)
\end{verbatim}

As the comment say, that time we start the simulation.  The four
numbers given are year, month, day and hour in that sequence.

\subsection{Management}
\label{sec:ex-man}

Back to the complex stuff.  \Daisy{} now knows about the piece of land
we want to simulate, the weather during the simulation, and when we
start.  However, since this is an agricultural system, we also need to
tell \daisy{} about the what management operations the farmer perform
during the simulation.

How to approach this problem depends on whether we are simulating a
real, already performed, experiment, or if we want to simulate a
hypothetical situation.  The difference is that in the first case, we
(hopefully) \emph{know} what the manager did, and when.  In the second
case, we need to build in some intelligence in the system, so the
actions of the hypothetical manager depend on the state of the system.
E.g.\ the manager is likely to wait for a crop to become ripe before
harvesting it.

In this example, we do both.

\begin{verbatim}
  (manager activity
    (wait (at 1987 3 20 1))
    (plowing)
    (wait (at 1987 4 4 1))
    (fertilize (mineral
                (weight 100.0 [kg N/ha])
                (NH4_fraction 0.5 [])))
    (wait (at 1987 4 5 1))
    (progn
      (sow "Grass")
      (sow "Spring Barley"))
    (wait (or (crop_ds_after "Spring Barley" 2.0)
              (at 1987 9 5 1)))
    (harvest "Spring Barley")
    (wait (at 1987 9 8 1))
    (fertilize (mineral
                (weight 80.0 [kg N/ha])
                (NH4_fraction 0.5 [])))
    (wait (at 1987 10 10 1))
    (harvest "Grass"
             (stub 8.0 [cm])              ;Leave 8 cm stub.
             (stem 1.00 []))             ;Harvest everything above stub.
    (wait (at 1988 4 1 1))
    (stop))
\end{verbatim}

First, \texttt{activity} is an order to perform the specified actions
in sequence.  When one action is done, we start the next.  Half of out
actions consist of waiting for some condition to come true.  The most
common condition is that we have reached a specific point in time.
For examples, the first line

\begin{verbatim}
    (wait (at 1987 3 20 1))
\end{verbatim}

mean wait until the simulation time is exactly 1:00~AM, March the
20th, 1987.  Our specific action is \texttt{wait} and the condition is
\texttt{at}.

The next action is \texttt{plowing}.  This happens one hour later, so
our poor farmer is out in the field plowing at 2:00~AM\@.  This activity
last (in the simulation) exactly one hour, after which he can go to
the next activity, which involves waiting once again.

I'll describe the actions used for this sample manager here, more
examples of management actions can be found in
section~\ref{sec:management}.

\begin{verbatim}
    (fertilize (mineral
                (weight 100.0 [kg N/ha])
                (NH4_fraction 0.5 [])))
\end{verbatim}

This is an order to fertilize.  The fertilization type is
\texttt{mineral}, and the amount of nitrogen is specified with the
\texttt{weight} parameter.  The nitrogen can be in the form of
ammonium (NH$_4^+$) or nitrate (NO$_3^-$).  We specify with the
\texttt{NH4\_fraction } parameter that half the nitrogen is ammonium,
and the rest is nitrate.  Read more about fertilizer types in
section~\ref{sec:fertilizer}.

\begin{verbatim}
    (progn
      (sow "Grass")
      (sow "Spring Barley"))
\end{verbatim}

Remember that \texttt{activity} was a directive to perform a list of
actions in sequence?  Well, \texttt{progn} is a directive to perform a
list of actions in parallel.  Here we sow two crops at once.  The
crops, \texttt{Grass} and \texttt{Spring Barley} are defined in the
file \texttt{crops.dai}, which we included in
section~\ref{sec:ex-input}.

\begin{verbatim}
    (wait (or (crop_ds_after "Spring Barley" 2.0)
              (at 1987 9 5 1)))
\end{verbatim}

This is the only place we wait for a condition more complex than a
specific point in time.  Actually, there are three conditions here.

\begin{verbatim}
              (at 1987 9 5 1)
\end{verbatim}

is just a point in time, as you have seen before, however

\begin{verbatim}
              (crop_ds_after "Spring Barley" 2.0)
\end{verbatim}

is new.  It is true when the development stage of a crop named
``\texttt{Spring Barley}'' is 2.0 or above.  The development stage, or
DS, is a key concept in describing the state of a crop.  It starts at
-1.0 when sowed, and increase with time.  When it reached 0.0, the
crop emerges.  At 1.0 it flowers, and at 2.0 it is ripe.   So this
condition is true when there is a ripe spring barley on the field.

The third conditions is named \texttt{or}.

\begin{verbatim}
          (or (crop_ds_after "Spring Barley" 2.0)
              (at 1987 9 5 1))
\end{verbatim}

It is true when any of the specified conditions are true.  I.e.\ it is
true when there is a ripe spring barley on the field, or when the
simulation has reached 1:00~AM, September the 5'th, 1987.

In effect, we wait for the spring barley to get ripe, but if it isn't
ripe at the specified date, we stop waiting anyway

\begin{verbatim}
    (harvest "Spring Barley")
\end{verbatim}

and harvest it.  This harvest is pretty brutal, we remove everything
above the soil, and leave only the roots to rot in the soil.  When we
later cut the grass, we are more gentle.

\begin{verbatim}
    (harvest "Grass"
             (stub 8.0 [cm])              ;Leave 8 cm stub.
             (stem 1.00 []))             ;Harvest everything above stub.
\end{verbatim}

Here, we leave 8 cm stub.  We still remove all the stem above that
height, the parameter to \texttt{stem} is the fraction to remove.  If
we remove a smaller fraction, the remainder will stay on top of the
soil as plant residuals, which will be incorporated into the soil
either gradually by earth worms and other insects, or at once by a
tillage operation.

The grass will most likely survive this cut, unlike the spring barley.

The last action, \texttt{stop}, is simply a command to stop the
simulation.

\subsection{Output}

Now \daisy{} know everything it needs to know in order to run the
simulation.  We just need to tell it what information about the
simulation is should save for later processing.

\begin{verbatim}
  ;; Create these log files.
  (output harvest
          ("N Balance" (when monthly)
                       (from 0 [m]) (to -1 [m]))
          ("Crop Production"
           (set "$crop" "Spring Barley")
           (where "sbarley.dlf"))
          (checkpoint (when (at 1987 8 7 6))))

  ;;; test.dai ends here.
\end{verbatim}
%$

The first entry in the list, \texttt{harvest}, produces a
\texttt{harvest.dlf} file containing information about everything that
has been harvested during the simulation.  The \emph{D}aisy \emph{L}og
\emph{F}ormat (\texttt{.dlf}) is understood by the
\texttt{ShowDaisyOutput.exe} program, but you can also view it with an
editor or load it into a spreadsheet.  The \texttt{harvest.dlf} file,
in particular, is best viewed as plain text.

Our next log model, \texttt{N~Balance}, will write enough information
about the nitrogen content of the soil to calculate a balance, in a
file name \texttt{n\_balance.dlf}.  By default, it will write this
information once a day, but by specifying \texttt{monthly} we reduce
this to once a month.  We also specify the soil layer to log to be the
top meter of the soil.  By default, it will log the nitrogen content
of all the soil down to and including the last numeric layer.

We have two crops simultaneously on the soil.  By default,
\texttt{Crop~Production} will add the content of both crops together,
and log it in a \texttt{crop\_prod.dlf} file.  The weird looking

\begin{verbatim}
           (set "$crop" "Spring Barley")
\end{verbatim}%$

line tells \daisy{} to only log information about the spring barley,
and by specifying a \texttt{where} parameter we tell it to store the
information in a file named \texttt{sbarley.dlf} instead.

Note that writing these log files tend to be much slower than actually
running the simulation, to the more information you tell \daisy{} to
log, the slower the simulation will run.

Finally, we tell \daisy{} To create a checkpoint file at 6:00~AM,
August the 7'th, 1987.  This file will be named
\texttt{checkpoint-1987-8-7+6.dai}.  As the extension hints, this is
not a log file, but a setup file containing all information about the
current state of the simulation.  In fact, you can restart the
simulation by typing
\begin{verbatim}
  daisy checkpoint-1987-8-7+6.dai
\end{verbatim}
You can also browse it with a text editor to get detailed information
about the state of the simulation at that point in time.

See section~\ref{sec:output} for more information about log models.

\section{Column}
\label{sec:column}

In this section we give examples of common customization of the
various subsystems of the column.

\subsection{Hydraulic properties}
\label{sec:hydraulic}

In section~\ref{sec:ex-hor}, we let \daisy{} guess the hydraulic
properties from the texture.  This kind of estimates (wild guessing)
are unreliable at best, and since the hydraulic properties are the key
most other aspects of the simulation depend on, getting them right
should be a priority.  \Daisy{} supports a number of models for
hydraulic properties, the most important are listed here, more can be
found in the reference manual~\cite{dina81}.

\begin{description}
\item[\texttt{B\_BaC}] Brooks and Corey retention curve model with
  Burdine theory.
\item[\texttt{B\_C}] Campbell retention curve model with Burdine
  theory.
\item[\texttt{B\_vG}] van Genuchten retention curve model with Burdine
  theory.
\item[\texttt{M\_BaC}] Brooks and Corey retention curve model with
  Mualem theory.
\item[\texttt{M\_C}] Campbell retention curve model with Mualem
  theory.
\item[\texttt{M\_vG}] van Genuchten retention curve model with Mualem
  theory.
\end{description}

Here is an example where we specify the van Genuchten retention curve
model with Mualem theory

\begin{verbatim}
  (defhorizon B default
    (clay 8.0) (silt 10.5) (coarse_sand 63.4) (fine_sand 16.5)
    (humus 1.12)
    (SOM_fractions 0.70 0.30)
    (hydraulic M_vG
               (K_sat 10 [cm/h])
               (Theta_res 0.05 [cm^3 H2O/cm^3])
               (Theta_sat 0.424 [cm^3 H2O/cm^3])
               (alpha 0.069 [cm^-1])
               (n 1.527)))
\end{verbatim}

The three parameters
\begin{description}
\item[\texttt{K\_sat}] Water conductivity of saturated soil.
\item[\texttt{Theta\_sat}] Saturation point (a.k.a.\ soil porosity).
\item[\texttt{Theta\_res}] Soil residual water.
\end{description}
are shared by all the models, except that \texttt{Theta\_res} must be
zero for Campbell.  The other parameters are model specific.

\subsection{Groundwater and Drainage}
\label{sec:groundwater}

Obviously, the groundwater is not always located ``far below'' the
root zone, as specified by
\begin{verbatim}
  (defcolumn Andeby default
    (Groundwater deep)
    ;; Other parameters...
\end{verbatim}
in the example.  We can therefore also specify a fixed groundwater
table as in
\begin{verbatim}
    (Groundwater fixed -100 [cm])
\end{verbatim}
where we specify groundwater in one meters depth.  Note that if we
specify a fixed groundwater table, it must be higher than our lowest
node.  A third common case is soil drainage
\begin{verbatim}
    (Groundwater pipe)
\end{verbatim}
which is often enough for drained soil.  A full specification would be
\begin{verbatim}
    (Groundwater pipe
                 (L 18 [m])
                 (x  9 [m])
                 (pipe_position -1.1 [m])
                 (K_aquitard 1e-4 [cm/h])
                 (Z_aquitard 2 [m])
                 (h_aquifer  2 [m]))
\end{verbatim}
where
\begin{description}
\item[\texttt{L}] Distance between pipes.
\item[\texttt{x}] Horizontal distance to nearest pipe. By default,
  this is $1/2 \mathtt{L}$, i.e.\ in the middle between the pipes.
\item[\texttt{pipe\_position}] Height pipes are placed in the soil.
\item[\texttt{K\_aquitard}] Conductivity of the aquitard.
\item[\texttt{Z\_aquitard}] Thickness of the aquitard.
\item[\texttt{h\_aquifer}] Pressure potential in the aquifer below the
  aquitard. By default this is equal to \texttt{Z\_aquitard}.
\end{description}
The above numbers are the default values, i.e.\ what you get if you
don't specify anything.  \texttt{L} and \texttt{pipe\_position} are
standard values for Danish fields.  The aquitard is water blocking
layer explained in~\cite{daisy-new}, the \texttt{K\_aquitard}
parameter is what you usually want to calibrate.

\textbf{Note:} When using drained soil, all numerical nodes will
contribute to the drain sink term, not just the node containing the
pipe.  This is because the pipe is assumed to be at some distance from
our node (as specified by the \texttt{x} or \texttt{L} parameter), and
the sink term is actually the simulated horizontal movement of water
towards the pipe.

\subsubsection{Aquitard horizons}
\label{sec:aquitard}

For some soils the groundwater table can fall into the aquitard on dry
years.  When this happens, \daisy{} will stop with an error.  In order
to proceed, you need to make the top of the aquitard part of the soil
profile.  You do this by defining an ``aquitard'' horizon, and place
that below your lowest horizon.  The only important parameter for the
aquitard horizon is \texttt{K\_sat}, which must be identical to
\texttt{\texttt{K\_aquitard}}.  You also need to \texttt{Z\_aquitard}
corresponding to the thickness of your new aquitard horizon.

\begin{verbatim}
  (defhorizon aquitard C
    (hydraulic M_vG
       ;; These parameters are not important.
       (Theta_res 0.05 [cm^3 H2O/cm^3])
       (Theta_sat 0.424 [cm^3 H2O/cm^3])
       (alpha 0.069 [cm^-1])
       (n 1.527)
       ;; Note: same as K_aquitard
       (K_sat 1e-4 [cm/h])))

  (defcolumn Andeby default
    (Soil (horizons (-20 [cm] Ap) (-2.5 [m] C) 
                    ;; Top 50 cm of aquitard.
                    (-3.0 [m] aquitard))
          (MaxRootingDepth 60.0 [cm]))
    (Groundwater pipe
                 ;; Use same K_sat as the aquitard horizon.
                 (K_aquitard 1e-4 [cm/h])
                 ;; Decrease thickness with 50 cm from default 
                 ;; of 2.0 m to compensate for aquitard horizon.
                 (Z_aquitard 1.5 [m]))) 
\end{verbatim}

\subsection{Macropores}
\label{sec:macro}

By default, \daisy{} will only calculate water flow through the
matrix, not through macropores.  This is usually adequate, since the
vast amount of water travel through the matrix.  However, when
calculating leaching of nitrogen and especially pesticides, water flow
through macropores become important, because it provides a must faster
route to the groundwater.

Here is an example where we specify macropores:
\begin{verbatim}
  (defcolumn Andeby default
    (SoilWater (macro default
                      (height_start 0 [cm])
                      (height_end -200 [cm])
                      (distribution (-200 1) (-80 0.1) (0 0))
                      (pressure_initiate -5 [cm])
                      (pressure_end -30 [cm])
                      (pond_max 0.5 [mm])))
    ;; Other parameters...
\end{verbatim}

We specify that the macropores start at the soil surface, and end in 2
meters depth with \texttt{height\_start} and \texttt{height\_end}.  All
macropores are assumed to start at \texttt{height\_start}, but their
endpoints are distributed as specified by the \texttt{distribution}
parameter.  Here, we specify that all the macropores have ended at 2
meters depth, 10\% of the macropores end above 80 cm depth, and none
end above the surface (doh!).  We use linear interpolation between
these points, so we can conclude that 5\% of the macropores end above
40 cm depth.  You do not have to specify \texttt{height\_start} and
\texttt{height\_end} explicitly, if you don't, the last and first
value point of \texttt{distribution} will be used.

The macropores can be activated in two different ways.  Macropores
that, like here, reach the surface, can be activated by ponding.  The
parameter \texttt{pond\_max} specifies that the macropores will be
activated when we have half a millimeter water on the soil surface.
Once activated, the macropores will empty the pond.  The other way
macropores can be activated is if the water pressure in any node
within the region with macropores reaches above
\texttt{pressure\_initiate}.  In that case, it will be drained
immediately down to \texttt{pressure\_end}.

The water, and any solutes in it will be distributed according to
where the macropores end.  I.e.\ if the macropores was activated by
ponding, 10\% of the ponded water would have entered the matrix at 80
cm depth, and all the ponded water would have entered the matrix at 2
meters depth.

\subsubsection{Log models}

The following log models are useful when dealing with macropores:
\begin{description}
\item[\texttt{Preferential Water Flux}] The preferential water flux
  between all numeric soil layers.  File: \texttt{pref\_flux.dlf}.
\item[\texttt{Preferential NO3 Transport}] The preferential nitrogen
  transport between all numeric soil layers.  File:
  \texttt{pref\_NO3.dlf}.
\item[\texttt{Preferential NH4 Transport}] The preferential ammonium
  transport between all numeric soil layers.  File:
  \texttt{pref\_NH4.dlf}.
\end{description}
See section~\ref{sec:output} for more information about log models. 

\subsection{Evapotranspiration}
\label{sec:bioclimate}

By default, Makkink is used for finding the reference
evapotranspiration.  It is a simple model, but adequate for simple
simulations.  It is specified as
\begin{verbatim}
  (defcolumn Andeby default
    (Bioclimate default (pet makkink))
    ;; Other parameters...
\end{verbatim}
except that you don't have to, since it is default.

Sometimes you have access to reference evapotranspiration data in the
weather file, in order to make \daisy{} use these you should specify
pet like
\begin{verbatim}
    (Bioclimate default (pet weather))
\end{verbatim}
\textbf{Important:}  If you specify reference evapotranspiration in
the weather file, but leave out the above line in the \daisy{} setup
file, \daisy{} will ignore the data in the weather file, and continue
to use Makkink for reference evapotranspiration.

\Daisy{} also support the more advanced model Penman-Monteith
\begin{verbatim}
    (Bioclimate default (pet PM))
\end{verbatim}
If you have reliable information about vapor pressure and wind, it is
much preferable to use Penman-Monteith over Makkink.  If the data are
missing of poor quality, the simple Makkink are more robust.

It is possible to use Penman-Monteith together with
Shuttleworth-Wallace for the most accurate model of the bioclimate
supported by \daisy{}.  Enable it with
\begin{verbatim}
    (Bioclimate default (pet PM) (svat PMSW))
\end{verbatim}
This requires hourly weather data to be useful.  

\subsubsection{Potential and Reference Evapotranspiration}

You may also want to change how potential evapotranspiration is
calculated from reference evapotranspiration.  Both the soil surface
and crops influence this, depending on how much of the soil is covered
by crops.  To change the soil surface factor, specify
\texttt{EpFactor} like this:
\begin{verbatim}
  (defcolumn Foulum default
    (Surface (EpFactor 1.0))
\end{verbatim}
The default value is 0.8.

For crops, derive new crop types like
\begin{verbatim}
  (defcrop "Andeby Spring Barley" "Spring Barley"
    "Spring Barley with modified EpFac as measured
  in Andeby, 2002, for the SuperGrow(TM) cooperation."
    (Canopy (EpFac 1.2)))
\end{verbatim}
overwriting the default \texttt{EpFac} of 1.0.  Remember that you need
to sow and harvest \texttt{Andeby Spring Barley} instead of
\texttt{Spring Barley}.  Or alternatively, overwrite the parameter at
the time of sowing, like this:
\begin{verbatim}
   (sow ("Spring Barley"
         (Canopy (EpFac 1.2))))
\end{verbatim}
If you want \texttt{EpFac} to depend on the development stage, specify
it like
\begin{verbatim}
   (sow ("Spring Barley"
        (Canopy (EpFacDS (0.0 1.0) (1.0 1.2) (2.0 1.0)))))
\end{verbatim}
Here, \texttt{EpFac} will be 1.0 at emergence (DS 0.0), gradually rise
to 1.2 at flowering (DS 1.0), and then fall towards 1.0 at ripeness (DS
2.0).

\subsection{Discretization}

Internally, \daisy{} divide the soil into a number of discrete layers.
During the simulation, it will keep track of the content (water, heat,
nitrogen, etc.) of each layer, and the flows between the layers.  A
fine discretization causes more exact results, but slows down the
simulation, and very fine discretization may cause numeric problems.
The default discretion represents a balance between these concerns,
but sometimes it will be advantageous to specify it manually.

A specific example is when you need to log the content of specific
intervals.  In that case it is best to be sure the intervals used by
\daisy{} does not cross any of the intervals you are logging.  If they
do, \daisy{} will attempt to interpolate, but the inexact values will
mean you won't get an exact mass balance.

We specify the a numeric discretization of the soil with the
\texttt{zplus} parameter.  
\begin{verbatim}
  (defcolumn Andeby default
    "Data collected by F.Guf at the B.And farm, Andeby, 2002."
    (Soil (horizons (-20 [cm] Ap) (-2.5 [m] C))
          (zplus -2.5 -5 -10 -15 -20
                 -25 -30 -40 -50 -60 -70
                 -80 -90 -100 -125 -150 -175
                 -200 -225 -250 [cm])
          (MaxRootingDepth 60.0 [cm]))
    (Groundwater deep))
\end{verbatim}
Like for the horizons, we list the endpoints of each numeric layer (or
\emph{node}) from the top to the bottom.  The \verb|[cm]| at the end
of the list indicates that all the previous numbers are given in cm.

A good rule of thump is to use small intervals at the top (one or two
cm) where most of the action is, and use larger steps (dozens of cm)
near the bottom, where things are more quiet, at least with a fixed
groundwater level.

A hard rule is that numeric layers may not cross horizon boundaries,
which mean the end points in the horizon list must also be somewhere
in the zplus list.

\section{Management}
\label{sec:management}

Writing management specifications that behave both semi-intelligently
and robustly to the simulation state is a challenge.  However,
specifying each and every little management operation for a 30 year
run can get both tedious and error prone.

In this section, we will therefore start by introducing you to a way
to organize your manager in ``crops'' and ``rotations'', which will
save you much work, and then continue to give examples of various
common management operations.

\subsection{Crop Managements}
\label{sec:crop-man}

Our first hint is to collect all management information about a
specific crop, as in the example below.
\begin{verbatim}
  ;; Spring Barley management.
  (defaction sbarley activity
    (wait_mm_dd 3 20)
    (plowing)
    (wait_mm_dd 4 10)
    (fertilize (pig_slurry (weight 100.0 [Mg w.w./ha])))
    (wait_mm_dd 4 15)
    (seed_bed_preparation)
    (sow "Spring Barley")
    (wait_mm_dd 7 1)
    (wait (or (crop_ds_after "Spring Barley" 2.0 [])      ;Ripe
              (mm_dd 10 1)))
    (harvest "Spring Barley"))
\end{verbatim}
Here, we define a new management operation (or \emph{action}) named
\texttt{sbarley}, which can be used like any other management
operation, e.g.\ plowing, it just last a bit longer.  Specifically, it
last the good part of a year.

We define \texttt{sbarley} as an \texttt{activity}, just like in the
example in section~\ref{sec:ex-man}.  The trick is to give it a name
(\texttt{sbarley}), and specify it in a way so it can be used any
year.  The key to the later is to use
\begin{verbatim}
    (wait_mm_dd 3 20)
\end{verbatim}
which, unlike (from the example)
\begin{verbatim}
    (wait (at 1987 3 20 1))
\end{verbatim}
does not specify a year\footnote{The observant reader will notice that
  the hour isn't specified either.  You can specify the hour,
  see~\cite{dina81}, but usually the default of 8:00~AM is fine.  It
  is certainly less stressful to the farmer than the 1:00~AM we send
  him plowing earlier.}.  It matches the 20'th of March every year,
allowing us to use in any year.  Later, when waiting for the crop to
become ripe, we similarity use an \texttt{mm\_dd} condition, which
become true at 8:00~AM October the 1'st every year, instead of waiting
testing for a more specific date as in the example.

\subsubsection{Rotations}
\label{sec:rotations}

If we have defined crop management actions as in the previous section
for spring barley, rye, and grass, we can use them in a
manager specification like this
\begin{verbatim}
  (mananger activity
    sbarley sbarley grass rye)
\end{verbatim}

As you see, we only have to specify the management operations
associated with growing spring barley once, even though we grow spring
barley twice in the simulation.  We can take this one step further, by
naming and reusing specific rotations.
\begin{verbatim}
  (defaction rotation_bbgr activity
    sbarley sbarley grass rye)

  (manager activity
    rotation_bbgr rotation_bbgr rotation_bbgr rotation_bbgr rotation_bbgr)
\end{verbatim}
Here, we define and name a whole rotation containing four seasons, and
repeat that rotation five times.  All in all, we have specified
management for 20 years in two lines.

This is how I in general recommend writing management specifications.
Break it down into management specifications for individual crops,
then combine the crops into rotations.  In the next sections I will
present more specific examples of common management operations.

\subsection{Tillage}
\label{sec:tillage}

The tillage operations defined in \texttt{tillage.dai} are
\texttt{plowing}, \texttt{rotavation}, \texttt{disk\_harrowing},
\texttt{stubble\_cultivation} and \texttt{seed\_bed\_preparation}.
You use them all like \texttt{plowing} in the example.

All the tillage operations kill any crops on the soil, and homogenize
the top soil content (be it water, organic or inorganic matter,
pesticides, or heat).  Plowing will additionally swap the content of
the top and bottom of the plowing layer.  The content on the soil
surface will be incorporated into the soil, for
\texttt{disk\_harrowing} and \texttt{stubble\_cultivation} this
incorporation will only be partial.

\subsubsection{Trafficability}
\label{sec:trafficable}

A real farmer might want to delay some operations if the soil is too
wet, or frozen.  The \texttt{trafficable} condition will test for
this.  You can use it like
\begin{verbatim}
  (wait trafficable)
  (plowing)
\end{verbatim}
but it might be a good idea to add a timeout for robustness.
\begin{verbatim}
  (wait (or trafficable (mm_dd 04 01)))
  (plowing)
\end{verbatim}
Here we wait until the soil is trafficable, or we have reached the
first of April.  This is important if we use it in a crop management
description like in section~\ref{sec:crop-man}.  Imagine we have the
following piece of management activity:
\begin{verbatim}
  (wait trafficable)
  (plowing)
  (wait_mm_dd 4 10)
\end{verbatim}
If the conditions doesn't get trafficable before the 10'th of April,
we will plow after the 10'th of April and then continue to wait for
the 10'th of April \emph{next year}.  If we only run a short
simulation, this might be OK, we can then discard the results and
investigate what went wrong.  However, if we are simulating a 30 year
period, it will be highly annoying to have to rerun because a single
year had an unusually later winter or wet spring.  For the sake of
such long simulations, it is a good idea to get get in the habit of
adding timeouts to such conditions.

\subsection{Sowing and Harvesting}
\label{sec:sow}

By including the \texttt{crops.dai} file, you get access to the
following crop types: \texttt{Fodder Beet}, \texttt{Grass},
\texttt{Maize}, \texttt{Pea}, \texttt{Potato}, \texttt{Rye},
\texttt{Spring Barley}, \texttt{Spring Rape}, \texttt{Sugar Beet},
\texttt{Spring Wheat}, \texttt{Winter Barley}, \texttt{Winter Rape}
and \texttt{Winter Wheat}.

In general, creating parameterizations for new crops requires field
experiments where you carefully measure the nitrogen content during
the development with different fertilization levels.  However, making
small modifications for specific purposes is feasible, as
demonstrated in section~\ref{sec:bioclimate}.

\subsubsection{Crop Residuals}

In general, we harvest when the crop is ripe, and kill the crops in
the process. The main decision is whether to remove the crop
residuals, or leave them as litter on the field.  Here is an example
where we remove the crop residuals
\begin{verbatim}
  (harvest "Spring Barley"
           (stub 8 [cm]))
\end{verbatim}
and one where we leave them as litter on the field.
\begin{verbatim}
  (harvest "Spring Barley"
           (stub 8 [cm])
           (stem 0.0 [])
           (leaf 0.0 []))
\end{verbatim}
In both cases, we leave the stub and roots on the field, and remove
the storage organ (that is, the grains).

\subsubsection{Multiple Cuts}
\label{sec:cuts}

For some crops, like grass, we will want to harvest or cut them
multiple times.  In such cases, the stub is essential, as that
determine how much leaf area there will be left to start regrowth.  If
you don't leave any stub (the default!), the plant will be unlikely to
recover.

Writing the generic management specification for grass is a lot more
complex than for barley, because we do not know in advance how many
times we are going to cut it.  It depends on how fast it grows.
Therefore, we start by defining two simpler actions.
\begin{verbatim}
  (defaction cut_grass activity
    (wait (or (crop_ds_after "Grass" 0.7 [])
              (crop_dm_over "Grass" 4000 [kg DM/ha])))
    (harvest "Grass"
             (stub 10.0 [cm])))
\end{verbatim}
Here, we define a \texttt{cut\_grass} operation which wait for the
grass to grow old or big enough, and then cut it.

\begin{verbatim}
  (defaction cut_grass_fertilize activity
    (cut_grass)
    (wait_days 1)
    (fertilize (N25S (weight 100.0 [kg N/ha]))))
\end{verbatim}
Here, we extend the \texttt{cut\_grass} operation by fertilizing after
the cut.

Now we can define our grass management action:
\begin{verbatim}
  (defaction grass activity
    (wait_mm_dd 3 20)
    (wait trafficable)
    (plowing)
    (wait_mm_dd 4 10)
    (fertilize (N25S (weight 100.0 [kg N/ha])))
    (wait_mm_dd 4 15)
    (seed_bed_preparation)
    (sow "Grass")
    (while (wait_mm_dd 9 1)
      (activity
        ;; Max 5 fertilized cuts.
        cut_grass_fertilize cut_grass_fertilize cut_grass_fertilize
        cut_grass_fertilize cut_grass_fertilize
        ;; As many unfertilized cuts as we want.
        (repeat cut_grass)))
    (while (wait_mm_dd 11 1)
      ;; Additional unfertilized cuts.
      (repeat cut_grass)))
\end{verbatim}
The initial part is quite conventional, as seen in
sections~\ref{sec:ex-man} and~\ref{sec:crop-man}, until the two
\texttt{while} actions.  A \texttt{while} action take two other
actions, and perform them both in parallel, until the first action is
done.  It does not matter whether the second action finish or not.

In the example above, the first action is (in both cases), a
\texttt{wait\_mm\_dd} operation.  This means that the \texttt{while}
will simply continue until we reach the specified date.  While we are
waiting for that date, it will do as specified by the second action.
This means that until the 1'st of September, we may see up to 5 cuts
followed by fertilization.  We might see less that that, if the grass
does not grow fast enough, but never more.  After the 5 fertilized
cuts, assuming it is not yet September, we may see any number of
unfertilized cuts.  The \texttt{repeat} action will perform the
specified operation over and over forever, which means it is only
useful inside a \texttt{while}, otherwise it will never break out of
the loop.

After that, we start a new round of unfertilized cuts, which last
until November.  You may wonder why we didn't just used a single
\texttt{while}, which waited until November.  What is the difference?
I'm glad you asked.  With the description above, we get a manager that
will apply fertilizer at most six times (including the initial
fertilization the 10'th of April), thus staying below a possibly
politically induced upper limit.  \emph{And}, this is the trick, never
apply fertilizer after August.  Even if we have only had, say, three
fertilized cuts before September, and later cuts will be unfertilized.
The idea being that we do not believe the grass will grow enough that
late in the year to justify the additional fertilization.

Read more about fertilization in the next section.

\subsection{Fertilization}
\label{sec:fertilizer}

We have two basic types of fertilizer, \emph{mineral} and
\emph{organic}.  These will be described in the two next sections,
followed by examples of some more advanced types of fertilization. 

\subsubsection{Mineral Fertilizer}

In section~\ref{sec:ex-man} we used mineral fertilizer, as below:
\begin{verbatim}
    (fertilize (mineral
                (weight 100.0 [kg N/ha])
                (NH4_fraction 0.5 [])))
\end{verbatim}
If we used this fertilizer type often, it might be nice to give it a
name.  We could do this with the following definition:
\begin{verbatim}
  (defam SuperGrow mineral
    (NH4_fraction 0.5 []))
\end{verbatim}
To use it, we would write:
\begin{verbatim}
  (fertilize (SuperGrow (weight 100.0 [kg N/ha])))
\end{verbatim}
It only saves one line, but the producers of SuperGrow\texttrademark{}
will be eternally grateful.  The \texttt{fertilizer.dai} file already
contain the following mineral fertilizers:
\begin{description}
\item[\texttt{Ammonia}] Pure NH$_4^+$.
\item[\texttt{AmmoniumNitrate}] A 50-50 mix of NH$_4^+$ and NO$_3^-$.
\item[\texttt{Nitrate}] Pure NO$_3^-$.
\item[\texttt{N25S}] 50.8\% NH$_4^+$.
\end{description}

\subsubsection{Organic Fertilizer}

We used organic fertilizer, specifically pig slurry, in the example in
section~\ref{sec:crop-man}.  The obvious difference is that the weight
is given in metric tons of wet weight, rather than kg nitrogen.  The
\texttt{fertilizer.dai} file contains the following ready to use
definitions of pig slurry (named \texttt{pig\_slurry}), cattle slurry
(\texttt{cattle\_slurry}), pig manure (\texttt{pig\_manure}), cattle
manure (\texttt{cattle\_manure}), and horse manure
(\texttt{horse\_manure}).

For purely hypothetical scenarios, using the predefined organic
fertilizer specifications are adequate, however, if you know more
about the system being simulated, getting a better description of the
slurry or manure being applied is important, as the quality varies a
lot.  You can define a new slurry like this:
\begin{verbatim}
  (defam Andeby_pig_slurry slurry
    "Data collected by F. Guf, 2002 at the B. And farm, Andeby."
    (dry_matter_fraction 0.073)
    (total_C_fraction 0.323)
    (total_N_fraction 0.0863)
    (NH4_fraction 0.7)
    (NO3_fraction 0.0)
    (volatilization 0.15))
\end{verbatim}
where
\begin{description}
\item[\texttt{dry\_matter\_fraction}] Dry matter fraction of total
  (wet) weight.
\item[\texttt{total\_C\_fraction}] Carbon fraction of dry matter.
\item[\texttt{total\_N\_fraction}] Nitrogen fraction of dry matter
\item[\texttt{NH4\_fraction}] Ammonium fraction of total N in
  fertilizer.
\item[\texttt{NO3\_fraction}] Nitrate fraction of total N in
  fertilizer, by default zero.
\item[\texttt{volatilization}] Fraction of NH$_4^+$ that evaporates on
  application.
\end{description}
The organic nitrogen is what is left having removed the two inorganic
fractions.

It is possible to specify other parameters such as the turnover rate
and efficiency for the organic fertilizer, but these require
specialized knowledge of both the fertilizer itself, and the organic
matter model in \daisy{} (see~\cite{daisy-new}).

\subsubsection{First and Second Year Utilization}

Sometimes organic fertilization plans are made from estimated first
and second year utilization of the nitrogen.  In Denmark, this even
have legal force.  None of the fertilizers in \texttt{fertilizer.dai}
contain such estimates, as they varies both in time and jurisdiction,
but you can specify them yourself, as below.
\begin{verbatim}
  (defam "Andeby Pig Slurry" pig_slurry
    (first_year_utilization 0.40)
    (second_year_utilization 0.15))
\end{verbatim}
Here, we estimate that 40\% of the nitrogen (organic or mineral) is
utilized the first year, and 15\% the second year.  The rest is
presumably lost.  Note again, that these estimates have absolutely no
effect on how much will actually be utilized during the simulation.

To use this information, we can write the following
\begin{verbatim}
  (fertilize ("Andeby Pig Slurry")
             (equivalent_weight 100.0 [kg N/ha]))
\end{verbatim}
in our management specification.  We let \daisy{} calculate the amount
of wet weight to apply, to get a first year utilization of 100 kg
N/ha.  Note that \texttt{equivalent\_weight} is outside the
parentheses around the \texttt{"Andeby Pig Slurry"}, unlike
\texttt{weight}.  The \texttt{weight} parameter is a property of the
fertilizer itself, while \texttt{equivalent\_weight} is a property of
the fertilize action.  Thus, the difference.

To use the \texttt{second\_year\_utilization} parameter, we should let
one harvest operation (usually the first) the next year have the form
\begin{verbatim}
  (fertilize ("Andeby Pig Slurry")
             (second_year_compensation true)
             (minimum_weight 50.0 [kg N/ha])
             (equivalent_weight 150.0 [kg N/ha]))
\end{verbatim}
The \texttt{second\_year\_compensation} directive means that \daisy{}
should subtract any second year effect accumulated from fertilizations
the previous year from \texttt{equivalent\_weight} before calculating
the amount of fertilizer to apply.  By specifying
\texttt{minimum\_weight} we put a minimum on the amount of fertilizer
we will bother applying.  That is, if the accumulated second year
effect is more than 100 kg N/ha, we won't fertilize at all.

\subsubsection{Precision Fertilization}

Precision fertilization allows you to let the amount of fertilizer you
apply depend on the nitrogen content of the soil in a specified zone.

\begin{verbatim}
  (fertilize (N25S)
             (precision 150.0 [kg N/ha] (from 0.0 [cm]) (to -100.0 [cm]))
             (minimum_weight 50.0 [kg N/ha]))
\end{verbatim}

Daisy will measure the amount of fertilizer in the specified zone (from
the soil surface to 1 m depth, which is also the default), and supply
enough extra nitrogen to reach the specified target (150 kg N/ha).  If
less than the specified minimum (50 kg N/ha) nitrogen is needed in
order to reach the target, no fertilizer will be applied.

You can use the \texttt{precision} parameter with organic fertilizer
as well, in that case you must also specify the
\texttt{first\_year\_utilization} fertilizer parameter.


\subsubsection{Incorporation}

By default, fertilizer will be applied directly on the soil surface.
There, organic fertilizer will be incorporated gardually by bioloical
means (earth worms and the like), or suddenly by machanical means,
i.e.\ by a tillage operation.  Inorganic fertilizer will penetrate the
soil along with the rain or irrigation water.

However, if a farmer has the equipment for it, he will prefer
incorporating the fertilizer directly into the soil.  It in general
increase utilization, and decrease complaints about pig slurry smell
from neighbors.

Use the \texttt{from} and \texttt{to} fertilize parameters to specify
incorporation, as below:
\begin{verbatim}
  (fertilize (pig_slurry (weight 100.0 [Mg w.w./ha]))
             (from -5 [cm]) (to -20 [cm]))
\end{verbatim}
Again, note the extra set of parentheses that binds \texttt{weight} to
\texttt{pig\_slurry}.  Unlike \texttt{from} and \texttt{to},
\texttt{weight} is a property of the \texttt{pig\_slurry}, not of the
\texttt{fertilize} operation.

\subsection{Irrigation}
\label{sec:irrigation}

Irrigation can be applied overhead the canopy (e.g.\ in the form of
sprinklers), on the soil surface, or incorporated directly into the
soil. 

To specify overhead irrigation, write
\begin{verbatim}
  (irrigate_overhead 30 [mm/h])
\end{verbatim}
This will irrigate with 30 mm/h for one hour, giving a total of 30 mm.
The water will first hit the canopy, where some will stay until it
eventually evaporate, and some may wash down any pesticides on the
canopy.

To specify surface irrigation, write
\begin{verbatim}
  (irrigate_surface 30 [mm/h] (temperature 10.0 [dg C]))
\end{verbatim}
Again, this will give a total of 30 mm.  The water will bypass the
canopy, and directly hit the soil surface where it might create a pond
that slowly evaporates or enter the soil.

If you don't specify the temperature parameter, as in the
\texttt{irrigate\_overhead} case, the air temperature will be used.

The syntax for incorporating the water directly into the soil is
quite different, here is an example:
\begin{verbatim}
  (set_subsoil_irrigation 1 [mm/h] (from -5.0 [cm]) (to -10.0 [cm])
                          (solute (NH4 0.1 [?mg N/l]) (NO3 0.7 [?mg N/l])))
  (wait_days 1 (hours 6))
  (stop_subsoil_irrigation)
\end{verbatim}
Here we specify subsoil irrigation with 1 mm/h for 30 hours, giving us
a total of 30 mm.  It will be incorporated into the soil between 5 and
10 cm's depth.  We also specify a content of inorganic nitrogen in the
water, something we can also do for overhead and surface irrigation.
We do not and cannot specify the water temperature for subsoil
irrigation, it is assumed to be the same temperature as the soil
itself.

\subsubsection{Conditions}

If we know exactly when the irrigation is to be done, we can simply
wait for the date, as here
\begin{verbatim}
  (wait_mm_dd 07 15)
  (irrigate_overhead 30 [mm/h])
\end{verbatim}

However, when writing generic crop managements as in
section~\ref{sec:crop-man}, we can't use that technique.  It might
rain the first two weeks of July for all we know, which will make
irrigating rather wasteful.  \Daisy{} therefore provides ways to test
for the soil water content.
\begin{verbatim}
  (wait (not (soil_water_pressure_above (height -30.0 [cm])
                                        (potential -1000 [cm]))))
  (irrigate_overhead 30 [mm/h])
\end{verbatim}
Here, we emulate a tensiometer placed in the field at 30 cm depth,
triggering irrigation when the potential gets below -1000 cm.


An alternative condition would be
\begin{verbatim}
  (wait (not (soil_water_content_above 200.0 [mm] 
                                       (from 0 [m]) (to 1 [m]))))
\end{verbatim}
where we wait until there is less than 200 mm water in the top meter
of the soil. 

%% TODO: water deficit condition

\subsubsection{Strategies}

However, conditions like these are not enough, since we cannot know in
advance how many times it gets necessary to irrigate, if any.  This
problem is similar to the problem of multiple cuts we had in
section~\ref{sec:cuts}, and the solution also build on our ``parallel
processing'' directive, \texttt{while}.  Imagine we already have a
crop management description, like \texttt{sbarley} in
section~\ref{sec:crop-man}.  We can then add an irrigation strategy
like this:
\begin{verbatim}
  (defaction irrigate_30 activity 
    (wait (not (soil_water_pressure_above (height -30.0 [cm])
                                          (potential -1000 [cm]))))
    (irrigate_overhead 30 [mm/h]))

  (manager activity
    (while sbarley
      (repeat irrigate_30))
    (stop))
\end{verbatim}

Here, we define a management action named \texttt{irrigate\_30}, which
consists of waiting for the the soil to dry out, and then irrigating
with 30 mm.  Our management specification says that as long as we are
managing our spring barley (as defined in \texttt{sbarley}), we will
repeatedly run \texttt{irrigate\_30}, that is, wait for the soil to
dry, and then irrigate.

An advantage of implementing our irrigation strategy like this is
that we can apply the strategy on any crop.  There are, however, two
problems we need to solve before it gets useful.  First, we should only
irrigate in the growth season, even if the soil somehow dries out
during the winter.  The second is that irrigation takes one hour, but
the water may very well take longer to reach the tensiometer.  This
means that the tensiometer will keep triggering irrigation several
time, until the water finally reaches it.

We solve the first problem by adding an ``in season'' test to the wait
conditions, and the second problem by adding a delay after
irrigation. 
\begin{verbatim}
  (defaction irrigate_30 activity 
    (wait (and (after_mm_dd 5 1)
               (before_mm_dd 9 1)
               (not (soil_water_pressure_above (height -30.0 [cm])
                                               (potential -1000 [cm])))))
    (irrigate_overhead 30 [mm/h])
    (wait_days 2))
\end{verbatim}

If there are limits on the total amount of water we are allowed to use
for irrigation, we can limit the macimal number of times we irrigate,
like below.
\begin{verbatim}
  (manager activity
    (while sbarley
      (activity irrigate_30 irrigate_30 irrigate_30 irrigate_30))
    (stop))
\end{verbatim}
Here, we will at most irrigate four times.

\subsection{Pesticides}
\label{sec:pesticides}

Pesticides have no effect in \daisy{} (i.e. \daisy{} crops are immune
to everything, and weeds are something that happens to other models).
Nonetheless, \daisy{} allows you to apply them to the field.  \Daisy{}
will then keep track on them, as the are transported with the water
through the system, or are removed from the system by either
decomposition, root uptake, or leaching to the groundwater.

The command to apply pesticides is
\begin{verbatim}
  (spray Bentazon 1000 [g/ha])
\end{verbatim}
with \texttt{Bentazon} being the name of the pesticide.  The currently
available pesticide parameterizations includes \texttt{Atrazine},
\texttt{Bentazon}, \texttt{IPU}, \texttt{MCPP},
\texttt{Pendimethalin}.  To use them, you must add
\begin{verbatim}
  (input file "chemistry.dai")
\end{verbatim}
to the start of the setup file.

\subsubsection{Defining New Pesticides}

You can easily introduce new pesticides by specifying a few
parameters.  The parameters may not be that easy to obtain, though.
Here is an example:
\begin{verbatim}
  (defchemical Atrazine default
    "From a FAO sponsored field experiment near Andeby, 2002."
    (canopy_dissipation_rate 8e-3 [h^-1])
    (canopy_washoff_coefficient 1.0 [])
    (crop_uptake_reflection_factor 0.8 [])
    (decompose_rate 7e-5 [h^-1])
    (diffusion_coefficient 8e-6 [cm^2/s])
    (solute (adsorption linear (K_clay 7 [cm^3/g]))))
\end{verbatim}
As we have seen before, in the first line we \emph{def}ine a new
\emph{chemical} name \emph{Atrazine}, using the \emph{default} (and
currently only) model for chemicals in \daisy{}.  In the second line,
we put some information about this chemical so we will know later
where these numbers come from.

\subsubsection{Canopy}

The pesticide will be introduced into the system with the
\texttt{spray} command.  Some of the pesticide will land directly on
the soil surface, while other will hit the canopy, depending on how
large a fraction of the soil is covered by the crops.  The fraction
that hits the canopy is assumed to stay there, until it either
dissipate entirely out of the system, or is washed down to the soil
surface by rain or irrigation.  

The parameter \texttt{canopy\_dissipation\_rate} indicates how large a
fraction of the pesticide dissipate from the canopy each hour, while
\linebreak{}\texttt{canopy\_washoff\_coefficient} is a measure of how
the pesticide disolve in the intercepted water.  If you set it 0.0 is
does not disolve at all, and will never be washed off.  If you set it
to 1.0 it will be fully disolved, and thus follow any water that fall
off the canopy.

\subsubsection{Soil Surface and Root Uptake}

Down on the soil surface, \daisy{} will pretend that the pesticide is
immediately fully dissolved in any water that hits the soil surface,
and follow the water when it enters the soil.  Nothing else will
happen to the pesticides when on the soil surface.  Pesticides in the
top of the soil can also move up to the soil surface in case of
ponding.  This effect can be significant when there are macropores
activated by the ponding, as pesticides that have already entered the
soil can later be transported with macropores from the surface.

In the soil, the pesticides can be adsorbed, transported, decomposed,
or taken up by the roots by transpiration.  The later is controlled by
\linebreak{}\texttt{crop\_uptake\_reflection\_factor}.  In the
example, we specify that 80\% of the pesticides dissolved in the water
extracted by the roots for transpiration will be reflected, that is
will stay in the soil water.  The 20\% that is uptaken will disappear
out of the system.

\subsubsection{Decomposition}

We set \texttt{decompose\_rate} to the fraction of pesticides assumed
to be decomposed each hour.  This rate is affected by a number of
factors, listed here:
\begin{description}
\item[\texttt{decompose\_heat\_factor},
  \texttt{decompose\_water\_factor}] Biological
  decomposition\linebreak{} tend to accelerate with increasing heat
  and moisture.  Unless you specify otherwise, the factors used for
  mineralization will be reused here.
\item[\texttt{decompose\_CO2\_factor}] CO$_2$ production is an
  indicator of an active soil biomass.  By specifying this parameter,
  you can let the general biological activity influence the
  decomposition. 
\item[\texttt{decompose\_conc\_factor}] Some pesticides will be hard
  to decompose at high or low concentrations, which can be specified
  with this parameter. 
\item[\texttt{decompose\_depth\_factor}] Sometimes the decompose rate
  depend on how deep in the soil we are, which is what this parameter
  is for. 
\item[\texttt{decompose\_lag\_increment}] Some pesticides require
  specialized organisms to decompose, which may take some time to
  appear.  \Daisy{} therefore keep track of a time lag variable for
  each pesticide, which starts at zero.  When it reaches one,
  decomposition begins.  You can set this parameter to specify the
  hourly increment, as a function of the concentration in the soil.
  By default, the increment is 1, which means decompositions starts
  immediately, with no time lag.
\end{description}
See~\cite{dina81} for more specific information about each of these
factors. 

\subsubsection{Transportation}

The pesticide in the soil can be transported with convection and
diffusion.  Set \texttt{diffusion\_coefficient} to specify the later.
Finally, the pesticide may adsorb to the soil.  \Daisy{} have a number
of advanced models for adsorption, but for pesticides it is usually
adequate to either disable adsorption \texttt{(adsorbtion none)}, or
use a simple linear correspondence between adsorbed and dissolved
matter, $A = K \rho C$, where $A$ is the amount of adsorbed matter,
$K$ is a pesticide and soil specific parameter, $\rho$ is the bulk
density of the soil, and $C$ is the amount of dissolved matter.  In
\daisy{}, we have split $K$ into a \texttt{K\_clay} parameter which is
multiplied with the clay fraction of the soil, and a \texttt{K\_OC}
parameter which is multiplied with the humus fraction of the soil.

% TODO: Transformations.

\subsubsection{Log models}

Here are some log models that are useful when dealing with pesticides:
\begin{description}
\item[\texttt{Surface Chemicals}] Summary information about chemicals
  on the soil surface. File: \texttt{surface\_chemicals.dlf}.
\item[\texttt{Soil Chemical}] Summary information about chemicals in
  the soil.  File: \texttt{soil\_chemical.dlf}. 
\item[\texttt{Chemical}] Amount of chemicals in each numeric soil
  layer.  Note that nothing will appear in the log file until the
  chemichals have entered the soil.  Not even zeros.  File:
  \texttt{chemical.dlf}.
\end{description}

In general, you will get the sum of all chemicals.  To log the content
of a specific pesticide, such as `Atrazine', specify the \verb|$chem|%$
parameter, as in:
\begin{verbatim}
    ("Soil Chemical" (set "$chem" Atrazine))
\end{verbatim}%$

\section{Log Files}
\label{sec:output}

In this section I list some useful log models not listed elsewhere,
but first lets refresh the general useful parameters.  The following
two parameters can be used for almost all log models:
\begin{description}
\item[\texttt{when}] Indicates how often to log.  In theory, you can
  use any condition, like e.g.\ \texttt{crop\_dm\_over}, but in
  practice \texttt{hourly}, \texttt{daily}, \texttt{weekly},
  \texttt{monthly} and \texttt{yearly} are the most useful ones.  The
  log models will automatically accumulate flux values between time
  steps.  When not otherwise indicated, log models log every hour.
\item[\texttt{where}] Select the file in which to log the results.
\end{description}

As explained in~\ref{ex:column}, the soil is split into a large number
of numeric layers for computational purposes.  Some log models, such
as \texttt{Soil Temperature}, will log the content of a single state
variable for each of these layers.  Other log models will log many
state variables by integrating the content of the variable in for all
layers in some interval, or for flux variables, use the values at the
top or bottom of the interval.  \texttt{N Balance} is an example of
such a log model.  You can control what interval to summarize by
setting the \texttt{from} and \texttt{to} parameters.

We use all of them in this example:
\begin{verbatim}
  (output ("N Balance" (when monthly)
                       (where "nitrogen-balance.dlf")
                       (from 0 [m]) (to -1 [m])))
\end{verbatim}

Note that writing a log file tend to be as slow as running the
simulation itself, and if you specify many log files \daisy{} will
become very slow indeed.  It also depend on the log file, as a general
rule, large log files are much more time consuming than small log
files.  So don't ask for hourly values of daily or even monthly values
will do.  But the `harvest' log model is nearly free.  And if you are
using \daisy{} interactively, don't specify a log file just because
you \emph{may} need the information.  If you later discover that you
need the information in that log file, you can rerun the simulation
specifying log file only.

\subsection{Log models}
\label{sec:log-models}

Here are some useful log models, you can find more in the
\texttt{log.dai} file, and scattered around this manual.
\begin{description}
\item[\texttt{Total Soil Content}] Information about the water, carbon
  and nitrogen content of the soil.  File: \texttt{soil\_content.dlf}.
\item[\texttt{Soil NO3 Concentration}] Amount of nitrate in all
  numeric soil layers.  This include both the nitrogen and oxygen
  atoms of the nitrate.  File: \texttt{soil\_NO3.dlf}.
\item[\texttt{Soil NO3-N Concentration}] Amount of nitrate nitrogen in
  all numeric soil layers.  This include only nitrogen atoms of the
  nitrate.  File: \texttt{soil\_NO3-N.dlf}.
\item[\texttt{Soil Temperature}] Temperature in all numeric soil
  layers. File: \texttt{soil\_temperature.dlf}.
\item[\texttt{Soil Water Potential}] Water potential in all numeric
  soil layers.  File: \texttt{soil\_water\_potential.dlf}.
\item[\texttt{Soil Water Sink}] The water sink in all numeric soil
  layers.  The water sink is an aggregate of root uptake, preferential
  flow, flow to drain pipes, and subsoil irrigation.  File:
  \texttt{soil\_water\_sink.dlf}.
\item[\texttt{Matrix Water Flux}] The matrix water flux between all
  numeric soil layers.  File: \texttt{matrix\_water\_flux.dlf}.
\item[\texttt{Surface Water Balance}] Enough information to create a
  water balance for the soil surface.  File:
  \texttt{surface\_water\_balance.dlf}.
\item[\texttt{Root Zone Water Balance}] Enough information to create a
  water balance for the root zone.  File:
  \texttt{root\_zone\_water\_balance.dlf}. 
\item[\texttt{N Balance}] Enough information to create a balance for
  the soil nitrogen.  File: \texttt{N\_balance.dlf}.
\item[\texttt{Soil Heat Balance}] All available information about the
  soil heat subsystem.  File: \texttt{soil\_heat\_balance.dlf}.
\item[\texttt{Snow}] All available information about the snow
  subsystem.  File: \texttt{snow.dlf}.
% TODO: \item[\texttt{}]   File: \texttt{}.
\end{description}

\section{Organizing Your Parameterizations}
\label{sec:files}

If you follow the advice in this tutorial, you will create a number of
parameterizations for things like horizons, columns, fertilizer types,
and crop management strategies.  Most likely, many of these
parameterizations will be generically useful, that is, useful in many
different simulations.  My advice is that you from the start create a
\emph{library} of such parameterizations, for later use.  This library
should be kept separate from both the standard library that comes with
\daisy{}, and from the files that are specific to your current
simulation.  By separate, I mean that you should keep the files in
three or more different directories.
\begin{itemize}
\item A directory containing the parameterization files distributed by
  \textsc{kvl}, i.e.\ the content of \texttt{daisy-lib.zip}.  You
  should never change these files, and never add new files yourself.
  If you want to add anything there, send your additions to
  \href{mailto:daisy@dina.kvl.dk}{\texttt{$<$daisy@dina.kvl.dk$>$}},
  and let me do it.  The advantage of this is that it will be easy to
  upgrade your \daisy{} installation when a new and improved version
  with new and improved features, and new and improved bugs, arrives.
  You just delete the old content of the directory, and install the
  new files instead.  Or better, rename the old directory so you can
  backup in case anything is horrible wrong with the new version.
\item A directory containing files with those parameterizations that
  are not specific to a single simulation.  In that directory you can
  organize the parameterizations by locations (e.g.\ put all horizon
  and column parameterizations containing data measured in Andeby in a
  single \texttt{andeby.dai} file) or by type (e.g.\ put all crop
  management specifications in a single \texttt{crop-man.dai} file) or
  both, depending on taste.
\item A directory for each simulation, containing a setup file, extra
  files with simulation specific parameterizations or other data, and
  all the output (\texttt{.dlf}) files produced by the simulation run.
\end{itemize}
From experience, such an organization saves time in the long run.  

\subsection{Don't Copy, Derive}

A related hint for your own parameterizations: Don't copy, derive.
For example, if you want a slightly different grass than the one
provided by the standard \daisy{} library, don't just copy the
definition from \texttt{grass.dai} and edit the parameter you want to
change.  And certainly don't change the \texttt{grass.dai} directly.
Instead, create a derived parameterization as we did in
section~\ref{sec:bioclimate}:
\begin{verbatim}
  (defcrop "Andeby Spring Barley" "Spring Barley"
    "Spring Barley with modified EpFac as measured
  in Andeby, 2002, for the SuperGrow(TM) cooperation."
    (Canopy (EpFac 1.2)))
\end{verbatim}
The advantage of doing it this way twofold.  First, if we fix or
improve the grass parameterization in \texttt{grass.dai}, you will
derived version will automatically get the improvements when you
install the updated \daisy{} version.  Furthermore, we may make
changes to the default crop model that requires corresponding changes
to the parameterizations.  Of course, we do that for the standard
parameterizations right away.  But if you have made a copy of the
parameterizations, it will break when you install the new version.

\subsection{\textsc{Daisypath}}
\label{sec:daisypath}

Now that you have followed my advice, and spread your files over three
directories, how do we tell \daisy{} where to find it?  Well, \daisy{}
uses two sources to find the files:
\begin{itemize}
\item A \emph{path} containing information about which directories to
  look, and
\item the \emph{current directory}, which is also the place it will
  put any log files. 
\end{itemize}

The preferred way to tell \daisy{} what path to use is through the
\textsc{daisypath} environment variable.  Unfortunately, how to set
this depend on what operating system you are using.  Under Unix
(including Linux), it has the same syntax as the \textsc{path}
environment variable, and is set the same place, i.e.\ your
\texttt{.bashrc} or \texttt{.cshrc} file, depending on your shell. 

On MS Windows, it still have the same syntax as
\textsc{path}\footnote{With the disclaimer that the syntax for
  \textsc{path} is different on Unix and MS Windows.}, and is set the
same way.  If you are one of the rare MS Windows users who know how to
do that, go ahead and do it.  It does depend on which exact version of
MS Windows you use, though.  

In \cite{dina81} there are some examples of how to set
\textsc{daisypath} on various systems.  However, you may prefer the
alternative of putting this information in the setup files themselves.
Insert
\begin{verbatim}
  ;; Search file files these places.
  (path "." "c:/My Files/daisy/lib/" "c:/Program Files/daisy/lib/")
\end{verbatim}
assuming you installed the \daisy{} standard parameter files
in\linebreak[4] ``\verb|c:\Program Files\daisy\lib|'' and your own
parameter files in\linebreak[4] ``\verb|c:\My Files\daisy\lib|''.
Note that ``\texttt{.}'' means ``look in the current directory'' and
that you should use forward slashes (\verb|/|) instead of backslashes
(\verb|\|) in the path names.

If you start \daisy{} from the command line, as in
section~\ref{sec:install}, the current directory will be the directory
where you run the command.  However, you can also specify it
explicitly in the setup file, like this:
\begin{verbatim}
  ;; Run program here.
  (directory "c:/My Files/Andeby/sim01")
\end{verbatim}
An advantage of doing it this way, is that you can start the simulation
by dragging the setup file to the \texttt{daisy.exe} icon in the file
manager.  If you don't specify \texttt{directory}, it will run the
program somewhere silly.

The reason I recommend the \textsc{daisypath} route, is that the setup
files become location independent.  That is, if you send it to someone
else who have installed \daisy{} in another location, he can still run
the setup file unchanged.  If you specify \texttt{path} in the setup
file, he has to change one line.  If you specify \texttt{directory},
he has to change two lines.  What I don't recommend is neither
specifying \textsc{daisypath}, nor setting \texttt{path}, but using
absolute file names.  E.g.
\begin{verbatim}
  (input file "c:/Program Files/daisy/lib/fertilizer.dai")
\end{verbatim}
It works, but gives a lot of lines that must be changed when moving
the files.

\appendix

\section{Experimental features}
\label{sec:exp}

These features are still experimental, and should not be used for
``normal'' simulations.  Use them for simulations where the main
purpose is to examine the effect of each feature.

\subsection{Ice}
\label{sec:ice}

% TODO

\subsection{Soil Compaction}
\label{sec:compact}

% TODO

\subsection{Phosphor}
\label{sec:phosphor}

% TODO: not supported, strongly adsorbed, solid, transformations.

\subsection{Dividing and Merging Fields}
\label{sec:divide}

% TODO: grazing

\subsection{Ridging}

A special tillage operation is ridging.  Ridging is common in tropical
agriculture, where it has various beneficial effects, such as
preventing erosion, controlling the soil water, and increasing the
effective rooting depth, depending on the situation.  It is also used
under temperate conditions, for example for potato farming in Denmark,
where ridging provide better growing conditions in the soil,
especially with regard to heat.

In \daisy{}, ridges only affects the flow of soil and surface water,
and a hard-to-penetrate crust is assumed, making it less relevant to
Danish potato farming.

You can specify that the field should contain ridges with the
\texttt{ridge} operation.
\begin{verbatim}
  (ridge (z (0.0 0.0) (0.4 0.0) (0.6 50) (1.0 50.0))
         (R_crust 50 [h]))
\end{verbatim}
where
\begin{description}
\item[\texttt{z}] The ridge height, formulated as $z (x)$, where $x$
  is the relative distance from the middle of the ridge.  $x = 0.0$ is
  in the middle of a ridge, while $x = 1.0$ is at the maximal
  distance.  The ridge is assumed to be symmetric.  $z (x)$ is
  measured in centimeter above the unridged soil surface, which means
  it is in the same reference system as the rest of the model.
\item[\texttt{R\_crust}] Resistance in crust.  Adjust this to control
  the rate in which water infiltrate.
\end{description}
The ridge must not be taller than the top horizon, as there should be
uniform hydraulic properties throughout the ridge.

To remove the ridges, specify any other tillage operation.

Technically, the ridge operation creates a special subsystem within
\daisy{} where water flow sideways into the ridge wall or down through
the ridge bottom.  You can log this subsystem with the following log
parameterization:
\begin{description}
\item[\texttt{Ridge}] Information about the Ridge subsystem.  File:
  \texttt{ridge.dlf}.
\end{description}

\subsection{Forced LAI}
\label{sec:forced-lai}

You can force the CAI to have a specific value for part of the year.
This is useful if you have measured the total CAI and want Daisy to
use those values for photosyntheses and transpiration (and
interception capacity).  Here is an example:
\begin{verbatim}
  (defcolumn Andeby default
    "The B.And farm, Andeby, 2003."
    (Vegetation crops
                (ForcedLAI (1987 (100 2.0) (200 4.0))
                           (1988 (100 0.0) (150 1.5) (200 5.0))))
    (Soil (horizons (-20 [cm] Ap) (-2.5 [m] C))
          (MaxRootingDepth 60.0 [cm]))
    (Groundwater deep))
\end{verbatim}
The code above should be read: In 1987 we start using forced CAI day
100 where it is 2.0 and end forced CAI in day 200 where it is 5.0.  We
use liniear interpolation between these two points, so in day 150 it
will be 4.0.  Before day 100 and after day 200 we use the simulated
CAI values.  In 1988 we start day 100 at CAI 0.0, increase linearly to
1.5 day 150, and from there again linearly to 5.0 day 200, where we
switch back to using simulated CAI\@.  All other years we use simulated
LAI.

Only paramterizations of the \texttt{default} crop model will be
affected by the \texttt{ForcedLAI} attribute.

If there are mulitple crops, the forced CAI will be distributed among
them corresponding to the relative size of their simulated CAI\@.  When
forced CAI is active, this is the only use of the simulated CAI\@.  

\subsubsection{Log models}

The simulated CAI will be calculated anyway and can be logged.  Most
log files will only log the CAI actually used which, when Forced LAI
is in effect, will not be the sum of Leaf, Stem and SOrg AI\@.  The new
"Forced LAI" log model will log both the used, forced and simulated
CAI\@.  By default, it will log the sum of all crops, set
\verb|$crop|%$
to log a particular crop alone as below:

\begin{verbatim}
  (output "Forced LAI"
          ("Forced LAI"
           (set "$crop" "Spring Barley")
           (where "sbarley.dlf"))
          ("Forced LAI"
           (set "$crop" "Grass to grain")
           (where "grass.dlf")))
\end{verbatim}

\section{\texttt{test.dai}}
\label{app:test}

\begin{verbatim}
  ;;; test.dai -- Sample file using the Daisy libraries.

  (description "Simulation for use in tutorial.")

  ;; Use standard parameterizations.
  (input file "tillage.dai")
  (input file "crop.dai")
  (input file "log.dai")

  ;; Weather data.
  (weather default "weather.dwf")

  ;; We have some very sandy soil.
  (defhorizon Ap default
    (clay 8.0 [])
    (silt 10.5 [])
    (coarse_sand 63.4 [])
    (fine_sand 16.5 [])
    (humus 1.12 [])
    (SOM_fractions 0.66 0.34 []))

  (defhorizon C Ap
    (humus 0.12 [])                     ;Less humus.
    (SOM_fractions 0.80 0.20 []))       ;Slower background mineralization.

  ;; We build the column from the horizons.
  (defcolumn Andeby default
    "The B.And farm, Andeby, 2002."
    (Soil (horizons (-20 [cm] Ap) (-2.5 [m] C))
          (MaxRootingDepth 60.0 [cm]))
    (Groundwater deep))

  ;; Use it.
  (column Andeby)

  ;; Simulation start date.
  (time 1986 12 1 1)

  (manager activity
    (wait (at 1987 3 20 1))
    (plowing)
    (wait (at 1987 4 4 1))
    (fertilize (mineral (weight 100.0 [kg N/ha])
                        (NH4_fraction 0.5 [])))
    (wait (at 1987 4 5 1))
    (progn
      (sow "Grass")
      (sow "Spring Barley"))
    (wait (or (crop_ds_after "Spring Barley" 2.0)
              (at 1987 9 5 1)))
    (harvest "Spring Barley")
    (wait (at 1987 9 8 1))
    (fertilize (mineral (weight 80.0 [kg N/ha])
                        (NH4_fraction 0.5 [])))
    (wait (at 1987 10 10 1))
    (harvest "Grass"
             (stub 8.0 [cm])              ;Leave 8 cm stub.
             (stem 1.00 []))             ;Harvest everything above stub.
    (wait (at 1988 4 1 1))
    (stop))

  ;; Create these log files.
  (output harvest
          ("N Balance" (when monthly)
                       (from 0 [m]) (to -1 [m]))
          ("Crop Production"
           (set "$crop" "Spring Barley")
           (where "sbarley.dlf"))
          (checkpoint (when (at 1987 8 7 6))))

  ;;; test.dai ends here.
\end{verbatim}
%$

\section{Bibliography}
\bibliography{daisy}

\end{document}

%%% Local Variables:
%%% mode: latex
%%% TeX-master: t
%%% End:
