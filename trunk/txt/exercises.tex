\documentclass[a4paper,11pt,twoside]{article}
\usepackage{a4}
\usepackage[T1]{fontenc}
\usepackage[latin1]{inputenc}
\usepackage{hyperref}
%\usepackage{times}
% Boxed listings.
\usepackage{boxedminipage}
\renewcommand{\figurename}{Box}

% Larger pages.
\addtolength{\textwidth}{4cm}
\addtolength{\hoffset}{-2cm}
\addtolength{\oddsidemargin}{1cm}
\addtolength{\evensidemargin}{-1cm}

% Per section figure and table numbering.
\makeatletter
\@addtoreset{table}{section}
\renewcommand\thetable{\thesection.\@arabic\c@table}
\@addtoreset{figure}{section}
\renewcommand\thefigure{\thesection.\@arabic\c@figure}
% Boldface, left adjusted captions.
\long\def\@makecaption#1#2{%
   \vskip\abovecaptionskip
   \setbox\@tempboxa\hbox{{\bf#1:} #2}%
   \ifdim \wd\@tempboxa >\hsize
     {\bf #1:} #2\par
   \else
     \global \@minipagefalse
     \hb@xt@\hsize{\box\@tempboxa\hfil}%
   \fi
  \vskip\belowcaptionskip}
\makeatother

\bibliographystyle{apalike}
\newcommand{\daisy}{Daisy}
\newcommand{\Daisy}{Daisy}
\newcommand{\cplusplus}%
{{\leavevmode{\rm{\hbox{C\hskip -0.1ex\raise 0.5ex\hbox{\tiny ++}}}}}}
\newcommand{\Cplusplus}{\cplusplus}
\newcommand{\mshe}{Mike/\textsc{she}}
\newcommand{\wintel}{\texttt{win32}}
\newcommand{\dll}{\textsc{dll}}
\newcommand{\Dll}{\textsc{Dll}}
\newcommand{\gui}{\textsc{gui}}
\newcommand{\Gui}{\textsc{Gui}}
\newcommand{\unix}{Unix}
\newcommand{\dhi}{\textsc{dhi}}
\newcommand{\Dhi}{\textsc{Dhi}}
\newcommand{\api}{\textsc{api}}
\newcommand{\Api}{\textsc{Api}}
\newcommand{\lai}{\textsc{lai}}
\newcommand{\Lai}{\textsc{Lai}}
%\newcommand{\url}[1]{\linebreak[4]\texttt{<URL:#1>}}

%%% Local Variables: 
%%% mode: latex
%%% TeX-master: t
%%% End: 


% External references.
\usepackage{xr}
\externaldocument[tut-]{tutorial}
\externaldocument[ref-]{reference}

% Constant references.
\newcommand{\daisytut}{\textit{\Daisy{} Tutorial}}
\newcommand{\daisyref}{\textit{\Daisy{} Program Reference Manual}}

% Easy of edit more important than ease of reading.
\sloppy

\begin{document}

\section*{General introduction to the application of the \Daisy{} model}

S�ren Hansen\\
The Royal Veterinary and Agricultural University\\
Department of Agricultural Sciences\\
Laboratory for Agrohydrology and Bioclimatology\\
\\
Updated by Per Abrahamsen, \today{}.

\begin{abstract}
  This introduction to the application of the \Daisy{} model comprises
  minor 12 exercises.  Exercise~\ref{ex:intro} illustrates how to set
  up a simple simulation using minimum information on the simulated
  system and how to extract simulation results.
  Exercises~\ref{ex:hydraulic}~\&~\ref{ex:retc} are dealing with soil
  hydraulic properties and discretization of the soil profile.
  Exercise~\ref{ex:irrigation} is introducing irrigation.
  Exercise~\ref{ex:groundwater} is dealing with tile drains and the
  calibration of the interface to the groundwater.
  Exercise~\ref{ex:rotation}~\&~\ref{ex:organic} are introducing the
  simulation of rotations and the use of organic fertilizers.
  Exercise~\ref{ex:warmup} is illustrating the importance of
  introducing a ``warm-up'' period. In Exercise~\ref{ex:mit} the
  MIT-model (Mineralization-Immobilization Turnover) is calibrated.
  Exercise~\ref{ex:weather} builds a \Daisy{} weather file.  In
  Exercise~\ref{ex:defam} an organic fertilizer is parameterised.  In
  Exercise~\ref{ex:croppar} a crop is parameterised.
  
  A technical description of the \Daisy{} model is given
  by \cite{daisy-ems} and \cite{daisy-new}.
\end{abstract}

\tableofcontents

\section*{Introduction}

In order to run a \daisy{} simulation the model needs
parameterisation. The required information is parsed on to the model
through setup-files (or \texttt{.dai} files), which are written in a special
input-language. \Daisy{} stores information in an internal database.
The information parsed on to the model need not occur in any special
order. The \daisy{} code contains a partial parameterisation, i.e.\ a
number of parameters is given default values at initialisation.
However, it is always possible to overwrite the default
parameterisation. Information on soil horizons is stored in the
internal database by the \texttt{horizon}--component:
\begin{verbatim}
  (defhorizon "Ap JB6" FAO3
    "An optional description."
    ;; More parameters
    (clay 15.5 [%]))
\end{verbatim}
where \texttt{defhorizon} is short for define horizon. Other similar
expressions are \texttt{defcolumn} (defining a soil column including
the soil surface and the bioclimate above it), \texttt{defcrop}
(defining crop growth parameters), \texttt{defam} (defining added
matter, e.g.\ fertilizers), \texttt{defaction} (defining an action
list of management actions), and \texttt{deflog} (defining
output-variables). \texttt{"\textit{Ap JB6}"} is the name which the
parameterisation will have in the internal database and which later
can be referred to when one wants to retrieve the stored information.
The \texttt{FAO3} keyword retrieves the structure of the internal
database and, hence, it specifies the type of information that now can
be expected and it also includes default values given to any of the
parameters of the horizon. Alternatively, the name of an existing
element in the database (a previously defined horizon) could be stated
instead of the expression \texttt{FAO3}\footnote{\texttt{(defhorizon "Ap
    Askov" "Ap JB6" (humus 1.5 [\%]))} this statement overwrites the
  humus content but retains all other information stored in ``Ap
  JP6''.}.  The expression ``\texttt{(clay 15.5 [\%])}\ldots'' refers
to an attribute list (an attribute list is an unordered set of
(\textit{name, value}) pairs), which contains the parameters to be
added or changed as compared to the selected database element (default
or a previously defined horizon). In the present example the component
``horizon'' has a member ``clay'' of type ``number'' and furthermore
the value 15.5 is assigned to this member.  The square brackets after
the number (15.5) is used to give the dimension of the number. In the
present example the clay content is given as percent. If the brackets
are left out, \daisy{} will use a default dimension, which may not be
what you think.  So don't leave out the dimension.

It is noted that parentheses are used to group information e.g.
\texttt{(clay 0.155 [])}. Similarly all the information defined by the
defhorizon statement and stored under the name ``\texttt{Ap JB6}'' are
embedded in parentheses. Parentheses are the main structural element
of the \daisy{} language syntax.

Furthermore it is noted that the name ``\texttt{Ap JB6}'' is imbedded
in quotations marks. Names may always be imbedded in quotation marks,
but it is only necessary if the name contains special symbols
including white spaces. The only special symbol, which can be used
without the use of quotation marks, is the underscore.

Comments, which are ignored by the parser, can be included in
\daisy{}-setup-files by use of semicolons (;). The parser ignores any
text occurring after a semicolon until end-of-line. The only exception
is when the semicolon is imbedded in quotation marks, i.e.\ is a part
of a name.

A comprehensive description of the \daisy{} input-language is given in
\daisyref{}~\cite{daisy-ref}.

\section{Writing a \Daisy{} Setup File}
\label{ex:intro}

The objective of this exercise is to illustrate how to write a
\daisy{}-setup-file, run a simple simulation, and extract simulation
results.

When writing a \daisy{}-setup-file it is practical first to define
computational environment and the external library-files, which is
going to be used by the simulation, as shown in box~\ref{box:header}.

\begin{figure}[!htbp]
  \begin{boxedminipage}{\textwidth}
    \caption{\Daisy{}-setup-file header}
    \label{box:header}
\begin{verbatim}
;; Setting "working directory", note that `/' is used and not `\'
(directory "C:/DaisySim/Exercise/01")
;; Setting search path for library-files
(path "." "C:/Daisy/Lib" "C:/Daisy/Weather")

;; Including external library-files
(input file "tillage.dai")
(input file "crop.dai")
(input file "fertilizer.dai")
(input file "log.dai")

;; Weather data
(weather default "Taastrup6201.dwf")

;; Description that will occur in all output files
(description "Spring Barley; Soil: Fine sandy loam; Weather: Taastrup")
\end{verbatim}
  \end{boxedminipage}
\end{figure}

It is noted that first the working directory is set. The reason is
that the \daisy{} code writes a log-file (\texttt{daisy.log}), where
all messages from the simulation, including error messages, is stored.
Hence if the working directory is set correctly then this log file
will occur in the selected directory.

The external library-files are written in the same \daisy{}
input-language as the \daisy{}-setup-file. In fact any information,
which is supplied by the \daisy{}-setup-file, could be moved to an
external file, which then could be included by the input file
statement. On the other hand any information supplied by the external
library-files could also be written directly to the
\daisy{}-setup-file. The use of external library-files is just a
convenient way of organising ones data. There is only one hard rule
that must be obeyed and that is that any information that is used must
be defined before any reference is made to it.

Weather data are driving variables in the simulation and is supplied
to the model through a special file written in a special format (daisy
weather format or \texttt{dwf}). This format shall be explained later.

Basic soil horizon information is given in table~\ref{tab:basic}. This
information is transferred to the internal library of the model using
defhorizon as shown in box~\ref{box:horizon}. 

\begin{table}[!htbp]
  %\centering
  \caption{Basic soil horizon parameters.}
  \label{tab:basic}
  \begin{tabular}{llllllllllllll}\hline
    Depth    & Dry bulk density & Humus & Clay     & Silt  & Sand \\
    cm       & g/cm$^{3}$       &       & <2$\mu$m &      & >50$\mu$m\\\hline
    00 -- 30 & 1.53             & 0.026 & 0.113    & 0.277 & 0.584 \\
    30 -- 80 & 1.51             & 0.005 & 0.235    & 0.253 & 0.507 \\
    80 --    & 1.57             & 0.002 & 0.244    & 0.283 & 0.471 \\\hline
  \end{tabular}
\end{table}

\begin{figure}[!htbp]
  \begin{boxedminipage}{\textwidth}
    \caption{Defining a horizon.}
    \label{box:horizon}
\begin{verbatim}
;;  Defining a soil horizon
(defhorizon "Ap F.S.L." FAO3
  "Data from Jacobsen, 1989"
  (dry_bulk_density 1.53 [g/cm^3])
  (clay 11.3 [%])
  (silt 27.7 [%])
  (sand 58.4 [%])
  (humus 2.6 [%])
  (C_per_N 11.0 [g C/g N]))
\end{verbatim}
  \end{boxedminipage}
\end{figure}

Using \texttt{defhorizon} many more parameters can be defined, see
\daisyref{}~\cite{daisy-ref}. When these parameters are not supplied
by the \texttt{defhorizon}-statements then the model uses either
predefined default values or pedotransfer functions to estimate the
missing parameters. E.g., turnover rates of organic matter have been
assigned default values, soil thermal properties are estimated by a
modified de Vries model~\cite{daisy-new}, and soil hydraulic
properties are estimated by the \textsc{hypres} pedotransfer function
\cite{hypres}.

Based on horizon information a column can now be build. In principle
the column comprises the bioclimate (upper boundary, exchange with the
atmosphere), the soil profile itself, and a groundwater condition
(lower boundary, exchange with the groundwater). However the
bioclimate component is fully parameterised by defaults, and hence it
is not necessary to include it in the setup. A column parameterisation
based on minimum requirements is shown in box~\ref{box:column}.
\begin{figure}[!htbp]
  \begin{boxedminipage}{\textwidth}
    \caption{Defining a column.}
    \label{box:column}
\begin{verbatim}
;;  Parameterisation of column (Fine sandy loam)
(defcolumn "Fine sandy loam" default
     (Soil (MaxRootingDepth 100 [cm])
           (horizons (  -30 [cm] "Ap F.S.L.")
                     (  -80 [cm] "B F.S.L.")
                     ( -400 [cm] "C F.S.L.")))
     (OrganicMatter original (init (input 3000 [kg C/ha/y])))
     (Groundwater deep))
\end{verbatim}
  \end{boxedminipage}
\end{figure}
Firstly, it is noticed, that the column is given the name
\texttt{"Fine sandy loam"}, secondly, that the parameterisation
comprises two sections, a soil section and a groundwater section, as
the bioclimate section is omitted. The soil section contains
information on the maximum rooting depth in the soil and the soil
horizons of the profile. The A-horizon extends down to 30 cm depth,
the B-horizon is found in the interval 30-80 cm and C-horizon starts
at 80 cm depth and continues down to at least 400 cm depth, where the
calculations stop. 

The organic matter section gives information used in initialization of
the MIT-model
(\textit{M}ineralization-\textit{I}mmobilization-\textit{T}urnover
model).  The figure 3000 kg C/ha/y indicated the average annual input
of organic carbon in the soil in the decade before the start of the
simulation.  The carbon input comprises carbon in organic fertilizer,
plant residuals, and rhizodeposition.  The model contains a procedure
that automatically parameterize the MIT-model based on this
information. 

The groundwater section tells that the position of the groundwater
table is located deep below the 400 cm limit of the calculations. This
means that a free drainage can be selected for the soil water dynamics
calculations. Again, when only this minimum information is supplied
then the model makes extensive use of default parameterisation.

The next step is to define a list of actions performed in order to
grow spring barley; in this case the list is named
\texttt{SBarley\_management}, box~\ref{box:sbarley}. 
\begin{figure}[!htbp]
  \begin{boxedminipage}{\textwidth}
    \caption{Spring barley management.}
    \label{box:sbarley}
\begin{verbatim}
;; Defining an action list related to management  
(defaction SBarley_management activity
  (description "Management plan for spring barley.")
  (wait_mm_dd 3 20)
  (plowing)
  (wait_mm_dd 4 15)
  (seed_bed_preparation) 
  (sow "Spring Barley")
  (wait_mm_dd 4 20) 
  (fertilize (N25S (weight 95 [kg N/ha]))) 
  (wait (or (crop_ds_after "Spring Barley" 2.0) ;Ripe
            (mm_dd 9 1))) 
  (harvest "Spring Barley"
           (stub 8 [cm])
           (stem 0.70 [])
           (leaf 0.70 [])))
\end{verbatim}
  \end{boxedminipage}
\end{figure}
It is noted that the action list comprises of activities viz.\ to wait,
to plow, to make a seedbed preparation, to sow, to fertilize and to
harvest. The two tillage operations, \texttt{plowing} and
\texttt{seed\_bed\_preparation}, are in fact lists of primitive
actions found in the library ``\texttt{tillage.dai}''. The command sow
takes one parameter, which is the name of crop that is to be sown; in
this case the name is \texttt{"Spring Barley"}. \texttt{"Spring
  Barley"} is the name of a list of crop parameters, which is found in
the library ``\texttt{crop.dai}'', or in fact the parameter list is
found in a file which is included (input file) in the
\texttt{crop.dai}-file. The command fertilize takes up to arguments.
The first argument comprises, a name, which in the present case is
N25S, and an attribute list.  However, only the weight parameter,
\texttt{(weight 95 [kg N/ha])}, is given here, the remaining of the
required parameters are to be found in the library file
``fertilizer.dai''.  The first argument of command harvest is the name
of the crop to be harvested, and then follows a list of attributes
specifying the stubble length (here 8 cm) and the fraction of straw
\texttt{(stem 0.70 [])} and leaf \texttt{(leaf 0.70 [])} removed from
the field at harvest. Defaults are in this case that the stubble
length is 0 cm and that all straw and leaf are removed.

Wait statements are used in order to govern the timing of the other
management actions. Two types of wait statements are used in the
present action list. The first statement tells the model to wait until
a certain day in the year defined by month number and day number in
the month, e.g.\ \texttt{(wait\_mm\_dd 3 20)}. The second type waits
until a certain condition becomes true. The or-statement becomes true
when one of its conditions becomes true. The \texttt{crop\_ds\_after}
statement takes two arguments; the first is the name of the considered
crop and the second is a boundary development state. If the
development stage of the crop is greater or equal to the boundary
development stage then the statement is true. The crop development
stage is 0 at emergence, 1 at flowering, and 2 at maturing. Hence the
statement gets true when the spring barley is ripe. The second
condition of the or-statement becomes true when the date is 1$^{st}$
of September \texttt{(mm\_dd 9 1)}.

Now everything is ready so we can actually begin a simulation. As
shown in box~\ref{box:activation} we activate a column, a stating date
and an agricultural management practice.
\begin{figure}[!htbp]
  \begin{boxedminipage}{\textwidth}
    \caption{Activate simulation.}
    \label{box:activation}
\begin{verbatim}
;;  Selecting a previously defined column.
(column ``Fine sandy loam'')

;; Selecting the of the beginning of the simulation.
(time 1993 1 1)

;; Selecting the agricultural management and the end of simulation
(manager activity
   SBarley_management
   (wait_mm_dd 1 1) 
   stop)
\end{verbatim}
  \end{boxedminipage}
\end{figure}
However, we have not selected any output. This is done with the output
statement as shown in box~\ref{box:output}. In the present case we
only make use of predefined output logs that can be found in the
external library file ``\texttt{log.dai}''.
\begin{figure}[!htbp]
  \begin{boxedminipage}{\textwidth}
    \caption{Select output files.}
    \label{box:output}
\begin{verbatim}
(output Harvest
        "Crop Production"
;;      Water balance of the upper 100 cm  
        ("Root Zone Water Balance" (to -100 [cm])(when daily)
          (where "Daily_WB.dlf"))  
        ("Root Zone Water Balance" (to -100 [cm])(when monthly)  
          (where "Monthly_WB.dlf"))  
;;      Nitrogen balance of the upper 100 cm  
        ("N Balance" (to -100 [cm])(when daily)
          (where "Daily_NB.dlf"))  
        ("N Balance" (to -100 [cm])(when monthly)  
          (where "Monthly_NB.dlf"))  
;;      Soil profile data
        ("Soil Water Content" (when daily))
        ("Soil Water Potential"(when daily))
        ("Soil NO3 Concentration"(when daily)))
\end{verbatim}
  \end{boxedminipage}
\end{figure}
When \texttt{Harvest} is selected then the harvested yields are
written to the output-file ``\texttt{Harvest.dlf}''. The extension
\texttt{.dlf} indicates that the file is written in
\daisy{}-log-format, which is a format that easily can be opened in a
spreadsheet. Furthermore it indicates that the output can be inspected
by use of the \daisy{} output-viewer (\texttt{ShowDaisyOutput.exe}).
However, the only exception from the latter is the
\texttt{Harvest.dlf}-file; all other output files can be viewed by the
output-viewer. Furthermore \texttt{"Crop Production"} is selected. In
this case daily (default) values of variables pertaining to crop
production is written to ``\texttt{crop\_prod.dlf}''. It is noted that
the water and nitrogen balances is of the upper 100 cm (the command:
\texttt{(to -100 [cm])}) and that both the output frequency daily and
monthly is selected (the when-command) and that the different output
frequencies are written to different output-files (the
where--command). In addition some profile data are selected as
daily-values.

Fill in the necessary information in the \daisy{}-setup-file
\texttt{Exercise01.dai} and run a simulation by dragging and dropping
the setup-file at the \daisy{} executeable. Inspect the created
output-files by use of the output-viewer and an
\textsc{excel}-spreadsheet.  Construct a water balance, a mineral
nitrogen balance, an ammonium balance and a nitrate balance.

\section{Setting Hydraulic Properties }
\label{ex:hydraulic}

The objective of this exercise is to utilize measured soil hydraulic
properties, e.g.\ the soil water retention curve and the hydraulic
conductivity curve. Based on measurements soil water retention and
unsaturated hydraulic conductivity table~\ref{tab:M-vG} has been
prepared.
\begin{table}[!htbp]
  %\centering
  \caption{Mualem-van Genuchten parameters of soil hydraulic
    properties based on data from~\cite{jyndevad}. Parameter fitting
    is done by using the \textsc{retc} software~\cite{retc}}.
  \label{tab:M-vG}
  \begin{tabular}{llllll}\hline
    Depth  & $\theta_{s}$ & $\theta_{r}$ & $\alpha$  & $n$   & $K_{s}$\\
    cm     &              &              & cm$^{-1}$ &       & cm/h \\\hline
    00--30 & 0.403        & 0.000        & 0.0385    & 1.211 & 7.52\\
    30--80 & 0.421        & 0.000        & 0.2605    & 1.135 & 14.5\\
    80--   & 0.401        & 0.000        & 0.0570    & 1.131 & 1.65\\\hline
  \end{tabular}
\end{table}

The soil hydraulic properties are a part of the soil horizon. The
\daisy{} code is able to use a number of different descriptions or
models of soil hydraulic properties~\cite{daisy-ref}. The soil hydraulic
properties parameters are written to the \daisy{}-setup-file in the
\texttt{defhorizon} section as illustrated in box~\ref{box:hydraulic}.
\begin{figure}[!htbp]
  \begin{boxedminipage}{\textwidth}
    \caption{Hydraulic parameters}
    \label{box:hydraulic}
\begin{verbatim}
(defhorizon "Ap F.S.L." FAO3
  (description "Data from O.H. Jacobsen (1989)")
  ;; Insert other parameters here.
  (hydraulic M_vG 
             (Theta_res 0.0 [%])
             (Theta_sat 40.3 [%])
             (alpha 0.0385 [cm^-1])
             (n 1.211 [])
             (K_sat 7.52 [cm/h])))
\end{verbatim}
  \end{boxedminipage}
\end{figure}

\texttt{M\_vG} is the name of the selected hydraulic model (Mualem-van
Genuchten model) and the following attribute list is the
parameterisation of this model.

Compare the water and nitrogen balance results obtained with the
parameterised Mualem--van Genuchhten model with the corresponding
results obtained with the pedotransfer function.

\section{Calculating Hydraulic Properties}
\label{ex:retc}

Measured basic soil data for a coarse sandy soil is shown in
table~\ref{tab:texture} through table~\ref{tab:conductivity}. Use the
\textsc{retc} software to estimate the necessary parameters in order
to build a parameterisation of this soil.

\begin{table}[!htbp]
  %\centering
  \caption{Basic soil horizon parameters of a coarse sandy
    soil~\cite{jyndevad}.}
  \label{tab:texture}
  \begin{tabular}{llllll}\hline
    Depth  & Dry bulk density & Humus & Clay     & Silt  & Sand\\
    cm     & g/cm$^{3}$       &       & <2$\mu$m &       & >50$\mu$m\\\hline
    00--30 & 1.49             & 0.023 & 0.039    & 0.072 & 0.866\\
    30--   & 1.49             & 0.004 & 0.029    & 0.023 & 0.944\\\hline
  \end{tabular}
\end{table}

\begin{table}[!htbp]
  %\centering
  \caption{Soil water retention data~\cite{jyndevad}.}
  \label{tab:retention}
  \begin{tabular}{lllllllll}\hline
    Depth & Porosity & pF & pF & pF & pF & pF & pF & pF\\
    cm &  & 1.0 & 1.2 & 1.7 & 2.0 & 2.2 & 2.7 & 4.2\\\hline
    00--30 & 0.427 & 0.375 & 0.364 & 0.211 & 0.168 & 0.149 & 0.119 & 0.050\\
    30--& 0.436 & 0.369 & 0.343 & 0.097 & 0.073 & 0.059 & 0.045 & 0.020\\\hline
  \end{tabular}
\end{table}

\begin{table}[!htbp]
  %\centering
  \caption{Soil hydraulic conductivity data [cm/h]~\cite{jyndevad}.}
  \label{tab:conductivity}
  \begin{tabular}{llrrrr}                              \hline
    Depth  & Saturation & pF    & pF   & pF   & pF   \\
    cm     & \%         & 1.0   & 1.2  & 1.4  & 1.7  \\\hline
    00--30 & 21.7       & 10.03 & 7.18 & 3.95 & 0.032\\
    30--   & 39.5       &  4.59 & 2.93 & 0.15 & 0.002\\\hline
  \end{tabular}
\end{table}

So far we have used the default discretization in our simulations.
Introduce your own discretization and run a simulation. It is hard
rule that numeric layers may not cross horizon boundaries. It is a
rule of thump that the size of a numeric layer should not exceed twice
the dispercivity of the soil (the dispercivity is a member of the soil
as defined within the column and its default value is 5 cm). The
numeric layers are defined by their lower boundary (zplus) starting
from the top. An example of a discretization is shown in
box~\ref{box:discretization}.
\begin{figure}[!htbp]
  \begin{boxedminipage}{\textwidth}
    \caption{Discretization}
    \label{box:discretization}
\begin{verbatim}
(defcolumn "Coarse sand" default
  ;; Insert other parameters here.
  (Soil (MaxRootingDepth 50 [cm])
        (dispercivity 5 [cm]) 
        (horizons (  -30 [cm] "Ap C.S")
                  ( -320 [cm] "C C.S")))
  (Movement vertical
    (Geometry (zplus -2 -4 -7 -10 -13 -19 -22 -25 -27 -30 -35 -40 -45 -50
                     -55 -60 -65 -70 -75 -80 -85 -90 -95 -100 -120 -130
                     -140 -150 -160 -170 -180 --190 -200 -210 --220 -230 
                     -240 -250 -260 -270 -280 -290 -300 -310 -320 [cm])))
\end{verbatim}
  \end{boxedminipage}
\end{figure}
In box~\ref{box:discretization} the specification of the dispercivity
might as well be omitted as it only specifies the default value.

Does the crop in this example experience water stress?

\section{Irrigation}
\label{ex:irrigation}

The objective of this exercise is to introduce irrigation in the
management practice.

Develop an irrigation strategy for the spring barley grown on a coarse
sandy soil (exercise~\ref{ex:retc}). The \daisytut{}
section~\ref{tut-sec:irrigation} describes how this can be done, and
further information can be obtained from the \daisyref{}.

\section{Groundwater}
\label{ex:groundwater}

The objective of this exercise is to introduce tile drains and the
calibration of the interface to the groundwater.

How tile drains can be introduced in the simulation is described in
\daisytut{}, section~\ref{tut-sec:groundwater}. The soil and the
management are characterised as in exercise~\ref{ex:hydraulic}.  We
know the distance between the drainpipes are 18 m (default value) and
that the pipes are located at 110 cm depth (default value.  Assume
further that an aquitard of thickness 2 m (default value) is located
at 2.20 -- 4.20 m depth and that the aquitard is overlaying an aquifer,
which is characterised by a pressure potential of 200 cm at the top of
the aquifer (the interface between the aquitard and the aquifer).
Measurement shows that approximately 2/3 of the net precipitation in
1993 was lost through the drainpipes.  Calibrate the hydraulic
conductivity of the aquitard so that the simulated value matches the
measured value.

In the present case the lowest soil horizon ends at 2.2 m depth (i.e.\
the top of the aquitard).  The new lower boundary condition is
introduced by entering 
\begin{verbatim}
  (Groundwater pipe (K_aquitard 0.01 [cm/h]))
\end{verbatim}
instead of 
\begin{verbatim}
  (Groundwater deep)
\end{verbatim}
The numerical value assigned to \texttt{K\_aquitard} is the value used
for the calibration. 

\section{Rotations}
\label{ex:rotation}

The objective of this exercise is to introduce the simulation of a
crop rotation.

Assume the same setup for the column as in exercise~\ref{ex:groundwater}.

A common crop rotation that is used at arable farms as well pig farms
are: spring barley, winter barley, winter rape, winter wheat and
winter wheat. A suggestion for management practices typical for Danish
conditions pertaining to this rotation is shown in
table~\ref{tab:rotation}.
\begin{table}[!htbp]
  %\centering
  \caption{Management practices of the rotation: spring
    barley, winter barley, winter rape, winter wheat and winter wheat.
    Typical dates for the field operations are given parentheses.}
  \label{tab:rotation}
  \begin{tabular}{lccccc}
    \hline
    Activity & S. Barley & W. Barley & W. Rape   & W. Wheat  & W. Wheat\\
    \hline
    Plowing  & Spring    & Autumn    & Autumn    & Autumn    & Autumn\\
             & (Mar. 20) & (Sep. 01) & (Aug. 15) & (Sep. 01) & (Sep. 01)\\
    \hline
    Seedbed preparation
             & Spring    & Autumn    & Autumn    & Autumn    & Autumn\\
    \& sowing& (Apr. 05) & (Sep. 10) & (Aug. 20) & (Sep. 10) & (Sep. 10)\\
    \hline
    Harvest  & Ripe      & Ripe      & Ripe      & Ripe      & Ripe\\
    Straw removal & Yes  & Yes       & No        & No        & No\\
    \hline
    Fertilization 
          & 121 kg N/ha & 60 kg N/ha & 25 kg N/ha & 55 kg N/ha & 60 kg N/ha\\
    & (Apr. 05) & (Mar. 25) & (Aug. 20) & (Apr. 05) & (Apr. 05)\\
    & & 81 kg N/ha & 130 kg N/ha & 74 kg N/ha & 101 kg N/ha\\
    & & (Apr. 25)  & (Mar. 05)   & (May 05)   & (May 05)\\
    \hline
  \end{tabular}
\end{table}

It is recommended to build the management practices for each crop
individually as in exercise~\ref{ex:intro} for the spring barley. Note
that winter wheat following a winter rape and winter wheat following a
winter wheat are not identical. When a management practice for each of
the individual crops has been build then the crop rotation can be
build as illustrated in box~\ref{box:rotation}.
\begin{figure}[!htbp]
  \begin{boxedminipage}{\textwidth}
    \caption{Rotation.}
    \label{box:rotation}
\begin{verbatim}
;; Spring Barley setup for an arable farm rotation.
(defaction sbarley_A_F_R activity
  (wait_mm_dd 3 20)(plowing)
  (wait_mm_dd 4 05)(fertilize (N25S (weight 121 [kg N/ha])))
  (seed_bed_preparation)(sow "Spring Barley")
  (wait (or (crop_ds_after "Spring Barley" 2.0)(mm_dd 08 30)))
  (harvest "Spring Barley" (stub 8 [cm])(stem 1.0 [])(leaf 1.0 [])))

;; Winter Barley setup for an arable farm rotation.
(defaction wbarley_A_F_R activity
  ;; Insert management actions here.
)

;; Winter Rape setup for an arable farm rotation.
(defaction wrape_A_F_R activity
  ;; Insert management actions here.
)

;; Winter Wheat setup for an arable farm rotation (after a rape).

(defaction wwheat_A_F_R_1 activity
  ;; Insert management actions here.
)

;; Winter Wheat setup for an arable farm rotation (after a cereal).
(defaction wwheat_A_F_R_2 activity
  ;; Insert management actions here.
)

;; Building an arable farm rotation.
(defaction ArableFarmRotation activity
  sbarley_A_F_R wbarley_A_F_R wrape_A_F_R wwheat_A_F_R_1 wwheat_A_F_R_2)
\end{verbatim}
  \end{boxedminipage}
\end{figure}

Simulate the suggested arable farm rotation (only mineral fertilizer)
begining the simulation period January 1$^{st}$ 1994.  While running
the simulation, try to guess when the simulation will end, i.e.\ the
year the second winter wheat is harvested.  Does the simulation run
over time?  If so, check that you always harvest the last crop before
sowing the next.

\section{Organic Fertilizer}
\label{ex:organic}

The objective of this exercise is to introduce the use of organic
fertilizer in the simulation of a rotation.

Modify the setup of exercise~\ref{ex:rotation} in order to simulate a
pig farm rotation where the main source of fertilizer is pig slurry.
The fertilization practice is shown in table~\ref{tab:pig-farm}.

\begin{table}[!htbp]
  %\centering
  \caption{Fertilization practices of the pig farm rotation:
spring barley, winter barley, winter rape, winter wheat and winter
wheat. Typical dates for the fertilization are given parentheses.
Ammonium loss during application is given in squared brackets.}
  \label{tab:pig-farm}
  \begin{tabular}{lccccc}
    \hline
    Activity & S. Barley & W. Barley & W. Rape & W. Wheat & W. Wheat\\
    \hline
    Plowing & Spring & Autumn & Autumn & Autumn & Autumn\\
    & (Mar. 05) & (Sep. 01) & (Aug. 15) & (Sep. 01) & (Sep. 01)\\
    \hline
    Fertilization & 30 T.w.w./ha & 18 T.w.w./ha & 20 T.w.w./ha & 18
    T.w.w./ha & 18 T.w.w./ha\\
    (pig slurry) & (Mar. 05) & (Apr. 20) & (Aug. 15) & (Apr. 20) &
    (Apr. 20) \\
    & [5\%] & [10\%] & [5\%] & [10\%] & [10\%]\\
    & & & 20 T.w.w./ha & & \\
    & & & (Mar. 20) & & \\
    & & & [10\%] & & \\
    \hline
    Fertilization & & 68 kg N/ha & 24 kg N/ha & 63 kg N/ha & 95 kg N/ha \\
    \multicolumn{2}{l}{(mineral fertilizer)} 
    & (Mar. 25) & (Mar. 05) & (May 05) & (Apr. 05)\\
    \hline
  \end{tabular}
\end{table}

The present version of the \daisy{} model has a deficiency in the way
it treats organic fertilizers that are surface broadcasted. The
mineral part of the organic fertilizer will eventually enter the soil
and can subsequently be utilized by the plants, but the organic part
will stay at the surface until the next soil tillage operation, and
while staying at the surface no turnover takes place. To compensate
for this it is recommended to incorporate the applied organic
fertilizer into the upper few centimetres of the soil.
Box~\ref{box:fertilizer} shows how this can be modelled. For further
information see \daisyref{} and \daisytut{}.
\begin{figure}[!htbp]
  \begin{boxedminipage}{\textwidth}
    \caption{Defining an organic fertilizer.}
    \label{box:fertilizer}
\begin{verbatim}
;; Defining a pig slurry where 5% of the ammonium content is lost at 
;; pplication (defam = defining added matter)
(defam "Slurry_Vol05%" pig_slurry 
  ;; Other parameters.
)

;; Spring Barley setup for a pig farm rotation.
(defaction sbarley_P_F_R activity
  (wait_mm_dd 3 05)
  (fertilize ("Slurry_Vol05%" (weight 30 [Mg w.w./ha]))(to --5 [cm]))
  (plowing)
  (wait_mm_dd 4 05)
  (seed_bed_preparation)
  (sow "Spring Barley")
  (wait (or (crop_ds_after "Spring Barley" 2.0)(mm_dd 08 30)))
  (harvest "Spring Barley" (stub 8 [cm])(stem 1.0 [])(leaf 1.0 [])))
\end{verbatim}
  \end{boxedminipage}
\end{figure}
Box~\ref{box:fertilizer} indicates that a pig slurry (name:
\texttt{pig\_slurry}) already has been defined (see
``\texttt{fertilizer.dai}''). How the volatilization is set can be
found in \daisyref{}.  Look up ``volatilization'' in the index.

When inspecting the simulation results please notice the temporal
development in soil microbial biomass pools. Does this development
look reasonable?

\section{Warm-Up Period}
\label{ex:warmup}

The objective of this exercise is to introduce a warm-up period, the
purpose of which is to make the simulation less sensitive to the
initial conditions.

Repeat the simulation of exercise~\ref{ex:organic} with the
modification that a warm-up period is inserted in front of the
simulation (1994--1998). It is a rule of thump to select one rotation
or at least 4 years of simulation as a warm-up period. Hence in the
present case the simulation should begin January 1$^{st}$ 1989. Be
sure to repeat the rotation; an example can be found in
section~\ref{tut-sec:rotations} of \daisytut{}.

Did the warm-up period improve the credibility of the simulation?

\section{The MIT Model}
\label{ex:mit}

The objective of this exercise is to calibrate the MIT-model
(Mineralization-Immobilization Turnover).

In box~\ref{box:column} organic matter section was introduced. The
total carbon content of the soil is calculated from the humus
contentwhich was defined in the horizon section
(box~\ref{box:horizon}).  The MIT model comproses several organic
matter pools and the toal carbon content is distributed among these
pools by the initialization procedure that makes use of anual input of
organic carbon.  This, in combination with a warm-up period, will
often give a reasonable initializatin and hence a reasonable
N-mineralization.  However this can be fine tuned by introducing the
\texttt{background\_mineralization} init parameter, like this:
\begin{verbatim}
       (OrganicMatter original 
                      (init (input 3000 [kg C/ha/y])
                            (background_mineralization 30 [kg N/ha/y])))
\end{verbatim}
Daisy will then attempt to initialize the organic matter pools so the
total mineralization is 30 kg N/ha/y higher than the amount of organic
bound nitrogen added to the system together with the 3000 kg C/ha/y we
specified above.  In other words, at the time we start the simulation
the amount of organic bound nitrogen in the soil is expected to
decrease (be released) at a rate of 30 kg N/ha/y.

In certain cases the model may not be able to initialize itself, and
it may be necessary to park some of the soil organic carbon  in an
inactive pool.  This is done be entreing the following parameter in
the \texttt{defhorizon} section
\begin{verbatim}
  (SOM_fractions -1 -1 xxx)
\end{verbatim}
where \emph{xxx} is the inactive fraction of the soil organic carbon.

A way to estimate the N mineralization level of the system is to grow
an unfertilized plot and measure the nitrogen content of the crop at
harvest. Consider the agro-ecosystem of exercise~\ref{ex:warmup} and
assume that in 1993 and 1994 an unfertilized plot were grown with
winter wheat. The above ground part of the crop was harvested in the
two years and the biomass was analyzed for its N content. The
harvested nitrogen was 102 kg N/ha and 60 kg N/ha in 1993 and 1994,
respectively. Use this information to calibrate the MIT model by
modifying the background mineralization parameter, and if necessary,
the \texttt{SOM\_fractions} parameter.

\section{Building a Weather File}
\label{ex:weather}

The objective of this exercise is to build a weather file.

The \daisy{} weather-file is written in the \daisy{} weather format
(extension \texttt{dwf}). In front the weather data is a header as
shown in Box~\ref{box:weather}
\begin{figure}[!htbp]
  \begin{boxedminipage}{\textwidth}
    \caption{Beginning of weather file.}
    \label{box:weather}
\begin{verbatim}
dwf-0.0 
Station: Taastrup
Elevation: 30 m
Longitude: 12 dgEast
Latitude: 55 dgNorth
TimeZone: 15 dgEast
Surface: reference
ScreenHeight: 2 m
Begin: 1962-04-01
End: 2001-10-07
NH4WetDep: 0.9 ppm 
NH4DryDep: 2.2 kgN/year
NO3WetDep: 0.6 ppm
NO3DryDep: 1.1 kgN/year
TAverage: 7.8 dgC
TAmplitude: 8.5 dgC
MaxTDay: 209 yday
Timestep: 24 hours
PrecipCorrect: 1.0 1.0 1.0 1.0 1.0 1.0 1.0 1.0 1.0 1.0 1.0 1.0
--------------------------------------------------------------
Year Month Day  GlobRad AirTemp Precip 
year month mday W/m^2  dgC     mm/d
1962 4     1    120.4   2.8     0.0
1962 4     2    130.7   4.4     5.3
1962 4     3    39.1    6.8     2.9
... more climate data here ...
\end{verbatim}
  \end{boxedminipage}
\end{figure}
The first line in the header tells \daisy{} that the present file is
written in the \daisy{} weather format version 0.0 (the only version
implemented so far). Then follows a number of keywords:
\texttt{Station}, \texttt{Elevation}, \texttt{Longitude},
\texttt{Latitude}, and \texttt{TimeZone}; defining name and location
of the weather station where the data were measured. \texttt{Surface}:
reference refers to the surface above which the weather data was
measured (reference is short grass common for most weather stations).
\texttt{ScreenHeight} gives the height at which air temperature (and
humidity and wind speed if measured) was recorded. \texttt{Begin} and
\texttt{End} defines the date of the fist and last record in the
dataset. \texttt{NH4WetDep}, \texttt{NH4DryDep}, \texttt{NO3WetDep},
and \texttt{NO3DryDep} define deposition parameters at the location.
\texttt{TAverage}, \texttt{TAmplitude}, and \texttt{MaxTDay} define
the thermal regime of locality (annual average air temperature, the
amplitude of the annual air temperature cycle, and the day of the year
when maximum air temperature is expected on average). The
\texttt{Timestep} keyword tells the model basic time-step of the data,
in the present case we are dealing with daily values.
\texttt{PrecipCorrect} gives twelve correction factors used to correct
the precipitation month by month. In the present case the
precipitation is measured at the soil surface, hence no correction is
needed for the aerodynamic effect. The next three lines form the
header for the actual weather data, which in the present example are
written in a TAB separated format.

The present example shows a minimum weather data required. However,
\daisy{} can make use of more detailed data (see \daisyref{}).

Build a \daisy{} weather file based on the data in the spreadsheet
\texttt{�rslevWeatherDat.xls}. �rslev is a locality located at the
center of Funen. The precipitation is measured at 1.5 m height and the
shelter category is B. The surface of weather station is short grass
and the screen height is 2 m.  The precipitation correction is found
in table~\ref{tab:precip}
\begin{table}[!htbp]
  %\centering
  \caption{Correction of point precipitation as function of shelter
    categories [per cent], \cite{percor}.}
  \label{tab:precip}
  \begin{tabular}{llllllllllllll}\hline
    Shelter & J & F & M & A & M & J & J & A & S & O & N & D & Year\\\hline
    A & 29 & 30 & 26 & 19 & 11 & 9 & 8 & 8 & 9 & 10 & 17 & 26 & 16\\
    B & 41 & 42 & 35 & 24 & 13 & 11 & 10 & 10 & 11 & 14 & 23 & 37 & 21\\
    C & 53 & 53 & 45 & 29 & 16 & 13 & 12 & 12 & 13 & 17 & 29 & 48 & 27\\\hline
  \end{tabular}
\end{table}

\section{A New Organic Fertilizer Parameterization}
\label{ex:defam}

The objective of this exercise is to introduce a new organic
fertilizer in fertilizer library.

The composition of a mixed slurry is: dry matter content 4.8\%; carbon
content 36\% (dry matter basis); nitrogen content 10.5 \% (dry matter
basis); ammonium content 8.5\% (dry matter basis); and nitrate contend
0.0\% (dry matter basis).  A batch experiment has shown that the
mineralization of the mixed slurry approximately can be described by
splitting the organic matter into two pools and assume first order
degradation kinetics for the pools. The resistant pool constituted
79\% of the C with a half-life of 150 days and the more easy
decomposable pool constituted 21\% of the C with a half-life of 15
days. From previous experience it was suggested that the C/N of the
easy decomposable pool should be around 5.

Include the mixed slurry in the fertilizer library. Help on how to
write a new parameterization of a new organic fertilizer can be found
in the \daisyref{}, section~\ref{ref-model:am-organic}
and~\ref{ref-fixed:AOM}. 

Compare the results obtained with the new mixed slurry with the
results obtained in exercise~\ref{ex:warmup}.

\section{Crop Parameterization}
\label{ex:croppar}

The objective of this exercise is to parameterize a maize crop.

The maize parameters found in the crop-file
``\texttt{PioneerMaize.dai}'' are mainly based on \cite{vries-maize}.
However, the measured data of LAI, and leaf, stem, and cob weight from
1998 found in the spreadsheet ``\texttt{MaizeProd.xls}'' are from a
silage maize. Define a new maize by assuming that the silage maize
parameters partly resembles the Pioneer Maize parameters. Calibrate
selected parameters making use of the measured data. When a new
slightly different parameterization is defined it recommended not
copying parameter settings and making changes, but instead to derive
the new parameterization from the existing one. This makes it easy to
see where the parameterizations differ.

The main elements of a crop definition are shown in box~\ref{box:cropdef}. 
\begin{figure}[!htbp]
  \begin{boxedminipage}{\textwidth}
    \caption{Crop definition.}
    \label{box:cropdef}
\begin{verbatim}
(defcrop "Pioneer Maize (WS)" default
  ;; Phenological development :
  ;; Emergence   DS = 0
  ;; Flowering   DS = 1
  ;; Maturity    DS = 2    
  (Devel default
    ;; Temperature sum calculated from sowing to emergence.
    (EmrTSum  350)
    ;; Development rate [d^-1] in the vegetative period 
    (DSRate1  0.0265)
    ;; Effect of temperature on development rate in the veg. period 
    (TempEff1 (8. 0.00) (10. 0.30) (15. 0.75) (25. 1.00) (35.  1.20))
    ;; Effect of photoperiod on development rate in the veg. period 
    (PhotEff1 (0. 0.90) (12. 1.00) (14. 0.95) (16. 0.90) (24.  0.90))
    ;; Development rate [d^-1] in the reproductive period 
    (DSRate2  0.017)
    ;; Effect of temperature on development rate in the rep. period 
    (TempEff2 (8. 0.00) (10. 0.30) (15. 0.75) (25. 1.00) (35.  1.20)))
  (LeafPhot 
    ;; Insert parameters here.
  )
  (Canopy
    ;; Insert parameters here.
  )
  (Root 
    ;; Insert parameters here.
  )
  (Partit
    ;; Insert parameters here.
  )
  (Prod 
    ;; Insert parameters here.
  )
  (CrpN 
    ;; Insert parameters here.
  )
  (Harvest 
    ;; Insert parameters here.
  ))
\end{verbatim}
  \end{boxedminipage}
\end{figure}
The crop definition comprises a number of subsections: \texttt{Devel}
contains parameters pertaining to the phenological development of the
crop; \texttt{LeafPhot} parameters govern the leaf photosynthesis;
\texttt{Canopy} parameters are describing LAI development and
distribution; \texttt{Root} contains root penetration and root
distribution parameters; \texttt{Partit} governs the partition of
assimilates; \texttt{Prod} are dealing with respiration and production
parameters; \texttt{CrpN} parameters are controlling the plant
relations; and \texttt{Harvest} parameters are dealing with the
harvest itself as well as the fate of plant residuals. All crop
parameters are defined in \daisyref. When consulting the manual it is
noticed that more than one crop model is implemented in the \daisy{}
code. However, in this case we use the standard crop model, which is
the default crop model. This crop model resembles the \textsc{sucros}
model~\cite{sucros}.

Parameters can be given in two ways: 1. as a single value (e.g.
\texttt{DSRate1}); 2. as a piecewise linear function (e.g.
\texttt{TempEff1}). The piecewise linear function is defined by a
number of points (\textit{x,y}) on the considered curve. In the
example, \texttt{TempEff1}, \textit{x} equals temperature and \textit{y} equals
temperature effect. In the crop definition it is common to make a
parameter a function of the development stage (DS). The only exception
is the subsection \texttt{Devel}, where the parameters governing the
DS are defined.  Notice that \texttt{Devel} requires the keyword
default because more that one model describing the development is
implemented.

We have no information on roots, plant nitrogen dynamics or plant
residues. Hence, in the calibration we need not consider the
\texttt{Root}, \texttt{CrpN} (the nitrogen supply is ample so we need
not consider N-stress), and \texttt{Harvest} sections. We have no
reason to believe that the present maize variety should differ from
the pioneer maize in respect to respiration, hence we may want to keep
these parameters without change (keeping the
\texttt{Prod}--parameters). The relevant \texttt{Canopy}-parameters in
this context are pertaining to the leaf area. The specific leaf area
(LAI per unit leaf weight) can be calculated directly from measured
data (it is noticed that the value is not constant during the growth
period, however the model can take this into account, see
\daisyref{}).

If we want to change the development rate of the maize in the vegetative period this can be done by writing:
\begin{verbatim}
  (defcrop "Exercise 12 Maize" "Pioneer Maize (WS)" 
    (Devel original  
      (DSRate1  0.0160)))
\end{verbatim}
The keyword original retains the original parameterization, hence only
the \texttt{DSRate1} is changed.
 

In this way the selected parameters can be changed one by one in order
to fit the data. We may have to do quite a number of simulations when
calibrating the model. In order to save time it is possible to make
use of the checkpoint facility (see \daisytut{}). By using this
facility it is possible to write a checkpoint at a predefined time and
then later continue the simulation from that point in time.  In the
present example we run the warm-up period and then make a checkpoint
just before the sowing of the maize (e.g.\ \texttt{(checkpoint (when
  (at yyyy mm dd hh)))}). The checkpoint is to be inserted among the
log files in the output section. Another facility which also often is
helpful is the \texttt{(activate\_output (after yyyy mm dd hh))},
which is used to suppress output during warm-up periods.

Use the calibrated setup of exercise~\ref{ex:mit} and assume that the
maize is sow the year after the second year winter wheat. 50 ton
w.w./ha was surface broadcasted and plowed in 27. April 1998. Maize
was sown 4.  May and 142 kg N/ha in ammonium nitrate was applied the
same day. The Maize was harvested for silage 3. November, however it
was not ripe at that date.

In the calibration procedure it is advisable to begin with the DS\@.
We have no information of when the different growth stages of the
maize occurred. But we have information on the onset of growth of the
cobs and we can use this in order to calibrate the development rate in
the vegetative period. We have no information that makes it possible
do a similar calibration for the reproductive period.

It is recommended that when the DS is calibrated we should continue
looking at the \texttt{Canopy}-parameters and afterwards at the
\texttt{LeafPhot}-parameters and \texttt{Partit}-parameters.

\nocite{daisy-tut,daisy-def,daisy-fertilizer}

\bibliography{daisy}

\end{document}
