\documentclass[a4paper,11pt,twoside]{article}
\usepackage{a4}
\usepackage[T1]{fontenc}
\usepackage[latin1]{inputenc}
\usepackage{amsmath}
\usepackage{hyperref}
\usepackage{natbib}
\usepackage{graphicx}
\newcommand{\daisy}{Daisy}
\newcommand{\Daisy}{Daisy}
\newcommand{\cplusplus}%
{{\leavevmode{\rm{\hbox{C\hskip -0.1ex\raise 0.5ex\hbox{\tiny ++}}}}}}
\newcommand{\Cplusplus}{\cplusplus}
\newcommand{\mshe}{Mike/\textsc{she}}
\newcommand{\wintel}{\texttt{win32}}
\newcommand{\dll}{\textsc{dll}}
\newcommand{\Dll}{\textsc{Dll}}
\newcommand{\gui}{\textsc{gui}}
\newcommand{\Gui}{\textsc{Gui}}
\newcommand{\unix}{Unix}
\newcommand{\dhi}{\textsc{dhi}}
\newcommand{\Dhi}{\textsc{Dhi}}
\newcommand{\api}{\textsc{api}}
\newcommand{\Api}{\textsc{Api}}
\newcommand{\lai}{\textsc{lai}}
\newcommand{\Lai}{\textsc{Lai}}
%\newcommand{\url}[1]{\linebreak[4]\texttt{<URL:#1>}}

%%% Local Variables: 
%%% mode: latex
%%% TeX-master: t
%%% End: 


\newcommand{\aba}[1]{\text{\textsc{aba}$_\text{\textsc{#1}}$}}

\begin{document}

\title{Stomata models and ABA signalling in Daisy}
\author{Per Abrahamsen}
\maketitle

\section{Stomata models}
\label{sec:stomata}

Stomata conductivity ($g_s$) is an improtant factor for both
transpiration and photosyntheis.  Daisy implements several models for
calculating $g_s$.

\subsection{Ball-Berry model adjusted with draught stress}

The Ball-Berry equation adjusted for draught stress, both based on
physical signalling (root crown potential, $\psi_c$) and chemical
signalling (ABA) is supported (Eq.~\ref{eq:gs}).
\begin{equation}
  g_s = m A_n \frac{h_s}{c_s} e^{-\beta \aba{xylem}} e^{-\delta |\psi_c|} + g_{s0}
  \label{eq:gs}
\end{equation}
With $\delta = 0$ is becomes equal to the equation proposed by
\citet{gs2002}, and with both $\delta = 0$ and $\beta = 0$ it becomes
equavalent to the unadjusted Ball-Berry model \citep{bb87}.

\subsection{Leuning model adjusted with draught stress}

The model described in \citet{leuning95} has been adjusted for draught
stress as in the previous section (Eq.~\ref{eq:leuning}).

\begin{equation}
  g_s = \frac{m A_n}{(c_s-\gamma)(1 + \frac{D_s}{D_0})} e^{-\beta \aba{xylem}} e^{-\delta |\psi_c|} + g_{s0}
  \label{eq:leuning}
\end{equation}

\subsection{Ahmadi models}

Two of the models introduced in \citet{Ahmadi20091541} have been
implemented in Daisy.  Eq.~\ref{eq:sha12}, corresponding to equation 12
in the original artcile, and Eq.~\ref{eq:sha14}, corresponding to
equation 14 in the original artcile
\begin{equation}
  g_s = m {A_n}^\lambda \frac{{h_s}^\alpha}{c_s} e^{-\beta \aba{xylem}} e^{-\delta |\psi_c|}
  \label{eq:sha12}
\end{equation}
\begin{equation}
  g_s = \frac{m}{c_s} e^{\lambda A_n} e^{\alpha h_s} e^{-\beta
    \aba{xylem}} e^{-\delta |\psi_c|}
  \label{eq:sha14}
\end{equation}

\subsection{Andersen model}

The model depicted in Eq.~\ref{eq:mna} is not documented.
\begin{equation}
  g_s = \frac{m}{c_s} e^{\lambda / A_n} e^{\alpha h_s} e^{-\beta
    \aba{xylem}} e^{-\delta |\psi_c|} + g_{s0}
  \label{eq:mna}
\end{equation}

\section{ABA}

ABA is a plant hormom which regulates, among other things, the closing
of stomata.  By triggering the production of ABA in crops it is
sometimes possible to combine a significant reduction in water use,
with a moderate reduction in yield.  By supporting ABA in Daisy, we
hope to be better at predicting when and how such savings can be
achieved.

ABA is produced in the root system and has its effect in the canopy.
In the present implementation in Daisy transport, storage, and
degradation of ABA are all ignored in favor of a system where the ABA
concentration in the canopy is solely function of the current water
status in the root system.

ABA also has other effects, such as on root growth and plant
phenology, but these are not implemented in Daisy and outside the
scope of this paper.

\subsection{ABA production in soil}

Let $V$ represent the soil volume we are simulating.  The soil volume
should as minimum include the entire root zone.  For every point in
$V$, Daisy will calculate three values: the volumetric water uptake
$S$, the root density $l$, and the water potential $h$.

Based on these values, we have several models for estimating ABA in
the stem (\aba{xylem}) from the conditions in the soil.

\subsubsection{Based on uptake location}

The first model is based directly on pot experiments where both
\aba{xylem} and the water potential ($h$) is measured.  The pots are
sufficiently small that the water potential is assumed to be the same
everywhere in the pot.  From this an empirical relationship
($\aba{uptake}(h)$) between $h$ and \aba{xylem} in the pots can be
developed.  By assuming the same relationship between ABA
concentration and uptaken water also holds in a field, we
get~\eqref{eq:aba_uptake}.

\begin{equation}
  \aba{xylem} = \frac{\int_V S \: \aba{uptake}(h) \: dV}{\int_V S \: dV}
  \label{eq:aba_uptake}
\end{equation}

The function $\aba{uptake}(h)$ must be supplied by the user.

\subsubsection{Based on production in roots}

ABA is produced in the roots with a rate that depends in the water
potential ($\aba{root}(h)$).  If we assume all the ABA produced ends up in
the stem, we can use $\aba{root}(h)$ to find \aba{xylem}, as
in~\eqref{eq:aba_root}.

\begin{equation}
  \aba{xylem} = \frac{\int_V l \: \aba{root}(h) \: dV}{\int_V S \: dV}
  \label{eq:aba_root}
\end{equation}

The function $\aba{root}(h)$ must be supplied by the user.

\subsubsection{Based on production in soil}

Both $S \aba{uptake}(h)$ and $l \aba{root}(h)$ are special cases of
calculating the ABA contribution from the soil water potential, root
uptake, and root density.  We call the generalized function
$\aba{soil}(S,l,h)$, and the generalized equation
for~\eqref{eq:aba_soil}.

\begin{equation}
  \aba{xylem} = \frac{\int_V \aba{soil}(S,l,h) \: dV}{\int_V S \: dV}
  \label{eq:aba_soil}
\end{equation}

We get \eqref{eq:aba_uptake} by setting $\aba{soil}(S,l,h) = S \:
\aba{uptake}(h)$ and \eqref{eq:aba_root} by setting $\aba{soil}(S,l,h) = l \:
\aba{root}(h)$.

The function $\aba{soil}(S,l,h)$ must be supplied by the user.

\subsubsection{No ABA production}

The simplest model for ABA production is to assume no ABA is produced,
which also gives the simlest equation (\eqref{eq:aba_none}).

\begin{equation}
  \aba{xylem} = 0
  \label{eq:aba_none}
\end{equation}

\subsection{ABA effect in canopy}

\subsubsection{ABA effect on photosynthesis}

The ABA effect on photosynthesis is through the stomata conductance,
as described in section~\ref{sec:stomata}.

The sun shade open canopy (ssoc) soil vegetation atmosphere (svat)
model will use the calculated stomata conductance as part of an energy
balance between soil, atmosphere, canopy, dark, and sunlit leaves.
The mechanics of the ssoc svat model is outside the scope of this
note.  As part of this energy balance, transpiration will be
estimated.  The transpiration will affect ABA production in soil,
which again will affect stomata conductance.  A one step iteration
model is used for finding a solution for both ABA production and
stomata conductance.

\section{List of symbols}

\begin{tabular}{lll}
  Symbol & Unit & Description\\\hline
  \aba{cond} & None
    & ABA effect on stomata conductance.\\
  $\aba{root}(h)$ & g \textsc{aba}/cm \textsc{root}/h
    & ABA production in roots\\
  $\aba{soil}(S,l,h)$ & g \textsc{aba}/cm$^3$ \textsc{soil}/h
    & ABA contribution from soil\\
  $\aba{uptake}(h)$ & g \textsc{aba}/cm$^3$ \textsc{water}
    & ABA concentration in water from roots\\
  \aba{xylem} & g \textsc{aba}/cm$^3$ \textsc{water}
    & ABA concentration in xylem\\
  $h$    & hPa              
    & Soil water potential\\
  $k$    & cm$^3$ \textsc{water}/g \textsc{aba}
    & Coefficient for calculating \aba{cond}\\
  $l$    & cm \textsc{root}/cm$^3$ \textsc{soil} 
    & Root density\\
  $S$    & cm$^3$ \textsc{water}/cm$^3$ \textsc{soil}/h
    & Volumetric water uptake\\
  $V$    & cm$^3$ \textsc{soil}
    & Soil volume\\
\end{tabular}

\bibliographystyle{elsart-harv}
\bibliography{daisy}

\end{document}
