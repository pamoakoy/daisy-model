\newcommand{\figrorrende}[1]{\includegraphics[trim=9mm 0mm 14mm 12mm,clip]{fig/#1}}
\newcommand{\figrorrendel}[1]{\hspace*{-2cm}\figrorrende{#1}}

\chapter{2D plots}
\label{app:plot-2d}

In this appendix we present simulated 2D plots for water, bromide,
pendimethalin, and ioxynil.  There are no measurements to compare
with, a major caveat for both the results and discussion.  We use two
kinds of graphs to capture the 2D structure.

The first kind depict static distribution in the soil.  Each graph has
horizontal distance from drain on the x-axis and height above surface
on the y-axis, using the same scale for both axes.  The graph
represents the the computational soil area used in the simulation.
The right side is the center between two drains (9 meter for
R{\o}rrende), and the bottom is 2 meter, where we use an aquitard
lower boundary with a calibrated aquifer.  The graphs are color coded,
where specific colors represent specific values for the soil at the
end of the month indicated by the graph title.  Each numeric cell in
the computation has a color representing the value within that cell.
Since cells are rectangular, the graphs appear blocky.

The second kind of graph depicts horizontal or vertical movement.  For
the graphs depicting horizontal movement, the y-axis specifies height
above surface (negative number) and the x-axis movement away from
drain (usually also negative).  The horizontal movement at different
distances from the drain pipes are shown as separate plots on each
graph.  For the graphs depicting vertical movement, the axes are
swapped.  The individual plots represent different depths.  We use the
same flow units as we used for the original input, so e.g. pesticide
transport is given in g/ha.

\FloatBarrier
\section{Water}

\subsection{Distribution}

\begin{figure}[htbp]\centering
  \begin{tabular}{ccc}
    \figrorrendel{Rorrende-pF-1998-5} & 
    \figrorrende{Rorrende-pF-1998-6} & 
    \figrorrende{Rorrende-pF-1998-7} \\
    \figrorrendel{Rorrende-pF-1998-8} & 
    \figrorrende{Rorrende-pF-1998-9} & 
    \figrorrende{Rorrende-pF-1998-10} \\
    \figrorrendel{Rorrende-pF-1998-11} & 
    \figrorrende{Rorrende-pF-1998-12} & 
    \figrorrende{Rorrende-pF-1999-1} \\
    \figrorrendel{Rorrende-pF-1999-2} & 
    \figrorrende{Rorrende-pF-1999-3} & 
    \figrorrende{Rorrende-pF-1999-4}\\
    \figrorrendel{Rorrende-pF-1999-5} & 
    \figrorrende{Rorrende-pF-1999-6} & 
    \figrorrende{Rorrende-pF-1999-7} \\
    \figrorrendel{Rorrende-pF-1999-8} & 
    \figrorrende{Rorrende-pF-1999-9} & 
    \figrorrende{Rorrende-pF-1999-10} \\
    \figrorrendel{Rorrende-pF-1999-11} & 
    \figrorrende{Rorrende-pF-1999-12} & 
    \figrorrende{Rorrende-pF-2000-1} \\
    \figrorrendel{Rorrende-pF-2000-2} & 
    \figrorrende{Rorrende-pF-2000-3} & 
    \figrorrende{Rorrende-pF-2000-4}\\
    \figrorrendel{Rorrende-pF-2000-5} & 
    \figrorrende{Rorrende-pF-2000-6} & 
    \figrorrende{Rorrende-pF-2000-7} \\
    \figrorrendel{Rorrende-pF-2000-8} & 
    \figrorrende{Rorrende-pF-2000-9} & 
    \figrorrende{Rorrende-pF-2000-10} \\
    \figrorrendel{Rorrende-pF-2000-11} & 
    \figrorrende{Rorrende-pF-2000-12} & 
    \figrorrende{Rorrende-pF-2001-1} 
  \end{tabular}
  
  \caption{Soil water pressure potential at the end of each month from
    May 1998 (top left) to January 2001 (bottom right).  The y-axis
    denotes depth, the x-axis distance from drain.  There are tick
    marks for every meter.  Blue denotes pF<0, white pF=1, yellow
    pF=2, orange pF=3, red pF=4, and black pF>5.}
\label{fig:Rorrende-pF}
\end{figure}

\subsection{Flow}

\begin{figure}[htbp]
  \centering
  \figtop{Rorrende-water-horizontal-1998}
  \figtop{Rorrende-water-horizontal-1999}
  \fig{Rorrende-water-horizontal-2000}
  
  \caption{R{\o}rrende total horizontal water flux between 2000-5-1 and
    2001-5-1 (top) and between 2001-5-1 and 2002-5-1 (bottom).  The
    flux is shown on the x-axis (positive away from drain) as a
    function of depth shown on the y-axis.  The graph labels are the
    distance from drain in centimeters.}
  \label{fig:Rorrende-water-horizontal}
\end{figure}

\begin{figure}[htbp]
  \centering
  \figtop{Rorrende-water-1998}
  \fig{Rorrende-water-biopore-1998}
  
  \caption{R{\o}rrende vertical water flux between 2000-5-1 and 2001-5-1.
    Top graph show total flux, bottom graph only biopores. The flux is
    shown on the y-axis (positive up) as a function of distance from
    drain shown on the x-axis.  The graph labels are depths in
    centimeters above surface.}
  \label{fig:Rorrende-water-2000}
\end{figure}
\begin{figure}[htbp]
  \centering
  \figtop{Rorrende-water-1999}
  \fig{Rorrende-water-biopore-1999}
  
  \caption{R{\o}rrende vertical water flux between 2000-5-1 and 2001-5-1.
    Top graph show total flux, bottom graph only biopores. The flux is
    shown on the y-axis (positive up) as a function of distance from
    drain shown on the x-axis.  The graph labels are depths in
    centimeters above surface.}
  \label{fig:Rorrende-water-2000}
\end{figure}
\begin{figure}[htbp]
  \centering
  \figtop{Rorrende-water-2000}
  \fig{Rorrende-water-biopore-2000}
  
  \caption{R{\o}rrende vertical water flux between 2000-5-1 and 2001-5-1.
    Top graph show total flux, bottom graph only biopores. The flux is
    shown on the y-axis (positive up) as a function of distance from
    drain shown on the x-axis.  The graph labels are depths in
    centimeters above surface.}
  \label{fig:Rorrende-water-2000}
\end{figure}

\FloatBarrier
\section{Bromide}

\subsection{Distribution}

Like for water, there is hardly any horizontal gradients worth
speaking of for bromide at Silstrup
(figure~\ref{fig:Silstrup-Bromide-2000}
and~\ref{fig:Silstrup-Bromide-2001}).  The bromide is mostly contained
within the plow layer the first summer, but at the end of the drain
season, the bromide is everywhere.  R{\o}rrende shows a different pattern
(figure~\ref{fig:Rorrende-Bromide-2000}
and~\ref{fig:Rorrende-Bromide-2001}).  At the end of summer, most of the
bromide has left the plow layer, and the upward direction of the water
flow below the drain pipes keep that part of the soil relatively clear
of bromide.  In the second year, the horizontal flow of of water in
the plow layer is resulting in the soil above drain pipes also being
cleared of bromide.

\subsection{Transport}

The most interesting thing to note about the horizontal bromide
transport is that the rather small horizontal flow of water depicted
on the top graph of figure~\ref{fig:Silstrup-water-horizontal}
translate into a much more significant transport of bromide shown on
figure~\ref{fig:Silstrup-Bromide-horizontal}.  This indicates that the
horizontal water flow happens early, when the bromide concentration of
the plow layer is still high.  The bottom graph of
figures~\ref{fig:Silstrup-Bromide-horizontal}
and~\ref{fig:Rorrende-Bromide-horizontal} both show less horizontal transport
the second year, especially in the plow layer.

Figure~\ref{fig:Silstrup-Bromide-2000-vertical} shows us that all the
bromide enter through the matrix, and only half the bromide leaver the
top 25 cm.  We also see the biopores being activated between -25 and
-50 cm, indicating the plow pan being significant.  The drain pipes
only visibly affect the transport right on top of them (-100 cm),
where most of the transport is through biopores.  The second year
(figure~\ref{fig:Silstrup-Bromide-2001-vertical}) does not show much
transport at all, except right above the pipes like the year before.  For
R{\o}rrende, we see a strong matrix transport with right above the pipes, with
some contributions from biopores
(figure~\ref{fig:Rorrende-Bromide-2000-vertical}).  The bromide leaching
from the top 25 cm is slightly higher than for Silstrup, and dominated
by matrix transport.  The second year
(figure~\ref{fig:Rorrende-Bromide-2001-vertical}) we get contribution to
the drains from both above and below, almost exclusively through
matrix transport.

\begin{figure}[htbp]\centering
  \begin{tabular}{ccc}
    \figrorrendel{Rorrende-M-Bromide-1998-5} & 
    \figrorrende{Rorrende-M-Bromide-1998-6} & 
    \figrorrende{Rorrende-M-Bromide-1998-7} \\
    \figrorrendel{Rorrende-M-Bromide-1998-8} & 
    \figrorrende{Rorrende-M-Bromide-1998-9} & 
    \figrorrende{Rorrende-M-Bromide-1998-10} \\
    \figrorrendel{Rorrende-M-Bromide-1998-11} & 
    \figrorrende{Rorrende-M-Bromide-1998-12} & 
    \figrorrende{Rorrende-M-Bromide-1999-1} \\
    \figrorrendel{Rorrende-M-Bromide-1999-2} & 
    \figrorrende{Rorrende-M-Bromide-1999-3} & 
    \figrorrende{Rorrende-M-Bromide-1999-4}
  \end{tabular}
  
  \caption{R{\o}rrende bromide soil content at the end of each month since
    first application of bromide.  The y-axis denotes depth, the x-axis distance from drain.  There are tick marks for every
    meter. The color scale is white<10 pg/l, yellow=1 ng/l,
    orange=0.1 $\mu$g/l, red=10 $\mu$g/l, and black>1 mg/l}
\label{fig:Rorrende-Bromide-2000}
\end{figure}

\begin{figure}[htbp]\centering
  \begin{tabular}{ccc}
    \figrorrendel{Rorrende-M-Bromide-1999-5} & 
    \figrorrende{Rorrende-M-Bromide-1999-6} & 
    \figrorrende{Rorrende-M-Bromide-1999-7} \\
    \figrorrendel{Rorrende-M-Bromide-1999-8} & 
    \figrorrende{Rorrende-M-Bromide-1999-9} & 
    \figrorrende{Rorrende-M-Bromide-1999-10} \\
    \figrorrendel{Rorrende-M-Bromide-1999-11} & 
    \figrorrende{Rorrende-M-Bromide-1999-12} & 
    \figrorrende{Rorrende-M-Bromide-2000-1} \\
    \figrorrendel{Rorrende-M-Bromide-2000-2} & 
    \figrorrende{Rorrende-M-Bromide-2000-3} & 
    \figrorrende{Rorrende-M-Bromide-2000-4}
  \end{tabular}
  
  \caption{R{\o}rrende bromide soil content at the end of each month second
    year after application of bromide.  The y-axis denotes depth, the
    x-axis distance from drain.  There are tick marks for every
    meter. The color scale is white<10 pg/l, yellow=1 ng/l, orange=0.1
    $\mu$g/l, red=10 $\mu$g/l, and black>1 mg/l}
\label{fig:Rorrende-Bromide-2001}
\end{figure}

\begin{figure}[htbp]
  \centering
  \figtop{Rorrende-Bromide-horizontal-1998}
  \fig{Rorrende-Bromide-horizontal-1999}
  
  \caption{R{\o}rrende total horizontal bromide transport between 2000-5-1 and
    2001-5-1 (top) and between 2001-5-1 and 2002-5-1 (bottom).  The
    transport is shown on the x-axis (positive away from drain) as a
    function of depth shown on the y-axis.  The graph labels are the
    distance from drain in centimeters.}
  \label{fig:Rorrende-Bromide-horizontal}
\end{figure}

\begin{figure}[htbp]
  \centering
  \figtop{Rorrende-Bromide-1998}
  \fig{Rorrende-Bromide-biopore-1999}
  
  \caption{R{\o}rrende total (top) and biopore (bottom) vertical bromide
    transport between 2000-5-1 and 2001-5-1.  The transport is shown on the
    y-axis (positive up) as a function of distance from drain shown on
    the x-axis.  The graph labels are depths in centimeters above
    surface.}
  \label{fig:Rorrende-Bromide-2000-vertical}
\end{figure}

\begin{figure}[htbp]
  \centering
  \figtop{Rorrende-Bromide-1998}
  \fig{Rorrende-Bromide-biopore-1999}
  
  \caption{R{\o}rrende total (top) and biopore (bottom) vertical bromide
    transport between 2001-5-1 and 2002-5-1.  The transport is shown on the
    y-axis (positive up) as a function of distance from drain shown on
    the x-axis.  The graph labels are depths in centimeters above
    surface.}
  \label{fig:Rorrende-Bromide-2001-vertical}
\end{figure}

\FloatBarrier
\section{Pendimethalin}

Unfortunately, the pendimethalin was applied on different years for the
sites, making them less comparable.  Nonetheless, comparing with the
rest of the data, the differences seem to be more a result of the
respective soils than difference in weather.

\subsection{Distribution}

On figure~\ref{fig:Silstrup-M-Pendimethalin-2001} (Silstrup) we can see
the pendimethalin entering the soil in three different places.  The soil
surface, the bottom of the short biopores that end right above the
plow pan, and the end of the deep biopores than end 1.2 meter below
the surface.  The pendimethalin within the plow layer is then mixed by a
soil tillage operation.  The leaching below 2 meter is hardly visible,
but there is clearly some redistribution within the biopore active
soil.  If we look at the concentration in soil
water~\ref{fig:Silstrup-C-Pendimethalin-2001} we see a clear decrease in
the plow layer, which can be explained by a combination of degradation
and dilution as the water content is increasing (see
figure~\ref{fig:Silstrup-pF-2001}).

At R{\o}rrende, the pendimethalin hardly even move out of the plow layer
(figure~\ref{fig:Rorrende-M-Pendimethalin-2000}).  If we look at the soil
water concentration (figure~\ref{fig:Rorrende-M-Pendimethalin-2000}), it is
only above the limit for drinking water within the plow layer, except
for the first month where it is near the limit in a area above the
drain pipes.  The reason for this is that the water table at the time
is lower above the drain pipes (see figure~\ref{fig:Rorrende-pF-2000}),
and the biopores will mainly empty in unsaturated soil.  Looking one
year further ahead (figure~\ref{fig:Rorrende-C-Pendimethalin-2001}) we see
the pendimethalin above 1 meter being degraded, and the pendimethalin below
1 meter going nowhere.

\subsection{Transport}

The horizontal transport (figure~\ref{fig:Pendimethalin-horizontal}) reflect
the location in the soil, at Silstrup we see some horizontal transport at
the top of the soil, at the bottom of the short biopores, and at the
bottom of the deep biopores.  At R{\o}rrende, we plow shortly after
application.  The plow operation as defined in Daisy distributes the
pendimethalin from the surface to the bottom half of the plow layer.
Which is where we see the horizontal transport.

At Silstrup (figure~\ref{fig:Silstrup-Pendimethalin-2001-vertical}) most
of the pendimethalin enters the soil through the matrix, but only the
part entering the soil through biopores is transported further down.
Unlike for metamitron
(figure~\ref{fig:Silstrup-Metamitron-2000-vertical}), less pendimethalin
enter the soil above the drain pipes, indicating that the pendimethalin
spend more time on the surface.  For R{\o}rrende
(figure~\ref{fig:Rorrende-Pendimethalin-2000}) there is no horizontal
variation in how much pendimethalin enter the soil, none of it does so
through the biopores.  There is some matrix transport 25 cm below surface
(the plowing operation put most pendimethalin 22 cm below surface),
further down there is some biopore facilitated transport above the
drains.

\begin{figure}[htbp]\centering
  \begin{tabular}{ccc}
    \figrorrendel{Rorrende-M-Pendimethalin-1999-5} & 
    \figrorrende{Rorrende-M-Pendimethalin-1999-6} & 
    \figrorrende{Rorrende-M-Pendimethalin-1999-7} \\
    \figrorrendel{Rorrende-M-Pendimethalin-1999-8} & 
    \figrorrende{Rorrende-M-Pendimethalin-1999-9} & 
    \figrorrende{Rorrende-M-Pendimethalin-1999-10} \\
    \figrorrendel{Rorrende-M-Pendimethalin-1999-11} & 
    \figrorrende{Rorrende-M-Pendimethalin-1999-12} & 
    \figrorrende{Rorrende-M-Pendimethalin-2000-1} \\
    \figrorrendel{Rorrende-M-Pendimethalin-2000-2} & 
    \figrorrende{Rorrende-M-Pendimethalin-2000-3} & 
    \figrorrende{Rorrende-M-Pendimethalin-2000-4}
  \end{tabular}
  
  \caption{R{\o}rrende pendimethalin soil content at the end of each month
    since first application of bromide.  The y-axis denotes depth, the
    x-axis distance from drain.  There are tick marks for every
    meter. The color scale is white<10 pg/l, yellow=1 ng/l, orange=0.1
    $\mu$g/l, red=10 $\mu$g/l, and black>1 mg/l}
\label{fig:Rorrende-M-Pendimethalin-1999}
\end{figure}

\begin{figure}[htbp]\centering
  \begin{tabular}{ccc}
    \figrorrendel{Rorrende-M-Pendimethalin-2000-5} & 
    \figrorrende{Rorrende-M-Pendimethalin-2000-6} & 
    \figrorrende{Rorrende-M-Pendimethalin-2000-7} \\
    \figrorrendel{Rorrende-M-Pendimethalin-2000-8} & 
    \figrorrende{Rorrende-M-Pendimethalin-2000-9} & 
    \figrorrende{Rorrende-M-Pendimethalin-2000-10} \\
    \figrorrendel{Rorrende-M-Pendimethalin-2000-11} & 
    \figrorrende{Rorrende-M-Pendimethalin-2000-12} & 
    \figrorrende{Rorrende-M-Pendimethalin-2001-1} \\
    \figrorrendel{Rorrende-M-Pendimethalin-2001-2} & 
    \figrorrende{Rorrende-M-Pendimethalin-2001-3} & 
    \figrorrende{Rorrende-M-Pendimethalin-2001-4}
  \end{tabular}
  
  \caption{R{\o}rrende pendimethalin soil content at the end of each month
    since first application of bromide.  The y-axis denotes depth, the
    x-axis distance from drain.  There are tick marks for every
    meter. The color scale is white<10 pg/l, yellow=1 ng/l, orange=0.1
    $\mu$g/l, red=10 $\mu$g/l, and black>1 mg/l}
\label{fig:Rorrende-M-Pendimethalin-1999}
\end{figure}

\begin{figure}[htbp]\centering
  \begin{tabular}{ccc}
    \figrorrendel{Rorrende-C-Pendimethalin-1999-5} & 
    \figrorrende{Rorrende-C-Pendimethalin-1999-6} & 
    \figrorrende{Rorrende-C-Pendimethalin-1999-7} \\
    \figrorrendel{Rorrende-C-Pendimethalin-1999-8} & 
    \figrorrende{Rorrende-C-Pendimethalin-1999-9} & 
    \figrorrende{Rorrende-C-Pendimethalin-1999-10} \\
    \figrorrendel{Rorrende-C-Pendimethalin-1999-11} & 
    \figrorrende{Rorrende-C-Pendimethalin-1999-12} & 
    \figrorrende{Rorrende-C-Pendimethalin-2000-1} \\
    \figrorrendel{Rorrende-C-Pendimethalin-2000-2} & 
    \figrorrende{Rorrende-C-Pendimethalin-2000-3} & 
    \figrorrende{Rorrende-C-Pendimethalin-2000-4}
  \end{tabular}
  
  \caption{R{\o}rrende pendimethalin soil water concentration at the end of
    each month since first application of bromide.  The y-axis denotes
    depth, the x-axis distance from drain.  There are tick marks for
    every meter. The color scale is white<10 pg/l, yellow=1 ng/l, orange=0.1
    $\mu$g/l, red=10 $\mu$g/l, and black>1 mg/l}
\label{fig:Rorrende-C-Pendimethalin-2000}
\end{figure}

\begin{figure}[htbp]\centering
  \begin{tabular}{ccc}
    \figrorrendel{Rorrende-C-Pendimethalin-2000-5} & 
    \figrorrende{Rorrende-C-Pendimethalin-2000-6} & 
    \figrorrende{Rorrende-C-Pendimethalin-2000-7} \\
    \figrorrendel{Rorrende-C-Pendimethalin-2000-8} & 
    \figrorrende{Rorrende-C-Pendimethalin-2000-9} & 
    \figrorrende{Rorrende-C-Pendimethalin-2000-10} \\
    \figrorrendel{Rorrende-C-Pendimethalin-2000-11} & 
    \figrorrende{Rorrende-C-Pendimethalin-2000-12} & 
    \figrorrende{Rorrende-C-Pendimethalin-2001-1} \\
    \figrorrendel{Rorrende-C-Pendimethalin-2001-2} & 
    \figrorrende{Rorrende-C-Pendimethalin-2001-3} & 
    \figrorrende{Rorrende-C-Pendimethalin-2001-4}
  \end{tabular}
  
  \caption{R{\o}rrende pendimethalin soil water concentration at the end of
    each month since first application of bromide.  The y-axis denotes
    depth, the x-axis distance from drain.  There are tick marks for
    every meter. The color scale is white<10 pg/l, yellow=1 ng/l, orange=0.1
    $\mu$g/l, red=10 $\mu$g/l, and black>1 mg/l}
\label{fig:Rorrende-C-Pendimethalin-2001}
\end{figure}

\begin{figure}[htbp]
  \centering
  \fig{Rorrende-Pendimethalin-horizontal-1999}
  \fig{Rorrende-Pendimethalin-horizontal-2000}

    
  \caption{Silstrup total horizontal pendimethalin transport between 2001-5-1
    and 2002-3-1 and R{\o}rrende total horizontal pendimethalin transport between
    2000-5-1 and 2001-5-1. The transport is shown on the x-axis (positive
    away from drain) as a function of depth shown on the y-axis.  The
    graph labels are the distance from drain in centimeters.}
  \label{fig:Pendimethalin-horizontal}
\end{figure}

\begin{figure}[htbp]
  \centering
  \figtop{Rorrende-Pendimethalin-1999}
  \fig{Rorrende-Pendimethalin-biopore-1999}
  
  \caption{R{\o}rrende total (top) and biopore (bottom) vertical pendimethalin
    transport between 2000-5-1 and 2001-5-1.  The transport is shown on the
    y-axis (positive up) as a function of distance from drain shown on
    the x-axis.  The graph labels are depths in centimeters above
    surface.}
  \label{fig:Rorrende-Pendimethalin-2000}
\end{figure}

\begin{figure}[htbp]
  \centering
  \figtop{Rorrende-Pendimethalin-2000}
  \fig{Rorrende-Pendimethalin-biopore-2000}
  
  \caption{R{\o}rrende total (top) and biopore (bottom) vertical pendimethalin
    transport between 2000-5-1 and 2001-5-1.  The transport is shown on the
    y-axis (positive up) as a function of distance from drain shown on
    the x-axis.  The graph labels are depths in centimeters above
    surface.}
  \label{fig:Rorrende-Pendimethalin-2000}
\end{figure}

\FloatBarrier
\section{Ioxynil}

Unfortunately, the ioxynil was applied on different years for the
sites, making them less comparable.  Nonetheless, comparing with the
rest of the data, the differences seem to be more a result of the
respective soils than difference in weather.

\subsection{Distribution}

On figure~\ref{fig:Silstrup-M-Ioxynil-2001} (Silstrup) we can see
the ioxynil entering the soil in three different places.  The soil
surface, the bottom of the short biopores that end right above the
plow pan, and the end of the deep biopores than end 1.2 meter below
the surface.  The ioxynil within the plow layer is then mixed by a
soil tillage operation.  The leaching below 2 meter is hardly visible,
but there is clearly some redistribution within the biopore active
soil.  If we look at the concentration in soil
water~\ref{fig:Silstrup-C-Ioxynil-2001} we see a clear decrease in
the plow layer, which can be explained by a combination of degradation
and dilution as the water content is increasing (see
figure~\ref{fig:Silstrup-pF-2001}).

At R{\o}rrende, the ioxynil hardly even move out of the plow layer
(figure~\ref{fig:Rorrende-M-Ioxynil-2000}).  If we look at the soil
water concentration (figure~\ref{fig:Rorrende-M-Ioxynil-2000}), it is
only above the limit for drinking water within the plow layer, except
for the first month where it is near the limit in a area above the
drain pipes.  The reason for this is that the water table at the time
is lower above the drain pipes (see figure~\ref{fig:Rorrende-pF-2000}),
and the biopores will mainly empty in unsaturated soil.  Looking one
year further ahead (figure~\ref{fig:Rorrende-C-Ioxynil-2001}) we see
the ioxynil above 1 meter being degraded, and the ioxynil below
1 meter going nowhere.

\subsection{Transport}

The horizontal transport (figure~\ref{fig:Ioxynil-horizontal}) reflect
the location in the soil, at Silstrup we see some horizontal transport at
the top of the soil, at the bottom of the short biopores, and at the
bottom of the deep biopores.  At R{\o}rrende, we plow shortly after
application.  The plow operation as defined in Daisy distributes the
ioxynil from the surface to the bottom half of the plow layer.
Which is where we see the horizontal transport.

At Silstrup (figure~\ref{fig:Silstrup-Ioxynil-2001-vertical}) most
of the ioxynil enters the soil through the matrix, but only the
part entering the soil through biopores is transported further down.
Unlike for metamitron
(figure~\ref{fig:Silstrup-Metamitron-2000-vertical}), less ioxynil
enter the soil above the drain pipes, indicating that the ioxynil
spend more time on the surface.  For R{\o}rrende
(figure~\ref{fig:Rorrende-Ioxynil-2000}) there is no horizontal
variation in how much ioxynil enter the soil, none of it does so
through the biopores.  There is some matrix transport 25 cm below surface
(the plowing operation put most ioxynil 22 cm below surface),
further down there is some biopore facilitated transport above the
drains.

\begin{figure}[htbp]\centering
  \begin{tabular}{ccc}
    \figrorrendel{Rorrende-M-Ioxynil-2000-5} & 
    \figrorrende{Rorrende-M-Ioxynil-2000-6} & 
    \figrorrende{Rorrende-M-Ioxynil-2000-7} \\
    \figrorrendel{Rorrende-M-Ioxynil-2000-8} & 
    \figrorrende{Rorrende-M-Ioxynil-2000-9} & 
    \figrorrende{Rorrende-M-Ioxynil-2000-10} \\
    \figrorrendel{Rorrende-M-Ioxynil-2000-11} & 
    \figrorrende{Rorrende-M-Ioxynil-2000-12} & 
    \figrorrende{Rorrende-M-Ioxynil-2001-1} \\
    \figrorrendel{Rorrende-M-Ioxynil-2001-2} & 
    \figrorrende{Rorrende-M-Ioxynil-2001-3} & 
    \figrorrende{Rorrende-M-Ioxynil-2001-4}
  \end{tabular}
  
  \caption{R{\o}rrende ioxynil soil content at the end of each month
    since first application of bromide.  The y-axis denotes depth, the
    x-axis distance from drain.  There are tick marks for every
    meter. The color scale is white<10 pg/l, yellow=1 ng/l, orange=0.1
    $\mu$g/l, red=10 $\mu$g/l, and black>1 mg/l}
\label{fig:Rorrende-M-Ioxynil-1999}
\end{figure}

\begin{figure}[htbp]\centering
  \begin{tabular}{ccc}
    \figrorrendel{Rorrende-C-Ioxynil-2000-5} & 
    \figrorrende{Rorrende-C-Ioxynil-2000-6} & 
    \figrorrende{Rorrende-C-Ioxynil-2000-7} \\
    \figrorrendel{Rorrende-C-Ioxynil-2000-8} & 
    \figrorrende{Rorrende-C-Ioxynil-2000-9} & 
    \figrorrende{Rorrende-C-Ioxynil-2000-10} \\
    \figrorrendel{Rorrende-C-Ioxynil-2000-11} & 
    \figrorrende{Rorrende-C-Ioxynil-2000-12} & 
    \figrorrende{Rorrende-C-Ioxynil-2001-1} \\
    \figrorrendel{Rorrende-C-Ioxynil-2001-2} & 
    \figrorrende{Rorrende-C-Ioxynil-2001-3} & 
    \figrorrende{Rorrende-C-Ioxynil-2001-4}
  \end{tabular}
  
  \caption{R{\o}rrende ioxynil soil water concentration at the end of
    each month since first application of bromide.  The y-axis denotes
    depth, the x-axis distance from drain.  There are tick marks for
    every meter. The color scale is white<10 pg/l, yellow=1 ng/l, orange=0.1
    $\mu$g/l, red=10 $\mu$g/l, and black>1 mg/l}
\label{fig:Rorrende-C-Ioxynil-2001}
\end{figure}

\begin{figure}[htbp]
  \centering
  \fig{Rorrende-Ioxynil-horizontal-2000}

    
  \caption{Silstrup total horizontal ioxynil transport between 2001-5-1
    and 2002-3-1 and R{\o}rrende total horizontal ioxynil transport between
    2000-5-1 and 2001-5-1. The transport is shown on the x-axis (positive
    away from drain) as a function of depth shown on the y-axis.  The
    graph labels are the distance from drain in centimeters.}
  \label{fig:Ioxynil-horizontal}
\end{figure}


\begin{figure}[htbp]
  \centering
  \figtop{Rorrende-Ioxynil-2000}
  \fig{Rorrende-Ioxynil-biopore-2000}
  
  \caption{R{\o}rrende total (top) and biopore (bottom) vertical ioxynil
    transport between 2000-5-1 and 2001-5-1.  The transport is shown on the
    y-axis (positive up) as a function of distance from drain shown on
    the x-axis.  The graph labels are depths in centimeters above
    surface.}
  \label{fig:Rorrende-Ioxynil-2000}
\end{figure}

%%% Local Variables: 
%%% TeX-master: "agrovand"
%%% End: 
