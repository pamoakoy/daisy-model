\subsection*{Estrup}

\begin{figure}[htbp]
  \begin{center}
    \includegraphics{fig/Estrup-weather}
  \end{center}
  \caption{Accumulated precipitation and hourly values for temperature
    measured at Estrup station.  Calculated potentiel and simulated
    actual evapotranspiration are also shown.}
  \label{fig:Estrup-weather}
\end{figure}\FloatBarrier

\begin{figure}[htbp] 
  \includegraphics{fig/Estrup-Ap-Theta}\includegraphics{fig/Estrup-Ap-K}\\
  \includegraphics{fig/Estrup-B-Theta}\includegraphics{fig/Estrup-B-K}\\
  \includegraphics{fig/Estrup-C1-Theta}\includegraphics{fig/Estrup-C1-K}\\
  \includegraphics{fig/Estrup-C2-Theta}\includegraphics{fig/Estrup-C2-K}
  \caption{Estrup soil hydraulic properties.  MACRO denotes the
    original parametrization, Daisy the modified parametrization, and
    HYPRES refers to parameters estimated according to \citet{hypres}.
    No cracks represent conductivity in the part of the horizon with
    no cracks.}
  \label{fig:Estrup-hor}
\end{figure}\FloatBarrier

\begin{figure}[htbp]
  \begin{center}
    \includegraphics{fig/Estrup-gw}
  \end{center}
  \caption{Estrup groundwater table.  Manual montly measurement until
    2000-09-19 are from P3, automatic daily measurements from
    2000-09-22 are from P1.  Simulated low value is calculated from
    pressure in lowest unsatured numeric cell, typically located near
    drain.  Simulated high value is calculated from pressure in
    highest saturated cell, typically farthest from drain.}
  \label{fig:Estrup-gw}
\end{figure}\FloatBarrier

\begin{figure}[htbp]
  \begin{center}
    \includegraphics{fig/Estrup-theta-SW025cm}
  \end{center}
  \caption{Soilstrup soil water content for measurement point S1.}
  \label{fig:Estrup-theta}
\end{figure}\FloatBarrier

\begin{figure}[htbp]
  \begin{center}
    \includegraphics[trim=0mm 5mm 0mm 0mm,clip]{fig/Estrup-sc-bromide}\\
    \includegraphics{fig/Estrup-Bromide-horizontal}
  \end{center}
  \caption{Estrup soil bromide content in 1.0 m depth (top) and 3.5
    m depth (bottom).  Sim (avg) is the average simulated
    concentration, Sim (fast) is the simulated concentration in the
    large (fast) pores.  S1 and S2 are suction cup measurements.
    H$1$.$m$ refer to measured values in different sections of
    horizontal filters.}
  \label{fig:Estrup-bromide}
\end{figure}\FloatBarrier


\begin{figure}[htbp]
  \begin{center}
    \includegraphics{fig/Estrup-horizontal}
  \end{center}
  \caption{Estrup pesticide concentration in soil water at 3.5 meters
    depth.}
  \label{fig:Estrup-horizontal}
\end{figure}\FloatBarrier

\begin{figure}[htbp]
  \begin{center}
    \includegraphics[trim=0mm 5mm 0mm 0mm,clip]{fig/Estrup-leak150bromide}\\
    \includegraphics[trim=0mm 5mm 0mm 0mm,clip]{fig/Estrup-leak150}\\
    \includegraphics{fig/Estrup-leak150acc}
  \end{center}
  \caption{Estrup simuleret leaching at 1.5 meter, 30 cm under bioporers.}
  \label{fig:Estrup-leak150}
\end{figure}\FloatBarrier

\begin{figure}[htbp]
  \begin{center}
    \includegraphics[trim=0mm 5mm 0mm 0mm,clip]{fig/Estrup-drain}\\
    \includegraphics{fig/Estrup-drain-acc}
  \end{center}
  \caption{Estrup drain flow, daily values and accumulated.}
  \label{fig:Estrup-drain}
\end{figure}\FloatBarrier

\begin{figure}[htbp]
  \begin{center}
    \includegraphics[trim=0mm 5mm 0mm 0mm,clip]{fig/Estrup-Bromide-weekly}\\
    \includegraphics{fig/Estrup-Bromide-acc}
  \end{center}
  \caption{Estrup weekly and accumulated drain flow of bromide.}
  \label{fig:Estrup-bromide-weekly}
\end{figure}\FloatBarrier


\begin{figure}[htbp]
  \begin{center}
    \includegraphics[trim=0mm 5mm 0mm 0mm,clip]{fig/Estrup-Dimethoate-weekly}\\
    \includegraphics[trim=0mm 5mm 0mm 0mm,clip]{fig/Estrup-Fenpropimorph-weekly}\\
    \includegraphics{fig/Estrup-Glyphosate-weekly}\\
  \end{center}
  \caption{Estrup weekly drain flow of selected pesticides.}
  \label{fig:Estrup-weekly}
\end{figure}\FloatBarrier

\begin{figure}[htbp]
  \begin{center}
    \includegraphics[trim=0mm 5mm 0mm 0mm,clip]{fig/Estrup-Dimethoate-acc}\\
    \includegraphics[trim=0mm 5mm 0mm 0mm,clip]{fig/Estrup-Fenpropimorph-acc}\\
    \includegraphics{fig/Estrup-Glyphosate-acc}\\
  \end{center}
  \caption{Estrup accumulated drain flow of selected pesticides.}
  \label{fig:Estrup-acc}
\end{figure}\FloatBarrier

\begin{figure}[htbp]
  \begin{center}
    \includegraphics{fig/Estrup-colloid}
  \end{center}
  \caption{Colloids in drain water.}
  \label{fig:Estrup-colloids}
\end{figure}\FloatBarrier

\begin{figure}[htbp]
  \begin{center}
    \includegraphics[trim=0mm 5mm 0mm 0mm,clip]{fig/Estrup-biopore}\\
    \includegraphics{fig/Estrup-biopore-acc}\\
  \end{center}
  \caption{Biopore activity in different soil layers.  The layers are
    ponded water, soil surface (top 3 cm), the rest of the plowing layer,
    the plow pan, and the the B horizon below plow pan down to 50 cm.}
  \label{fig:Estrup-biopore}
\end{figure}\FloatBarrier

\begin{figure}[htbp]
  \begin{center}
    \includegraphics[trim=0mm 5mm 0mm 0mm,clip]{fig/Estrup-biopore-drain}\\
    \includegraphics{fig/Estrup-biopore-drain-acc}
  \end{center}
  \caption{Drain contribution through biopores from different soil
    layers.  The layers are ponded water, soil surface (top 3 cm), the
    rest of the plowing layer, the plow pan, and the the B horizon
    below plow pan down to 50 cm.}
  \label{fig:Estrup-biopore-drain}
\end{figure}\FloatBarrier
