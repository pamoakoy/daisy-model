\documentclass[a4paper]{article}

%\usepackage[left=1cm,top=1cm,right=1cm,nohead,nofoot]{geometry}
\usepackage[left=1cm,top=2cm,right=1cm]{geometry}
\usepackage[latin1]{inputenc}
\usepackage[T1]{fontenc}
\usepackage{natbib}
\bibliographystyle{apalike}
\usepackage{graphicx}
\usepackage{hyperref}
\usepackage{fancyhdr}
\usepackage{placeins}
\usepackage{comment}
\pagestyle{fancy}
\rhead{\today}

%\includecomment{text}
\excludecomment{text}

\begin{document}

\begin{text}
* Introduction

** About PLAP

** Role of PLAP simulations in R2D2

** Presentation of sites

** Selection criteria

* Model

* Data, model setup, and calibration

** Weather

** Management

** Soil Lindhart, MACRO

** Measurements

* Results

\end{text}

% \section*{Daisy simulation figures for Silstrup og Estrup}

\subsection*{Silstrup}

\begin{figure}[htbp]
  \begin{center}
    \includegraphics{fig/Silstrup-gw}
  \end{center}
  \caption{Silstrup groundwater table.  Automatic daily measurements
    are from P3.  Simulated low value is calculated from pressure in
    lowest unsatured numeric cell, typically located near drain.
    Simulated high value is calculated from pressure in highest
    saturated cell, typically farthest from the drain.}
  \label{fig:Silstrup-gw}
\end{figure}\FloatBarrier

\begin{figure}[htbp]
  \begin{center}
    \includegraphics[trim=0mm 5mm 0mm 0mm,clip]{fig/Silstrup-theta-SW025cm}\\
    \includegraphics[trim=0mm 5mm 0mm 0mm,clip]{fig/Silstrup-theta-SW060cm}\\
    \includegraphics{fig/Silstrup-theta-SW110cm}
  \end{center}
  \caption{Silstrup soil water content for measurement point S1.}
  \label{fig:Silstrup-theta}
\end{figure}\FloatBarrier

\begin{figure}[htbp]
  \begin{center}
    \includegraphics[trim=0mm 5mm 0mm 0mm,clip]{fig/Silstrup-sc-bromide}\\
    \includegraphics{fig/Silstrup-Bromide-horizontal}
  \end{center}
  \caption{Silstrup soil bromide content in 1.0 m depth (top) and 3.5
    m depth (bottom).  Sim (avg) is the average simulated
    concentration, Sim (fast) is the simulated concentration in the
    large (fast) pores.  S1 and S2 are suction cup measurements.
    H$n$.$m$ refer to measured values in different sections of
    horizontal filters.}
  \label{fig:Silstrup-bromide}
\end{figure}\FloatBarrier

\begin{figure}[htbp]
  \begin{center}
    \includegraphics{fig/Silstrup-horizontal}
  \end{center}
  \caption{Silstrup pesticide concentration in soil water at 3.5 meters depth.}
  \label{fig:Silstrup-horizontal}
\end{figure}\FloatBarrier

\begin{figure}[htbp]
  \begin{center}
    \includegraphics[trim=0mm 5mm 0mm 0mm,clip]{fig/Silstrup-leak150bromide}\\
    \includegraphics[trim=0mm 5mm 0mm 0mm,clip]{fig/Silstrup-leak150}\\
    \includegraphics{fig/Silstrup-leak150acc}
  \end{center}
  \caption{Silstrup simuleret leaching at 1.5 meter, 30 cm under bioporers.}
  \label{fig:Silstrup-leak150}
\end{figure}\FloatBarrier

\begin{figure}[htbp]
  \begin{center}
    \includegraphics[trim=0mm 5mm 0mm 0mm,clip]{fig/Silstrup-drain}\\
    \includegraphics{fig/Silstrup-drain-acc}
  \end{center}
  \caption{Silstrup drain flow, daily values and accumulated.}
  \label{fig:Silstrup-drain}
\end{figure}\FloatBarrier

\begin{figure}[htbp]
  \begin{center}
    \includegraphics[trim=0mm 5mm 0mm 0mm,clip]{fig/Silstrup-Bromide-weekly}\\
    \includegraphics{fig/Silstrup-Metamitron-weekly}
  \end{center}
  \caption{Weekly drain flow of bromide and metamitron.}
  \label{fig:Silstrup-weekly}
\end{figure}\FloatBarrier

\begin{figure}[htbp]
  \begin{center}
    \includegraphics[trim=0mm 5mm 0mm 0mm,clip]{fig/Silstrup-Dimethoate-weekly}\\
    \includegraphics[trim=0mm 5mm 0mm 0mm,clip]{fig/Silstrup-Fenpropimorph-weekly}\\
    \includegraphics{fig/Silstrup-Glyphosate-weekly}
  \end{center}
  \caption{Weekly drain flow of selected pesticides.}
  \label{fig:Silstrup-weekly2}
\end{figure}\FloatBarrier

\begin{figure}[htbp]
  \begin{center}
    \includegraphics[trim=0mm 5mm 0mm 0mm,clip]{fig/Silstrup-Bromide-acc}\\
    \includegraphics{fig/Silstrup-Metamitron-acc}
  \end{center}
  \caption{Accumulated drain flow of bromide and metamitron.}
  \label{fig:Silstrup-bromide-acc}
\end{figure}\FloatBarrier

\begin{figure}[htbp]
  \begin{center}
    \includegraphics[trim=0mm 5mm 0mm 0mm,clip]{fig/Silstrup-Dimethoate-acc}\\
    \includegraphics[trim=0mm 5mm 0mm 0mm,clip]{fig/Silstrup-Fenpropimorph-acc}\\
    \includegraphics{fig/Silstrup-Glyphosate-acc}
  \end{center}
  \caption{Accumulated drain flow of selected pesticides.}
  \label{fig:Silstrup-acc}
\end{figure}\FloatBarrier

\begin{figure}[htbp]
  \begin{center}
    \includegraphics{fig/Silstrup-colloid}
  \end{center}
  \caption{Colloids in drain water.}
  \label{fig:Silstrup-colloids}
\end{figure}\FloatBarrier

\begin{figure}[htbp]
  \begin{center}
    \includegraphics[trim=0mm 5mm 0mm 0mm,clip]{fig/Silstrup-biopore}\\
    \includegraphics{fig/Silstrup-biopore-acc}\\
  \end{center}
  \caption{Biopore activity in different soil layers.  The layers are
    ponded water, soil surface (top 3 cm), the rest of the plowing layer,
    the plow pan, and the the B horizon below plow pan down to 50 cm.}
  \label{fig:Silstrup-biopore}
\end{figure}\FloatBarrier

\begin{figure}[htbp]
  \begin{center}
    \includegraphics[trim=0mm 5mm 0mm 0mm,clip]{fig/Silstrup-biopore-drain}\\
    \includegraphics{fig/Silstrup-biopore-drain-acc}
  \end{center}
  \caption{Drain contribution through biopores from different soil
    layers.  The layers are ponded water, soil surface (top 3 cm), the
    rest of the plowing layer, the plow pan, and the the B horizon
    below plow pan down to 50 cm.}
  \label{fig:Silstrup-biopore-drain}
\end{figure}\FloatBarrier

\begin{figure}[htbp]
  \begin{center}
    \includegraphics[trim=0mm 5mm 0mm 0mm,clip]{fig/Silstrup-weather-glyphosate}\\
    \includegraphics[trim=0mm 5mm 0mm 0mm,clip]{fig/Silstrup-water-glyphosate}\\
    \includegraphics{fig/Silstrup-first-glyphosate}
  \end{center}
  \caption{Silstrup surface water and glyphosate in the first week
    after application.  Top graph shows fluxes affecting surface
    water.  Middle graph shows water storage on surface, as well as
    the water holding capcaity of the litter pack.  Bottom graph track
    the fate of glyphosate on the surface.}
  \label{fig:Silstrup-weather-glyphosate}
\end{figure}\FloatBarrier



\newcommand{\figsilstrup}[1]{\includegraphics[trim=8mm 0mm 12mm 7mm,clip]{fig/#1}}

\subsection*{Silstrup 2D}

\begin{figure}[htbp]\centering
  \begin{tabular}{ccc}
    \figsilstrup{Silstrup-pF-2000-5} & 
    \figsilstrup{Silstrup-pF-2000-6} & 
    \figsilstrup{Silstrup-pF-2000-7} \\
    \figsilstrup{Silstrup-pF-2000-8} & 
    \figsilstrup{Silstrup-pF-2000-9} & 
    \figsilstrup{Silstrup-pF-2000-10} \\
    \figsilstrup{Silstrup-pF-2000-11} & 
    \figsilstrup{Silstrup-pF-2000-12} & 
    \figsilstrup{Silstrup-pF-2001-1} \\
    \figsilstrup{Silstrup-pF-2001-2} & 
    \figsilstrup{Silstrup-pF-2001-3} & 
    \figsilstrup{Silstrup-pF-2001-4}
  \end{tabular}
  
  \caption{Silstrup soil potential at the end of each month since
    first application of bromide.  The z-axis denotes depth, the
    x-axis distance from drain.  There are tick marks for every meter.
    Blue denotes pF<0, white pF=1, yellow pF=2, orange pF=3, red pF=4,
    and black pF>5.}
\label{fig:Silstrup-pF-2000}
\end{figure}\FloatBarrier

\begin{figure}[htbp]\centering
  \begin{tabular}{ccc}
    \figsilstrup{Silstrup-pF-2001-5} & 
    \figsilstrup{Silstrup-pF-2001-6} & 
    \figsilstrup{Silstrup-pF-2001-7} \\
    \figsilstrup{Silstrup-pF-2001-8} & 
    \figsilstrup{Silstrup-pF-2001-9} & 
    \figsilstrup{Silstrup-pF-2001-10} \\
    \figsilstrup{Silstrup-pF-2001-11} & 
    \figsilstrup{Silstrup-pF-2001-12} & 
    \figsilstrup{Silstrup-pF-2002-1} \\
    \figsilstrup{Silstrup-pF-2002-2} & &
  \end{tabular}
  
  \caption{Silstrup soil potential at the end of each month second year
    after application of bromide.  The z-axis denotes depth, the
    x-axis distance from drain.  There are tick marks for every meter.
    Blue denotes pF<0, white pF=1, yellow pF=2, orange pF=3, red pF=4,
    and black pF>5.}
\label{fig:Silstrup-pF-2001}
\end{figure}\FloatBarrier

\begin{figure}[htbp]
  \centering
  \includegraphics{fig/Silstrup-water-horizontal-2000}
  
  \caption{Silstrup total horizontal water flux between 2000-5-1 and
    2001-5-1.  The flux is shown on the x-axis (positive away from
    drain) as a function of depth shown on the y-axis.  The graph
    labels are the distance from drain in centimeters.}
  \label{fig:Silstrup-water-2000-horizontal}
\end{figure}\FloatBarrier

\begin{figure}[htbp]
  \centering
  \includegraphics{fig/Silstrup-water-2000}
  
  \caption{Silstrup total vertical water flux between 2000-5-1 and
    2001-5-1.  The flux is shown on the y-axis (positive up) as a
    function of distance from drain shown on the y-axis.  The graph
    labels are depths in centimeters above surface.}
  \label{fig:Silstrup-water-2000}
\end{figure}\FloatBarrier

\begin{figure}[htbp]
  \centering
  \includegraphics{fig/Silstrup-water-biopore-2000}
  
  \caption{Silstrup total biopore water flux between 2000-5-1 and
    2001-5-1.  The flux is shown on the y-axis (positive up) as a
    function of distance from drain shown on the y-axis.  The graph
    labels are depths in centimeters above surface.}
  \label{fig:Silstrup-water-biopore-2000}
\end{figure}\FloatBarrier

\begin{figure}[htbp]
  \centering
  \includegraphics{fig/Silstrup-water-horizontal-2001}
  
  \caption{Silstrup total horizontal water flux between 2001-5-1 and
    2002-3-1.  The flux is shown on the x-axis (positive away from
    drain) as a function of depth shown on the y-axis.  The graph
    labels are the distance from drain in centimeters.}
  \label{fig:Silstrup-water-2001-horizontal}
\end{figure}\FloatBarrier

\begin{figure}[htbp]
  \centering
  \includegraphics{fig/Silstrup-water-2001}
  
  \caption{Silstrup total vertical water flux between 2001-5-1 and
    2002-3-1.  The flux is shown on the y-axis (positive up) as a
    function of distance from drain shown on the y-axis.  The graph
    labels are depths in centimeters above surface.}
  \label{fig:Silstrup-water-2001}
\end{figure}\FloatBarrier

\begin{figure}[htbp]
  \centering
  \includegraphics{fig/Silstrup-water-biopore-2001}
  
  \caption{Silstrup total biopore water flux between 2001-5-1 and
    2002-3-1.  The flux is shown on the y-axis (positive up) as a
    function of distance from drain shown on the y-axis.  The graph
    labels are depths in centimeters above surface.}
  \label{fig:Silstrup-water-biopore-2001}
\end{figure}\FloatBarrier

\begin{figure}[htbp]\centering
  \begin{tabular}{ccc}
    \figsilstrup{Silstrup-M-Bromide-2000-5} & 
    \figsilstrup{Silstrup-M-Bromide-2000-6} & 
    \figsilstrup{Silstrup-M-Bromide-2000-7} \\
    \figsilstrup{Silstrup-M-Bromide-2000-8} & 
    \figsilstrup{Silstrup-M-Bromide-2000-9} & 
    \figsilstrup{Silstrup-M-Bromide-2000-10} \\
    \figsilstrup{Silstrup-M-Bromide-2000-11} & 
    \figsilstrup{Silstrup-M-Bromide-2000-12} & 
    \figsilstrup{Silstrup-M-Bromide-2001-1} \\
    \figsilstrup{Silstrup-M-Bromide-2001-2} & 
    \figsilstrup{Silstrup-M-Bromide-2001-3} & 
    \figsilstrup{Silstrup-M-Bromide-2001-4}
  \end{tabular}
  
  \caption{Silstrup bromide soil content at the end of each month
    since first application of bromide.  The z-axis denotes depth, the
    x-axis distance from drain.  There are tick marks for every
    meter. The color scale is white<10 pg/l, yellow=1 ng/l, orange=0.1
    $\mu$g/l, red=10 $\mu$g/l, and black>1 mg/l}
\label{fig:Silstrup-Bromide-2000}
\end{figure}\FloatBarrier

\begin{figure}[htbp]\centering
  \begin{tabular}{ccc}
    \figsilstrup{Silstrup-M-Bromide-2001-5} & 
    \figsilstrup{Silstrup-M-Bromide-2001-6} & 
    \figsilstrup{Silstrup-M-Bromide-2001-7} \\
    \figsilstrup{Silstrup-M-Bromide-2001-8} & 
    \figsilstrup{Silstrup-M-Bromide-2001-9} & 
    \figsilstrup{Silstrup-M-Bromide-2001-10} \\
    \figsilstrup{Silstrup-M-Bromide-2001-11} & 
    \figsilstrup{Silstrup-M-Bromide-2001-12} & 
    \figsilstrup{Silstrup-M-Bromide-2002-1} \\
    \figsilstrup{Silstrup-M-Bromide-2002-2} &  & 
  \end{tabular}
  
  \caption{Silstrup bromide soil content at the end of each month
    second year after application of bromide.  The z-axis denotes
    depth, the x-axis distance from drain.  There are tick marks for
    every meter. The color scale is white<10 pg/l, yellow=1 ng/l,
    orange=0.1 $\mu$g/l, red=10 $\mu$g/l, and black>1 mg/l}
\label{fig:Silstrup-Bromide-2001}
\end{figure}\FloatBarrier

\begin{figure}[htbp]
  \centering
  \includegraphics{fig/Silstrup-Bromide-horizontal-2000}
  
  \caption{Silstrup total horizontal bromide flow between 2000-5-1 and
    2001-5-1.  The flow is shown on the x-axis (positive away from
    drain) as a function of depth shown on the y-axis.  The graph
    labels are the distance from drain in centimeters.}
  \label{fig:Silstrup-Bromide-2000-horizontal}
\end{figure}\FloatBarrier

\begin{figure}[htbp]
  \centering
  \includegraphics{fig/Silstrup-Bromide-2000}
  
  \caption{Silstrup total vertical bromide flow between 2000-5-1 and
    2001-5-1.  The flow is shown on the y-axis (positive up) as a
    function of distance from drain shown on the y-axis.  The graph
    labels are depths in centimeters above surface.}
  \label{fig:Silstrup-Bromide-2000-vertical}
\end{figure}\FloatBarrier

\begin{figure}[htbp]
  \centering
  \includegraphics{fig/Silstrup-Bromide-biopore-2000}
  
  \caption{Silstrup total biopore bromide flow between 2000-5-1 and
    2001-5-1.  The flow is shown on the y-axis (positive up) as a
    function of distance from drain shown on the y-axis.  The graph
    labels are depths in centimeters above surface.}
  \label{fig:Silstrup-Bromide-biopore-2000}
\end{figure}\FloatBarrier

\begin{figure}[htbp]
  \centering
  \includegraphics{fig/Silstrup-Bromide-horizontal-2001}
  
  \caption{Silstrup total horizontal bromide flow between 2001-5-1 and
    2002-3-1.  The flow is shown on the x-axis (positive away from
    drain) as a function of depth shown on the y-axis.  The graph
    labels are the distance from drain in centimeters.}
  \label{fig:Silstrup-Bromide-2001-horizontal}
\end{figure}\FloatBarrier

\begin{figure}[htbp]
  \centering
  \includegraphics{fig/Silstrup-Bromide-2001}
  
  \caption{Silstrup total vertical bromide flow between 2001-5-1 and
    2002-3-1.  The flow is shown on the y-axis (positive up) as a
    function of distance from drain shown on the y-axis.  The graph
    labels are depths in centimeters above surface.}
  \label{fig:Silstrup-Bromide-2001-vertical}
\end{figure}\FloatBarrier

\begin{figure}[htbp]
  \centering
  \includegraphics{fig/Silstrup-Bromide-biopore-2001}
  
  \caption{Silstrup total biopore bromide flow between 2001-5-1 and
    2002-3-1.  The flow is shown on the y-axis (positive up) as a
    function of distance from drain shown on the y-axis.  The graph
    labels are depths in centimeters above surface.}
  \label{fig:Silstrup-Bromide-biopore-2001}
\end{figure}\FloatBarrier

\begin{figure}[htbp]
  \centering
  \includegraphics{fig/Silstrup-Metamitron-horizontal-2000}
  
  \caption{Silstrup total horizontal metamitron flow between 2000-5-1 and
    2001-5-1.  The flow is shown on the x-axis (positive away from
    drain) as a function of depth shown on the y-axis.  The graph
    labels are the distance from drain in centimeters.}
  \label{fig:Silstrup-Metamitron-2000-horizontal}
\end{figure}\FloatBarrier

\begin{figure}[htbp]
  \centering
  \includegraphics{fig/Silstrup-Metamitron-2000}
  
  \caption{Silstrup total vertical metamitron flow between 2000-5-1 and
    2001-5-1.  The flow is shown on the y-axis (positive up) as a
    function of distance from drain shown on the y-axis.  The graph
    labels are depths in centimeters above surface.}
  \label{fig:Silstrup-Metamitron-2000-vertical}
\end{figure}\FloatBarrier

\begin{figure}[htbp]
  \centering
  \includegraphics{fig/Silstrup-Metamitron-biopore-2000}
  
  \caption{Silstrup total biopore metamitron flow between 2000-5-1 and
    2001-5-1.  The flow is shown on the y-axis (positive up) as a
    function of distance from drain shown on the y-axis.  The graph
    labels are depths in centimeters above surface.}
  \label{fig:Silstrup-Metamitron-biopore-2000}
\end{figure}\FloatBarrier

\begin{figure}[htbp]\centering
  \begin{tabular}{ccc}
    \figsilstrup{Silstrup-M-Glyphosate-2001-5} & 
    \figsilstrup{Silstrup-M-Glyphosate-2001-6} & 
    \figsilstrup{Silstrup-M-Glyphosate-2001-7} \\
    \figsilstrup{Silstrup-M-Glyphosate-2001-8} & 
    \figsilstrup{Silstrup-M-Glyphosate-2001-9} & 
    \figsilstrup{Silstrup-M-Glyphosate-2001-10} \\
    \figsilstrup{Silstrup-M-Glyphosate-2001-11} & 
    \figsilstrup{Silstrup-M-Glyphosate-2001-12} & 
    \figsilstrup{Silstrup-M-Glyphosate-2002-1} \\
    \figsilstrup{Silstrup-M-Glyphosate-2002-2} & & 
  \end{tabular}
  
  \caption{Silstrup glyphosate soil content at the end of each month
    since first application of bromide.  The z-axis denotes depth, the
    x-axis distance from drain.  There are tick marks for every
    meter. The color scale is white<10 pg/l, yellow=1 ng/l, orange=0.1
    $\mu$g/l, red=10 $\mu$g/l, and black>1 mg/l}
\label{fig:Silstrup-M-Glyphosate-2001}
\end{figure}\FloatBarrier

\begin{figure}[htbp]\centering
  \begin{tabular}{ccc}
    \figsilstrup{Silstrup-C-Glyphosate-2001-5} & 
    \figsilstrup{Silstrup-C-Glyphosate-2001-6} & 
    \figsilstrup{Silstrup-C-Glyphosate-2001-7} \\
    \figsilstrup{Silstrup-C-Glyphosate-2001-8} & 
    \figsilstrup{Silstrup-C-Glyphosate-2001-9} & 
    \figsilstrup{Silstrup-C-Glyphosate-2001-10} \\
    \figsilstrup{Silstrup-C-Glyphosate-2001-11} & 
    \figsilstrup{Silstrup-C-Glyphosate-2001-12} & 
    \figsilstrup{Silstrup-C-Glyphosate-2002-1} \\
    \figsilstrup{Silstrup-C-Glyphosate-2002-2} &  & 
  \end{tabular}
  
  \caption{Silstrup glyphosate soil water concentration at the end of
    each month since one year after first application of bromide.  The
    z-axis denotes depth, the x-axis distance from drain.  There are
    tick marks for every meter. The color scale is white<10 pg/l,
    yellow=1 ng/l, orange=0.1 $\mu$g/l, red=10 $\mu$g/l, and black>1
    mg/l}
\label{fig:Silstrup-C-Glyphosate-2001}
\end{figure}\FloatBarrier

\begin{figure}[htbp]
  \centering
  \includegraphics{fig/Silstrup-Glyphosate-horizontal-2001}
  
  \caption{Silstrup total horizontal glyphosate flow between 2001-5-1 and
    2002-3-1.  The flow is shown on the x-axis (positive away from
    drain) as a function of depth shown on the y-axis.  The graph
    labels are the distance from drain in centimeters.}
  \label{fig:Silstrup-Glyphosate-2001-horizontal}
\end{figure}\FloatBarrier

\begin{figure}[htbp]
  \centering
  \includegraphics{fig/Silstrup-Glyphosate-2001}
  
  \caption{Silstrup total vertical glyphosate flow between 2001-5-1 and
    2002-3-1.  The flow is shown on the y-axis (positive up) as a
    function of distance from drain shown on the y-axis.  The graph
    labels are depths in centimeters above surface.}
  \label{fig:Silstrup-Glyphosate-2001-vertical}
\end{figure}\FloatBarrier

\begin{figure}[htbp]
  \centering
  \includegraphics{fig/Silstrup-Glyphosate-biopore-2001}
  
  \caption{Silstrup total biopore glyphosate flow between 2001-5-1 and
    2002-3-1.  The flow is shown on the y-axis (positive up) as a
    function of distance from drain shown on the y-axis.  The graph
    labels are depths in centimeters above surface.}
  \label{fig:Silstrup-Glyphosate-biopore-2001}
\end{figure}\FloatBarrier


%% \subsection*{Estrup}

\begin{figure}[htbp]
  \begin{center}
    \includegraphics{fig/Estrup-theta-SW025cm}
  \end{center}
  \caption{Soilstrup soil water content for measurement point S1.}
  \label{fig:Estrup-theta}
\end{figure}\FloatBarrier

\begin{figure}[htbp]
  \begin{center}
    \includegraphics[trim=0mm 5mm 0mm 0mm,clip]{fig/Estrup-sc-bromide}\\
    \includegraphics{fig/Estrup-Bromide-horizontal}
  \end{center}
  \caption{Estrup soil bromide content in 1.0 m depth (top) and 3.5
    m depth (bottom).  Sim (avg) is the average simulated
    concentration, Sim (fast) is the simulated concentration in the
    large (fast) pores.  S1 and S2 are suction cup measurements.
    H$1$.$m$ refer to measured values in different sections of
    horizontal filters.}
  \label{fig:Estrup-bromide}
\end{figure}\FloatBarrier


\begin{figure}[htbp]
  \begin{center}
    \includegraphics{fig/Estrup-horizontal}
  \end{center}
  \caption{Estrup pesticide concentration in soil water at 3.5 meters
    depth.}
  \label{fig:Estrup-horizontal}
\end{figure}\FloatBarrier

\begin{figure}[htbp]
  \begin{center}
    \includegraphics[trim=0mm 5mm 0mm 0mm,clip]{fig/Estrup-leak150bromide}\\
    \includegraphics[trim=0mm 5mm 0mm 0mm,clip]{fig/Estrup-leak150}\\
    \includegraphics{fig/Estrup-leak150acc}
  \end{center}
  \caption{Estrup simuleret leaching at 1.5 meter, 30 cm under bioporers.}
  \label{fig:Estrup-leak150}
\end{figure}\FloatBarrier

\begin{figure}[htbp]
  \begin{center}
    \includegraphics[trim=0mm 5mm 0mm 0mm,clip]{fig/Estrup-drain}\\
    \includegraphics{fig/Estrup-drain-acc}
  \end{center}
  \caption{Estrup drain flow, daily values and accumulated.}
  \label{fig:Estrup-drain}
\end{figure}\FloatBarrier

\begin{figure}[htbp]
  \begin{center}
    \includegraphics[trim=0mm 5mm 0mm 0mm,clip]{fig/Estrup-Bromide-weekly}\\
    \includegraphics{fig/Estrup-Bromide-acc}
  \end{center}
  \caption{Estrup weekly and accumulated drain flow of bromide.}
  \label{fig:Estrup-bromide-weekly}
\end{figure}\FloatBarrier


\begin{figure}[htbp]
  \begin{center}
    \includegraphics[trim=0mm 5mm 0mm 0mm,clip]{fig/Estrup-Dimethoate-weekly}\\
    \includegraphics[trim=0mm 5mm 0mm 0mm,clip]{fig/Estrup-Fenpropimorph-weekly}\\
    \includegraphics{fig/Estrup-Glyphosate-weekly}\\
  \end{center}
  \caption{Estrup weekly drain flow of selected pesticides.}
  \label{fig:Estrup-weekly}
\end{figure}\FloatBarrier

\begin{figure}[htbp]
  \begin{center}
    \includegraphics[trim=0mm 5mm 0mm 0mm,clip]{fig/Estrup-Dimethoate-acc}\\
    \includegraphics[trim=0mm 5mm 0mm 0mm,clip]{fig/Estrup-Fenpropimorph-acc}\\
    \includegraphics{fig/Estrup-Glyphosate-acc}\\
  \end{center}
  \caption{Estrup accumulated drain flow of selected pesticides.}
  \label{fig:Estrup-acc}
\end{figure}\FloatBarrier

\begin{figure}[htbp]
  \begin{center}
    \includegraphics{fig/Estrup-colloid}
  \end{center}
  \caption{Colloids in drain water.}
  \label{fig:Estrup-colloids}
\end{figure}\FloatBarrier

\begin{figure}[htbp]
  \begin{center}
    \includegraphics[trim=0mm 5mm 0mm 0mm,clip]{fig/Estrup-biopore}\\
    \includegraphics{fig/Estrup-biopore-acc}\\
  \end{center}
  \caption{Biopore activity in different soil layers.  The layers are
    ponded water, soil surface (top 3 cm), the rest of the plowing layer,
    the plow pan, and the the B horizon below plow pan down to 50 cm.}
  \label{fig:Estrup-biopore}
\end{figure}\FloatBarrier

\begin{figure}[htbp]
  \begin{center}
    \includegraphics[trim=0mm 5mm 0mm 0mm,clip]{fig/Estrup-biopore-drain}\\
    \includegraphics{fig/Estrup-biopore-drain-acc}
  \end{center}
  \caption{Drain contribution through biopores from different soil
    layers.  The layers are ponded water, soil surface (top 3 cm), the
    rest of the plowing layer, the plow pan, and the the B horizon
    below plow pan down to 50 cm.}
  \label{fig:Estrup-biopore-drain}
\end{figure}\FloatBarrier


%% \newcommand{\figestrupl}[1]{\hspace*{-1cm}\includegraphics[trim=12mm 0mm 17mm 9mm,clip]{fig/#1}}
\newcommand{\figestrup}[1]{\includegraphics[trim=12mm 0mm 17mm 9mm,clip]{fig/#1}}

\chapter{Estrup 2D dynamics}

In this appendix the simulated 2D dynamics for water, bromide and
pesticides of the Estrup site is presented.  There are no
measurements to compare with.

\begin{figure}[htbp]\centering
  \begin{tabular}{ccc}
    \figestrupl{Estrup-pF-2000-5} & 
    \figestrup{Estrup-pF-2000-6} & 
    \figestrup{Estrup-pF-2000-7} \\
    \figestrupl{Estrup-pF-2000-8} & 
    \figestrup{Estrup-pF-2000-9} & 
    \figestrup{Estrup-pF-2000-10} \\
    \figestrupl{Estrup-pF-2000-11} & 
    \figestrup{Estrup-pF-2000-12} & 
    \figestrup{Estrup-pF-2001-1} \\
    \figestrupl{Estrup-pF-2001-2} & 
    \figestrup{Estrup-pF-2001-3} & 
    \figestrup{Estrup-pF-2001-4}
  \end{tabular}
  
  \caption{Estrup soil potential at the end of each month since first
    application of bromide.  The z-axis denotes depth, the x-axis
    distance from drain.  There are tick marks for every meter.  Blue
    denotes pF<0, white pF=1, yellow pF=2, orange pF=3, red pF=4, and
    black pF>5.}
\label{fig:Estrup-pF-2000}
\end{figure}\FloatBarrier

\begin{figure}[htbp]\centering
  \begin{tabular}{ccc}
    \figestrupl{Estrup-pF-2001-5} & 
    \figestrup{Estrup-pF-2001-6} & 
    \figestrup{Estrup-pF-2001-7} \\
    \figestrupl{Estrup-pF-2001-8} & 
    \figestrup{Estrup-pF-2001-9} & 
    \figestrup{Estrup-pF-2001-10} \\
    \figestrupl{Estrup-pF-2001-11} & 
    \figestrup{Estrup-pF-2001-12} & 
    \figestrup{Estrup-pF-2002-1} \\
    \figestrupl{Estrup-pF-2002-2} & 
    \figestrup{Estrup-pF-2002-3} & 
    \figestrup{Estrup-pF-2002-4}
  \end{tabular}
  
  \caption{Estrup soil potential at the end of each month second year
    after application of bromide.  The z-axis denotes depth, the
    x-axis distance from drain.  There are tick marks for every meter.
    Blue denotes pF<0, white pF=1, yellow pF=2, orange pF=3, red pF=4,
    and black pF>5.}
\label{fig:Estrup-pF-2001}
\end{figure}\FloatBarrier

\begin{figure}[htbp]
  \centering
  \fig{Estrup-water-horizontal-2000}
  
  \caption{Estrup total horizontal water flux between 2000-5-1 and
    2001-5-1.  The flux is shown on the x-axis (positive away from
    drain) as a function of depth shown on the y-axis.  The graph
    labels are the distance from drain in centimeters.}
  \label{fig:Estrup-water-2000-horizontal}
\end{figure}\FloatBarrier

\begin{figure}[htbp]
  \centering
  \fig{Estrup-water-2000}
  
  \caption{Estrup total vertical water flux between 2000-5-1 and
    2001-5-1.  The flux is shown on the y-axis (positive up) as a
    function of distance from drain shown on the y-axis.  The graph
    labels are depths in centimeters above surface.}
  \label{fig:Estrup-water-2000}
\end{figure}\FloatBarrier

\begin{figure}[htbp]
  \centering
  \fig{Estrup-water-biopore-2000}
  
  \caption{Estrup total biopore water flux between 2000-5-1 and
    2001-5-1.  The flux is shown on the y-axis (positive up) as a
    function of distance from drain shown on the y-axis.  The graph
    labels are depths in centimeters above surface.}
  \label{fig:Estrup-water-biopore-2000}
\end{figure}\FloatBarrier

\begin{figure}[htbp]
  \centering
  \fig{Estrup-water-horizontal-2001}
  
  \caption{Estrup total horizontal water flux between 2001-5-1 and
    2002-5-1.  The flux is shown on the x-axis (positive away from
    drain) as a function of depth shown on the y-axis.  The graph
    labels are the distance from drain in centimeters.}
  \label{fig:Estrup-water-2001-horizontal}
\end{figure}\FloatBarrier

\begin{figure}[htbp]
  \centering
  \fig{Estrup-water-2001}
  
  \caption{Estrup total vertical water flux between 2001-5-1 and
    2002-5-1.  The flux is shown on the y-axis (positive up) as a
    function of distance from drain shown on the y-axis.  The graph
    labels are depths in centimeters above surface.}
  \label{fig:Estrup-water-2001}
\end{figure}\FloatBarrier

\begin{figure}[htbp]
  \centering
  \fig{Estrup-water-biopore-2001}
  
  \caption{Estrup total biopore water flux between 2001-5-1 and
    2002-5-1.  The flux is shown on the y-axis (positive up) as a
    function of distance from drain shown on the y-axis.  The graph
    labels are depths in centimeters above surface.}
  \label{fig:Estrup-water-biopore-2001}
\end{figure}\FloatBarrier

\begin{figure}[htbp]\centering
  \begin{tabular}{ccc}
    \figestrupl{Estrup-M-Bromide-2000-5} & 
    \figestrup{Estrup-M-Bromide-2000-6} & 
    \figestrup{Estrup-M-Bromide-2000-7} \\
    \figestrupl{Estrup-M-Bromide-2000-8} & 
    \figestrup{Estrup-M-Bromide-2000-9} & 
    \figestrup{Estrup-M-Bromide-2000-10} \\
    \figestrupl{Estrup-M-Bromide-2000-11} & 
    \figestrup{Estrup-M-Bromide-2000-12} & 
    \figestrup{Estrup-M-Bromide-2001-1} \\
    \figestrupl{Estrup-M-Bromide-2001-2} & 
    \figestrup{Estrup-M-Bromide-2001-3} & 
    \figestrup{Estrup-M-Bromide-2001-4}
  \end{tabular}
  
  \caption{Estrup bromide soil content at the end of each month since
    first application of bromide.  The z-axis denotes depth, the x-axis distance from drain.  There are tick marks for every
    meter. The color scale is white<10 pg/l, yellow=1 ng/l,
    orange=0.1 $\mu$g/l, red=10 $\mu$g/l, and black>1 mg/l}
\label{fig:Estrup-Bromide-2000}
\end{figure}\FloatBarrier

\begin{figure}[htbp]\centering
  \begin{tabular}{ccc}
    \figestrupl{Estrup-M-Bromide-2001-5} & 
    \figestrup{Estrup-M-Bromide-2001-6} & 
    \figestrup{Estrup-M-Bromide-2001-7} \\
    \figestrupl{Estrup-M-Bromide-2001-8} & 
    \figestrup{Estrup-M-Bromide-2001-9} & 
    \figestrup{Estrup-M-Bromide-2001-10} \\
    \figestrupl{Estrup-M-Bromide-2001-11} & 
    \figestrup{Estrup-M-Bromide-2001-12} & 
    \figestrup{Estrup-M-Bromide-2002-1} \\
    \figestrupl{Estrup-M-Bromide-2002-2} & 
    \figestrup{Estrup-M-Bromide-2002-3} & 
    \figestrup{Estrup-M-Bromide-2002-4}
  \end{tabular}
  
  \caption{Estrup bromide soil content at the end of each month second
    year after application of bromide.  The z-axis denotes depth, the
    x-axis distance from drain.  There are tick marks for every
    meter. The color scale is white<10 pg/l, yellow=1 ng/l, orange=0.1
    $\mu$g/l, red=10 $\mu$g/l, and black>1 mg/l}
\label{fig:Estrup-Bromide-2001}
\end{figure}\FloatBarrier

\begin{figure}[htbp]
  \centering
  \fig{Estrup-Bromide-horizontal-2000}
  
  \caption{Estrup total horizontal bromide flow between 2000-5-1 and
    2001-5-1.  The flow is shown on the x-axis (positive away from
    drain) as a function of depth shown on the y-axis.  The graph
    labels are the distance from drain in centimeters.}
  \label{fig:Estrup-Bromide-2000-horizontal}
\end{figure}\FloatBarrier

\begin{figure}[htbp]
  \centering
  \fig{Estrup-Bromide-2000}
  
  \caption{Estrup total vertical bromide flow between 2000-5-1 and
    2001-5-1.  The flow is shown on the y-axis (positive up) as a
    function of distance from drain shown on the y-axis.  The graph
    labels are depths in centimeters above surface.}
  \label{fig:Estrup-Bromide-2000-vertical}
\end{figure}\FloatBarrier

\begin{figure}[htbp]
  \centering
  \fig{Estrup-Bromide-biopore-2000}
  
  \caption{Estrup total biopore bromide flow between 2000-5-1 and
    2001-5-1.  The flow is shown on the y-axis (positive up) as a
    function of distance from drain shown on the y-axis.  The graph
    labels are depths in centimeters above surface.}
  \label{fig:Estrup-Bromide-biopore-2000}
\end{figure}\FloatBarrier

\begin{figure}[htbp]
  \centering
  \fig{Estrup-Bromide-horizontal-2001}
  
  \caption{Estrup total horizontal bromide flow between 2001-5-1 and
    2002-5-1.  The flow is shown on the x-axis (positive away from
    drain) as a function of depth shown on the y-axis.  The graph
    labels are the distance from drain in centimeters.}
  \label{fig:Estrup-Bromide-2001-horizontal}
\end{figure}\FloatBarrier

\begin{figure}[htbp]
  \centering
  \fig{Estrup-Bromide-2001}
  
  \caption{Estrup total vertical bromide flow between 2001-5-1 and
    2002-5-1.  The flow is shown on the y-axis (positive up) as a
    function of distance from drain shown on the y-axis.  The graph
    labels are depths in centimeters above surface.}
  \label{fig:Estrup-Bromide-2001-vertical}
\end{figure}\FloatBarrier

\begin{figure}[htbp]
  \centering
  \fig{Estrup-Bromide-biopore-2001}
  
  \caption{Estrup total biopore bromide flow between 2001-5-1 and
    2002-5-1.  The flow is shown on the y-axis (positive up) as a
    function of distance from drain shown on the y-axis.  The graph
    labels are depths in centimeters above surface.}
  \label{fig:Estrup-Bromide-biopore-2001}
\end{figure}\FloatBarrier

\begin{figure}[htbp]\centering
  \begin{tabular}{ccc}
    \figestrupl{Estrup-M-Glyphosate-2000-5} & 
    \figestrup{Estrup-M-Glyphosate-2000-6} & 
    \figestrup{Estrup-M-Glyphosate-2000-7} \\
    \figestrupl{Estrup-M-Glyphosate-2000-8} & 
    \figestrup{Estrup-M-Glyphosate-2000-9} & 
    \figestrup{Estrup-M-Glyphosate-2000-10} \\
    \figestrupl{Estrup-M-Glyphosate-2000-11} & 
    \figestrup{Estrup-M-Glyphosate-2000-12} & 
    \figestrup{Estrup-M-Glyphosate-2001-1} \\
    \figestrupl{Estrup-M-Glyphosate-2001-2} & 
    \figestrup{Estrup-M-Glyphosate-2001-3} & 
    \figestrup{Estrup-M-Glyphosate-2001-4}
  \end{tabular}
  
  \caption{Estrup glyphosate soil content at the end of each month
    since first application of bromide.  The z-axis denotes depth, the
    x-axis distance from drain.  There are tick marks for every
    meter. The color scale is white<10 pg/l, yellow=1 ng/l, orange=0.1
    $\mu$g/l, red=10 $\mu$g/l, and black>1 mg/l}
\label{fig:Estrup-M-Glyphosate-2000}
\end{figure}\FloatBarrier

\begin{figure}[htbp]\centering
  \begin{tabular}{ccc}
    \figestrupl{Estrup-C-Glyphosate-2000-5} & 
    \figestrup{Estrup-C-Glyphosate-2000-6} & 
    \figestrup{Estrup-C-Glyphosate-2000-7} \\
    \figestrupl{Estrup-C-Glyphosate-2000-8} & 
    \figestrup{Estrup-C-Glyphosate-2000-9} & 
    \figestrup{Estrup-C-Glyphosate-2000-10} \\
    \figestrupl{Estrup-C-Glyphosate-2000-11} & 
    \figestrup{Estrup-C-Glyphosate-2000-12} & 
    \figestrup{Estrup-C-Glyphosate-2001-1} \\
    \figestrupl{Estrup-C-Glyphosate-2001-2} & 
    \figestrup{Estrup-C-Glyphosate-2001-3} & 
    \figestrup{Estrup-C-Glyphosate-2001-4}
  \end{tabular}
  
  \caption{Estrup glyphosate soil water concentration at the end of
    each month since first application of bromide.  The z-axis denotes
    depth, the x-axis distance from drain.  There are tick marks for
    every meter. The color scale is white<10 pg/l, yellow=1 ng/l, orange=0.1
    $\mu$g/l, red=10 $\mu$g/l, and black>1 mg/l}
\label{fig:Estrup-C-Glyphosate-2000}
\end{figure}\FloatBarrier

\begin{figure}[htbp]
  \centering
  \fig{Estrup-Glyphosate-horizontal-2000}
  
  \caption{Estrup total horizontal glyphosate flow between 2000-5-1 and
    2001-5-1.  The flow is shown on the x-axis (positive away from
    drain) as a function of depth shown on the y-axis.  The graph
    labels are the distance from drain in centimeters.}
  \label{fig:Estrup-Glyphosate-2000-horizontal}
\end{figure}\FloatBarrier

\begin{figure}[htbp]
  \centering
  \fig{Estrup-Glyphosate-2000}
  
  \caption{Estrup total vertical glyphosate flow between 2000-5-1 and
    2001-5-1.  The flow is shown on the y-axis (positive up) as a
    function of distance from drain shown on the y-axis.  The graph
    labels are depths in centimeters above surface.}
  \label{fig:Estrup-Glyphosate-2000}
\end{figure}\FloatBarrier

\begin{figure}[htbp]
  \centering
  \fig{Estrup-Glyphosate-biopore-2000}
  
  \caption{Estrup total biopore glyphosate flow between 2000-5-1 and
    2001-5-1.  The flow is shown on the y-axis (positive up) as a
    function of distance from drain shown on the y-axis.  The graph
    labels are depths in centimeters above surface.}
  \label{fig:Estrup-Glyphosate-biopore-2000}
\end{figure}\FloatBarrier



\begin{text}
\section{Discussion}

* Silstrup drain period may be too short due to heterogenity.

* Silstrup drain flow follow first rain, indicate surface flow and anisotropy

* Silstrup Metamitron not p�virket af halverinsgtid

\section{Conclusion and further work}

It is possible to explain the measured data based on the processes
included in the present model, with some caveats
\begin{itemize}
\item The high degree of heterogenity found in the Estrup site would
  require a detailed 3D model of the entire area to model
  mechanistically, the current 2D model setup can at best be viewed as
  ``effective parameters''.
\item The drain are activated more abrubtly (first year) and sooner
  (second year) in reality, than what we have been able to simulate
  for the Silsrup site.  This is particularily noticable for the
  Bromide measurements.  
\item Bromide is found in some of the horizontal filters at 3.5 meters
  depth at both sites in the first measurements after application of
  Bromide.  No pesticides are found in those depth though.  It does
  indicate a transport way for non-sorbing solutes that we cannot
  model.  One possibility is large scale fractures, this suggestion is
  supported by other work at GEUS.
\end{itemize}

Furthermore, the work presented in thus report cannot count as a
validation of either the parameterization or the conceptual model,
there are too many unknowns where we have had to guess or calibrate
parameters, and there are likely many different setups that would have
resultet in as good or better fits to the measured results.  It is not
certain that a proper validation is possible, but by applying the
model on more datasets we should be able to gain confidence in it it.

Apart from applying the model on more and larger datasets, two areas
in particular need further work.
\begin{itemize}
\item The surface processes (flow, litter storage, decomposition) are
  very important, especially for the Silstrup site, but only the
  minimal work on those to get an effect have been done in this project.
\item The pesticide processes (colloid transport, sorbtion sites,
  sorption kinetics, and docposition) are parameterized from
  litterature values and ``best guesses'', many of those parameters
  should be adabted to direct local measurements.
\end{itemize}
\end{text}
  
\end{document}
